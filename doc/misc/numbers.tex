%%%%%%%%%%%%%%%%%%%%%%%%%%%%%%%%%%%%%%%%%%%%%%%%%%
\subsection*{\Condor{shadow} Exit Codes}
%%%%%%%%%%%%%%%%%%%%%%%%%%%%%%%%%%%%%%%%%%%%%%%%%%

%detailed in Table~\ref{shadow-exit-codes}. 

% shadow error codes
\begin{center}
\begin{table}[hbt]
\begin{tabular}{|c|c|c|} \hline
\emph{Value} & \emph{Error Name} & \emph{Description} \\ \hline \hline
4   &   JOB\_EXCEPTION    & the job exited with an exception \\ \hline
44  &   DPRINTF\_ERROR    & there was a fatal error with dprintf() \\ \hline
100 &   JOB\_EXITED       & the job exited (not killed)  \\ \hline
101 &   JOB\_CKPTED       & the job did produce a checkpoint  \\ \hline
102 &   JOB\_KILLED       & the job was killed     \\ \hline
103 &   JOB\_COREDUMPED   & the job was killed and a core file was produced  \\ \hline
105 &   JOB\_NO\_MEM      & not enough memory to start the condor\_shadow \\ \hline
106 &   JOB\_SHADOW\_USAGE & incorrect arguments to condor\_shadow \\ \hline
107 &   JOB\_NOT\_CKPTED  & the job vacated without a checkpoint \\ \hline
107 &   JOB\_SHOULD\_REQUEUE  & (!) same number used, to achieve the same behavior. However, JOB\_NOT\_CKPTED is irrelevant unless it is a standard universe job. This exit code implies that we want the job to be put back in the job queue and run again. \\ \hline
108 &   JOB\_NOT\_STARTED  & can not connect to the \Condor{startd} or request refused \\ \hline
109 &   JOB\_BAD\_STATUS  & job status != RUNNING on start up \\ \hline
110 &   JOB\_EXEC\_FAILED & exec failed for some reason other than ENOMEM \\ \hline
111 &   JOB\_NO\_CKPT\_FILE & there is no checkpoint file (as it was lost) \\ \hline
112 &   JOB\_SHOULD\_HOLD & the job should be put on hold \\ \hline
113 &   JOB\_SHOULD\_REMOVE & the job should be removed \\ \hline
\end{tabular}
%\caption{\label{shadow-exit-codes}\Condor{shadow} exit codes}
\label{shadow-exit-codes}
\end{table}
\end{center}

