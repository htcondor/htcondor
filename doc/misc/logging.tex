%%%%%%%%%%%%%%%%%%%%%%%%%%%%%%%%%%%%%%%%%%%%%%%%%%%%%%%%%%%%%%%%%%%%%%
\section{\label{sec:logging}Logging in HTCondor}
%%%%%%%%%%%%%%%%%%%%%%%%%%%%%%%%%%%%%%%%%%%%%%%%%%%%%%%%%%%%%%%%%%%%%%
\index{logging|(}
HTCondor records many types of information in a variety of logs.
Administration may require locating and using the
contents of a log to debug issues.
Listed here are details of the logs, to aid in identification.

%%%%%%%%%%%%%%%%%%%%%%%%%%%%%%%%%%%%%%%%%%%%%%%%%%%%%%%%%%%%%%%%%%%%%%
\subsection{\label{sec:job-systemlogs}Job and Daemon Logs}
%%%%%%%%%%%%%%%%%%%%%%%%%%%%%%%%%%%%%%%%%%%%%%%%%%%%%%%%%%%%%%%%%%%%%%
\begin{description}
\item[job event log]
  The job event log is an optional, chronological list of events that occur
  as a job runs.
  The job event log is written on the submit machine.
  The submit description file for the job requests a job event log with
  the submit command \SubmitCmd{log}.
  The log is created and remains on the submit machine.
  Contents of the log are detailed in section~\ref{sec:job-log-events}.
  Examples of events are that the job is running,
  that the job  is placed on hold, or that the job completed.
\item[daemon logs]
  Each daemon configured to have a log writes events relevant to that daemon.
  Each event written consists of a timestamp and message.
  The name of the log file is set by the value of configuration variable
  \Macro{<SUBSYS>\_LOG}, where \MacroNI{<SUBSYS>} is replaced by the name
  of the daemon.
  The log is not permitted to grow without bound;
  log rotation takes place after a configurable maximum size or length of time
  is encountered.
  This maximum is specified by configuration variable 
  \Macro{MAX\_<SUBSYS>\_LOG}.

  Which events are logged for a particular daemon are determined by 
  the value of configuration variable \Macro{<SUBSYS>\_DEBUG}.
  The possible values for \MacroNI{<SUBSYS>\_DEBUG} categorize events,
  such that it is possible to control the level and quantity of events
  written to the daemon's log.

  Configuration variables that affect daemon logs are 
  \begin{description}
  \item [\Macro{MAX\_NUM\_<SUBSYS>\_LOG}] 
  \item [\Macro{TRUNC\_<SUBSYS>\_LOG\_ON\_OPEN}] 
  \item [\Macro{<SUBSYS>\_LOG\_KEEP\_OPEN}] 
  \item [\Macro{<SUBSYS>\_LOCK}] 
  \item [\Macro{FILE\_LOCK\_VIA\_MUTEX}] 
  \item [\Macro{TOUCH\_LOG\_INTERVAL}] 
  \item [\Macro{LOGS\_USE\_TIMESTAMP}] 
  \end{description}

  Daemon logs are often investigated to accomplish administrative debugging.
  \Condor{config\_val} can be used to determine the location and file name
  of the daemon log.
  For example, to display the location of the log for the \Condor{collector} 
  daemon, use
\begin{verbatim}
  condor_config_val COLLECTOR_LOG
\end{verbatim}

\item[job queue log]
  The job queue log is a transactional representation of the current job queue. 
  If the \Condor{schedd} crashes, the job queue can be rebuilt using
  this log.
  The file name is set by configuration variable \Macro{JOB\_QUEUE\_LOG},
  and defaults to \File{\$(SPOOL)/job\_queue.log}.

  Within the log,
  each transaction is identified with an integer value and followed where
  appropriate with other values relevant to the transaction.
  To reduce the size of the log and remove any transactions that are 
  no longer relevant,
  a copy of the log is kept
  by renaming the log at each time interval defined by configuration variable
  \MacroNI{QUEUE\_CLEAN\_INTERVAL}, 
  and then a new log is written with only current and relevant transactions. 

  Configuration variables that affect the job queue log are 
  \begin{description}
  \item [\Macro{SCHEDD\_BACKUP\_SPOOL}] 
  \item [\Macro{ROTATE\_HISTORY\_DAILY}] 
  \item [\Macro{ROTATE\_HISTORY\_MONTHLY}] 
  \item [\Macro{QUEUE\_CLEAN\_INTERVAL}] 
  \item [\Macro{MAX\_JOB\_QUEUE\_LOG\_ROTATIONS}] 
  \end{description}

\item[\Condor{schedd} audit log]
  The optional \Condor{schedd} audit log records user-initiated events
  that modify the job queue, such as invocations of \Condor{submit},
  \Condor{rm}, \Condor{hold} and \Condor{release}.
  Each event has a time stamp and a message that describes details of
  the event.

  This log exists to help administrators track the activities of pool users.

  The file name is set by configuration variable \Macro{SCHEDD\_AUDIT\_LOG}.

  Configuration variables that affect the audit log are 
  \begin{description}
  \item [\Macro{MAX\_SCHEDD\_AUDIT\_LOG}] 
  \item [\Macro{MAX\_NUM\_SCHEDD\_AUDIT\_LOG}] 
  \end{description}

\item[\Condor{shared\_port} audit log]
  The optional \Condor{shared\_port} audit log records connections made
  through the \Macro{DAEMON\_SOCKET\_DIR}.  Each record includes the source
  address, the socket file name, and the target process's PID, UID, GID,
  executable path, and command line.

  This log exists to help administrators track the activities of pool users.

  The file name is set by configuration variable 
  \Macro{SHARED\_PORT\_AUDIT\_LOG}.

  Configuration variables that affect the audit log are
  \begin{description}
  \item [\Macro{MAX\_SHARED\_PORT\_AUDIT\_LOG}]
  \item [\Macro{MAX\_NUM\_SHARED\_PORT\_AUDIT\_LOG}]
  \end{description}

\item[event log]
  The event log is an optional, chronological list of events that occur
  for all jobs and all users.
  The events logged are the same as those that would go into a job event
  log.
  The file name is set by configuration variable \Macro{EVENT\_LOG}.
  The log is created only if this configuration variable is set.

  Configuration variables that affect the event log, 
  setting details such as
  the maximum size to which this log may grow and details of file rotation
  and locking are
  \begin{description}
  \item [\Macro{EVENT\_LOG\_MAX\_SIZE}] 
  \item [\Macro{EVENT\_LOG\_MAX\_ROTATIONS}]
  \item [\Macro{EVENT\_LOG\_LOCKING}]
  \item [\Macro{EVENT\_LOG\_FSYNC}]
  \item [\Macro{EVENT\_LOG\_ROTATION\_LOCK}]
  \item [\Macro{EVENT\_LOG\_JOB\_AD\_INFORMATION\_ATTRS}]
  \item [\Macro{EVENT\_LOG\_USE\_XML}]
  \end{description}

  
\item[accountant log]
  The accountant log is a transactional representation of the 
  \Condor{negotiator} daemon's database of accounting information,
  which are user priorities.
  The file name of the accountant log is \File{\$(SPOOL)/Accountantnew.log}.
  Within the log, users are identified by  \verb|username@uid_domain|. 

  To reduce the size and remove information that is no longer relevant,
  a copy of the log is made when its size hits the number of bytes
  defined by configuration variable \MacroNI{MAX\_ACCOUNTANT\_DATABASE\_SIZE},
  and then a new log is written in a more compact form. 

  Administrators can change user priorities kept in this log by using
  the command line tool \Condor{userprio}.

\item[negotiator match log]
  The negotiator match log is a second daemon log from the \Condor{negotiator}
  daemon. 
  Events written to this log are those with debug level of \Expr{D\_MATCH}.
  The file name is set by configuration variable 
  \Macro{NEGOTIATOR\_MATCH\_LOG},
  and defaults to \File{\$(LOG)/MatchLog}.
  
\item[history log]
  This optional log contains information about all jobs that have been
  completed.
  It is written by the \Condor{schedd} daemon.
  The file name is \File{\$(SPOOL)/history}.

  Administrators can change view this historical information by using
  the command line tool \Condor{history}.

  Configuration variables that affect the history log, 
  setting details such as
  the maximum size to which this log may grow are
  \begin{description}
  \item [\Macro{ENABLE\_HISTORY\_ROTATION}] 
  \item [\Macro{MAX\_HISTORY\_LOG}] 
  \item [\Macro{MAX\_HISTORY\_ROTATIONS}] 
  \end{description}

\end{description}

%%%%%%%%%%%%%%%%%%%%%%%%%%%%%%%%%%%%%%%%%%%%%%%%%%%%%%%%%%%%%%%%%%%%%%
\subsection{\label{sec:DAGMan-logs}DAGMan Logs}
%%%%%%%%%%%%%%%%%%%%%%%%%%%%%%%%%%%%%%%%%%%%%%%%%%%%%%%%%%%%%%%%%%%%%%
\begin{description}

\item[default node log]
  A job event log of all node jobs within a single DAG.
  It is used to enforce the dependencies of the DAG.

  The file name is set by configuration variable 
  \Macro{DAGMAN\_DEFAULT\_NODE\_LOG}, 
  and the full path name of this file must be unique while any and 
  all submitted DAGs  and other jobs from the submit host run.
  The syntax used in the definition of this configuration variable is
  different to enable the setting of a unique file name.
  See section~\ref{param:DAGManDefaultNodeLog} for the complete definition.

  Configuration variables that affect this log are
  \begin{description}
  \item [\Macro{DAGMAN\_ALWAYS\_USE\_NODE\_LOG}] 
  \end{description}

\item[the \File{.dagman.out} file]
  A log created or appended to for each DAG submitted with timestamped
  events and extra information about the configuration applied to the DAG.
  The name of this log is formed by appending \Expr{.dagman.out} to
  the name of the DAG input file.  The file remains after the DAG completes.

  This log may be helpful in debugging what has happened in the execution
  of a DAG, as well as help to determine the final state of the DAG.

  Configuration variables that affect this log are
  \begin{description}
  \item [\Macro{DAGMAN\_VERBOSITY}] 
  \item [\Macro{DAGMAN\_PENDING\_REPORT\_INTERVAL}] 
  \end{description}

\item[the \File{jobstate.log} file]
  This optional, machine-readable log enables automated monitoring of DAG.
  Section~\ref{sec:DAGJobstateLog} details this log.
  

\end{description}

\index{logging|)}
