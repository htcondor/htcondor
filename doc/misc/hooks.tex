%%%%%%%%%%%%%%%%%%%%%%%%%%%%%%%%%%%%%%%%%%%%%%%%%%%%%%%%%%%%%%%%%%%%%%
\section{\label{sec:hooks}Hooks}
%%%%%%%%%%%%%%%%%%%%%%%%%%%%%%%%%%%%%%%%%%%%%%%%%%%%%%%%%%%%%%%%%%%%%%
\index{Hooks|(}
A \Term{hook} is an external program or script invoked by HTCondor.

Job hooks that fetch work allow sites to write their own programs or scripts, 
and allow HTCondor to invoke these hooks at the right moments 
to accomplish the desired outcome.
This eliminates the expense of the matchmaking and scheduling provided 
by the \Condor{schedd} and the \Condor{negotiator},
although at the price of the flexibility they offer. 
Therefore, job hooks that fetch work allow HTCondor to more easily and directly
interface with external scheduling systems. 

Hooks may also behave as a Job Router.

The Daemon ClassAd hooks permit the \Condor{startd} and 
the \Condor{schedd} daemons to execute hooks once or on a periodic basis.

Note that standard universe jobs execute different \Condor{starter} and
\Condor{shadow} daemons that do not implement any hook mechanisms.


%%%%%%%%%%%%%%%%%%%%%%%%%%%%%%%%%%%%%%%%%%%%%%%%%%%%%%%%%%%%%%%%%%%%%%
\input{misc/job-hooks.tex}

%%%%%%%%%%%%%%%%%%%%%%%%%%%%%%%%%%%%%%%%%%%%%%%%%%%%%%%%%%%%%%%%%%%%%%
\subsection{\label{sec:job-hooks-JR-overview}
Hooks for a Job Router}
%%%%%%%%%%%%%%%%%%%%%%%%%%%%%%%%%%%%%%%%%%%%%%%%%%%%%%%%%%%%%%%%%%%%%%
\index{Hooks!Job Router hooks} 

Job Router Hooks allow for an alternate transformation and/or 
monitoring than the \Condor{job\_router} daemon implements.
Routing is still managed by the \Condor{job\_router} daemon,
but if the Job Router Hooks are specified,
then these hooks will be used to transform
and monitor the job instead.

Job Router Hooks are similar in concept to Fetch Work Hooks,
but they are limited in their scope.
A hook is an external program or script invoked by the \Condor{job\_router}
daemon at various points during the life cycle of a routed job.

The following sections describe how and when these hooks are used,
what hooks are invoked at various stages of the job's life, 
and how to configure HTCondor to use these Hooks.

%%%%%%%%%%%%%%%%%%%%%%%%%%%%%%%%%%%%%%%%%%%%%%%%%%%%%%%%%%%%%%%%%%%%%%
\subsubsection{\label{sec:job-hooks-JR}
Hooks Invoked for Job Routing}
%%%%%%%%%%%%%%%%%%%%%%%%%%%%%%%%%%%%%%%%%%%%%%%%%%%%%%%%%%%%%%%%%%%%%%
\index{Job Router}

The Job Router Hooks allow for replacement of the transformation engine used
by HTCondor for routing a job.
Since the external transformation engine is not controlled by HTCondor,
additional hooks provide a means to update the job's
status in HTCondor, and to clean up upon exit or failure cases.
This allows one job to be transformed to just about any other type of job
that HTCondor supports,
as well as to use execution nodes not normally available to HTCondor.

It is important to note that if the Job Router Hooks are utilized, 
then HTCondor will not ignore or work around a failure in any hook execution.
If a hook is configured,
then HTCondor assumes its invocation is required and will not
continue by falling back to a part of its internal engine.
For example,
if there is a problem transforming the job using the hooks,
HTCondor will not fall back on its transformation accomplished without the hook
to process the job.

There are 2 ways in which the Job Router Hooks may be enabled.
A job's submit description file may cause the hooks to be invoked with 
\footnotesize
\begin{verbatim}
  +HookKeyword = "HOOKNAME"
\end{verbatim}
\normalsize
Adding this attribute to the job's ClassAd causes the \Condor{job\_router}
daemon on the submit machine to invoke hooks prefixed with the defined keyword.
\Expr{HOOKNAME} is a string chosen as an example; any string may be used.

The job's ClassAd attribute definition of \Attr{HookKeyword} takes
precedence,
but if not present, hooks may be enabled by defining on the submit machine
the configuration variable 
\footnotesize
\begin{verbatim}
 JOB_ROUTER_HOOK_KEYWORD = HOOKNAME
\end{verbatim}
\normalsize
Like the example attribute above,
\Expr{HOOKNAME} represents a chosen name for the hook, 
replaced as desired or appropriate.

There are 4 hooks that the Job Router can be configured to use.
Each hook will be described below along with data passed 
to the hook and expected output.
All hooks must exit successfully.

\begin{itemize}
\index{Job hooks!Job Router Hooks!Translate Job}
\item[Hook: Translate]

  The hook defined by the configuration variable 
  \Macro{<Keyword>\_HOOK\_TRANSLATE\_JOB}
  is invoked when the Job Router has determined that a job
  meets the definition for a route.  This hook is responsible for doing the
  transformation of the job and configuring any resources that are external to
  HTCondor if applicable.

\begin{description}
\item[Command-line arguments passed to the hook]
  None.
\item[Standard input given to the hook]
  The first line will be the route that the job matched as
  defined in HTCondor's configuration files followed by the job ClassAd,
  separated by the string \verb@"------"@ and a new line.
\item[Expected standard output from the hook]
  The transformed job.
\item[Exit status of the hook]
  0 for success, any non-zero value on failure.
\end{description}


\index{Job hooks!Job Router Hooks!Update Job Info}
\item[Hook: Update Job Info]

  The hook defined by the configuration variable 
  \Macro{<Keyword>\_HOOK\_UPDATE\_JOB\_INFO}
  is invoked to provide status on the specified routed job
  when the Job Router polls the status of routed jobs at intervals
  set by \Macro{JOB\_ROUTER\_POLLING\_PERIOD}.

\begin{description}
\item[Command-line arguments passed to the hook]
  None.
\item[Standard input given to the hook]
  The routed job ClassAd that is to be updated.
\item[Expected standard output from the hook]
   The job attributes to be updated in the routed job,
   or nothing, if there was no update.
   To prevent clashing with HTCondor's management of job attributes,
   only attributes that are not managed by HTCondor should be output
   from this hook.
\item[Exit status of the hook]
  0 for success, any non-zero value on failure.
\end{description}


\index{Job hooks!Job Router Hooks!Job Finalize}
\item[Hook: Job Finalize]

  The hook defined by the configuration variable 
  \Macro{<Keyword>\_HOOK\_JOB\_FINALIZE}
  is invoked when the Job Router has found that the job has completed.
  Any output from the hook is treated as an update to the source job.

\begin{description}
\item[Command-line arguments passed to the hook]
  None.
\item[Standard input given to the hook]
  The source job ClassAd, followed by the routed copy Classad that completed,
  separated by the string \verb@"------"@ and a new line.
\item[Expected standard output from the hook]
  An updated source job ClassAd, or nothing if there was no update.
\item[Exit status of the hook]
  0 for success, any non-zero value on failure.
\end{description}

\index{Job hooks!Job Router Hooks!Job Cleanup}
\item[Hook: Job Cleanup]

  The hook defined by the configuration variable 
  \Macro{<Keyword>\_HOOK\_JOB\_CLEANUP}
  is invoked when the Job Router finishes managing the job.
  This hook will be invoked regardless of whether the job
  completes successfully or not,
  and must exit successfully.

\begin{description}
\item[Command-line arguments passed to the hook]
  None.
\item[Standard input given to the hook]
  The job ClassAd that the Job Router is done managing.
\item[Expected standard output from the hook]
  None.
\item[Exit status of the hook]
  0 for success, any non-zero value on failure.
\end{description}

\end{itemize}

%%%%%%%%%%%%%%%%%%%%%%%%%%%%%%%%%%%%%%%%%%%%%%%%%%%%%%%%%%%%%%%%%%%%%%
\subsection{\label{sec:daemon-classad-hooks}
Daemon ClassAd Hooks}
%%%%%%%%%%%%%%%%%%%%%%%%%%%%%%%%%%%%%%%%%%%%%%%%%%%%%%%%%%%%%%%%%%%%%%
\index{Hooks!Daemon ClassAd Hooks} 
\index{Daemon ClassAd Hooks} 
\index{Hawkeye!see Daemon ClassAd Hooks} 
\index{Startd Cron functionality!see Daemon ClassAd Hooks} 
\index{Schedd Cron functionality!see Daemon ClassAd Hooks} 

The \Prog{Daemon ClassAd Hook} mechanism is
used to run executables (called jobs) directly from the
\Condor{startd} and \Condor{schedd} daemons. 
The output from these jobs is incorporated into the machine ClassAd
generated by the respective daemon.
The mechanism and associated jobs have been identified by various
names, including the \Term{Startd Cron}, dynamic attributes,
and a distribution of executables collectively known as \Term{Hawkeye}.

Pool management tasks can be enhanced by using a 
daemon's ability to periodically run executables.
The executables are expected to generate ClassAd attributes as their output,
which are incorporated into the machine ClassAd.
Policy expressions may then reference the dynamic attributes.

Configuration variables related to Daemon ClassAd Hooks are defined
within section ~\ref{sec:Config-hooks}.

The output of the job is incorporated into one or more ClassAds when
the job exits.
When the job outputs the specialized line:
\begin{verbatim}
  - update:true
\end{verbatim}
the output of the job goes is merged into all proper ClassAds,
and an update goes to the \Condor{collector} daemon.

When the output of the job contains a dash as the first character,
following the dash may be a tag composed of alphanumeric characters,
which uniquely identifies a buffer corresponding to a specific ClassAd.
The contents of the output attributes from the job specify which ClassAd.
The use of the tag intends to permit output with different attributes 
and values for different slots of the same machine.
An output line from the job with a dash and optionally a tag
causes the associated buffer to be merged with appropriate ClassAd.

Here is a complete configuration example.
It defines all three of the available types of jobs:
ones that use the \Condor{startd}, benchmark jobs,
and ones that use the \Condor{schedd}.
\footnotesize
\begin{verbatim}
#
# Startd Cron Stuff
#
# auxiliary variable to use in identifying locations of files
MODULES = $(ROOT)/modules

STARTD_CRON_CONFIG_VAL = $(RELEASE_DIR)/bin/condor_config_val
STARTD_CRON_MAX_JOB_LOAD = 0.2
STARTD_CRON_JOBLIST =

# Test job
STARTD_CRON_JOBLIST = $(STARTD_CRON_JOBLIST) test
STARTD_CRON_TEST_MODE = OneShot
STARTD_CRON_TEST_RECONFIG_RERUN = True
STARTD_CRON_TEST_PREFIX = test_
STARTD_CRON_TEST_EXECUTABLE = $(MODULES)/test
STARTD_CRON_TEST_KILL = True
STARTD_CRON_TEST_ARGS = abc 123
STARTD_CRON_TEST_SLOTS = 1
STARTD_CRON_TEST_JOB_LOAD = 0.01

# job 'date'
STARTD_CRON_JOBLIST = $(STARTD_CRON_JOBLIST) date
STARTD_CRON_DATE_MODE = Periodic
STARTD_CRON_DATE_EXECUTABLE = $(MODULES)/date
STARTD_CRON_DATE_PERIOD = 15s
STARTD_CRON_DATE_JOB_LOAD = 0.01

# Job 'foo'
STARTD_CRON_JOBLIST = $(STARTD_CRON_JOBLIST) foo
STARTD_CRON_FOO_EXECUTABLE = $(MODULES)/foo
STARTD_CRON_FOO_PREFIX = Foo
STARTD_CRON_FOO_MODE = Periodic
STARTD_CRON_FOO_PERIOD = 10m
STARTD_CRON_FOO_JOB_LOAD = 0.2

#
# Benchmark Stuff
#
BENCHMARKS_JOBLIST = mips kflops

# MIPS benchmark
BENCHMARKS_MIPS_EXECUTABLE = $(LIBEXEC)/condor_mips
BENCHMARKS_MIPS_JOB_LOAD = 1.0

# KFLOPS benchmark
BENCHMARKS_KFLOPS_EXECUTABLE = $(LIBEXEC)/condor_kflops
BENCHMARKS_KFLOPS_JOB_LOAD = 1.0

#
# Schedd Cron Stuff 
#
SCHEDD_CRON_CONFIG_VAL = $(RELEASE_DIR)/bin/condor_config_val
SCHEDD_CRON_JOBLIST =

# Test job
SCHEDD_CRON_JOBLIST = $(SCHEDD_CRON_JOBLIST) test
SCHEDD_CRON_TEST_MODE = OneShot
SCHEDD_CRON_TEST_RECONFIG_RERUN = True
SCHEDD_CRON_TEST_PREFIX = test_
SCHEDD_CRON_TEST_EXECUTABLE = $(MODULES)/test
SCHEDD_CRON_TEST_PERIOD = 5m
SCHEDD_CRON_TEST_KILL = True
SCHEDD_CRON_TEST_ARGS = abc 123

\end{verbatim}
\normalsize

\index{Hooks|)}
