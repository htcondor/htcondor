%%%%%%%%%%%%%%%%%%%%%%%%%%%%%%%%%%%%
\section{\label{API-Python} Python Bindings}
%%%%%%%%%%%%%%%%%%%%%%%%%%%%%%%%%%%%
\index{API!Python bindings}
\index{Python bindings}

The Python module provides bindings to the client-side APIs for HTCondor.  
It tries to provide functionality similar to the HTCondor command line tools.
Further improvements are planned for the difficult to use submission API.

%%%%%%%%%%%%%%%%%%%%%%%%%%%%%%%%%%%%
\subsection{\label{Python-OtherModule} Other Module}
%%%%%%%%%%%%%%%%%%%%%%%%%%%%%%%%%%%%
This module attempts to wrap a C++ API that is
far less formal when compared to the ClassAd language.

Module functions:
%
%  NOTE FROM KAREN:  Yes, this formatting looks somewhat poor in both
%  the html and pdf versions.  However, please do not change it, and
%  carefully follow the existing formatting, as the LaTeX to pdf and
%  LaTeX to html translators do quite different things for lists.  This
%  particular formatting at least works in both versions.
%
\begin{description}
\item [\Code{platform}]
  Returns the platform of HTCondor this module is running on.
  
\item [\Code{version}]
  Returns the version of HTCondor this module is linked against.

\item [\Code{reload\_config}]
  Reload the HTCondor configuration from disk.

\item [\Code{send\_command}]
  Send a command to an HTCondor daemon specified by a location ClassAd.

  \Code{send\_command( ad, (DaemonCommands)dc, (str)target = None)  }
  \begin{description}
    \item[ Parameters]
    \begin{itemize}
      \item \Code{ad} A ClassAd specifying the location of the daemon; 
      typically, found by using \Code{Collector.locate(...)}.
      \item \Code{dc} A command type; 
      must be a member of the enum \Code{DaemonCommands}.
      \item \Code{target} Some commands require additional arguments; 
      for example, sending \Code{DaemonOff} to a \Condor{master} requires 
      one to specify which subsystem to turn off.  
      If this parameter is given, the daemon is sent an additional argument.
    \end{itemize}
  \end{description}  

\end{description}

Module object:
\begin{description}
  \item[\Code{param}]
  A dictionary-like object providing access to the configuration variables
  in the current HTCondor configuration.
\end{description}

The \Condor{schedd} class:
\Todo

The \Condor{collector} class:
\Todo

The class to access the internal security object:
\Todo

%%%%%%%%%%%%%%%%%%%%%%%%%%%%%%%%%%%%
\subsection{\label{Python-ClassAd} ClassAd Module}
%%%%%%%%%%%%%%%%%%%%%%%%%%%%%%%%%%%%

The ClassAd module class provides a dictionary-like mechanism for interacting
with the ClassAd language. 
ClassAd objects implement the iterator interface to iterate 
through the ClassAd's attributes.

Module functions:

\begin{description}
\item [\Code{parse}]
  Parse input into a ClassAd.

  \Code{parse( input )}
  \begin{description}
    \item[ Parameters]
    \begin{itemize}
      \item \Code{input} A string-like object or a file pointer. 
    \end{itemize}
    \item[ Return Value]
      A ClassAd object.
  \end{description}  

\item [\Code{parseOld}]
  Parse old ClassAd format input into a ClassAd.

\item [\Code{version}]
  Return the version of the linked ClassAd library.
\end{description}

Standard Python object methods for the ClassAd class:

\begin{description}

\item [\Code{\_\_init\_\_}]
  Create a ClassAd object from a string.
  The string must be formatted in the new ClassAd format.

  \Code{\_\_init\_\_( str )}
  \begin{description}
    \item[ Parameters]
    \begin{itemize}
      \item \Code{str} A string formatted in the new ClassAd format. 
    \end{itemize}
    \item[ Return Value]
      ?? .
  \end{description}  

\item [\Code{\_\_len\_\_}]
  Returns the number of attributes in the ClassAd; 
  allows \Code{len(object)} semantics for ClassAds.

  \Code{\_\_len\_\_( )}
  \begin{description}
    \item[ Return Value]
      Returns the integer number of attributes in the ClassAd.
  \end{description}  

\item [\Code{\_\_str\_\_}]
  Converts the ClassAd to a string; the formatting style is new ClassAd,
  with square brackets and semi-colons.  
  For example, \Expr{[ Foo = "bar"; ]} may be returned.

  \Code{\_\_str\_\_( )}
  \begin{description}
    \item[ Return Value]
      Returns the converted string.
  \end{description}  

\end{description}

The ClassAd object has the following dictionary-like methods:
\begin{description}

\item [\Code{items}]
  Returns an iterator, where each item returned by the iterator 
  is a tuple representing a pair (attribute,value) in the ClassAd object.

  \Code{items( )}
  \begin{description}
    \item[ Return Value]
      Returns an iterator of tuples.
  \end{description}  

\item [\Code{keys}]
  Returns an iterator, where each item returned by the iterator 
  is an attribute string in the ClassAd.

  \Code{keys( )}
  \begin{description}
    \item[ Return Value]
      Returns an iterator of strings.
  \end{description}  

\item [\Code{values}]
  Returns an iterator, where each item returned by the iterator is a value 
  in the ClassAd.  If the value is a literal, 
  it will be cast to a native python object, 
  so a ClassAd string will be returned as a Python string.
  \Code{values( )}
  \begin{description}
    \item[ Return Value]
      Returns an iterator of objects.
  \end{description}  

\item [\Code{\_\_getitem\_\_}]
  Returns as an object the value corresponding to an attribute.

  \Code{\_\_getitem\_\_( attr )}
  \begin{description}
    \item[ Parameters]
    \begin{itemize}
      \item \Code{attr} ?? . 
    \end{itemize}
    \item[ Return Value]
      ClassAd values will be returned as Python objects; 
      ClassAd expressions will be returned as \Code{ExprTree} objects.
  \end{description}  

\item [\Code{\_\_setitem\_\_}]
  Sets the ClassAd attribute to the given value.
  Python objects will be converted to ClassAd values; 
  \Expr{ExprTree} objects will not be evaluated.

  \Code{\_\_setitem\_\_( attr, value )}
  \begin{description}
    \item[ Parameters]
    \begin{itemize}
      \item \Code{attr} ?? . 
      \item \Code{value} ?? . 
    \end{itemize}
    \item[ Return Value]
      ClassAd values will be returned as Python objects; 
      ClassAd expressions will be returned as \Code{ExprTree} objects.
  \end{description}  


\end{description}

Additional methods:
\begin{description}

\item [\Code{eval}]
  Evaluate the value given a ClassAd attribute.
  Throws \Expr{ValueError} if unable to evaluate the object.

  \Code{eval ( attr )}
  \begin{description}
    \item[ Parameters]
    \begin{itemize}
      \item \Code{attr} ?? . 
    \end{itemize}
    \item[ Return Value]
      Returns the Python object corresponding to the evaluated ClassAd
      attribute.
  \end{description}  

\item [\Code{lookup}]
  Lookup the \Expr{ExprTree} object associated with the given attribute.
  No attempt will be made to convert to a Python object.

  \Code{lookup ( attr )}
  \begin{description}
    \item[ Parameters]
    \begin{itemize}
      \item \Code{attr} ?? . 
    \end{itemize}
    \item[ Return Value]
      Returns an \Expr{ExprTree} object.
  \end{description}  

\item [\Code{printOld}]
  Print the ClassAd in the old ClassAd format.

  \Code{printOld( )}
  \begin{description}
    \item[ Return Value]
      ?? Returns a string.
  \end{description}  

\end{description}

%Class Object:
%\begin{description}
%  \item[\Code{ExprTree}]
%  An object representing an expression in the ClassAd language.
%\end{description}


 \MoreTodo
