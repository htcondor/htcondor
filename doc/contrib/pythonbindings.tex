%%%%%%%%%%%%%%%%%%%%%%%%%%%%%%%%%%%%
\section{\label{API-Python} Python Bindings}
%%%%%%%%%%%%%%%%%%%%%%%%%%%%%%%%%%%%
\index{API!Python bindings}
\index{Python bindings}

The Python module provides bindings to the client-side APIs for HTCondor.  
It tries to provide functionality similar to the HTCondor command line tools.
Further improvements are planned for the difficult to use submission API.

%%%%%%%%%%%%%%%%%%%%%%%%%%%%%%%%%%%%
\subsection{\label{Python-ClassAd} ClassAd Module}
%%%%%%%%%%%%%%%%%%%%%%%%%%%%%%%%%%%%
This module attempts to wrap a C++ API that is
far less formal when compared to the ClassAd language.

Module functions:
%
%  NOTE FROM KAREN:  Yes, this formatting looks somewhat poor in both
%  the html and pdf versions.  However, please do not change it, and
%  carefully follow the existing formatting, as the LaTeX to pdf and
%  LaTeX to html translators do quite different things for lists.  This
%  particular formatting at least works in both versions.
%
\begin{description}
\item [\Code{platform}]
  Returns the platform of HTCondor this module is running on.
  
\item [\Code{version}]
  Returns the version of HTCondor this module is linked against.

\item [\Code{reload\_config}]
  Reload the HTCondor configuration from disk.

\item [\Code{send\_command}]
  Send a command to an HTCondor daemon specified by a location ClassAd.

  \Code{send\_command( ad, (DaemonCommands)dc, (str)target = None) : }
  \begin{description}
    \item[ Parameters]
    \begin{itemize}
      \item \Code{ad} A ClassAd specifying the location of the daemon; 
      typically, found by using \Code{Collector.locate(...)}.
      \item \Code{dc} A command type; 
      must be a member of the enum \Code{DaemonCommands}.
      \item \Code{target} Some commands require additional arguments; 
      for example, sending \Code{DaemonOff} to a \Condor{master} requires 
      one to specify which subsystem to turn off.  
      If this parameter is given, the daemon is sent an additional argument.
    \end{itemize}
  \end{description}  

\end{description}

Module object:
\begin{description}
  \item[\Code{param}]
  A dictionary-like object providing access to the configuration variables
  in the current HTCondor configuration.
\end{description}

ClassAd Class functions:

\begin{description}
\item [\Code{parse}]
  Parse input into a ClassAd.
  \Code{parse( input )}
  \begin{description}
    \item[ Parameters]
    \begin{itemize}
      \item \Code{input} A string-like object or a file pointer. 
    \end{itemize}
    \item[ Return Value]
      A ClassAd object.
  \end{description}  

\item [\Code{parseOld}]
  Parse old ClassAd format input into a ClassAd.
\end{description}
 \MoreTodo

%%%%%%%%%%%%%%%%%%%%%%%%%%%%%%%%%%%%
\subsection{\label{Python-OtherModule} Other Module}
%%%%%%%%%%%%%%%%%%%%%%%%%%%%%%%%%%%%

The \Condor{schedd} class:
\Todo

The \Condor{collector} class:
\Todo

The class to access the internal security object:
\Todo
