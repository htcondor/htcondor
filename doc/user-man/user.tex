%%%%%%%%%%%%%%%%%%%%%%%%%%%%%%%%%%%%%%%%%%%%%%%%%%%%%%
\section{Welcome to Condor}  
%
% .... or alternatively called the 'warm fuzzies' section
% <smirk>  
% 
%
% Warning: much of what you are about to read was very 
% hastily written by a very tired Todd.... Good Luck.  
%%%%%%%%%%%%%%%%%%%%%%%%%%%%%%%%%%%%%%%%%%%%%%%%%%%%%

\label{sec:usermanual}
\index{Condor!user manual|(}
\index{user manual|(}
Presenting Condor \VersionNotice! Condor is developed by
the Condor Team at the University of Wisconsin-Madison (UW-Madison), and
was first installed as a production system in the UW-Madison Computer
Sciences department more than 10 years ago. This Condor pool has since
served as a major source of computing cycles to UW faculty and students.
For many, it has revolutionized the role computing plays in their
research. An increase of one, and sometimes even two, orders of
magnitude in the computing throughput of a research organization can
have a profound impact on its size, complexity, and scope. Over the
years, the Condor Team has established collaborations with scientists
from around the world, and it has provided them with access to surplus
cycles (one scientist has consumed 100 CPU years!). Today, our
department's pool consists of more than 700 desktop Unix workstations
and more than 100 Windows 2000 machines.
On a typical day, our pool delivers more than 500 CPU days to UW
researchers. Additional Condor pools have been established over the
years across our campus and the world. Groups of researchers, engineers,
and scientists have used Condor to establish compute pools ranging in
size from a handful to hundreds of workstations. We hope that Condor
will help revolutionize your compute environment as well.


%%%%%%%%%%%%%%%%%%%%%%%%%%%%%%%%%%%%%%%%%%%%%%%%%%%%%%%
\section{Introduction}
%%%%%%%%%%%%%%%%%%%%%%%%%%%%%%%%%%%%%%%%%%%%%%%%%%%%%%%


In a nutshell, Condor is a specialized batch system 
\index{batch system}
for managing compute-intensive jobs.
Like most batch systems, Condor provides a
queuing mechanism, scheduling policy, priority scheme, and resource
classifications.  Users submit their compute jobs to Condor, Condor puts
the jobs in a queue, runs them, and then informs the user as to the
result.

Batch systems normally operate only with dedicated machines.  Often 
termed compute servers, these dedicated machines are typically owned by
one organization and dedicated to the sole purpose of running compute
jobs.  Condor can schedule jobs on dedicated machines.  But unlike traditional 
batch systems, Condor is also designed to effectively 
utilize non-dedicated machines to run jobs.  By being told to only
run compute jobs on machines which are currently not being used (no keyboard
activity, no load average, no active telnet users, etc), Condor can
effectively harness otherwise idle machines throughout a pool of machines.
This is important because often times the amount of
compute power represented by the aggregate total of all the non-dedicated 
desktop workstations sitting on people's desks throughout the
organization is far greater than the compute power of a dedicated
central resource.

Condor has several unique capabilities at its disposal which are geared 
toward effectively utilizing non-dedicated resources that are not owned or
managed by a centralized resource. These include transparent process
checkpoint and migration, remote system calls, and ClassAds.
Read section~\ref{sec:what-is-condor} for a general 
discussion of these features before reading any further.


%%%%%%%%%%%%%%%%%%%%%%%%%%%%%%%%%%%%%%%%%%%%%%%%%%%%%%%%
\section{Matchmaking with ClassAds}
\label{sec:matchmaking-with-classads}
%%%%%%%%%%%%%%%%%%%%%%%%%%%%%%%%%%%%%%%%%%%%%%%%%%%%%%%%

Before you learn about how to submit a job, it is important to
understand how Condor allocates resources. 
\index{Condor!resource allocation}
Understanding the
unique framework by which Condor matches submitted jobs with machines is
the key to getting the most from Condor's scheduling algorithm. 

Condor simplifies job submission by acting as a matchmaker of ClassAds.
Condor's ClassAds
\index{ClassAd}
are analogous to the classified advertising section of the
newspaper. Sellers advertise specifics about what they have to sell,
hoping to attract a buyer. Buyers may advertise specifics about what
they wish to purchase. Both buyers and sellers list constraints that
need to be satisfied.
For instance, a buyer has a maximum spending limit, 
and a seller requires a minimum purchase price.
Furthermore, both want to rank requests to their own advantage.
Certainly a seller would rank
one offer of \$50 dollars higher than a different
offer of \$25.
In Condor, users submitting
jobs can be thought of as buyers of compute resources and machine owners
are sellers. 

All machines in a Condor pool advertise their attributes,
\index{ClassAd!attributes}
such as
available RAM memory, CPU type and speed, virtual memory size, current
load average, along with other static and dynamic properties.
This machine ClassAd
\index{ClassAd!machine}
also advertises under what conditions it is
willing to run a Condor job and what type of job it would prefer. These
policy attributes can reflect the individual terms and preferences by
which all the different owners have graciously allowed their machine to
be part of the Condor pool. 
You may
advertise that your machine is only willing to run jobs at night
and when there is no keyboard activity on your machine.
In addition, you may
advertise a preference (rank) for running jobs submitted by you
or one of your co-workers. 

Likewise, when submitting a job, you specify a ClassAd with
your requirements and preferences.
The ClassAd
\index{ClassAd!job}
includes the
type of machine you  wish to use. For instance, perhaps you are
looking for the fastest floating point performance available.
You want Condor to rank available machines
based upon floating point performance. Or, perhaps you
care only that the machine has a minimum of 128 Mbytes of RAM.
Or, perhaps you will
take any machine you can get! These job attributes and requirements
are bundled up into a job ClassAd.

Condor plays the role of a matchmaker by continuously reading
all the job ClassAds and all the machine ClassAds, 
matching and ranking job ads with machine ads.
Condor makes certain that all
requirements in both ClassAds are satisfied. 

%%%%%
\subsection{Inspecting Machine ClassAds with \condor{status}}
%%%%%

\index{Condor commands!condor\_status}
Once Condor is installed,
you will get a feel for what
a machine ClassAd does by trying
the \Condor{status} command.
Try the \Condor{status} command to get
a summary of information from
ClassAds about the resources available in your pool.
Type \Condor{status} and hit enter to see a summary 
similar to the following:
\begin{center}
%\small       too big
%\tiny        too small
\footnotesize
\begin{verbatim}
Name       Arch     OpSys        State      Activity   LoadAv Mem  ActvtyTime

adriana.cs INTEL    SOLARIS251   Claimed    Busy       1.000  64    0+01:10:00
alfred.cs. INTEL    SOLARIS251   Claimed    Busy       1.000  64    0+00:40:00
amul.cs.wi SUN4u    SOLARIS251   Owner      Idle       1.000  128   0+06:20:04
anfrom.cs. SUN4x    SOLARIS251   Claimed    Busy       1.000  32    0+05:16:22
anthrax.cs INTEL    SOLARIS251   Claimed    Busy       0.285  64    0+00:00:00
astro.cs.w INTEL    SOLARIS251   Claimed    Busy       0.949  64    0+05:30:00
aura.cs.wi SUN4u    SOLARIS251   Owner      Idle       1.043  128   0+14:40:15
\end{verbatim}
\normalsize
\Dots 
\end{center}


The \Condor{status} command has options that summarize machine ads 
in a variety of ways.
For example,
\begin{description}
\item[\Condor{status -available}] shows only machines which are
willing to run jobs now. 
\item[\Condor{status -run}] shows only machines
which are currently running jobs.  
\item[\Condor{status -l}] lists the machine ClassAds for all machines
in the pool.
\end{description}

Refer to the \Condor{status} command 
reference page located on page~\pageref{man-condor-status}
for a complete description of the \Condor{status} command.

Figure~\ref{fig:CondorStatusL} shows the complete machine ClassAd
\index{ClassAd!machine example}
\index{machine ClassAd}
for a single workstation: alfred.cs.wisc.edu. Some of the listed
attributes are used by
Condor for scheduling. Other attributes are for information purposes.
An important point is that \emph{any} of the attributes in a
machine ad can be utilized at job submission time as part of a request
or preference on what machine to use. Additional attributes
can be easily added. For example, your site administrator can
add a physical location attribute to your machine ClassAds.

%
% figures for this section
%
% condor_status -l alfred
%
\begin{center}
\begin{figure}
\small
\begin{verbatim}
MyType = "Machine"
TargetType = "Job"
Name = "alfred.cs.wisc.edu"
Machine = "alfred.cs.wisc.edu"
StartdIpAddr = "<128.105.83.11:32780>"
Arch = "INTEL"
OpSys = "SOLARIS251"
UidDomain = "cs.wisc.edu"
FileSystemDomain = "cs.wisc.edu"
State = "Unclaimed"
EnteredCurrentState = 892191963
Activity = "Idle"
EnteredCurrentActivity = 892191062
VirtualMemory = 185264
Disk = 35259
KFlops = 19992
Mips = 201
LoadAvg = 0.019531
CondorLoadAvg = 0.000000
KeyboardIdle = 5124
ConsoleIdle = 27592
Cpus = 1
Memory = 64
AFSCell = "cs.wisc.edu"
START = LoadAvg - CondorLoadAvg <= 0.300000 && KeyboardIdle > 15 * 60
Requirements = TRUE
Rank = Owner == "johndoe" || Owner == "friendofjohn" 
CurrentRank =  - 1.000000
LastHeardFrom = 892191963
\end{verbatim}
\normalsize
\caption{\label{fig:CondorStatusL}Sample output from \Condor{status -l alfred}}
\end{figure}
\end{center}


%%%%%%%%%%%%%%%%%%%%%%%%%%%%%%%%%%%%%%%%%%%%%%%%%%%%%%%%%%%%%
\section{Road-map for Running Jobs}
%%%%%%%%%%%%%%%%%%%%%%%%%%%%%%%%%%%%%%%%%%%%%%%%%%%%%%%%%%%%%

\index{job!preparation}
The road to using Condor effectively is a short one.  The basics
are quickly and easily learned.

Here are all the steps needed to run a job using Condor.
\begin{description}

\item[Code Preparation.]
A job run under Condor must be able to 
run as a background batch job.
\index{job!batch ready}
Condor runs the program unattended and in the background. 
A program that runs in the background will not be able
to do interactive input and output.
Condor can redirect console output (stdout and stderr)
and keyboard input (stdin)
to and from files for you.
Create any needed files that contain
the proper keystrokes needed for program input.
Make certain the program will run correctly with the files.

\item[The Condor Universe.]
Condor has several 
runtime environments (called a \Term{universe}) from which to choose.
Of the universes, two are likely choices when learning
to submit a job to Condor: the standard universe and the vanilla universe.
The standard universe allows a job running under Condor to
handle system calls by returning them to the machine where the
job was submitted.
The standard universe also provides the mechanisms necessary
to take a checkpoint and migrate a partially completed job,
should the machine on which the job is executing become
unavailable.
To use the standard universe, it is necessary to
relink the program with the Condor library using the
\Condor{compile} command.
The manual page for \Condor{compile} on page~\pageref{man-condor-compile} has details.

The vanilla universe provides a way to run jobs that cannot be
relinked.
There is no way to take a checkpoint or migrate a job executed
under the vanilla universe.
For access to input and output files, jobs must either use a shared
file system, or use Condor's File Transfer mechanism.

Choose a universe under which to run the Condor program,
and re-link the program if necessary.

\item[Submit description file.]
Controlling the details of a job submission is a
submit description file.
The file contains information
about the job such as what executable to run, the
files to use for keyboard and screen data,
the platform type required to run the program, and
where to send e-mail when the job completes.
You can also tell Condor how many times to run a program;
it is simple to run the same program
multiple times with multiple data sets.

Write a submit description file to go with the job, using
the examples provided in section~\ref{sec:sample-submit-files}
for guidance.

\item[Submit the Job.]Submit the program to Condor with
the \Condor{submit} command.
\index{Condor commands!condor\_submit}

\end{description}

Once submitted, Condor does the rest toward running
the job.
Monitor the job's progress with the \Condor{q}
\index{Condor commands!condor\_q}
and \Condor{status} commands.
\index{Condor commands!condor\_status}
You may modify the order in which Condor will run your jobs with
\Condor{prio}. If desired, Condor can even inform you in a log file 
every time your job is checkpointed and/or migrated to a different machine. 

When your program completes, Condor will tell you
(by e-mail, if preferred) the exit status of your program and various
statistics about its performances, including time used and I/O performed.
If you are using a log file for the job(which is recommended) the exit
status will be recorded in the log file.
You can remove a job from the
queue prematurely with \Condor{rm}. 
\index{Condor commands!condor\_rm}


%%%%%%%%%%%%%%%%%%%%%%%%%%%%%%%%%%%%%%%%%%%%%%%%
\subsection{\label{sec:Choosing-Universe}
Choosing a Condor Universe}
%%%%%%%%%%%%%%%%%%%%%%%%%%%%%%%%%%%%%%%%%%%%%%%%

A \Term{universe} in Condor
\index{universe}
\index{Condor!universe}
defines an execution environment. 
Condor \VersionNotice\ supports several different
universes for user jobs:
\begin{itemize}
	\item Standard
	\item Vanilla
	\item PVM
 	\item MPI
	\item Globus or Grid
	\item Java
	\item Scheduler
	\item Local
\end{itemize}

The \AdAttr{Universe} attribute is specified in the submit description file.
If a universe is not specified, the default is \Expr{standard}.

\index{universe!standard}
The \Expr{standard} universe provides migration and reliability, but has some
restrictions on the programs that can be run. 
\index{universe!vanilla}
The \Expr{vanilla} universe provides fewer services, but has very few
restrictions.
\index{universe!PVM}
The \Expr{PVM} universe is for programs written to the Parallel Virtual
Machine interface.  See section~\ref{sec:PVM} for more about PVM and Condor.
\index{universe!MPI}
The \Expr{MPI} universe is for programs written to the MPICH interface.
See section~\ref{sec:MPI} for more about MPI and Condor.
\index{universe!Globus}
\index{universe!Grid}
The Globus or Grid universe allows users to submit 
jobs using Condor's interface.
These jobs are submitted for execution on grid resources.
For Globus jobs,
see \URL{http://www.globus.org} for more information.
\index{universe!Java}
\index{Java}
\index{Java Virtual Machine}
\index{JVM}
The Java universe allows users to run jobs written for the
Java Virtual Machine (JVM).
%\index{universe!Scheduler}
The \Expr{scheduler} universe allows users to submit lightweight jobs
to be spawned by the \Condor{schedd} on the submit host itself.
%\index{universe!Local}
%The local universe . . .

%%%%%%%%%%%%%%%%%%%%%%%%%%%%%%%%%%%%%%%%%%%%%%%%%%%%%%%%%%%%%%%%%%%%%%
\subsubsection{\label{sec:standard-universe}Standard Universe}
%%%%%%%%%%%%%%%%%%%%%%%%%%%%%%%%%%%%%%%%%%%%%%%%%%%%%%%%%%%%%%%%%%%%%%

\index{universe!standard}
In the standard universe, Condor provides \Term{checkpointing} and
\Term{remote system calls}.  These features make a job more reliable
and allow it uniform access to resources from anywhere in the pool.
To prepare a program as a standard universe job, it must be relinked
with \Condor{compile}.  Most programs can be prepared as a standard
universe job, but there are a few restrictions.

\index{checkpoint}
\index{checkpoint image}
Condor checkpoints a job at regular intervals.
A \Term{checkpoint image} is essentially a snapshot of the current
state of a job. 
If a job must be migrated from one machine to another,
Condor makes a checkpoint image, copies the image to the new machine,
and restarts the job continuing the job from where it left off.
If a machine should
crash or fail while it is running a job, Condor can restart the job on
a new machine using the most recent checkpoint image.
In this way, jobs
can run for months or years even in the face of occasional computer failures.

\index{remote system call}
\index{shadow}
Remote system calls make a job perceive that it is executing on its home
machine, even though the job may execute on many different machines over its
lifetime.
When a job runs on a remote machine, a second process, called
a \Condor{shadow} runs on the machine where the job was submitted.
\index{condor\_shadow}
\index{agents!condor\_shadow}
\index{Condor daemon!condor\_shadow}
\index{remote system call!condor\_shadow}
When the job attempts a system call, the \Condor{shadow} performs
the system call instead and sends the results to the remote
machine.
For example, if a job attempts to open a file that is
stored on the submitting machine,
the \Condor{shadow} will find the file,
and send the data to the machine where
the job is running.

To convert your program into a standard universe job, you must use
\Condor{compile} to relink it with the Condor libraries.
Put \Condor{compile} in front of your usual link command.
You do not need to modify the program's source code,
but you do need access to the unlinked object files.
A commercial program that is packaged as a single executable file cannot be
converted into a standard universe job.

For example, if you would have linked the job by executing:
\begin{verbatim}
% cc main.o tools.o -o program
\end{verbatim}

Then, relink the job for Condor with:
\begin{verbatim}
% condor_compile cc main.o tools.o -o program
\end{verbatim}

There are a few restrictions on standard universe jobs:


\begin{enumerate}

\index{Unix!fork}
\index{Unix!exec}
\index{Unix!system}
\item Multi-process jobs are not allowed.  This includes system calls such as
\Syscall{fork}, \Syscall{exec}, and \Syscall{system}.

\index{Unix!pipe}
\index{Unix!semaphore}
\index{Unix!shared memory}
\item Interprocess communication is not allowed.  This includes pipes, semaphores, and shared memory.

\index{Unix!socket}
\index{network}
\item Network communication must be brief.  A job \emph{may} make network
connections using system calls such as \Syscall{socket}, but a network
connection left open for long periods will delay checkpointing and migration.

\index{signal}
\index{signal!SIGUSR2}
\index{signal!SIGTSTP}
\item Sending or receiving the SIGUSR2 or SIGTSTP signals is not allowed.
HTCondor reserves these signals for its own use.  Sending or receiving all
other signals \emph{is} allowed.

\index{Unix!alarm}
\index{Unix!timer}
\index{Unix!sleep}
\item Alarms, timers, and sleeping are not allowed.  This includes system
calls such as \Syscall{alarm}, \Syscall{getitimer}, and \Syscall{sleep}.

\index{thread!kernel-level}
\index{thread!user-level}
\item Multiple kernel-level threads are not allowed.  However,
multiple user-level threads \emph{are} allowed.

\index{file!memory-mapped}
\index{Unix!mmap}
\item Memory mapped files are not allowed.  This includes system calls such
as \Syscall{mmap} and \Syscall{munmap}.

\index{file!locking}
\index{Unix!flock}
\index{Unix!lockf}
\item File locks are allowed, but not retained between checkpoints.

\index{file!read only}
\index{file!write only}
\item All files must be opened read-only or write-only.  A file opened
for both reading and writing will cause trouble if a job must be rolled back
to an old checkpoint image.  For compatibility reasons, a file opened
for both reading and writing will result in a warning but not an error.

\item A fair amount of disk space must be available on the submitting machine
for storing a job's checkpoint images.  A checkpoint image is approximately
equal to the virtual memory consumed by a job while it runs.  If disk space
is short, a special \Term{checkpoint server} can be designated for storing
all the checkpoint images for a pool.

\index{linking!dynamic}
\index{linking!static}
\item On Linux, the job must be statically linked. 
\Condor{compile} does this by default.

\index{Unix!large files} 
\item Reading to or writing from files larger than 2 GBytes is only supported
when the submit side \Condor{shadow} and the standard universe user job
application itself are both 64-bit executables.

\end{enumerate}






%%%%%%%%%%%%
\subsubsection{Vanilla Universe}
%%%%%%%%%%%%

\index{universe!vanilla}
The vanilla universe in Condor is intended
for programs which cannot
be successfully re-linked.
Shell scripts are another case where the vanilla universe
is useful.
Unfortunately, jobs run under the vanilla universe cannot checkpoint or use
remote system calls. 
This has unfortunate consequences for a job that is partially
completed 
when the remote machine running a job must be returned
to its owner.
Condor has only two choices.  It can suspend the job, hoping to
complete it at a later time,
or it can give up and restart the job \emph{from the beginning} 
on another machine in the pool.

Since Condor's remote system call features cannot be used with the
vanilla universe, access to the job's input and output files becomes a
concern.
One option is for Condor to rely on a shared file system, such as NFS
or AFS. 
Alternatively, Condor has a mechanism for transferring files on behalf
of the user.
In this case, Condor will transfer any files needed by a job to the
execution site, run the job, and transfer the output back to the
submitting machine.

Under Unix, the Condor presumes a shared file system for vanilla jobs. 
However, if a shared file system is unavailable, a user can enable the
Condor File Transfer mechanism.
On Windows platforms, the default is to use the File Transfer
mechanism.
For details on running a job with a shared file system, see
section~\ref{sec:shared-fs} on page~\pageref{sec:shared-fs}.
For details on using the Condor File Transfer mechanism, see 
section~\ref{sec:file-transfer} on page~\pageref{sec:file-transfer}.


%%%%%%%%%%%%
\subsubsection{PVM Universe}
%%%%%%%%%%%%

\index{universe!PVM}
The PVM universe allows programs written for the Parallel Virtual Machine
interface to be used within the opportunistic Condor environment.
Please see section~\ref{sec:PVM} for more details.

%%%%%%%%%%%%
\subsubsection{MPI Universe}
%%%%%%%%%%%%
\index{universe!MPI}
The MPI universe allows programs written to the MPICH
interface to be used within the opportunistic Condor environment.
Please see section~\ref{sec:MPI} for more details.

%%%%%%%%%%%%
\subsubsection{Globus Universe}
%%%%%%%%%%%%

\index{universe!Globus}
The Globus universe in Condor is intended to provide the standard
Condor interface to users who wish to start Globus system jobs
from Condor. Each job queued in the job submission file is translated
into the Globus Resource Specification Language (RSL) and subsequently 
submitted to Globus via the GRAM
protocol. You benefit from using the Globus universe because Condor
maintains your jobs in a queue on your submit machine and can deal
with a wide variety of common errors you might
experience. Section~\ref{sec:Condor-G} on page~\pageref{sec:Condor-G}
has details on using the Globus universe.
The manual page for \Condor{submit}
on page~\pageref{man-condor-submit}
has detailed descriptions of
the Globus-related attributes.

%%%%%%%%%%%%
\subsubsection{Java Universe}
%%%%%%%%%%%%

\index{universe!Java}

A program submitted to the Java universe may run on any sort of machine
with a JVM regardless of its location, owner, or JVM version.  Condor
will take care of all the details such as finding the JVM binary and
setting the classpath.

%%%%%%%%%%%%
\subsubsection{Scheduler Universe}
%%%%%%%%%%%%

\index{universe!Scheduler}

The \Expr{scheduler} universe allows users to submit lightweight jobs
to be run immediately alongside the \Condor{schedd} on the submit host
itself.

\Expr{Scheduler} universe jobs are not matched with a remote machine,
and will never be preempted.  They do not obey \Expr{requirements}.

Originally intended for meta-schedulers like \Condor{dagman} (hence
the name ``scheduler''), the \Expr{scheduler} universe can also be
used to manage jobs of any sort that need to run on the submit host.

However, unlike the \Expr{local} universe, the \Expr{scheduler}
universe does not use a \Condor{starter} to manage the job, and thus
offers limited features and policy support.  The \Expr{local} universe
is a better choice for most jobs which must run on the submit host, as
it offers a richer set of job-management features, and is more
consistent with other remote universes like \Expr{vanilla}.

The \Expr{scheduler} universe may be deprecated in the future, in
favor of the newer \Expr{local} universe.

%%%%%%%%%%%%%%%%%%%%%%%%%%%%%%%%%%%%%%%%%%%%%%%%%%%%%%%%%%%%%%%%%%%%%%
\subsubsection{\label{sec:local-universe}Local Universe}
%%%%%%%%%%%%%%%%%%%%%%%%%%%%%%%%%%%%%%%%%%%%%%%%%%%%%%%%%%%%%%%%%%%%%%

\index{universe!local}
The local universe allows a Condor job to be submitted and
executed with different assumptions for the execution conditions
of the job.
The job does not wait to be matched with a machine.
It instead executes right away, on the machine where the job
is submitted.
The job will never be preempted.
The machine requirements are not considered for local universe
jobs.


%%%%%%%%%%%%%%%%%%%%%%%%%%%%%%%%%%%%%%%%%%%%%%%%%%%%%%%%%%%%%%
\section{Submitting a Job}
%%%%%%%%%%%%%%%%%%%%%%%%%%%%%%%%%%%%%%%%%%%%%%%%%%%%%%%%%%%%%%

\index{job!submitting}
A job is submitted for execution to Condor using the
\Condor{submit} command.
\index{Condor commands!condor\_submit}
\Condor{submit} takes as an argument the name of a
file called a submit description file.
\index{submit description file}
\index{file!submit description}
This file contains commands and keywords to direct the queuing of jobs.
In the submit description file, Condor finds everything it needs
to know about the job.  Items such as the name of the executable to run,
the initial working directory, and command-line arguments to the
program all go into
the submit description file.  \Condor{submit} creates a job
ClassAd based upon the information,
and Condor
works toward running the job.

The contents of a submit file
\index{submit description file!contents of}
can save time for Condor users.
It is easy to submit multiple runs of a program to
Condor. To run the same program 500 times on 500
different input data sets, arrange your data files
accordingly so that each run reads its own input, and each run
writes its own output.
Each individual run may have its own initial
working directory, stdin, stdout, stderr, command-line arguments, and
shell environment.
A program that directly opens its own
files will read the file names to use either from stdin
or from the command line. 
A program that opens a static filename every time
will need to use a separate subdirectory for the output of each run.

The \Condor{submit} manual page 
is on page~\pageref{man-condor-submit} and
contains a complete and full description of how to use \Condor{submit}.

%%%%%%%%%%%%%%%%%%%%
\subsection{\label{sec:sample-submit-files}Sample submit description files}  
%%%%%%%%%%%%%%%%%%%%

In addition to the examples of submit description files given
in the 
\Condor{submit} manual page, here are a few more.
\index{submit description file!examples|(}

\subsubsection{Example 1} 

Example 1 is the simplest submit description
file possible. It queues up one copy of the program \Prog{foo}(which had been
created by \Condor{compile}) for execution
by Condor.
Since no platform is specified, Condor will use its default,
which is to run the job on a machine which has the
same architecture and operating system as the machine from which it was
submitted. 
No 
\AdAttr{input},
\AdAttr{output}, and
\AdAttr{error}
commands are given in the submit
description file, so the
files \File{stdin}, \File{stdout}, and \File{stderr} will all refer to 
\File{/dev/null}.
The program may produce output by explicitly opening a file and writing to
it.
A log file, \File{foo.log}, will also be produced that contains events
the job had during its lifetime inside of Condor.
When the job finishes, its exit conditions will be noted in the log file.
It is recommended that you always have a log file so you know what
happened to your jobs.
\begin{verbatim}
  ####################                                                    
  # 
  # Example 1                                                            
  # Simple condor job description file                                    
  #                                                                       
  ####################                                                    
                                                                          
  Executable     = foo                                                    
  Log            = foo.log                                                    
  Queue    
\end{verbatim}

\subsubsection{Example 2}

Example 2 queues two copies of the program \Prog{mathematica}. The
first copy will run in directory \File{run\_1}, and the second will run in
directory \File{run\_2}. For both queued copies, 
\File{stdin} will be \File{test.data},
\File{stdout} will be \File{loop.out}, and
\File{stderr} will be \File{loop.error}.
There will be two sets of files written,
as the files are each written to their own directories.
This is a convenient way to organize data if you
have a large group of Condor jobs to run. The example file 
shows program submission of
\Prog{mathematica} as a vanilla universe job.
This may be necessary if the source
and/or object code to program \Prog{mathematica} is not available.
\begin{verbatim}
  ####################     
  #                       
  # Example 2: demonstrate use of multiple     
  # directories for data organization.      
  #                                        
  ####################                    
                                         
  Executable     = mathematica          
  Universe = vanilla                   
  input   = test.data                
  output  = loop.out                
  error   = loop.error             
  Log     = loop.log                                                    
                                  
  Initialdir     = run_1         
  Queue                         
                               
  Initialdir     = run_2      
  Queue                     
\end{verbatim}

\subsubsection{Example 3}

The submit description file for Example 3 queues 150
\index{running multiple programs}
runs of program \Prog{foo} which has been compiled and linked for
Silicon Graphics workstations running IRIX 6.5. 
This job requires Condor to run the program on machines which have
greater than 32 megabytes of physical memory, and expresses a
preference to run the program on machines with more than 64 megabytes,
if such machines are available.  It also advises Condor that it will
use up to 28 megabytes of memory when running.
Each of the 150 runs of the program is given its own process number,
starting with process number 0.
So, files 
\File{stdin}, \File{stdout}, and \File{stderr} will
refer to \File{in.0}, \File{out.0}, and \File{err.0} for the first run
of the program,
\File{in.1}, \File{out.1},
and \File{err.1} for the second run of the program, and so forth.
A log file containing entries
about when and where Condor runs, checkpoints, and migrates processes for
the 150 queued programs
will be written into file \File{foo.log}.
\begin{verbatim}
  ####################                    
  #
  # Example 3: Show off some fancy features including
  # use of pre-defined macros and logging.
  #
  ####################                                                    

  Executable     = foo                                                    
  Requirements   = Memory >= 32 && OpSys == "IRIX65" && Arch =="SGI"     
  Rank		 = Memory >= 64
  Image_Size     = 28 Meg                                                 

  Error   = err.$(Process)                                                
  Input   = in.$(Process)                                                 
  Output  = out.$(Process)                                                
  Log = foo.log

  Queue 150
\end{verbatim}

\index{submit description file!examples|)}

%%%%%%%%%%%%%%%%%
\subsection{About Requirements and Rank}
%%%%%%%%%%%%%%%%%

The 
\AdAttr{requirements} and \AdAttr{rank} commands in the submit description file
are powerful and flexible. 
\index{submit commands!requirements}
\index{requirements attribute}
\index{rank attribute}
\index{ClassAd attribute!requirements}
\index{ClassAd attribute!rank}
Using them effectively requires care, and this section presents
those details.

Both \AdAttr{requirements} and \AdAttr{rank} need to be specified 
as valid Condor ClassAd expressions, however, default values are set by the
\Condor{submit} program if these are not defined in the submit description file.
From the \Condor{submit} manual page and the above examples, you see
that writing ClassAd expressions is intuitive, especially if you
are familiar with the programming language C.  There are some
pretty nifty expressions you can write with ClassAds.
A complete description of ClassAds and their expressions
can be found in section~\ref{classad-reference} on 
page~\pageref{classad-reference}.

All of the commands in the submit description file are case insensitive, 
\emph{except} for the ClassAd attribute string values.
ClassAds attribute names are
case insensitive, but ClassAd string
values are always \emph{case sensitive}.
The correct specification for an architecture is
\begin{verbatim}
        requirements = arch == "ALPHA"
\end{verbatim}
so an accidental specification of
\begin{verbatim}
        requirements = arch == "alpha"
\end{verbatim}
will not work due to the incorrect case.

The allowed
ClassAd attributes are those 
that appear in a machine or a job ClassAd.
To see all of the machine ClassAd attributes for all machines in
the Condor pool, run \Condor{status -l}.  
\index{Condor commands!condor\_status}
The \Arg{-l} argument to
\Condor{status} means to display all the complete machine ClassAds.
The job ClassAds, if there jobs in the queue, can be seen
with the \Condor{q -l} command.
This
will show you all the available attributes you can play with.

To help you out with what these attributes all signify,
descriptions follow for the attributes which will be common to every
machine ClassAd. Remember that because ClassAds are flexible, the
machine ads in your pool may include additional attributes specific
to your site's installation and policies. 

\subsubsection{\label{user-man-machad}ClassAd Machine Attributes}
\begin{description}
%
\item[Activity] : String which describes Condor job activity on the machine.
Can have one of the following values:
	\begin{description}
	\item[``Idle''] : There is no job activity
	\item[``Busy''] : A job is busy running
	\item[``Suspended''] : A job is currently suspended
	\item[``Vacating''] : A job is currently checkpointing
	\item[``Killing''] : A job is currently being killed
	\item[``Benchmarking''] : The startd is running benchmarks
	\end{description}
%
\item[AFSCell] : If the machine is running AFS, this is a string
containing the AFS cell name.
%
\item[Arch] : String with the architecture of the machine.  Typically
one of the following: 
	\begin{description}
	\item[``INTEL''] : Intel CPU (Pentium, Pentium II, etc).
	\item[``ALPHA''] : Digital Alpha CPU
	\item[``SGI''] : Silicon Graphics MIPS CPU
	\item[``SUN4u''] : Sun UltraSparc CPU
	\item[``SUN4x''] : A Sun Sparc CPU other than an UltraSparc, i.e.
sun4m or sun4c CPU found in older Sparc workstations such as the Sparc~10, 
Sparc~20, IPC, IPX, etc.
	\item[``HPPA1''] :  Hewlett Packard PA-RISC 1.x CPU (i.e. PA-RISC    
                      7000 series CPU) based workstation
	\item[``HPPA2''] :  Hewlett Packard PA-RISC 2.x CPU (i.e. PA-RISC    
                      8000 series CPU) based workstation
	\end{description}
%
\item[ClockDay] : The day of the week, where 0 = Sunday, 1 = Monday, \Dots, 6 = Saturday. 
%
\item[ClockMin] : The number of minutes passed since midnight.
%
\item[CondorLoadAvg] : The load average generated by Condor (either
from remote jobs or running benchmarks).
%
\item[ConsoleIdle] : The number of seconds since activity on the system
console keyboard or console mouse has last been detected.
%
\item[Cpus] : Number of CPUs in this machine, i.e. 1 = single CPU machine, 2 = dual
CPUs, etc.
%
\item[CurrentRank] : A float which represents this machine owner's affinity
for running the Condor job which it is currently hosting.  If not
currently hosting a Condor job, CurrentRank is -1.0.
%
\item[Disk] : The amount of disk space on this machine available for
the job in kbytes ( e.g. 23000 = 23 megabytes ).  Specifically, this
is the amount of disk space available in the directory specified in
the Condor configuration files by the \Macro{EXECUTE} macro, minus any
space reserved with the \Macro{RESERVED\_DISK} macro.
%
\item[EnteredCurrentActivity] : Time at which the machine entered the 
current Activity (see \AdAttr{Activity} entry above).  Measured in the
number of seconds since the epoch (00:00:00 UTC, Jan 1, 1970).
%
\item[FileSystemDomain] : a domain name configured by the Condor 
administrator which describes a cluster of machines which all access 
the same networked filesystems usually via NFS or AFS.  
%
\item[KeyboardIdle] : The number of seconds since activity on any
keyboard or mouse associated with this machine has last been detected.
Unlike \AdAttr{ConsoleIdle}, \AdAttr{KeyboardIdle} also takes activity 
on pseudo-terminals into
account (i.e. virtual ``keyboard'' activity from telnet and rlogin
sessions as well).  Note that \AdAttr{KeyboardIdle} will always be equal to or
less than \AdAttr{ConsoleIdle}.
%
\item[KFlops] : Relative floating point performance as determined via a
linpack benchmark.
%
\item[LastHeardForm] : Time when the Condor Central Manager last
received a status update from this machine.  
Expressed as seconds since the epoch (integer value).
Note: This attribute is only inserted by the Central Manager once it
receives the ClassAd.
It is not present in the startd's copy of the ClassAd.
Therefore, you couldn't use this attribute in defining startd
expressions (which you wouldn't want to, anyway).
%
\item[LoadAvg] : A floating point number with the machine's current load
average.
%
\item[Machine] : A string with the machine's fully qualified hostname.
%
\item[Memory] : The amount of RAM in megabytes.
%
\item[Mips] : Relative integer performance as determined via a dhrystone
benchmark.
%
\item[MyType] : The ClassAd type; always set to the literal string ``Machine''.
%
\item[Name] : The name of this resource; typically the same value as
the \AdAttr{Machine} attribute, but could be customized by the site
administrator.
On SMP machines, the startd will divide the CPUs up into seperate
virtual machines, each with with a unique name.
These names will be of the form ``vm\#@full.hostname'', for example,
``vm1@vulture.cs.wisc.edu'', which signifies virtual machine 1 from
vulture.cs.wisc.edu. 
%
\item[OpSys] : String describing the operating system running on this
machine.  For Condor \VersionNotice\ typically one of the following:
	\begin{description}
	\item ``HPUX10'' (for HPUX 10.20)
	\item ``IRIX6''  (for IRIX 6.2, 6.3, or 6.4)
	\item ``LINUX''  (for LINUX 2.0.x kernel systems)
	\item ``LINUX-GLIBC''  (for LINUX systems, using GNU's libc)
	\item ``OSF1''	 (for Digital Unix 4.x)
	\item ``SOLARIS251''
	\item ``SOLARIS26''
	\end{description}
%
\item[Requirements] : A boolean which, when evaluated within the context
of the Machine ClassAd and a Job ClassAd, must evaluate to
TRUE before Condor will allow the job to use this machine.
%
\item[StartdIpAddr] : String with the IP and port address of the
\Condor{startd} daemon which is publishing this Machine ClassAd.
%
\item[State] : String which publishes the machine's Condor state, which
can be:
	\begin{description}
	\item[``Owner''] : The machine owner is using the machine, and
it is unavailable to Condor.
	\item[``Unclaimed''] : The machine is available to run Condor jobs,
but a good match (i.e. job to run here) is either not available or not 
yet found.
	\item[``Matched''] : The Condor Central Manager has found a good
match for this resource, but a Condor scheduler has not yet claimed it.
	\item[``Claimed''] : The machine is claimed by a remote
\Condor{schedd} and is probably running a job.
	\item[``Preempting''] : A Condor job is being preempted (possibly
via checkpointing) in order to clear the machine for either a higher
priority job or because the machine owner wants the machine back.
	\end{description}   % of State
%
\item[TargetType] : Describes what type of ClassAd to match with.
Always set to the string literal ``Job'', because Machine ClassAds
always want to be matched with Jobs, and vice-versa.
%
\item[UidDomain] : a domain name configured by the Condor 
administrator which describes a cluster of machines which all have 
the same "passwd" file entries, and therefore all have the same logins.
%
\item[VirtualMemory] : The amount of currently available virtual memory 
(swap space) expressed in kbytes.

\end{description}


\subsubsection{\label{user-man-jobad}ClassAd Job Attributes}
\begin{description}
%
\index{ClassAd!job attributes}
%
\item[\AdAttr{CkptArch}] : String describing the architecture of the machine
where this job last checkpointed.  If the job has never checkpointed,
this attribute is UNDEFINED.
%
\item[\AdAttr{CkptOpSys}] : String describing the operating system of
the machine where this job last checkpointed.  If the job has never
checkpointed, this attribute is UNDEFINED.
%
\item[\AdAttr{ClusterId}] : Integer cluster identifier for this job.
%
\item[\AdAttr{ExecutableSize}] : Size of the executable in kbytes.
%
\item[\AdAttr{ImageSize}] : Estimate of the memory image size of the
job in kbytes.  The initial estimate may be specified in the job
submit file.  Otherwise, the initial value is equal to the size of the
executable.  When the job checkpoints, the \AdAttr{ImageSize}
attribute is set to the size of the checkpoint file (since the
checkpoint file contains the job's memory image).
%
\item[\AdAttr{JobPrio}] : Integer priority for this job, set by
\Condor{submit} or \Condor{prio}.  The default value is 0.
%
\item[\AdAttr{JobStatus}] : Integer which indicates the current
status of the job, where 1 = Idle, 2 = Running, 3 = Removed, 4 =
Completed, and 5 = Held.
%
\item[\AdAttr{JobUniverse}] : Integer which indicates the job
universe, where 1 = Standard, 4 = PVM, 5 = Vanilla, and 7 = Scheduler.
%
\item[\AdAttr{LastCkptServer}] : Hostname of the last checkpoint
server used by this job.  When a pool is using multiple checkpoint
servers, this tells the job where to find its checkpoint file.
%
\item[\AdAttr{NiceUser}] : Boolean value which indicates whether
this is a nice-user job.
%
\item[\AdAttr{Owner}] : String describing the user who submitted this
job.
%
\item[\AdAttr{ProcId}] : Integer process identifier for this job.  In
a cluster of many jobs, each job will have the same ClusterId but will
have a unique ProcId.
%
\item[\AdAttr{QDate}] : Time at which the job was submitted to the job
queue.  Measured in the
number of seconds since the epoch (00:00:00 UTC, Jan 1, 1970).
%
\end{description}


\subsubsection{\label{rank-examples}Rank Expression Examples}

\index{rank attribute!examples}
\index{ClassAd attribute!rank examples}
\index{submit commands!rank}
When considering the match between a job and a machine, rank is used
to choose a match from among all machines that satisfy the job's
requirements and are available to the user, after accounting for
the user's priority and the machine's rank of the job.
The rank expressions, simple or complex, define a numerical value
that expresses preferences.

The job's \AdAttr{rank} expression evaluates to one of three values.
It can be UNDEFINED, ERROR, or a floating point value.
If \AdAttr{rank} evaluates to a floating point value,
the best match will be the one with the largest, positive value.
If no \AdAttr{rank} is given 
in the submit description file,
then Condor substitutes a default value of 0.0 when considering
machines to match.
If the job's \AdAttr{rank} of a given machine evaluates
to UNDEFINED or ERROR,
this same value of 0.0 is used.
Therefore, the machine is still considered for a match,
but has no rank above any other.

A boolean expression evaluates to the numerical value of 1.0
if true, and 0.0 if false.

The following \AdAttr{rank} expressions provide examples to
follow.

For a job that desires the machine with the most available memory:
\begin{verbatim}
   Rank = memory
\end{verbatim}

For a job that prefers to run on a friend's machine
on Saturdays and Sundays:
\begin{verbatim}
   Rank = ( (clockday == 0) || (clockday == 6) )
          && (machine == "friend.cs.wisc.edu")
\end{verbatim}

For a job that prefers to run on one of three specific machines:
\begin{verbatim}
   Rank = (machine == "friend1.cs.wisc.edu") ||
          (machine == "friend2.cs.wisc.edu") ||
          (machine == "friend3.cs.wisc.edu")
\end{verbatim}

For a job that wants the machine with the best floating point
performance (on Linpack benchmarks):
\begin{verbatim}
   Rank = kflops
\end{verbatim}
This particular example highlights a difficulty with rank expression
evaluation as currently defined.
While all machines have floating point processing ability,
not all machines will have the \AdAttr{kflops} attribute defined.
For machines where this attribute is not defined,
\AdAttr{Rank} will evaluate to the value UNDEFINED, and
Condor will use a default rank of the machine of 0.0.
The \AdAttr{rank} attribute will only rank machines where
the attribute is defined.
Therefore, the machine with the highest floating point
performance may not be the one given the highest rank.

So, it is wise when writing a \AdAttr{rank} expression to check
if the expression's evaluation will lead to the expected
resulting ranking of machines.
This can be accomplished using the \Condor{status} command with the
\oArg{-constraint} argument.  This allows the user to see a list of
machines that fit a constraint.
To see which machines in the pool have \AdAttr{kflops} defined,
use
\begin{verbatim}
condor_status -constraint kflops
\end{verbatim}
Alternatively, to see a list of machines where 
\AdAttr{kflops} is not defined, use
\begin{verbatim}
condor_status -constraint "kflops=?=undefined"
\end{verbatim}

For a job that prefers specific machines in a specific order:
\begin{verbatim}
   Rank = ((machine == "friend1.cs.wisc.edu")*3) +
          ((machine == "friend2.cs.wisc.edu")*2) +
           (machine == "friend3.cs.wisc.edu")
\end{verbatim}
If the machine being ranked is \AdStr{friend1.cs.wisc.edu}, then the
expression
\begin{verbatim}
   (machine == "friend1.cs.wisc.edu")
\end{verbatim}
is true, and gives the value 1.0.
The expressions
\begin{verbatim}
   (machine == "friend2.cs.wisc.edu")
\end{verbatim}
and
\begin{verbatim}
   (machine == "friend3.cs.wisc.edu")
\end{verbatim}
are false, and give the value 0.0.
Therefore, \AdAttr{rank} evaluates to the value 3.0.
In this way, machine \AdStr{friend1.cs.wisc.edu} is ranked higher than
machine \AdStr{friend2.cs.wisc.edu},
machine \AdStr{friend2.cs.wisc.edu}
is ranked higher than 
machine \AdStr{friend3.cs.wisc.edu},
and all three of these machines are ranked higher than others.

%%%%%%%%%%%% 
\subsection{\label{sec:shared-fs}
Submitting Jobs Using a Shared File System} 
%%%%%%%%%%%%
\index{job!submission using a shared file system}
\index{shared file system!submission of jobs}

If vanilla, Java or MPI
jobs are submitted without using the File Transfer mechanism, 
Condor must use a shared file system to access input and output
files. 
In this case, the job \emph{must} be able to access the data files
from any machine on which it could potentially run.

As an example, suppose a job is submitted from blackbird.cs.wisc.edu,
and the job requires a particular data file called
\File{/u/p/s/psilord/data.txt}.  If the job were to run on
cardinal.cs.wisc.edu, the file \File{/u/p/s/psilord/data.txt} must be
available through either NFS or AFS for the job to run correctly.

Condor allows users to ensure their jobs have access to the right
shared files by using the \AdAttr{FileSystemDomain} and
\AdAttr{UidDomain} machine ClassAd attributes.
These attributes specify which machines have access to the same shared
file systems.
All machines that mount the same shared directories in the same
locations are considered to belong to the same file system domain.
Similarly, all machines that share the same user information (in
particular, the same UID, which is important for file systems like
NFS) are considered part of the same UID domain.

The default configuration for Condor places each machine
in its own UID domain and file system domain, using the full hostname of the
machine as the name of the domains.
So, if a pool \emph{does} have access to a shared file system,
the pool administrator \emph{must} correctly configure Condor 
such that all
the machines mounting the same files have the same
\AdAttr{FileSystemDomain} configuration.
Similarly, all machines that share common user information must be
configured to have the same \AdAttr{UidDomain} configuration.

When a job relies on a shared file system,
Condor uses the
\AdAttr{requirements} expression to ensure that the job runs
on a machine in the
correct \AdAttr{UidDomain} and \AdAttr{FileSystemDomain}.
In this case, the default \AdAttr{requirements} expression specifies
that the job must run on a machine with the same \AdAttr{UidDomain}
and \AdAttr{FileSystemDomain} as the machine from which the job
is submitted.
This default is almost always correct.
However, in a pool spanning multiple \AdAttr{UidDomain}s and/or
\AdAttr{FileSystemDomain}s, the user may need to specify a different
\AdAttr{requirements} expression to have the job run on the correct
machines.

For example, imagine a pool made up of both desktop workstations and a
dedicated compute cluster.
Most of the pool, including the compute cluster, has access to a
shared file system, but some of the desktop machines do not.
In this case, the administrators would probably define the
\AdAttr{FileSystemDomain} to be \File{cs.wisc.edu} for all the machines
that mounted the shared files, and to the full hostname for each
machine that did not. An example is \File{jimi.cs.wisc.edu}.

In this example,
a user wants to submit vanilla universe jobs from her own desktop
machine (jimi.cs.wisc.edu) which does not mount the shared file system
(and is therefore in its own file system domain, in its own world).
But, she wants the jobs to be able to run on more than just her own
machine (in particular, the compute cluster), so she puts the program
and input files onto the shared file system.
When she submits the jobs, she needs to tell Condor to send them to
machines that have access to that shared data, so she specifies a
different \AdAttr{requirements} expression than the default:
\begin{verbatim}
   Requirements = UidDomain == "cs.wisc.edu" && \
                  FileSystemDomain == "cs.wisc.edu"
\end{verbatim}

\Warn If there is \emph{no} shared file system, or the Condor pool
administrator does not configure the \AdAttr{FileSystemDomain}
setting correctly (the default is that each machine in a pool is in
its own file system and UID domain), a user submits a job that cannot
use remote system calls (for example, a vanilla universe job), and the
user does not enable Condor's File Transfer mechanism, the job will
\emph{only} run on the machine from which it was submitted.


%%%%%%%%%%%% 
\subsection{\label{sec:file-transfer}
Submitting Jobs Without a Shared File System:
Condor's File Transfer Mechanism} 
%%%%%%%%%%%%

\index{job!submission without a shared file system}
\index{shared file system!submission of jobs without one}
\index{file transfer mechanism}
\index{transferring files}

Condor works well without a shared file system.
The Condor file transfer mechanism is utilized by the user
when the user submits jobs.
Condor will transfer any files needed by a job from
the machine where the job was submitted into a
temporary working directory on the machine where the
job is to be executed.
Condor executes the job
and transfers output back to the submitting machine.
The user specifies which files to transfer,
and at what point the output files should be copied back to the
submitting machine.
This specification is done within the job's submit description file.

The default behavior of the file transfer mechanism
varies across the
different Condor universes, and it differs between UNIX and Windows machines.

%%%%%%%%%%%% 
\subsubsection{Default Behavior across Condor Universes and Platforms}
%%%%%%%%%%%%

For jobs submitted under the standard universe,
the existence of a shared file system is not relevant.
Access to files (input and output) is handled through Condor's
remote system call mechanism.
The executable and checkpoint files are transfered automatically, when
needed. 
Therefore, the user does not need to change the submit description
file if there is no shared file system.

For the vanilla, Java, and MPI universes, access to files (including
the executable) through a shared file system is presumed as a default
on UNIX machines.
If there is no shared file system, then Condor's file transfer
mechanism must be explicitly enabled.
When submitting a job from a Windows machine,
Condor presumes the opposite: no access to a shared file system.
It instead enables the file transfer mechanism by default.
Submission of a job might need to specify which files to
transfer, and/or when to transfer the output files back.

For the grid universe,
jobs are to be executed on remote machines, so there would never
be a shared file system between machines.
See section~\ref{sec:Using-Condor-G} for more details.

For the PVM universe,
file transfer other than the master's executable and files given in
\SubmitCmd{input},
\SubmitCmd{output}, and
\SubmitCmd{error} commands is not supported.
This is not usually an impediment (shared file system or not), since
PVM jobs are set up to have the master direct the workers, and I/O
from the workers is usually passed back to the master via PVM
messages, not files.

For the scheduler universe,
Condor is only using the machine from which the job is submitted.
Therefore, the existence of a shared file system is not relevant.


%%%%%%%%%%%% 
\subsubsection{Specifying If and When to Transfer Files
\label{sec:file-transfer-if-when}}
%%%%%%%%%%%%

To enable the file transfer mechanism, two commands are
placed in the job's submit description file:
\SubmitCmd{should\_transfer\_files} and \SubmitCmd{when\_to\_transfer\_output}.
\index{submit commands!should\_transfer\_files}
\index{submit commands!when\_to\_transfer\_output}
An example is:

\begin{verbatim}
  should_transfer_files = YES
  when_to_transfer_output = ON_EXIT
\end{verbatim}

The \SubmitCmd{should\_transfer\_files} command specifies
whether Condor should
transfer input files from the submit machine to the remote machine
where the job executes.
It also specifies whether the output files are transferred 
back to the submit machine.
The command takes on one of three possible values:
\begin{enumerate}

\item \verb@YES@: Condor always transfers both input and output files.

\item \verb@IF_NEEDED@: Condor transfers files if the job is
matched with (and to be executed on) a machine in a
different \Attr{FileSystemDomain} than the
one the submit machine belongs to.
If the job is matched with a machine in the local \Attr{FileSystemDomain},
Condor will not transfer files and relies
on a shared file system.

\item \verb@NO@: Condor's file transfer mechanism is disabled. 

\end{enumerate}

The \SubmitCmd{when\_to\_transfer\_output} command tells Condor when output
files are to be transferred back to the submit machine after the job
has executed on a remote machine.
The command takes on one of two possible values:

\begin{enumerate}
\item \verb@ON_EXIT@: Condor transfers output files back to the submit
machine only when the job exits on its own.

\item \verb@ON_EXIT_OR_EVICT@:
Condor will always do the transfer,
whether the job completes on its own, is preempted by another job, 
vacates the machine, or is killed.
As the job completes on its own, files are transferred back
to the directory where the job was submitted, as expected.
For the other cases, \emph{files are transferred back at eviction time}.
These files are placed in
the directory defined by the configuration
variable \MacroNI{SPOOL}, not the directory from which the
job was submitted.
The transferred files are named using the
\Attr{ClusterId} and \Attr{ProcId} job ClassAd attributes. 
The file name takes the form:
\begin{verbatim}
   cluster<X>.proc<Y>.subproc0
\end{verbatim}
where \verb@<X>@ is the value of \Attr{ClusterId}, and 
\verb@<Y>@ is the value of \Attr{ProcId}. 
As an example, job 735.0 may produce the file
\begin{verbatim}
   $(SPOOL)/cluster735.proc0.subproc0
\end{verbatim}

This is only useful if partial runs of the job are valuable.
An example of valuable partial runs is when the application
produces its own checkpoints.
\end{enumerate}

There is no default value for \SubmitCmd{when\_to\_transfer\_output}.
If using the file transfer mechanism, 
this command must be defined.
If \SubmitCmd{when\_to\_transfer\_output} is specified in the submit
description file,
but \SubmitCmd{should\_transfer\_files} is not, Condor assumes a
value of \verb@YES@ for \SubmitCmd{should\_transfer\_files}.

\Note The combination of:
\begin{verbatim}
  should_transfer_files = IF_NEEDED
  when_to_transfer_output = ON_EXIT_OR_EVICT
\end{verbatim}
would produce undefined file access semantics.
Therefore, this combination is prohibited by \Condor{submit}.

When submitting from a Unix platform,
the file transfer mechanism is unused by default.
If neither \SubmitCmd{when\_to\_transfer\_output} or \SubmitCmd{should\_transfer\_files} 
are defined, Condor assumes
\verb@should_transfer_files = NO@.

When submitting from a Windows platform,
Condor does not provide any way to use a shared file
system for jobs. 
Therefore, if 
neither \SubmitCmd{when\_to\_transfer\_output} or \SubmitCmd{should\_transfer\_files}
are defined, the file
transfer mechanism is enabled by default with the following values:

\begin{verbatim}
  should_transfer_files = YES
  when_to_transfer_output = ON_EXIT
\end{verbatim}

\Note Prior to Condor version 6.5.2, different attributes were used to
control when and if files should be transferred.
Previously, a single attribute was used to control both things, and
the \verb@IF_NEEDED@ value was not supported.
This older attribute is still allowed in newer versions of Condor but
it is now deprecated.
\SubmitCmd{when\_to\_transfer\_output} and
\SubmitCmd{should\_transfer\_files} should be used instead.
However, beware that these settings will not work with Condor versions
older than 6.5.2. 


%%%%%%%%%%%% 
\subsubsection{Specifying What Files to Transfer}
%%%%%%%%%%%%

If the file transfer mechanism is enabled,
Condor will transfer the following files before the job
is run on a remote machine.
\begin{enumerate}
  \item the executable
  \item the input, as defined with the \SubmitCmd{input} command
  \item any jar files (for the Java universe)
\end{enumerate}
If the job requires any other input files,
the submit description file should utilize the
\SubmitCmd{transfer\_input\_files} command.
This comma-separated list specifies any other files that Condor is to
transfer to a remote site to set up the execution environment for the
job before it is run.
These files are placed in the same temporary working directory
as the job's executable.
At this time, directories can not be transferred in this way.
For example:

\begin{verbatim}
  transfer_input_files = file1,file2 
\end{verbatim}

As a default, for jobs other than those submitted to the grid universe,
any files that are modified or created by the job in the
temporary directory at the remote site are transferred back
to the machine from which the job was submitted.
Most of the time, this is the best option.
To restrict the files that are transferred,
specify the exact list of files with  \SubmitCmd{transfer\_output\_files}.
Delimite these file names with a comma.
When this list is defined, and any of the files do not exist as the
job exits, Condor considers this an error, and re-runs the job.

\Warn Do not specify \SubmitCmd{transfer\_output\_files} (for other than
grid universe jobs) unless there is a
really good reason -- it is best to let Condor figure things out by
itself based upon what output the job produces.

For grid universe jobs, files to be transferred 
(other than standard output and standard error)
must be specified using \SubmitCmd{transfer\_output\_files}
in the submit description file. 

%%%%%%%%%%%%
\subsubsection{File Paths for File Transfer}
%%%%%%%%%%%%

% Note: it might be nice to get the initialdir entry in
% the index to refer to something in here.

% Note: a Windows-based example would be good, too.

The file transfer mechanism specifies file names and/or paths on
both the file system of the submit machine and on the
file system of the execute machine.
Care must be taken to know which machine (submit or execute)
is utilizing the file name and/or path. 

Files in the \SubmitCmd{transfer\_input\_files} command
are specified as they are accessed on the submit machine.
The program (as it executes) accesses files as they are
found on the execute machine.

There are three ways to specify files and paths
for \SubmitCmd{transfer\_input\_files}:
\begin{enumerate}
\item Relative to the submit directory, if the submit command
\SubmitCmd{initialdir} is not specified.
\item Relative to the initial directory, if the submit command 
\SubmitCmd{initialdir} is specified.
\item Absolute.
\end{enumerate}

Before executing the program, Condor copies the
executable, an input file as specified
by the submit command \SubmitCmd{input},
along with any input files specified 
by \SubmitCmd{transfer\_input\_files}.
All these files are placed into
a temporary directory (on the execute machine)
in which the program runs.
Therefore,
the executing program must access input files \emph{without} paths.
Because all transferred files are placed into a single,
flat directory,
input files must be uniquely named to
avoid collision when transferred.
A collision causes the last file in the list to
overwrite the earlier one.

If the program creates output files during execution,
it must create them within the temporary working directory.
Condor transfers back all files within the temporary
working directory that have been modified or created.
To transfer back only a subset of these files,
the submit command
\SubmitCmd{transfer\_output\_files}
is defined.
Transfer of files that exist,
but are not within the temporary working directory is not supported.
Condor's behavior in this instance is undefined.

It is
okay to create files outside the temporary working directory
on the file system of the execute machine,
(in a directory such as \File{/tmp})
if this directory is guaranteed to exist and be
accessible on all possible execute machines.
However,
transferring such a file back after execution completes
may not be done.

Here are several examples to illustrate the use of file transfer.
The program executable is called \Prog{my\_program},
and it uses three command-line arguments as it executes: 
two input file names and an output file name.
The program executable and the submit description file 
for this job are located in directory
\File{/scratch/test}. 

The directory tree for all these examples:
\begin{verbatim}
/scratch/test (directory)
      my_program.condor (the submit description file)
      my_program (the executable)
      files (directory)
          logs2 (directory)
          in1 (file)
          in2 (file)
      logs (directory)
\end{verbatim}

%--------------------------
\begin{description}
\item[Example 1]

This simple example explicitly transfers input files.
These input files to be transferred
are specified relative to the directory where the job is submitted.
The single output file, \File{out1}, created when the job is executed
will be transferred back into the directory
\File{/scratch/test}, \emph{not} the \File{files} directory. 

\footnotesize
\begin{verbatim}
# file name:  my_program.condor
# Condor submit description file for my_program
Executable      = my_program
Universe        = vanilla
Error           = logs/err.$(cluster)
Output          = logs/out.$(cluster)
Log             = logs/log.$(cluster)

should_transfer_files = YES
when_to_transfer_output = ON_EXIT
transfer_input_files = files/in1, files/in2

Arguments       = in1 in2 out1
Queue
\end{verbatim}
\normalsize

%--------------------------
\item[Example 2]

This second example is identical to Example 1,
except that absolute paths to the input files are specified,
instead of relative paths to the input files.

\footnotesize
\begin{verbatim}
# file name:  my_program.condor
# Condor submit description file for my_program
Executable      = my_program
Universe        = vanilla
Error           = logs/err.$(cluster)
Output          = logs/out.$(cluster)
Log             = logs/log.$(cluster)

should_transfer_files = YES
when_to_transfer_output = ON_EXIT
transfer_input_files = /scratch/test/files/in1, /scratch/test/files/in2

Arguments       = in1 in2 out1
Queue
\end{verbatim}
\normalsize

%--------------------------
\item[Example 3]

This third example illustrates the use of the 
submit command \SubmitCmd{initialdir}, and its effect
on the paths used for the various files.
The expected location of the 
executable is not affected by the 
\SubmitCmd{initialdir} command.
All other files
(specified by \SubmitCmd{input},
\SubmitCmd{output},
\SubmitCmd{transfer\_input\_files},
as well as files modified or created by the job
and automatically transferred back)
are located relative to the specified \SubmitCmd{initialdir}.
Therefore, the output file, \File{out1},
will be placed in the \verb@files@ directory.
Note that the \File{logs2} directory
exists to make this example work correctly.

\footnotesize
\begin{verbatim}
# file name:  my_program.condor
# Condor submit description file for my_program
Executable      = my_program
Universe        = vanilla
Error           = logs2/err.$(cluster)
Output          = logs2/out.$(cluster)
Log             = logs2/log.$(cluster)

initialdir      = files

should_transfer_files = YES
when_to_transfer_output = ON_EXIT
transfer_input_files = in1, in2

Arguments       = in1 in2 out1
Queue
\end{verbatim}
\normalsize

%--------------------------
\item[Example 4 -- Illustrates an Error]

This example illustrates a job that will fail.
The files specified using the
\SubmitCmd{transfer\_input\_files} command work
correctly (see Example 1).
However,
relative paths to files in the
\SubmitCmd{arguments} command
cause the executing program to fail.
The file system on the submission side may utilize
relative paths to files,
however those files are placed into a single,
flat, temporary directory on the execute machine.

Note that this specification and submission will cause the
job to fail and reexecute.

\footnotesize
\begin{verbatim}
# file name:  my_program.condor
# Condor submit description file for my_program
Executable      = my_program
Universe        = vanilla
Error           = logs/err.$(cluster)
Output          = logs/out.$(cluster)
Log             = logs/log.$(cluster)

should_transfer_files = YES
when_to_transfer_output = ON_EXIT
transfer_input_files = files/in1, files/in2

Arguments       = files/in1 files/in2 files/out1
Queue
\end{verbatim}
\normalsize

This example fails with the following error:
\footnotesize
\begin{verbatim}
err: files/out1: No such file or directory.
\end{verbatim}
\normalsize

%--------------------------
\item[Example 5 -- Illustrates an Error]

As with Example 4,
this example illustrates a job that will fail.
The executing program's use of 
absolute paths cannot work.

\footnotesize
\begin{verbatim}
# file name:  my_program.condor
# Condor submit description file for my_program
Executable      = my_program
Universe        = vanilla
Error           = logs/err.$(cluster)
Output          = logs/out.$(cluster)
Log             = logs/log.$(cluster)

should_transfer_files = YES
when_to_transfer_output = ON_EXIT
transfer_input_files = /scratch/test/files/in1, /scratch/test/files/in2

Arguments = /scratch/test/files/in1 /scratch/test/files/in2 /scratch/test/files/out1
Queue
\end{verbatim}
\normalsize

The job fails with the following error:
\footnotesize
\begin{verbatim}
err: /scratch/test/files/out1: No such file or directory.
\end{verbatim}
\normalsize

% Karen has editted to this point in the file transfer sections.

%--------------------------
\item[Example 6 -- Illustrates an Error]

This example illustrates a failure case
where the executing program creates an output file in a directory
other than within the single, flat, temporary directory that the 
program executes within.
The file creation may or may not cause an error,
depending on the existence and permissions
of the directories on the remote file system.

Further incorrect usage is seen during
the attempt to transfer the output file back 
using the \SubmitCmd{transfer\_output\_files} command.
The behavior of Condor for this case is undefined.

\footnotesize
\begin{verbatim}
# file name:  my_program.condor
# Condor submit description file for my_program
Executable      = my_program
Universe        = vanilla
Error           = logs/err.$(cluster)
Output          = logs/out.$(cluster)
Log             = logs/log.$(cluster)

should_transfer_files = YES
when_to_transfer_output = ON_EXIT
transfer_input_files = files/in1, files/in2
transfer_output_files = /tmp/out1

Arguments       = in1 in2 /tmp/out1
Queue
\end{verbatim}
\normalsize

\end{description}

%%%%%%%%%%%% 
\subsubsection{Requirements and Rank for File Transfer}
%%%%%%%%%%%%

\index{submit commands!requirements}
The \Attr{requirements} expression for a job must depend
on the \verb@should_transfer_files@ command.
The job must specify the correct logic to ensure that the job is matched
with a resource that meets the file transfer needs.
If no \Attr{requirements} expression is in the submit description file,
or if the expression specified does not refer to the
attributes listed below, \Condor{submit} adds an
appropriate clause to the \Attr{requirements} expression for the job.
\Condor{submit} appends these clauses with a logical AND, \verb@&&@,
to ensure that the proper conditions are met.
Here are the default clauses corresponding to the different values of
\verb@should_transfer_files@:

\begin{enumerate}

\item 
\verb@should_transfer_files = YES@ results in the addition of
the clause \verb@(HasFileTransfer)@.
  If the job is always going to transfer files, it is required to 
  match with a machine that has the capability to transfer files.
  This is a backward compatibility issue, since all versions
  of Condor since version 6.3.3 support file transfer and have
  \Attr{HasFileTransfer} defined to \verb@TRUE@.

\item 
\verb@should_transfer_files = NO@ results in the addition of
  \verb@(TARGET.FileSystemDomain == MY.FileSystemDomain)@.
  In addition, Condor automatically adds the
  \Attr{FileSystemDomain} attribute to the job ad, with whatever
  string is defined for the \Condor{schedd} to which the job is
  submitted.
  If the job is not using the file transfer mechanism, Condor assumes
  it will need a shared file system, and therefore, a machine in the
  same \Attr{FileSystemDomain} as the submit machine.

\item \verb@should_transfer_files = IF_NEEDED@ results in the addition of
\begin{verbatim}
  (HasFileTransfer || (TARGET.FileSystemDomain == MY.FileSystemDomain))
\end{verbatim}
  If Condor will optionally transfer files, it must require
  that the machine is \emph{either} capable of transferring files
  \emph{or} in the same file system domain.

\end{enumerate}

To ensure that the job is matched to a machine with enough local disk
space to hold all the transfered files, Condor automatically adds the
\Attr{DiskUsage} job attribute.
This attribute includes the total
size of the job's executable and all input files to be transferred.
Condor then adds an additional clause to the \Attr{Requirements}
expression that states that the remote machine must have at least
enough available disk space to hold all these files:
\begin{verbatim}
  && (Disk >= DiskUsage)
\end{verbatim}

If \verb@should_transfer_files = IF_NEEDED@ and the job prefers
to run on a machine in the local file system domain
over transferring files,
(but are still willing to allow the job to run remotely and transfer
\index{submit commands!rank}
files), the \Attr{rank} expression works well.  Use:

\begin{verbatim}
rank = (TARGET.FileSystemDomain == MY.FileSystemDomain)
\end{verbatim}

The \Attr{rank} expression is a floating point number, so if 
other items are considered in ranking the possible machines this job
may run on, add the items:

\begin{verbatim}
rank = kflops + (TARGET.FileSystemDomain == MY.FileSystemDomain)
\end{verbatim}

The value of \Attr{kflops} can vary widely among machines,
so this \Attr{rank} expression will likely not do as it intends.
To place emphasis on the job running in the same file
system domain,
but still consider kflops among the machines in the file system domain,
weight the part of the rank expression that is matching the file system domains.
For example: 

\begin{verbatim}
rank = kflops + (10000 * (TARGET.FileSystemDomain == MY.FileSystemDomain))
\end{verbatim}

%%%%%%%%%%%% 
\subsubsection{Old Attributes for File Transfer}
%%%%%%%%%%%%

The \verb@should_transfer_files@ and \verb@when_to_transfer_output@
commands in the submit description file result in two corresponding string
attributes in the job ClassAd: \Attr{ShouldTransferFiles} and
\Attr{WhenToTransferOutput}.
These attributes are only defined when the job is matched with an
execute machine running Condor version 6.5.3 or a more recent version.
So, for backward compatibility, \Condor{submit} also includes the old
attribute used to control this feature: \Attr{TransferFiles}.
If you examine a job with the \Opt{-long} option to \Condor{q}, and
you see \Attr{TransferFiles}, that attribute is only there for
backward compatibility, and it is ignored if matched with a machine
running version 6.5.3 or greater.  
There were problems with this old attribute, since it
was not flexible enough to handle the new \verb@IF_NEEDED@
functionality, and it was confusing for users.
Therefore, \Attr{TransferFiles} is deprecated, and we will no longer
document its use.
If your submit file refers to \verb@transfer_files@,
consider switching it to use the settings described here.

%%%%%%%%%%%% 
\subsection{Environment Variables}
%%%%%%%%%%%% 

\index{environment variables}
\index{execution environment}
The environment under which a job executes often contains
information that is potentially useful to the job.
Condor allows a user to both set and reference environment
variables for a job or job cluster.

Within a submit description file, the user may define environment
variables for the job's environment by using the 
\Opt{environment} command.
See the \Condor{submit} manual page at
section~\ref{man-condor-submit} for more details about this command.

The submittor's entire environment can be copied into the job
ClassAd for the job at job submission.
The \Opt{getenv} command within the submit description file
does this.
See the \Condor{submit} manual page at
section~\ref{man-condor-submit} for more details about this command.

Commands within the submit description file may reference the
environment variables of the submitter as a job is submitted.
Submit description file commands use \verb@$ENV(EnvironmentVariableName)@
to reference the value of an environment variable.
Again,
see the \Condor{submit} manual page at
section~\ref{man-condor-submit} for more details about this usage.

Condor sets several additional environment variables for each executing
job that may be useful for the job to reference.

\begin{itemize}
\item \Env{CONDOR\_SCRATCH\_DIR}
\index{CONDOR\_SCRATCH\_DIR}
\index{environment variables!CONDOR\_SCRATCH\_DIR}
 gives the directory
where the job may place temporary data files. 
This directory is unique for
every job that is run, and it's contents are deleted by Condor
when the job stops running on a machine, no matter how the job completes.

\item \Env{CONDOR\_VM}
\index{CONDOR\_VM}
\index{environment variables!CONDOR\_VM}
gives the name of the virtual machine (for SMP machines),
on which the job is run.  This setting is only available in the
standard universe.
See 
section~\ref{sec:Configuring-SMP} for more details about SMP
machines and their configuration.

\item \Env{X509\_USER\_PROXY}
\index{X509\_USER\_PROXY}
\index{environment variables!X509\_USER\_PROXY}
gives the full path to the X509 user proxy file if one is
associated with the job.  (Typically a user will specify
\SubmitCmd{x509userproxy} in the submit file.)
This setting is currently available in the
local, java, and vanilla universes.

\end{itemize}



%%%%%%%%%%%% 
\subsection{Heterogeneous Submit: Execution on Differing Architectures} 
%%%%%%%%%%%%

\index{job!heterogeneous submit}
\index{running a job!on a different architecture}
\index{heterogeneous pool!submitting a job to}
If executables are available for the different platforms of machines
in the Condor pool,
Condor can be allowed the choice of a larger number of machines
when allocating a machine for a job.
Modifications to the submit description file allow this choice
of platforms.

A simplified example is a cross submission.
An executable is available for one platform, but
the submission is done from a different platform.
Given the correct executable, the \AdAttr{requirements} command in
the submit description file specifies the target architecture.
For example, an executable compiled for a Sun 4, submitted
from an Intel architecture running Linux would add the 
\AdAttr{requirement}
\begin{verbatim}
  requirements = Arch == "SUN4x" && OpSys == "SOLARIS251"
\end{verbatim}
Without this \AdAttr{requirement}, \Condor{submit}
will assume that the program is to be executed on
a machine with the same platform as the machine where the job
is submitted.

Cross submission works for both
\Expr{standard} and \Expr{vanilla} universes.
The burden is on the user to both obtain and specify
the correct executable for the target architecture.
To list the architecture and operating systems of the machines
in a pool, run \Condor{status}.

%%%%%%%%%%%% 
\subsubsection{Vanilla Universe Example for Execution on Differing Architectures} 
%%%%%%%%%%%%

A more complex example of a heterogeneous submission
occurs when a job may be executed on
many different architectures to gain full
use of a diverse architecture and operating system pool.
If the executables are available for the different architectures,
then a modification to the submit description file
will allow Condor to choose an executable after an
available machine is chosen.

A special-purpose Machine Ad substitution macro can be used in
the \AdAttr{executable}, \AdAttr{environment},  and \AdAttr{arguments}
attributes in the submit description file.
The macro has the form
\begin{verbatim}
  $$(MachineAdAttribute)
\end{verbatim}
Note that this macro is ignored in all other submit description attributes.
The \$\$() informs Condor to substitute the requested 
\AdAttr{MachineAdAttribute} 
from the machine where the job will be executed.

An example of the heterogeneous job submission
has executables available for three platforms:
LINUX Intel, Solaris26 Intel, and Irix 6.5 SGI machines.
This example uses \Prog{povray}
to render images using a popular free rendering engine.

The substitution macro chooses a specific executable after
a platform for running the job is chosen.
These executables must therefore be named based on the
machine attributes that describe a platform.
The executables named \begin{verbatim}
  povray.LINUX.INTEL
  povray.SOLARIS26.INTEL
  povray.IRIX65.SGI
\end{verbatim}
will work correctly for the macro
\begin{verbatim}
  povray.$$(OpSys).$$(Arch)
\end{verbatim}

The executables or links to executables with this name
are placed into the initial working directory so that they may be
found by Condor. 
A submit description file that queues three jobs for this example:

\begin{verbatim}
  ####################
  #
  # Example of heterogeneous submission
  #
  ####################

  universe     = vanilla
  Executable   = povray.$$(OpSys).$$(Arch)
  Log          = povray.log
  Output       = povray.out.$(Process)
  Error        = povray.err.$(Process)

  Requirements = (Arch == "INTEL" && OpSys == "LINUX") || \
                 (Arch == "INTEL" && OpSys =="SOLARIS26") || \
                 (Arch == "SGI" && OpSys == "IRIX65")

  Arguments    = +W1024 +H768 +Iimage1.pov
  Queue 

  Arguments    = +W1024 +H768 +Iimage2.pov
  Queue 

  Arguments    = +W1024 +H768 +Iimage3.pov
  Queue 
\end{verbatim}

These jobs are submitted to the vanilla universe
to assure that once a job is started on a specific platform,
it will finish running on that platform.
Switching platforms in the middle of job execution cannot
work correctly.

There are two common errors made with the substitution macro.
The first is the use of a non-existent \AdAttr{MachineAdAttribute}.
If the specified \AdAttr{MachineAdAttribute} does not
exist in the machine's ClassAd, then Condor will place
the job in the machine state of hold until the problem is resolved.

The second common error occurs due to an incomplete job set up.
For example, the submit description file given above specifies
three available executables.
If one is missing, Condor report back that an
executable is missing when it happens to match the
job with a resource that requires the missing binary.

%%%%%%%%%%%% 
\subsubsection{Standard Universe Example for Execution on Differing Architectures} 
%%%%%%%%%%%%

Jobs submitted to the standard universe may produce checkpoints.
A checkpoint can then be used to start up and continue execution
of a partially completed job.
For a partially completed job, the checkpoint and the job are specific
to a platform.
If migrated to a different machine, correct execution requires that
the platform must remain the same.

In previous versions of Condor, the author of the heterogeneous
submission file would need to write extra policy expressions in the
\AdAttr{requirements} expression to force Condor to choose the
same type of platform when continuing a checkpointed job.
However, since it is needed in the common case, this
additional policy is now automatically added
to the \AdAttr{requirements} expression.
The additional expression is added
provided the user does not use
\AdAttr{CkptArch} in the \AdAttr{requirements} expression.
Condor will remain backward compatible for those users who have explicitly
specified \AdAttr{CkptRequirements}--implying use of \AdAttr{CkptArch},
in their \AdAttr{requirements} expression.

The expression added when the attribute \AdAttr{CkptArch} is not specified 
will default to

\footnotesize
\begin{verbatim}
  # Added by Condor
  CkptRequirements = ((CkptArch == Arch) || (CkptArch =?= UNDEFINED)) && \
                      ((CkptOpSys == OpSys) || (CkptOpSys =?= UNDEFINED))

  Requirements = (<user specified policy>) && $(CkptRequirements)
\end{verbatim}
\normalsize

The behavior of the \AdAttr{CkptRequirements} expressions and its addition to
\AdAttr{requirements} is as follows.
The \AdAttr{CkptRequirements} expression guarantees correct operation
in the two possible cases for a job.
In the first case, the job has not produced a checkpoint.
The ClassAd attributes \Attr{CkptArch} and \Attr{CkptOpSys}
will be undefined, and therefore the meta operator (\verb@=?=@)
evaluates to true.
In the second case, the job has produced a checkpoint.
The Machine ClassAd is restricted to require further execution
only on a machine of the same platform.
The attributes \Attr{CkptArch} and \Attr{CkptOpSys}
will be defined, ensuring that the platform chosen for further
execution will be the same as the one used just before the
checkpoint.

Note that this restriction of platforms also applies to platforms where
the executables are binary compatible.

The complete submit description file for this example:

\begin{verbatim}
  ####################
  #
  # Example of heterogeneous submission
  #
  ####################

  universe     = standard
  Executable   = povray.$$(OpSys).$$(Arch)
  Log          = povray.log
  Output       = povray.out.$(Process)
  Error        = povray.err.$(Process)

  # Condor automatically adds the correct expressions to insure that the
  # checkpointed jobs will restart on the correct platform types.
  Requirements = ( (Arch == "INTEL" && OpSys == "LINUX") || \
                 (Arch == "INTEL" && OpSys =="SOLARIS26") || \
                 (Arch == "SGI" && OpSys == "IRIX65") )

  Arguments    = +W1024 +H768 +Iimage1.pov
  Queue 

  Arguments    = +W1024 +H768 +Iimage2.pov
  Queue 

  Arguments    = +W1024 +H768 +Iimage3.pov
  Queue 
\end{verbatim}

%%%%%%%%%%%%%%%%%%%%%%%%%%%%%%%%%%%%%%%%%%
\section{Managing a Job}
This section provides a brief summary of what can be done once jobs
are submitted. The basic mechanisms for monitoring a job are
introduced, but the commands are discussed briefly.
You are encouraged to
look at the man pages of the commands referred to (located in
Chapter~\ref{command-reference} beginning on
page~\pageref{command-reference}) for more information. 

When jobs are submitted, Condor will attempt to find resources
to run the jobs. 
A list of all those with jobs submitted
may be obtained through \Condor{status}
\index{Condor commands!condor\_status}
with the 
\Arg{-submitters} option. 
An example of this would yield output similar to:
\begin{verbatim}
%  condor_status -submitters

Name                 Machine      Running IdleJobs HeldJobs

ballard@cs.wisc.edu  bluebird.c         0       11        0
nice-user.condor@cs. cardinal.c         6      504        0
wright@cs.wisc.edu   finch.cs.w         1        1        0
jbasney@cs.wisc.edu  perdita.cs         0        0        5

                           RunningJobs           IdleJobs           HeldJobs

 ballard@cs.wisc.edu                 0                 11                  0
 jbasney@cs.wisc.edu                 0                  0                  5
nice-user.condor@cs.                 6                504                  0
  wright@cs.wisc.edu                 1                  1                  0

               Total                 7                516                  5
\end{verbatim}

\subsection{Checking on the progress of jobs}
At any time, you can check on the status of your jobs with the \Condor{q}
command.
\index{Condor commands!condor\_q}
This command displays the status of all queued jobs.
An example of the output from \Condor{q} is
\begin{verbatim}
%  condor_q

-- Submitter: froth.cs.wisc.edu : <128.105.73.44:33847> : froth.cs.wisc.edu
 ID      OWNER            SUBMITTED    CPU_USAGE ST PRI SIZE CMD               
 125.0   jbasney         4/10 15:35   0+00:00:00 I  -10 1.2  hello.remote      
 127.0   raman           4/11 15:35   0+00:00:00 R  0   1.4  hello             
 128.0   raman           4/11 15:35   0+00:02:33 I  0   1.4  hello             

3 jobs; 2 idle, 1 running, 0 held

\end{verbatim} 
This output contains many columns of information about the
queued jobs.
\index{status!of queued jobs}
The \verb@ST@ column (for status) shows the status of
current jobs in the queue. An \verb@R@ in the status column
means the the job is currently running.
An \verb@I@ stands for idle. The job is not running right
now, because it is waiting for a machine to become available. 
The status
\verb@H@ is the hold state. In the hold state,
the job will not be scheduled to
run until it is released (see condor\_hold and condor\_release man pages).
Older versions of Condor used a
\verb@U@ in the status column to stand for unexpanded.
In this state,
a job has never 
checkpointed and when it starts running, it will start running from the
beginning.
Newer versions of Condor do not use the \verb@U@ state.

The \verb@CPU_USAGE@ time reported for a job is the time that has been
committed to the job.  It is not updated for a job until
the job checkpoints. At that time, the job has made guaranteed forward 
progress.  Depending upon how the site administrator configured the pool,
several hours may pass between checkpoints, so do not worry if you do
not observe the \verb@CPU_USAGE@ entry changing by the hour.
Also note that this is actual CPU
time as reported by the operating system; it is not time as
measured by a wall clock.

Another useful method of tracking the progress of jobs is through the
user log.  If you have specified a \AdAttr{log} command in 
your submit file, the progress of the job may be followed by viewing the
log file.  Various events such as execution commencement, checkpoint, eviction 
and termination are logged in the file.
Also logged is the time at which the event occurred.

% Karen's note:  degraded performance where?
When your job begins to run, Condor starts up a \Condor{shadow} process
\index{condor\_shadow}
\index{remote system call!condor\_shadow}
on the submit machine.  The shadow process is the mechanism by which the
remotely executing jobs can access the environment from which it was
submitted, such as input and output files.  

It is normal for a machine which has submitted hundreds of jobs to have 
hundreds of shadows running on the machine.  Since the text segments of 
all these processes is the same, the load on the submit machine is usually 
not significant.  If, however, you notice degraded performance, you can limit 
the number of jobs that can run simultaneously through the 
\Macro{MAX\_JOBS\_RUNNING} configuration parameter.  Please talk to your 
system administrator for the necessary configuration change.

You can also find all the machines that are running your job through the
\Condor{status} command.
\index{Condor commands!condor\_status}
For example, to find all the machines that are
running jobs submitted by ``breach@cs.wisc.edu,'' type:
\begin{verbatim}
%  condor_status -constraint 'RemoteUser == "breach@cs.wisc.edu"'

Name       Arch     OpSys        State      Activity   LoadAv Mem  ActvtyTime

alfred.cs. INTEL    SOLARIS251   Claimed    Busy       0.980  64    0+07:10:02
biron.cs.w INTEL    SOLARIS251   Claimed    Busy       1.000  128   0+01:10:00
cambridge. INTEL    SOLARIS251   Claimed    Busy       0.988  64    0+00:15:00
falcons.cs INTEL    SOLARIS251   Claimed    Busy       0.996  32    0+02:05:03
happy.cs.w INTEL    SOLARIS251   Claimed    Busy       0.988  128   0+03:05:00
istat03.st INTEL    SOLARIS251   Claimed    Busy       0.883  64    0+06:45:01
istat04.st INTEL    SOLARIS251   Claimed    Busy       0.988  64    0+00:10:00
istat09.st INTEL    SOLARIS251   Claimed    Busy       0.301  64    0+03:45:00
...
\end{verbatim}
To find all the machines that are running any job at all, type:
\begin{verbatim}
%  condor_status -run

Name       Arch     OpSys        LoadAv RemoteUser           ClientMachine  

adriana.cs INTEL    SOLARIS251   0.980  hepcon@cs.wisc.edu   chevre.cs.wisc.
alfred.cs. INTEL    SOLARIS251   0.980  breach@cs.wisc.edu   neufchatel.cs.w
amul.cs.wi SUN4u    SOLARIS251   1.000  nice-user.condor@cs. chevre.cs.wisc.
anfrom.cs. SUN4x    SOLARIS251   1.023  ashoks@jules.ncsa.ui jules.ncsa.uiuc
anthrax.cs INTEL    SOLARIS251   0.285  hepcon@cs.wisc.edu   chevre.cs.wisc.
astro.cs.w INTEL    SOLARIS251   1.000  nice-user.condor@cs. chevre.cs.wisc.
aura.cs.wi SUN4u    SOLARIS251   0.996  nice-user.condor@cs. chevre.cs.wisc.
balder.cs. INTEL    SOLARIS251   1.000  nice-user.condor@cs. chevre.cs.wisc.
bamba.cs.w INTEL    SOLARIS251   1.574  dmarino@cs.wisc.edu  riola.cs.wisc.e
bardolph.c INTEL    SOLARIS251   1.000  nice-user.condor@cs. chevre.cs.wisc.
...
\end{verbatim}

\subsection{Removing a job from the queue}
A job can be removed from the queue at any time by using the \Condor{rm}
\index{Condor commands!condor\_rm}
command.  If the job that is being removed is currently running, the job
is killed without a checkpoint, and its queue entry is removed.  
The following example shows the queue of jobs before and after
a job is removed.
\begin{verbatim}
%  condor_q

-- Submitter: froth.cs.wisc.edu : <128.105.73.44:33847> : froth.cs.wisc.edu
 ID      OWNER            SUBMITTED    CPU_USAGE ST PRI SIZE CMD               
 125.0   jbasney         4/10 15:35   0+00:00:00 I  -10 1.2  hello.remote      
 132.0   raman           4/11 16:57   0+00:00:00 R  0   1.4  hello             

2 jobs; 1 idle, 1 running, 0 held

%  condor_rm 132.0
Job 132.0 removed.

%  condor_q

-- Submitter: froth.cs.wisc.edu : <128.105.73.44:33847> : froth.cs.wisc.edu
 ID      OWNER            SUBMITTED    CPU_USAGE ST PRI SIZE CMD               
 125.0   jbasney         4/10 15:35   0+00:00:00 I  -10 1.2  hello.remote      

1 jobs; 1 idle, 0 running, 0 held
\end{verbatim}

%%%%%%%%%%%%%%%%%%%%%%%%%%%%%%%%%%%%%%%%%%%%%%%%%%%%%%%%%%%%%%%%%%%%%%
\subsection{\label{sec:job-prio}Changing the priority of jobs}
%%%%%%%%%%%%%%%%%%%%%%%%%%%%%%%%%%%%%%%%%%%%%%%%%%%%%%%%%%%%%%%%%%%%%%

\index{job!priority}
\index{priority!of a job}
In addition to the priorities assigned to each user, Condor also provides
each user with the capability of assigning priorities to each submitted job.
These job priorities are local to each queue and range from -20 to +20, with
higher values meaning better priority.

The default priority of a job is 0, but can be changed using the \Condor{prio}
command.
\index{Condor commands!condor\_prio}
For example, to change the priority of a job to -15,
\begin{verbatim}
%  condor_q raman

-- Submitter: froth.cs.wisc.edu : <128.105.73.44:33847> : froth.cs.wisc.edu
 ID      OWNER            SUBMITTED    CPU_USAGE ST PRI SIZE CMD               
 126.0   raman           4/11 15:06   0+00:00:00 I  0   0.3  hello             

1 jobs; 1 idle, 0 running, 0 held

%  condor_prio -p -15 126.0

%  condor_q raman

-- Submitter: froth.cs.wisc.edu : <128.105.73.44:33847> : froth.cs.wisc.edu
 ID      OWNER            SUBMITTED    CPU_USAGE ST PRI SIZE CMD               
 126.0   raman           4/11 15:06   0+00:00:00 I  -15 0.3  hello             

1 jobs; 1 idle, 0 running, 0 held
\end{verbatim}

It is important to note that these \emph{job} priorities are completely 
different from the \emph{user} priorities assigned by Condor.  Job priorities
do not impact user priorities.  They are only a mechanism for the user to
identify the relative importance of jobs among all the jobs submitted by the
user to that specific queue.

\subsection{Why does the job not run?}
\index{job!analysis}
\index{job!not running}
Users sometimes find that their jobs do not run.  There are several reasons why
a specific job does not run.  These reasons include failed job or machine
constraints, bias due to preferences, insufficient priority, and the preemption
throttle that is implemented by the \Condor{negotiator} to prevent
thrashing.  Many of these reasons can be diagnosed by using the \Arg{-analyze}
option of \Condor{q}.
\index{Condor commands!condor\_q}
For example, the following job submitted by user
jbasney was found to have not run for several days.
\begin{verbatim}
% condor_q

-- Submitter: froth.cs.wisc.edu : <128.105.73.44:33847> : froth.cs.wisc.edu
 ID      OWNER            SUBMITTED    CPU_USAGE ST PRI SIZE CMD               
 125.0   jbasney         4/10 15:35   0+00:00:00 I  -10 1.2  hello.remote      

1 jobs; 1 idle, 0 running, 0 held
\end{verbatim}

Running \Condor{q}'s analyzer provided the following information:

\begin{verbatim}
%  condor_q 125.0 -analyze

-- Submitter: froth.cs.wisc.edu : <128.105.73.44:33847> : froth.cs.wisc.edu
---
125.000:  Run analysis summary.  Of 323 resource offers,
          323 do not satisfy the request's constraints
            0 resource offer constraints are not satisfied by this request
            0 are serving equal or higher priority customers
            0 are serving more preferred customers
            0 cannot preempt because preemption has been held
            0 are available to service your request

WARNING:  Be advised:
   No resources matched request's constraints
   Check the Requirements expression below:

Requirements = Arch == "INTEL" && OpSys == "IRIX6" && 
  Disk >= ExecutableSize && VirtualMemory >= ImageSize
\end{verbatim}

%%%%%%%%%%%%%%%%%%%
%condor_status -total lists the Arch/OS combinations in our pool:
%
%                     Machines Owner Claimed Unclaimed Matched Preempting
%
%           SGI/IRIX6       14     3       0        11       0          0
%          ALPHA/OSF1        8     6       1         1       0          0
%     SUN4u/SOLARIS26       84    38      46         0       0          0
%    SUN4u/SOLARIS251        8     0       1         7       0          0
%     SUN4x/SOLARIS26      104    47      56         1       0          0
%    SUN4x/SOLARIS251        1     0       1         0       0          0
%     INTEL/SOLARIS26      214    63     144         7       0          0
%       INTEL/WINNT40        6     0       0         6       0          0
%
%               Total      439   157     249        33       0          0
%
%So, one example of a platform that does not exist would be:
%
% requirements = Arch == "INTEL" && OpSys == "IRIX6"
%
%%%%%%%%%%%%%%%%%%%

For this job,
the \Attr{Requirements}
\index{ClassAd attribute!requirements}
expression specifies a platform that does not exist.
Therefore, the expression always evaluates to false.

While the analyzer can diagnose most common problems, there are some situations
that it cannot reliably detect due to the instantaneous and local nature of the
information it uses to detect the problem.  Thus, it may be that the analyzer
reports that resources are available to service the request, but the job still 
does not run.  In most of these situations, the delay is transient, and the
job will run during the next negotiation cycle.

If the problem persists and the analyzer is unable to detect the situation, it
may be that the job begins to run but immediately terminates due to some 
problem.  Viewing the job's error and log files
(specified in the submit command file) and Condor's \Macro{SHADOW\_LOG} file
may assist in tracking down the problem.  If the cause is still unclear, please
contact your system administrator.

\subsection{\label{sec:job-completion}Job Completion}
\index{job!completion}

When your Condor job completes(either through normal means or abnormal
termination by signal), Condor will remove it from the job queue (i.e.,
it will no longer appear in the output of \Condor{q}) and insert it into
the job history file.  You can examine the job history file with the
\Condor{history} command. If you specified a log file in your submit
description file, then the job exit status will be recorded there as well.

By default, Condor will send you an email message
when your job completes.  You can modify this behavior with the
\Condor{submit} ``notification'' command.
The message will include the exit status of your job (i.e., the
argument your job passed to the exit system call when it completed) or
notification that your job was killed by a signal.  It will also
include the following statistics (as appropriate) about your job:

\begin{description}

\item[Submitted at:] when the job was submitted with \Condor{submit}

\item[Completed at:] when the job completed

\item[Real Time:] elapsed time between when the job was submitted and
when it completed (days hours:minutes:seconds)

\item[Run Time:] total time the job was running (i.e., real time minus
queueing time)

\item[Committed Time:] total run time that contributed to job
completion (i.e., run time minus the run time that was lost because
the job was evicted without performing a checkpoint)

\item[Remote User Time:] total amount of committed time the job spent
executing in user mode

\item[Remote System Time:] total amount of committed time the job spent
executing in system mode 

\item[Total Remote Time:] total committed CPU time for the job

\item[Local User Time:] total amount of time this job's
\Condor{shadow} (remote system call server) spent executing in user
mode

\item[Local System Time:] total amount of time this job's
\Condor{shadow} spent executing in system mode

\item[Total Local Time:] total CPU usage for this job's \Condor{shadow}

\item[Leveraging Factor:] the ratio of total remote time to total
system time (a factor below 1.0 indicates that the job ran
inefficiently, spending more CPU time performing remote system calls
than actually executing on the remote machine)

\item[Virtual Image Size:] memory size of the job, computed when the
job checkpoints

\item[Checkpoints written:] number of successful checkpoints performed
by the job

\item[Checkpoint restarts:] number of times the job successfully
restarted from a checkpoint

\item[Network:] total network usage by the job for checkpointing and
remote system calls

\item[Buffer Configuration:] configuration of remote system call I/O
buffers

\item[Total I/O:] total file I/O detected by the remote system call
library

\item[I/O by File:] I/O statistics per file produced by the remote
system call library

\item[Remote System Calls:] listing of all remote system calls
performed (both Condor-specific and Unix system calls) with a count of
the number of times each was performed

\end{description}

%%%%%%%%%%%%%%%%%%%%%%%%%%%%%%%%%%%%%%%%%%

%%%%%%%%%%%%%%%%%%%%%%%%%%%%%%%%%%%%%%%%
\section{\label{sec:Priorities}Priorities and Preemption}
%%%%%%%%%%%%%%%%%%%%%%%%%%%%%%%%%%%%%%%%

Condor has two independent priority controls: \Term{job}
priorities and \Term{user} priorities.  

\subsection{Job Priority}

\index{job!priority}
\index{priority!of a job}
Job priorities allow the assignment of a priority level to
each submitted Condor job in order to
control order of execution.
To set a job priority, use the \Condor{prio} command
\index{Condor commands!condor\_prio}
--- see the example in section~\ref{sec:job-prio}, or the
command reference page on page~\pageref{man-condor-prio}.
Job priorities do not impact user priorities in any fashion.
A job priority can be any integer, and higher values are ``better''.

%%%%%%%%%%%%%%%%%%%%%%%%%%%%%%%%%%%%%%%%%%%%%%%%%%%%%%%%%%%%%%%%%%%%%%
\subsection{\label{sec:user-priority-explained}User priority}
%%%%%%%%%%%%%%%%%%%%%%%%%%%%%%%%%%%%%%%%%%%%%%%%%%%%%%%%%%%%%%%%%%%%%%

\index{preemption!priority}
\index{user!priority}
\index{priority!of a user}
Machines are allocated to users based upon a user's priority.
A lower numerical value for user priority means higher priority,
so a user with priority 5 will get more resources than
a user with priority 50.
User priorities in Condor can be examined with the \Condor{userprio}
command (see page~\pageref{man-condor-userprio}).
\index{Condor commands!condor\_userprio}
Condor administrators can set and change individual user priorities
with the same utility.

Condor continuously calculates the share of available machines that each
user should be allocated.    This share is inversely related to the ratio
between user priorities.
For example, a user with a priority of 10 will get twice as many
machines as a user with a priority of 20.
The priority of each individual user changes according to
the number of resources the individual is using.
Each user starts out with the best possible priority: 0.5.
If the number of machines a user currently has is greater than 
the user priority,
the user priority will worsen by numerically increasing over time.
If the number of machines is less then the priority,
the priority will improve by numerically decreasing over time. 
The long-term result is fair-share access across all users.
The speed at which Condor adjusts the priorities is
controlled with the configuration macro \Macro{PRIORITY\_HALFLIFE},
an exponential half-life value.
The default is one day.
If a user that has user priority of 100 and is
utilizing 100 machines removes all his/her jobs,
one day later that user's
priority will be 50, and two days later the priority will be 25.

Condor enforces that each user gets his/her fair share of machines
according to user priority both when allocating machines which become
available and by priority preemption of currently allocated machines.
For instance, if a low priority user is utilizing all available machines
and suddenly a higher priority user submits jobs, Condor will
immediately checkpoint and vacate jobs belonging to the lower priority
user. This will free up machines that Condor will then give over to the
higher priority user. Condor will not starve the lower priority user; it
will preempt only enough jobs so that the higher priority user's fair
share can be realized (based upon the ratio between user priorities). To
prevent thrashing of the system due to priority preemption, the Condor 
site administrator can define a \Macro{PREEMPTION\_REQUIREMENTS} expression in Condor's configuration.
The default expression that ships with Condor is configured to only preempt 
lower priority jobs that have run
for at least one hour. So in the previous example, in the worse case it
could take up to a maximum of one hour until the higher priority user
receives his fair share of machines. 

User priorities are keyed on ``username@domain'', for example
``johndoe@cs.wisc.edu''. The domain name to use, if any, is configured by
the Condor site administrator.  Thus, user priority and therefore resource
allocation is not impacted by which machine the user submits from or
even if the user submits jobs from multiple machines.

\index{nice job}
\index{priority!nice job}
An extra feature is the ability to submit a job as
a ``nice'' job (see page~\pageref{man-condor-submit-nice}).
Nice jobs artificially boost the user priority 
by one million just for the nice job.
This effectively means that nice jobs will only run on
machines that no other Condor job (that is, non-niced job) wants.
In a similar fashion, a Condor administrator could set
the user priority of any specific Condor user very high.
If done, for example, with a guest account,
the guest could only use cycles not wanted by other users of the system.


%%%%%%%%%%%%%%%%%%%%%%%%%%%%%%%%%%%%%%%%%%%%%%%%%%%%%%%%%%%%%%%%%%%%%%
\subsection{\label{sec:Vacate-Explained}
Details About How Condor Jobs Vacate Machines}
%%%%%%%%%%%%%%%%%%%%%%%%%%%%%%%%%%%%%%%%%%%%%%%%%%%%%%%%%%%%%%%%%%%%%%

\index{vacate}
\index{preemption!vacate}
When Condor needs a job to vacate a machine for whatever reason, it
sends the job an asynchronous signal specified in the \AdAttr{KillSig}
attribute of the job's ClassAd.
The value of this attribute can be specified by
the user at submit time by placing the \Opt{kill\_sig} option in the
Condor submit description file.  

If a program wanted to do some special work when required
to vacate a machine, the program may set up a
signal handler to use a trappable signal as an indication
to clean up.
When submitting this job, this clean up signal is specified to be used with
\Opt{kill\_sig}.
Note that the clean up work needs to be quick.
If the job takes too long to go away, Condor
follows up with a SIGKILL signal which immediately terminates the
process.

\index{Condor commands!condor\_compile}
A job that is linked using \Condor{compile}
and is subsequently submitted into the standard universe, 
will checkpoint and exit upon receipt of a SIGTSTP signal.
Thus, SIGTSTP is
the default value for \AdAttr{KillSig} when submitting to the standard
universe.
The user's code may still checkpoint itself at any time
by calling one of the following functions exported by the Condor libraries:
\begin{description}
\item[\Procedure{ckpt()}] Performs a checkpoint and then returns.
\item[\Procedure{ckpt\_and\_exit()}] Checkpoints and exits; Condor will then
restart the process again later, potentially on a different machine.
\end{description}

For jobs submitted into the vanilla universe, the default value for
\AdAttr{KillSig} is SIGTERM,
the usual method to nicely terminate a Unix program.

%%%%%%%%%%%%%%%%%%%%%%%%%%%%%%%%%%%%%%%%%%%%%%%%%%%%%%%%%%%%%%%%%%%%%%
%%%%%%%%%%%%%%%%%%%%%%%%%%%%%%%%%%%%%%%%%%%%%%%%%%%%%%%%%%%%%%%%%%%%%%
\section{\label{sec:java-install}Java Support Installation}
%%%%%%%%%%%%%%%%%%%%%%%%%%%%%%%%%%%%%%%%%%%%%%%%%%%%%%%%%%%%%%%%%%%%%%

\index{installation!Java}
\index{Java}

Compiled Java programs may be executed (under HTCondor) on
any
execution site with a
\index{Java Virtual Machine}
\index{JVM}
Java Virtual Machine (JVM).
To do this,
HTCondor must be informed of some details of the
JVM installation.

Begin by installing a Java distribution according to the vendor's
instructions.
We have successfully used the Sun Java Developer's Kit,
but any distribution should suffice.
Your machine may have
been delivered with a JVM already installed -- installed code
is frequently found in \File{/usr/bin/java}.

HTCondor's configuration includes the location of the installed
JVM.
Edit the configuration file.
Modify the \Macro{JAVA} entry to point to the JVM binary,
typically \File{/usr/bin/java}.
Restart the \Condor{startd} daemon on that host.  For example,

\begin{verbatim}
% condor_restart -startd bluejay
\end{verbatim}

The \Condor{startd} daemon takes a few moments to exercise the Java
capabilities of the \Condor{starter}, query its properties,
and then advertise the machine
to the pool as Java-capable.
If the set up succeeded, then \Condor{status} will
tell you the host is now Java-capable by printing the Java
vendor and the version number:

\begin{verbatim}
% condor_status -java bluejay
\end{verbatim}

After a suitable amount of time, if this command does not give any output,
then the \Condor{starter}  is having difficulty executing the JVM.
The exact cause of the problem depends on the details of the
JVM, the local installation, and a variety of other factors.
We can offer only limited advice on these matters,
but here is an approach to solving the problem.

To reproduce the test that the \Condor{starter} is attempting,
try running the Java \Condor{starter} directly.  To find
where the \Condor{starter} is installed, run this command:

\begin{verbatim}
% condor_config_val STARTER
\end{verbatim}

This command prints out the path to the \Condor{starter},
perhaps something like this:

\begin{verbatim}
/usr/condor/sbin/condor_starter
\end{verbatim}

Use this path to execute the \Condor{starter} directly
with the \Arg{-classad} argument.
This tells the starter to run its tests and display its properties.

\begin{verbatim}
/usr/condor/sbin/condor_starter -classad
\end{verbatim}

This command will display a short list of cryptic properties, such as:

\begin{verbatim}
IsDaemonCore = True
HasFileTransfer = True
HasMPI = True
CondorVersion = "$CondorVersion: 7.1.0 Mar 26 2008 BuildID: 80210 $"
\end{verbatim}

If the Java configuration is correct, there will also
be a short list of Java properties, such as:

\begin{verbatim}
JavaVendor = "Sun Microsystems Inc."
JavaVersion = "1.2.2"
JavaMFlops = 9.279696
HasJava = True
\end{verbatim}

If the Java installation is incorrect, then any error
messages from the shell or Java will be printed
on the error stream instead.

The Sun JVM sets a value of 64 Mbytes for the Java Maxheap Argument,
which HTCondor uses.
This value is often too small for the application.
The administrator can change this value through configuration by setting
a different value for \Macro{JAVA\_EXTRA\_ARGUMENTS}.

\footnotesize
\begin{verbatim}
JAVA_EXTRA_ARGUMENTS = -Xmx1024m
\end{verbatim}
\normalsize
Note that if a specific job sets the value in the submit description
file, using the submit command \SubmitCmd{java\_vm\_args},
this job's value takes precedence over a configured value.



%%%%%%%%%%%%%%%%%%%%%%%%%%%%%%%%%%%%%%%%%%%%%%%%%%%%%%%%%%%%%%%%%%%%%%

%%%%%%%%%%%%%%%%%%%%%%%%%%%%%%%%%%%%%%%%%%%%%%%%%%%%%%%%%%%%%%%%%%%%%%

\newcommand{\func}[1]{\texttt{#1}}

Condor has a PVM submit Universe which allows the user to submit PVM jobs to
the Condor pool.  In this section, we will first
discuss the differences between running under normal PVM and running PVM under the Condor
environment.  Then we give some hints on how to write good PVM
programs to suit the Condor environment via an example program.  In the
end, we illustrate how to submit PVM jobs to Condor by examining a
sample Condor submit-description file which submits a PVM job.

Note that Condor-PVM is an optional Condor module.  To check and see if
it has been installed at your site, enter the following command:
\begin{verbatim}
        ls -l `condor_config_val PVMD`
\end{verbatim}
(notice the use of backticks in the above command).  If this shows the
file ``condor\_pvmd'' on your system, Condor-PVM is installed.  If not,
ask your site administrator to download Condor-PVM from
\Url{http://www.cs.wisc.edu/condor/condor-pvm} and install it.

\subsection{What does Condor-PVM do?}

Condor-PVM provides a framework to run parallel applications written to
PVM in Condor's opportunistic environment.  This means that you no
longer need a
set of dedicated machines to run PVM applications; Condor can be used to dynamically 
construct PVM virtual machines out of non-dedicated desktop machines on your network
which would have otherwise been idle.   In Condor-PVM, Condor acts as the
resource manager for the PVM daemon.  Whenever your PVM program asks
for nodes (machines), the request is re-mapped to Condor.  Condor then
finds a machine in the Condor pool via the usual mechanisms, and adds it
to the PVM virtual machine.  If a machine needs to leave the pool, your
PVM program is notified of that as well via the normal PVM mechanisms.

\subsection{The Master-Worker Paradigm}

There are several different parallel programming paradigms.  One of the
more common is the \Term{master-worker} (or \Term{pool of tasks})
arrangement.  In a master-worker program model, one node acts as the
controlling master for the parallel application and sends pieces work out to worker nodes.  The
worker node does some computation, and sends the result back to the
master node.  The master has a pool of work that needs to be
done, and simply assigns the next piece of work out to the next worker
that becomes available.  

Not all parallel programming paradigms lend themselves to an
opportunistic environment. In such an environment, any of the nodes
could be preempted and therefore disappear at any moment. The
master-worker model, on the other hand, is a model that can work well.
The idea is the master needs to keep track of which piece of work it
sends to each worker. If the master node is then informed that a worker
has disappeared, it puts the piece of work it assigned to that worker
back into the pool of tasks, and sends it out again to the next
available worker. If the master notices that the number of workers has
dropped below an acceptable level, it could request for more workers
(via \func{pvm\_addhosts()}). Or perhaps perhaps the master will request
a replacement node every single time it is notified that a worker has
gone away. The point is that in this paradigm, the number of workers is
not important (although more is better!) and changes in the size of
the virtual machine can be handled naturally.

Condor-PVM is designed to run PVM applications which follow the
master-worker paradigm.  Condor runs the master application on the
machine where the job was submitted and will not preempt it.  Workers
are pulled in from the Condor pool as they become available.

\subsection{Binary Compatibility}

Condor-PVM does not define a new API (application program interface);
programs can simply use the existing resource management PVM calls such
as \func{pvm\_addhosts()} and \func{pvm\_notify()}.  Because of this, some
master-worker PVM applications are ready to run under Condor-PVM with no
changes at all.  Regardless of using Condor-PVM or not, it is good
master-worker design to handle the case of a worker node disappearing,
and therefore many programmers have already constructed their master program
with all the necessary logic for fault-tolerance purposes.  

In fact, regular PVM and Condor-PVM are \underline{binary compatible}
with each other.  The same binary which runs under regular PVM will run
under Condor, and vice-versa.  There is no need to re-link for Condor-PVM.
This permits easy application development
(develop your PVM application interactively with the regular PVM console, XPVM,
etc) as well as binary sharing between Condor and some dedicated MPP systems.

\subsection{Runtime differences between Condor-PVM and regular PVM}

This release of the Condor-PVM is based on PVM 3.3.11.  The vast majority of the PVM
library functions under Condor maintain the same semantics as in
PVM 3.3.11, including messaging operations, group operations, and 
pvm\_catchout().

We summarize the changes and new features of PVM under running in the
Condor environment in the following list:

\begin{itemize}

\item Concept of machine class.  Under Condor-PVM, machines of
  different architectures attributes belong to different machine classes.  Machine
  classes are are numbered 0, 1, \Dots, etc.  A machine class can be
  specified by the user in the submit-description file when the job
  is submitted to Condor.

\item \func{pvm\_addhosts()}.  When the application
  needs to add a host machine, it should call \func{pvm\_addhosts()}
  with the first argument as a string that specifies the machine
  class.  For example, to specify class 0, a pointer to string ``0''
  should be used as the first argument.  Condor will find a machine
  that satisfies the requirements of class 0 and adds it to the PVM
  virtual machine.

  Furthermore, \func{pvm\_addhosts()} no longer blocks under Condor.  It
  will return immediately, before the hosts are actually added to the virtual
  machine.  After all, in a non-dedicated environment the amount of time it takes until
  a machine becomes available is not bound. The user should simply call 
  \func{pvm\_notify()} before calling
  \func{pvm\_addhosts()}, so that when a host is added later, the user
  will be notified via Condor in the usual PVM 
  fashion (via a PvmHostAdd notification message).
    
\item \func{pvm\_notify()}.  Under Condor, we added two additional 
  possible notification requests, \func{PvmHostSuspend} and
  \func{PvmHostResume}, to the function \func{pvm\_notify()}.  When a
  host is suspended (or resumed) by Condor, if the user has called
  \func{pvm\_notify()} with that host tid and with
  \func{PvmHostSuspend} (or \func{PvmHostResume}) as arguments, then
  the application will receive a notification for the corresponding
  event.  Note that after you receive one of these notifications, the notify
  is 'consumed' for that pid.  So if you set up a \func{PvmHostSuspend} 
  notification request for tid 4 and you get a \func{PvmHostSuspend}
  message for tid 4, you won't get a \func{PvmHostSuspend} message for
  that tid again unless you set up another notification request.

  The easiest way to handle this is the following:  When a worker
  node starts up, set up a notification for \func{PvmHostSuspend} on
  its tid.  When that node gets suspended, set up a \func{PvmHostResume}
  notification.  When it resumes, set up a \func{PvmHostSuspend}
  notification.


\item \func{pvm\_spawn()}.  If  the flag in \func{pvm\_spawn()} is 
  PvmTaskArch, then the string specifying the desired architecture
  class should be used.  Typically, if you are using only one class of
  machine in your virtual machine, specify ``0'' as the desired architecture.

  Furthermore, under Condor we currently only allow one
  PVM task to be spawned per node, since Condor's typical setup at most 
  sites will suspend or vacate
  a job if the load on its machine is higher than a specified
  threshold.

  A good fault-tolerant program will, of course, be able to deal with
  \func{pvm\_spawn()} failing.  This happens more often in opportunistic 
  environments like condor than in dedicated ones.

\end{itemize}

\subsection{A Sample PVM program for Condor-PVM}

Normal PVM applications assume dedicated machines.  However, when running a
PVM application under Condor, since Condor's environment is an
opportunistic environment, machines can be suspended and even removed
from the PVM virtual machine during the life-time of the PVM
application.  

Here, we include an extensively commented skeleton of a sample PVM
program \Prog{master\_sum.c}, which, we hope, will help you to
write PVM code that is better suited for a non-dedicated opportunistic
environment like Condor.

\CondorVerySmall
\begin{verbatim}
/* 
 * master_sum.c
 *
 * master program to perform parallel addition - takes a number n 
 * as input and returns the result of the sum 0..(n-1).  Addition 
 * is performed in parallel by k tasks, where k is also taken as 
 * input.  The numbers 0..(n-1) are stored in an array, and each 
 * worker adds a portion of the array, and returns the sum to the 
 * master.  The Master adds these sums and prints final sum.  
 *
 * To make the program fault-tolerant, the master has to monitor 
 * the tasks that exited without sending the result back.  The 
 * master creates some new tasks to do the work of those tasks 
 * that have exited. 
 */

#define NOTIFY_NUM 5  /* number of items to notify */

#define HOSTDELETE 12
#define HOSTSUSPEND 13
#define HOSTRESUME 14
#define TASKEXIT 15
#define HOSTADD 16
    
/* send the pertask and start number to the worker task i */
void send_data_to_worker(int i, int *tid, int *num, int pertask, 
            FILE *fp, int round)
{
     int status;
     int start_val;
     
     /* send the round number */
     pvm_initsend(PvmDataDefault); /* XDR format */
     pvm_pkint(&round, 1, 1);    /* number of numbers to add */
     status = pvm_send(tid[i], ROUND_TAG);

     pvm_initsend(PvmDataDefault); /* XDR format */
     pvm_pkint(&pertask, 1, 1);    /* number of numbers to add */
     status = pvm_send(tid[i], NUM_NUM_TAG);

     pvm_initsend(PvmDataDefault); /* XDR format */
     start_val = i * pertask; /* initial number for this task */
     pvm_pkint(&start_val, 1, 1);     /* the initial number */
     status = pvm_send(tid[i], START_NUM_TAG);   

     fprintf(fp, "Round %d: Send data %d to worker task %d, ``
           ``tid =%x. status %d \n", round, start_val, i, tid[i], status);
}

/* 
 * to see if more hosts are needed 
 * 1 = yes; 0 = no 
 */
int need_more_hosts(int i)
{
     int nhost, narch;
     char *hosts="0";  /* any host in arch class 0 */
     struct pvmhostinfo *hostp = (struct pvmhostinfo *) 
                     calloc (1, sizeof(struct pvmhostinfo));

     /* get the current configuration */
     pvm_config(&nhost, &narch, &hostp);
     
     if (nhost > i)
        return 0;
     else 
        return 1;
}

/* 
 * Add a new host until success, assuming that request for 
 * PvmAddHost notification has already been sent 
 */
void add_a_host(FILE *fp)
{
     int done = 0;
     int buf_id;
     int success = 0;
     int tid;
     int msg_len, msg_tag, msg_src;
     char *hosts="0";  /* any host in arch class 0 */
     int infos[1];

     while (done != 1) {
        /* 
        * add one host - no specific machine named 
        * add host will asynchronously, so we need
        * to receive the notification before go on.
        */
        pvm_addhosts(&hosts,1 , infos);
      
        /* receive hostadd notification from anyone */
        buf_id = pvm_recv(-1, HOSTADD);
      
        if (buf_id < 0) {
            fprintf(fp, "Error with buf_id = %d\n", buf_id);
            done = 0;
            continue;
        }
        done = 1;
     
        pvm_bufinfo(buf_id, &msg_len , &msg_tag, &msg_src);
        pvm_upkint(&tid, 1, 1);

        pvm_notify(PvmHostDelete, HOSTDELETE, 1, &tid);

        fprintf(fp, "Received HOSTADD: ");
        fprintf(fp, "Host %x added from %x\n", tid, msg_src);
        fflush(fp);
    }
}

/* 
 * Spawn a worker task until success.  
 * Return its tid, and the tid of its host. 
 */
void spawn_a_worker(int i, int* tid, int * host_tid, FILE *fp)
{
     int numt = 0;
     int status;

     while (numt == 0){
          /* spawn a worker on a host belonging to arch class 0 */
          numt = pvm_spawn ("worker_sum", NULL, PvmTaskArch, "0", 1, &tid[i]);

          fprintf(fp, "master spawned %d task tid[%d] = %x\n",numt,i,tid[i]);
          fflush(fp);
         
          /* if the spawn is successful */
          if (numt == 1) {
               /* notify when the task exits */
               status = pvm_notify(PvmTaskExit, TASKEXIT, 1, &tid[i]);
               
               fprintf(fp, "Notify status for exit = %d\n", status);
               
               if (pvm_pstat(tid[i]) != PvmOk) numt = 0;
          }
          
          if (numt != 1) {
               fprintf(fp, "!! Failed to spawn task[%d]\n", i);
               
               /* 
                * currently Condor-pvm allows only one task running on 
                * a host
                */
                while (need_more_hosts(i) == 1)
                    add_a_host(fp);
          }
     }
}


main()
{
    int n;                  /* will add <n> numbers n .. n-1 */
    int ntasks;             /* need <ntask> workers to do the addition. */
    int pertask;            /* numbers to add per task */
    int tid[MAX_TASKS];     /* tids of tasks */ 
    int deltid[MAX_TASKS];  /* tids monitored for deletion */
    int sum[MAX_TASKS];     /* hold the reported sum */
    int num[MAX_TASKS];     /* the initial numbers the workers should add */
    int host_tid[MAX_TASKS];/* the tids of the host that the *
                             * tasks <0..ntasks> are running on*/
    
    int i, numt, nhost, narch, status;
    int result;
    int mytid;    /* task id of master */
    int mypid;    /* process id of master */
    int buf_id;   /* id of recv buffer */
    int msg_leg, msg_tag, msg_src, msg_len;
    int int_val;  

    int infos[MAX_TASKS];
    char * hosts[MAX_TASKS];
    struct pvmhostinfo *hostp = (struct pvmhostinfo *) 
                    calloc (MAX_TASKS, sizeof(struct pvmhostinfo));

    FILE *fp;
    char outfile_name[100];

    char *codes[NOTIFY_NUM] = {"HostDelete", "HostSuspend", 
            "HostResume", "TaskExit", "HostAdd"};
    
    int count;   /* the number of times that while loops */
    int round_val;
    int correct = 0;
    int wrong = 0;

    mypid = getpid();

    sprintf(outfile_name, "out_sum.%d", mypid);
    fp = fopen(outfile_name, "w"); 

    /* redirect all children tasks' stdout to fp */
    pvm_catchout(stderr);  

    if (pvm_parent() == PvmNoParent){
        fprintf(fp, "I have no parent!\n");
        fflush(fp);
    }

    /* will add <n> numbers 0..(n-1) */
    fprintf(fp, "How many numbers? ");
    fflush(fp);
    scanf("%d", &n);
    fprintf(fp, "%d\n", n);
    fflush(fp);

    /* will spawn ntasks workers to perform addition */
    fprintf(fp, "How many tasks? ");
    fflush(fp);
    scanf("%d", &ntasks);
    fprintf(fp, "%d\n\n", ntasks);
    fflush(fp);

    /* will iterate count loops */
    fprintf(fp, "How many loops? ");
    fflush(fp);
    scanf("%d", &count);
    fprintf(fp, "%d\n", count);
    fflush(fp);

    /* set the hosts to be in arch class 0 */
    for (i = 0; i< ntasks; i++) hosts[i] = "0";

    /* numbers to be added by each worker */
    pertask = n/ntasks;

    /* get the master's TID */
    mytid = pvm_mytid();
    fprintf(fp, "mytid = %x; mypid = %d\n", mytid, mypid);

    /* get the current configuration */
    pvm_config(&nhost, &narch, &hostp);

    fprintf(fp, "current number of hosts = %d\n", nhost);
    fflush(fp);

    /* 
     * notify request for host addition, with tag HOSTADD, 
     * no tids to monitor.  
     *
     * -1 turns the notification request on;
     * 0 turns it off;
     * a positive integer n will generate at most n 
     * notifications.
     */     
    pvm_notify(PvmHostAdd, HOSTADD, -1, NULL);

    /* add more hosts - no specific machine named */
    i = ntasks - nhost;
    if (i > 0) {
        status = pvm_addhosts(hosts, i , infos);
      
        fprintf(fp, "master: addhost status = %d\n", status);
        fflush(fp);
    }
     
    /* if not enough hosts, loop and call pvm_addhosts */
    for (i = nhost; i < ntasks; i++) {
        /* receive notification from anyone, with HostAdd tag */
        buf_id = pvm_recv(-1, HOSTADD);

        if (buf_id < 0) {
           fprintf(fp, "Error with buf_id = %d\n", buf_id);
        } else {
           fprintf(fp, "Success with buf_id = %d\n", buf_id);
        }

        pvm_bufinfo(buf_id, &msg_len , &msg_tag, &msg_src);
        if (msg_tag==HOSTADD) {
            pvm_upkint(&int_val, 1, 1);

            fprintf(fp, "Received HOSTADD: ");
            fprintf(fp, "Host %x added from %x\n", int_val, msg_src);
           fflush(fp);
        } else {
           fprintf(fp, "Received unexpected message with tag: %d\n", msg_tag);
        }
    }

    /* get current configuration */
    pvm_config(&nhost, &narch, &hostp);

    /* notify all exceptional conditions about the hosts*/
    status = pvm_notify(PvmHostDelete, HOSTDELETE, ntasks, deltid);
    fprintf(fp, "Notify status for delete = %d\n", status);
     
    status = pvm_notify(PvmHostSuspend, HOSTSUSPEND, ntasks, deltid);
    fprintf(fp, "Notify status for suspend = %d\n", status);
     
    status = pvm_notify(PvmHostResume, HOSTRESUME, ntasks, deltid);
    fprintf(fp, "Notify status for resume = %d\n", status);

    /* spawn <ntasks> */
    for (i = 0; i < ntasks ; i++) {
        /* spawn the i-th task, with notifications. */
        spawn_a_worker(i, tid, host_tid, fp);
    }

    /* add the result <count> times */
    while (count > 0) {
        /* 
         * if array length was not perfectly divisible by ntasks, 
         *    some numbers are remaining. Add these yourself 
         */
        result = 0;
        for ( i = ntasks * pertask ; i < n ; i++)
           result += i;
     
        /* initialize the sum array with -1 */
        for (i = 0; i< ntasks; i++) 
            sum[i] = -1;
 
        /* send array partitions to each task */
        for (i = 0; i < ntasks ; i++) {
           send_data_to_worker(i, tid, num, pertask, fp, count);
        }

        /* 
        * Wait for results.  If a task exited without 
        * sending back the result, start another task to do
        * its job. 
        */
        for (i = 0; i< ntasks; ) {   
            buf_id = pvm_recv(-1, -1);
            pvm_bufinfo(buf_id, &msg_len , &msg_tag, &msg_src);
            fprintf(fp, "Receive: task %x returns mesg tag %d, ``
                ``buf_id = %d\n", msg_src, msg_tag, buf_id);
            fflush(fp);
           
            /* is a result returned by a worker */
            if(msg_tag == RESULT_TAG)  {
                int j;
                
                pvm_upkint(&round_val, 1, 1);
                fprintf(fp, "  round_val = %d\n", round_val);
                fflush(fp);
             
                if (round_val != count) continue;

                pvm_upkint(&int_val, 1, 1);
                for (j=0; (j<ntasks) && (tid[j] != msg_src); j++)
                    ;
                fprintf(fp, "  Data from task %d, tid = %x : %d\n", 
                    j, msg_src, int_val);
                fflush(fp);
                
                if (sum[j] == -1) {
                    sum[j] = int_val; /* store the sum */
                    i++;
                }
           } else if (msg_tag == TASKEXIT) {
                /* A task has exited. */
                /* Find out which task has exited. */ 
                int which_tid, j;          
                pvm_upkint(&which_tid, 1, 1);
                for (j=0; (j<ntasks) && (tid[j] != which_tid); j++)
                 ;
                fprintf(fp, "  from tid %x : task %d, tid =  %x, ``
                    ``exited.\n", 
                    msg_src, j, which_tid);
                fflush(fp);
                /* 
                 * If a task exited before sending back the message,
                 * create another task to do the same job.
                 */
                if (j < ntasks && sum[j] == -1) {
                     /* spawn the j-th task */
                     spawn_a_worker(j, tid, host_tid, fp);
                     
                     /* send unfinished work to the new task */
                     send_data_to_worker(j, tid, num, pertask, fp, count);
                }
            } else if (msg_tag == HOSTDELETE) {
                /* 
                * If a host has been deleted, check to see if 
                * the tasks running on it has been finished.  
                * If not, should create  new worker tasks to do 
                * the work on some other  hosts.
                */
                int which_tid, j;
                    
                /* get which host has been suspended/deleted */
                pvm_upkint(&which_tid, 1, 1);
                    
                fprintf(fp, "  from tid %x : %x %s\n", msg_src, which_tid, 
                    codes[msg_tag - HOSTDELETE]);
                fflush(fp);
                    
                /* 
                 * If the task on that host has not finished its
                 * work, then create new task to do the work.
                 */
                for (j = 0; j < ntasks; j++) {
                     if (host_tid[j] == which_tid && sum[j] == -1) {
                          fprintf(fp, "host_tid[%d] = %x, ``
                            ``need new task\n",
                              j, host_tid[j]);
                          fflush(fp);
                          
                          /* spawn the i-th task, with notifications. */
                          spawn_a_worker(j, tid, host_tid, fp);
                          
                          /* send the unfinished work to the new task */
                          send_data_to_worker(j,tid,num,pertask,fp,count);
                     }
                }
            } else {
                /* print out some other notifications or messages */
                int which_tid;
                pvm_upkint(&which_tid, 1, 1);
        
                fprintf(fp, "  from tid %x : %x %s\n", msg_src,
                        which_tid,   codes[msg_tag - HOSTDELETE]);
                fflush(fp);
            }
        }        
      
        /* add up the sum */
        for (i=0; i<ntasks; i++)
           result += sum[i];
          
        fprintf(fp, "Sum from  0 to %d is %d\n", n-1 , result);
        fflush(fp);
          
        /* check correctness */
        if (result == (n-1)*n/2) {
           correct++;
           fprintf(fp, "*** Result Correct! ***\n");
        } else {
           wrong++;
           fprintf(fp, "*** Result WRONG! ***\n");
        }

        fflush(fp);
        count--;
    }
     
     fprintf(fp, "correct = %d; wrong = %d\n", correct, wrong);
     fflush(fp);

     pvm_exit();
     exit(0);
}

\end{verbatim}
\normalsize

There is also a larger and far more robust application available
at \Url{http://www.cs.wisc.edu/condor/condor-pvm/}.

\subsection{Sample PVM submit file}
\label{submit}

Like submitting jobs in any other universe,
to submit a PVM job, the user needs to specify the requirements and
options in the submit-desciption file and run \Condor{submit}.  Figure~\ref{pvm_submit} on
page~\pageref{pvm_submit} is an example of
a submit-description file for a PVM job.  This job has a master PVM program called
\func{master\_pvm}.

\begin{figure}[hbt]
\CondorSmall
\begin{verbatim}
###########################################################
# sample_submit
# Sample submit file for PVM jobs. 
###########################################################

# The job is a PVM universe job.
universe = PVM  

# The executable of the master PVM program is ``master_pvm''.
executable = master_pvm

In = "in_sum"
Out = "stdout_sum"
Err = "err_sum"

###################  Architecture class 0  ##################

Requirements = (Arch == "INTEL") && (OpSys == "SOLARIS251") 

# We want at least 2 machines in class 0 before starting the 
# program.  We can use up to 4 machines.
machine_count = 2..4  
queue

###################  Architecture class 1  ##################

Requirements = (Arch == "SUN4x") && (OpSys == "SOLARIS251") 

# We need at least 1 machine in class 1 before starting the 
# executable.  We can use up to 3 to start with.
machine_count = 1..3
queue

###############################################################
# note: the program will not be started until the least 
#       requirements in all classes are satisfied.
###############################################################
\end{verbatim}
\normalsize

\label{pvm_submit}
\caption{A sample submit file for PVM jobs.}
\end{figure}

In this sample submit file, the command \func{universe = PVM}
specifies that the jobs should be submitted into PVM universe.

The command \func{executable = master\_pvm} tells Condor that the PVM
master program is \func{master\_sum}.  This program will be started on
the submitting machine.  The workers should be spawned by this master
program during execution.

This submit file also tells Condor that the PVM virtual machine is
consisted of two different classes of machine architectures.  Class
0 contains machines with INTEL architecture running SOLARIS251; class
1 contains machines with SUN4x (SPARC) architecture running SOLARIS251.

By using \func{machine\_count = <min>..<max>}, the submit file tells
Condor that before the PVM program, there should be at least \verb@<min>@
number of machines of the current class.  It also asks Condor to give
it as many as \verb@<max>@ machines.  During the execution of the program,
the application can get more machines of each of the class by calling
\func{pvm\_addhosts()} with a string specifying the desired architecture
class.  (See the sample program in this section for details.)

The \func{queue} command should be inserted after the specifications of
each class.







%%%%%%%%%%%%%%%%%%%%%%%%%%%%%%%%%%%%%%%%%%%%%%%%%%%%%%%%%%%%%%%%%%%%%%

%%%%%%%%%%%%%%%%%%%%%%%%%%%%%%%%%%%%%%%%%%%%%%%%%%%%%%%%%%%%%%%%%%%%%%
%%%%%%%%%%%%%%%%%%%%%%%%%%%%%%%%%%%%%%%%%%%%%%%%%%%%%%%%%%%%%%%%%%%%%%
\section{\label{sec:MPI}Running MPICH jobs in Condor}
%%%%%%%%%%%%%%%%%%%%%%%%%%%%%%%%%%%%%%%%%%%%%%%%%%%%%%%%%%%%%%%%%%%%%%
In addition to PVM, Condor also supports the execution of parallel jobs
that utilize MPI.
Our current implementation supports the following features:
\begin{itemize}
\item There are no alterations to the MPICH implementation.  You can directly
use the version from Argonne National Labs.

\item You do not have to re-compile or re-link your MPICH job.  Just
compile it using the regular \Prog{mpicc}.  Note that you have to be using
the ch\_p4 subsystem provided by Argonne.

\item The communication speed of the MPI nodes is not affected by 
running it under Condor.
\end{itemize}
However, there are some limitations to our current implementation.

\subsection{\label{sec:MPI-caveats}Caveats}
\begin{description}
\item[MPICH] Your MPI job must be compiled with MPICH, Argonne National
Labs' implementation of MPI.  Specifically, you must use the ``ch\_p4'' 
device for MPICH.  For information on MPICH, see Argonne's web page
at \Url{http://www-unix.mcs.anl.gov/mpi/mpich/}

\item[Dedicated Resources] You must make sure that your MPICH jobs
will be running on machines that will not vacate the job before the job
terminates naturally.  (This is a limitation of MPICH and the MPI 
specification.) Unlike PVM (Section~\ref{sec:PVM}), the current MPICH
implementation does not support dynamic resource management.  That is, 
processes in the virtual machine may NOT join or leave the computation at 
any time.  If you start an MPI job with 4 nodes, for example, none of those 4 
nodes can be preempted by other Condor jobs or the machine's owner.

\item[Scheduling] We do not yet have a sophisticated scheduling 
algorithm in place for MPI jobs.  If you set things up properly, 
there shouldn't be much of a problem.  However, if there are several
users trying to run MPI jobs on the same machines, it may be the case 
that no jobs will run at all and Condor's scheduling will deadlock.  
Writing a good scheduler for this environment is high on the priority 
list for Condor version 6.5.

\item[``New'' shadow and starter] We have been developing new versions
of the \Condor{shadow} and the \Condor{starter}.  You have to use these
new versions to run MPI jobs.  For information on obtaining these 
binaries, see below.

\item[Shared File System] The machines where you want your MPI job
to run must have a shared file system.  There is no remote I/O for
our MPI support like there is for our Standard Universe jobs.

\item[Condor Version 6.1.11+] You must be running this version of 
the Condor distribution (or greater) in order to use this contrib 
module.
\end{description}


%%%%%%%%%%%%%%%%%%%%%%%%%%%%%%%%%%%%%%%%%%%%%%%%%%%%%%%%%%%%%%%%%%%
\subsection{\label{sec:MPI-binaries}Getting the Binaries}

There is now an MPI ``contrib'' module available with Condor.  It can
be found in the contrib section of the downloads.  When you un-tar the 
tarfile, there will be three files:

\begin{itemize}
\item\Condor{starter.v61}
\item\Condor{shadow.v61}
\item\Prog{rsh}
\end{itemize}

The last item is named \File{rsh}, but it is not the \Prog{rsh} utility you're 
familiar with --- it's a wrapper that is required for our implementation to 
function correctly. 
These three binaries should go in Condor's \File{sbin} directory, where
many other files like them reside.

%%%%%%%%%%%%%%%%%%%%%%%%%%%%%%%%%%%%%%%%%%%%%%%%%%%%%%%%%%%%%%%%%%%
\subsection{\label{sec:MPI-config}Configuring Condor }

Now that you've got the necessary binaries, you'll have to configure Condor
to use MPI.  Insert the following lines in the main \condor{config} file:
\begin{verbatim}
ALTERNATE_STARTER_2	= $(SBIN)/condor_starter.v61
STARTER_2_IS_DC		= TRUE
MPI_CONDOR_RSH_PATH	= $(SBIN)
SHADOW_MPI		= $(SBIN)/condor_shadow.v61
\end{verbatim}
Reconfigure your pool by typing
\begin{verbatim}
condor_reconfig `condor_status -m`
\end{verbatim}
The -m argument tells \Condor{status} to return just the names of all
the running \Condor{master} daemons in your pool.  Note that you have to do 
this from a machine with administrator privileges.

%%%%%%%%%%%%%%%%%%%%%%%%%%%%%%%%%%%%%%%%%%%%%%%%%%%%%%%%%%%%%%%%%%%
\subsection{\label{sec:MPI-machines}Managing Dedicated Machines}

There are several ways that you can set up a pool to run MPI jobs without 
interruption.  We will cover two methods that will work, although more
sophisticated solutions are possible. 
Familiarity with Startd policy configuration
(Section~\ref{sec:Configuring-Policy}) is necessary to understand the
following examples.

For the first example, let's assume that you have a cluster of machines which
do not have regular users on them.  Let's also assume that these machines are 
solely dedicated to the use of Condor.  
The simplest way to set up your policy is as follows:
\begin{verbatim}
START       = TRUE
CONTINUE    = TRUE
SUSPEND     = FALSE
PREEMPT     = FALSE
KILL        = FALSE
\end{verbatim}

With the above configuration, the machines will accept any Condor job, and the
jobs will never be suspended, preempted, or killed.  You will never have to 
worry about an MPI job (or any job, for that matter) being evicted from the 
machines.

For a more complex example, let us assume you have machines with sophisticated 
policies already in place, and you'd like the machines to manage MPI jobs 
differently.  The following macros (which should be specified
near other Startd policy support macros) allow you to accomplish the task 
easily.
\begin{verbatim}
MPI	  = 8
IsMPI = (JobUniverse == $(MPI))
\end{verbatim}
Now change your configuration from
\begin{verbatim}
START	= /* your interesting policy here */
\end{verbatim}
to 
\begin{verbatim}
FORMER_START	= /* your interesting policy here */
\end{verbatim}
Similarly, the \Macro{CONTINUE}, \Macro{SUSPEND}, \Macro{PREEMPT}, and 
\Macro{KILL} expressions should be changed to macros named 
\MacroNI{FORMER\_CONTINUE}, etc.  The following configuration will ensure that
MPI jobs are never suspended or evicted while implementing your former policy
for all other jobs.
\begin{verbatim}
START		= ( $(FORMER_START) )
CONTINUE	= ( $(FORMER_CONTINUE) )
SUSPEND		= ( $(FORMER_SUSPEND) && ((IsMPI) == FALSE ) )
PREEMPT		= ( $(FORMER_PREEMPT) && ((IsMPI) == FALSE ) )
KILL		= ( $(FORMER_KILL) && ((IsMPI) == FALSE ) )
\end{verbatim}
Thus, Condor will never attempt to vacate an MPI job from a machine once 
it starts running on that machine.  
Some machine owners may not like this setup, so you may need to customize 
your configuration to suit your needs.  
The most important point to remember when creating your Startd policy is 
that MPI jobs are immediately killed if one or more nodes of the job leave
the computation.

%%%%%%%%%%%%%%%%%%%%%%%%%%%%%%%%%%%%%%%%%%%%%%%%%%%%%%%%%%%%%%%%%%%
\subsection{\label{sec:MPI-submit}Submitting to Condor}

Here is a minimal submit file to submit an MPI job to Condor.  For more 
information on writing submit files, see Section~\ref{sec:sample-submit-files}.

\begin{verbatim}
universe = MPI
executable = your_mpi_program
machine_count = 4
queue 
\end{verbatim}

This tells Condor to start the executable named \File{your\_mpi\_program}
on four machines.  These four machines will be of the same architechture
and operating system as the submitting machine.  Note the 
\verb+universe = MPI+ line tells Condor that an MPICH job is being submitted.  

Now let's try a more sophisticated submit file:
\begin{verbatim}
###################################################################
## submitfile                                                    ##
###################################################################
universe = MPI
executable = simplempi
log = logfile
input = infile.$(NODE)
output = outfile.$(NODE)
error = errfile.$(NODE)
machine_count = 4
queue
\end{verbatim}

Notice the \MacroU{NODE} macro, which is expanded when the job starts so that
it becomes equivalent to the MPI ``id'' of the MPICH job.  The first 
process started becomes ``0'', the second is ``1'', etc.  For example, 
let's say I prepared four input files, named \File{infile.0} through 
\File{infile.3}:
\begin{verbatim}
infile.0: 
Hello number zero.

infile.1: 
Hello number one.
\end{verbatim}
etc.  I then created a simple MPI job, named \File{simplempi.c}
\begin{verbatim}
/******************************************************************
 * simplempi.c
 ******************************************************************/
#include <stdio.h>
#include "mpi.h"

int main(argc,argv)
    int argc;
    char *argv[];
{
    int myid;
    char line[128];

    MPI_Init(&argc,&argv);
    MPI_Comm_rank(MPI_COMM_WORLD,&myid);

    fprintf ( stdout, "Printing to stdout...%d\n", myid );
    fprintf ( stderr, "Printing to stderr...%d\n", myid );
    fgets ( line, 128, stdin );
    fprintf ( stdout, "From stdin: %s", line );

    MPI_Finalize();
    return 0;
}
\end{verbatim}
And to complete the demonstration, here's the \File{Makefile}:
\begin{verbatim}
###################################################################
## This is a very basic Makefile                                 ##
###################################################################

# Change this part to your mpicc, obviously....
CC          = /usr/local/bin/mpicc
CLINKER     = $(CC)

CFLAGS    = -g
EXECS     = simplempi

all: $(EXECS)

simplempi: simplempi.o
        $(CLINKER) -o simplempi simplempi.o -lm

.c.o:
        $(CC) $(CFLAGS) -c $*.c
\end{verbatim}

Once \File{simplempi} is built, use \Condor{submit} to submit your job.
This job should finish pretty quickly once it finds machines to run on,
and the results will be what you expect:  8 files will be created:  
\File{errfile.[0-3]} and \File{outfile.[0-3]}.  For example, \File{outfile.0}
will contain
\begin{verbatim}
Printing to stdout...0
From stdin: Hello number zero.
\end{verbatim}
and \File{errfile.0} will contain
\begin{verbatim}
Printing to stderr...0
\end{verbatim}

Of course, individual tasks may open other files; this example was 
constructed to demonstrate the \MacroUNI{NODE} feature and the setup of
the expected \File{stdin}, \File{stdout}, and \File{stderr} files in the MPI
universe.  

%%%%%%%%%%%%%%%%%%%%%%%%%%%%%%%%%%%%%%%%%%%%%%%%%%%%%%%%%%%%%%%%%%%%%%

%%%%%%%%%%%%%%%%%%%%%%%%%%%%%%%%%%%%%%%%%%%%%%%%%%%%%%%%%%%%%%%%%%%%%%
\section{Interjob Dependencies: DAGMan Meta-Scheduler}
\label{sec:DAGMan}

The Directed Acyclic Graph Manager (DAGMan) is a meta-scheduler for Condor
jobs.  DAGMan is responsible for submitting batch jobs in a predefined order
and processing the results. A configuration file is defined prior to execution
of DAGMan in which the jobs, their \textit{CondorConfigFile}, and job
dependencies are declared.

The importance of such a tool lies in the fact that the user is able to define
the execution order of a number of Condor Jobs. Just as Condor schedules
condor jobs, DAGMan schedules a system of jobs. In essence, it defines a
problem. Solving a problem may require multiple condor jobs that need data
from each other. This is best represented using a Directed Acyclic Graph
(DAG), which represents the flow of control from one node to another (i.e.,
from one condor job to another) through arrows.

From the point of view of the user, the scheduler is initialized with the
order of execution of jobs, and then started. DAGMan is responsible for all
scheduling, recovery and reporting activities of the submitted system of jobs.

The following sections explain the use of DAGMan in full detail.  However, if
the user only wants the bare essentials, please read
section~\ref{dagman:essentials} to get started more quickly.

\subsection{DAG Input File}

For Unix users, a useful analogy might be to think of the DAGMan input file as
a makefile, and DAGMan itself as the make executable.  However, DAGMan differs
from make.  Instead of looking at file modification timestamps, DAGMan reads
the Condor log file generated by each Condor job to find out which jobs are
unsubmitted, submitted, or complete.  DAGMan also makes a guarantee that a DAG
is recoverable, even if the machine running DAGMan goes down during execution.

\subsubsection{Description}
\label{dagman:dagdesc}

Job dependencies are defined prior to execution of the DAGMan program, using a
DAG input file.  An example input configuration file name is \File{diamond.dag}.
The input file is read completely, and the DAG data structure is constructed
in memory before the first job is submitted.  With the exception of the
\textit{CondorCommandFile} (see below), the input file is case insensitive.

Throughout the input file, comments can be placed.  Legal comments exist on a
single line which immediately starts with a `\texttt{\#}' character, followed
by any characters up to the newline `\texttt{$\backslash$n}'.

It is interesting to note that the DAGMan input file does not contain any
specifics about the individual jobs. Each condor job by itself is handled as
if DAGMan was not present (this includes compiling and linking of the
job). The executable and the input/output parameters for each job are
contained in the CondorCommandFile.  The DAG file merely describes the
relationship between the different condor jobs using the semantics just
described.

\begin{description}

\item[Signature]

The first line of a DAG input file is the signature, which precisely
identifies which DAG file format follows.  As of this writing, only one DAG
format exists, and thus only one signature is possible.

\begin{verbatim}
  ### DAGMan 6.1.0
\end{verbatim}

This line must appear as it is written here, character for character.
Anything different will be rejected by DAGMan.  Having a precise signature tag
will enable future versions of DAGMan to remain backward compatible with older
DAG input file formats.

\item[Job Section]

The Job Section of the input DAG file declares all the jobs that will appear
in the DAG.  Each job is described by a single line called a Job Entry.  The
following syntax is used:

\begin{verbatim}
	JOB <JobName> <CondorCommandFile>
\end{verbatim}

The \texttt{JOB} keyword (shown here in upper case only for clarity) declares
this line will map a \textit{JobName} to a Condor Command File.  The
\textit{JobName} is used by DAGMan to uniquely identify jobs throughout the
input file and to name them in output messages.  The
\textit{CondorCommandFile} is the input file used by \Condor{submit} to run
the individual condor job.  Because the Unix file system is case sensitive,
the case of the \textit{CondorCommandFile} is preserved.

The JobName can be any string that contains no white space.  The JobName is
not case sensitive, so ``JobA'' is equivalent to ``joba''.  An example
\textit{CondorCommandFile} name is \File{a.condor}.  Some important
restrictions are placed on the contents of the \textit{CondorCommandFile},
which will be discussed later.

The user can also have the option of declaring a job as being already
completed in the DAG input file. This may be useful in situations where the
user wishes to verify results, but does not need the entire job dependency
graph to be executed. This is done by adding the word "DONE" to the end of the
Job declaration line.

\begin{verbatim}
	JOB <JobName> <CondorCommandFile> DONE
\end{verbatim}

\item[Dependency Section]

The dependency section of the DAG input file follows the Job Section and
describes the dependencies between the jobs listed in the Job Section.  The
notion of a ``parent'' and ``child'' job is introduced here.  A parent job
produces output which is required by one or more child jobs.  None of the
children can run until the parent successfully terminates.  A child job is one
whose input is taken from one or more parent jobs.  The child job cannot run
until all of its parents have successfully terminated.

A single line in the input file can specify the dependencies from one or more
parents to one or more children.

\begin{verbatim}
	PARENT <ParentJobName>* CHILD <ChildJobName>*
\end{verbatim}

The \texttt{PARENT} keyword is followed by one or more
\textit{ParentJobName}s.  Those are followed by the \texttt{CHILD} keyword,
which is followed by one or more \textit{ChildJobName}s.  Each child job
depends on each and every parent job on this line.  So the line
``\texttt{PARENT p1 p2 CHILD c1 c2}'' would produce four dependencies.

\end{description}

\subsubsection{Example}

The following \File{diamond.dag} DAG input file shown below is illustrated in
Figure~\ref{fig:dagman-diamond}.

\begin{verbatim}
  ### DAGMan 6.1.0
  # Filename: diamond.dag
  #
  Job  A  A.condor 
  Job  B  B.condor 
  Job  C  C.condor	
  Job  D  D.condor
  PARENT A CHILD B C
  PARENT B C CHILD D
\end{verbatim}

\begin{figure}[hbt]
\centering
\includegraphics{user-man/dagman-diamond.eps}
\caption{\label{fig:dagman-diamond}Diamond DAG}
\end{figure}

With \File{diamond.dag}, job A must execute first, because all other jobs
directly or indirectly depend on it.  After job A successfully completes, both
job B and C are eligible to run.  In fact, they will be submitted at the same
time and hopefully Condor will find two remote hosts that can run them in
parallel.  Since job D depends on both B and C, it must wait for both to
complete successfully before it can be submitted.

\subsection{Execution}

\subsubsection{Preparing Jobs}
\label{dagman:prepjob}

Each individual job in a DAG is free to be a unique executable, with a unique
\textit{CondorCommandFile}.  The DAG can contain a mixture of standard and
vanilla jobs, or even other meta-scheduler jobs, like DAGMan.  On the other
hand, the jobs in the DAG could all use the same executable, or even the same
\textit{CondorCommandFile}.  Anything between both extremes is possible.
However, two limits are imposed.

First, each \textit{CondorCommandFile} must submit a cluster of size one.
There cannot be multiple \texttt{queue} lines.  The reasoning is long winded,
so a brief summary will be attempted.  If multi-job clusters were allowed,
DAGMan would have to parse the \textit{CondorCommandFile} to find out how many
jobs belong to that cluster.  Otherwise, DAGMan would not know for sure if a
cluster had terminated based on seeing the event from one job of that
cluster.  This restriction may be lifted in future DAGMan version, depending
on the design and implementation issues.

Second, all \textit{CondorCommandFile}s of a DAG must specify the same log.
In order for DAGMan to follow the order of events correctly, all events from
all jobs in the DAG must be sent to the same log file.  This restriction will
be loosened in later versions (see section~\ref{dagman:version}).

For this example, we will write a single \textit{CondorCommandFile} to be used
by all three jobs in the DAG.  Thus, each job will run the same executable.
This example is very artificial, because normally separate jobs would need
output for their child jobs to go to unique output and error files.
Otherwise, the jobs would be clobbering each other's output.  However, since
we are sending output and error to \File{/dev/null}, sharing the
\textit{CondorCommandFile} is OK.

\begin{verbatim}
  # Filename: diamond_job.condor
  #
  executable   = /path/diamond.exe
  output       = /dev/null
  error        = /dev/null
  log          = diamond_condor.log
  universe     = vanilla
  notification = NEVER
  queue
\end{verbatim}

Note that notification is set to \texttt{NEVER}.  This is recommended if you
prefer not to have Condor send you e-mail for every job in a large DAG.

\subsubsection{Writing the DAG File}
\label{dagman:writedag}

The DAG file names the jobs, associates jobs with their
\textit{CondorCommandFile}, and declares job dependencies.  For our artificial
DIAMOND example, all three jobs will use the same diamond\_job.condor file
written earlier.  However, a more typical DAG file would have unique
\textit{CondorCommandFile} for every job.

\begin{verbatim}
  ### DAGMan 6.1.0
  # Filename: diamond.dag
  # DIAMOND DAG File for DAGMan
  #
  Job  A  diamond_job.condor
  Job  B  diamond_job.condor
  Job  C  diamond_job.condor
  Job  D  diamond_job.condor
  PARENT A CHILD B C
  PARENT B C CHILD D
\end{verbatim}

This DAG file will be the input file for the \Condor{dagman} program.

\subsubsection{Submitting the DAG to Condor}
\label{dagman:submitdag}

In order to guarantee recoverability, the DAGMan program itself is run as a
Condor job.  However, DAGMan is not submitted as a standard universe or
vanilla universe job.  Instead, it is run as a meta-scheduler.  Standard and
vanilla universe jobs are usually submitted to the local schedd, which
schedules them for execution on some remote machine in the pool that is idle.
A meta-scheduler is also submitted to the local schedd, but runs on the local
schedd.  The meta-scheduler then submits jobs, according to its design, to the
same local schedd, just as if the user submitted them manually.  In fact, the
local schedd does not know the difference between DAGMan submitting a job, and
the user who originally submitted DAGMan, and could have submitted the DAG
jobs manually.

A DAG is submitted using the \Condor{submit\_dag} script.  For example, to
submit the \File{diamond.dag} DAG to Condor, simply type
``\Condor{submit\_dag} \File{diamond.dag}''.  This script will generate the
\File{diamond.dag.condor} \textit{CondorCommandFile} for the DAG, and submit
it to Condor.

If the user prefers to edit the \File{diamond.dag.condor} file before it is
submitted to Condor (for example, to change the pre-chosen filenames), she can
issue ``\Condor{submit\_dag} -n \File{diamond.dag}'', which specifies that
\File{diamond.dag.condor} is generated, but not submitted to Condor.  To run
the DAG, issue the command \Condor{submit} diamond.dag.condor.

\subsection{Removal}

After submitting a DAG, the user may change her mind and wish to remove the
entire DAG, plus any jobs submitted by that DAG which happen to currently be
running.  DAG removal is easily accomplished by issuing a \Condor{rm} on the
DAGMan job itself.  The schedd sends a special signal to the meta-scheduler,
telling it to remove any of its condor jobs (using \Condor{rm}) that are
currently running.

However, if the machine is scheduled to go down, and the schedd receives a
shutdown command from the master, the schedd will send a running DAGMan job a
similar shutdown, which instructs DAGMan to clean up memory and exit.
However, in this case, DAGMan does not remove its submitted jobs, but rather
expects them to persistently exist in the Condor queue after restart.

The important thing to remember is that DAGMan will not explicitly run
\Condor{rm} on its jobs except as a result of the user running \Condor{rm} on
the DAGMan job.

\subsection{Recovery}

The Condor system offers the benefit of recoverability, in that if any host
crashes, Condor jobs that were running can be recovered, either by continuing
from the last checkpoint, or rerunning from scratch.  In any event, Condor
guarantees that once a job is successfully submitted, the Condor system will
not loose it.

DAGMan makes the same guarantee about the DAG as a whole.  If the machine
running DAGMan goes down or crashes, upon restart DAGMan will be restarted,
and the state of the DAG jobs will be recovered from the log file
(\File{diamond.dag.condor.log} from our example before).  DAGMan knows to
recover a DAG (as apposed to starting a new one) because it will detect the
existance of a lock file that was not removed from the last run.  If DAGMan
successfully finishes a DAG, the lock file is removed, so that the next run
will not go into recover mode.  The lock file is specified via command-line
argument to DAGMan in the \textit{CondorCommandFile}.  Refer to
section~\ref{dagman:submitdag}.

\subsection{Essentials}
\label{dagman:essentials}

This section is written for those users looking for the boiled down,
absolutely essential steps to successfully submit a DAG.

\begin{description}

\item[Prepare Jobs] Each job in the DAG must have its own
\textit{CondorCommandFile}.  Each \textit{CondorCommandFile} can only submit
one job.  Multi-job clusters (multiple \texttt{queue} lines) are not
supported.  The \texttt{log=} for all \textit{CondorCommandFile}s must point
to the same Condor log file, otherwise, DAGMan will not see all the Condor log
entries for every job in the DAG.  Refer to section~\ref{dagman:prepjob} for
details on how to prepare jobs.

\item[Write DAG File] Write the DAG file, so that JOB entries refer to the
\textit{CondorCommandFile}s you wrote in the previous step.  Refer to
section~\ref{dagman:writedag} to learn about writing a DAG file.

\item[Submit the DAG] Finally, you submit the DAG written in the previous step
using the \Condor{submit\_dag} script.  Refer to
section~\ref{dagman:submitdag}.

\end{description}


\subsection{Version Summary}
\label{dagman:version}

This section addresses the features and limitations that exist in the current
version of DAGMan, and how they may change in future versions.

This first public release of DAGMan was written and tested in the Condor 6.1.0
environment.  It is shipped separate from the main Condor system as a
contribution program.  As such, it is not as rigorously tested as the core
components of Condor.  A reasonable effort has been made to test large DAGS
(on the order of 5000 jobs) on Solaris x86 and Sparc.  However, the DAGMan is
not arrogant enough to claim itself bug free.  Users are encouraged to send
e-mail to \Email{condor-admin@cs.wisc.edu}.

The following feature summary compares the current version with possible
versions of DAGMan still to come.

\begin{description}
\item[Feature] : Command Socket
\item[Version 6.1.0] : Unsupported
\item[Future Versions] : A general purpose command socket will be used to
direct Dagman while it's running.  Commands like CANCEL\_JOB X or DELETE\_ALL
would be supported, as well as notification messages like JOB\_SUBMIT or
JOB\_TERMINATE, etc.  Eventually, a Java Gui would graphically represent the
Dag's state, and offer buttons and dials for graphic Dag manipulation.
\end{description}

\begin{description}
\item[Feature]: DAG removal
\item[Version 6.1.0]: Supported via \Condor{rm} of the DAG.
\item[Future Versions]: Supported by a command socket such as DELETE\_ALL
\end{description}

\begin{description}
\item[Feature]: Condor Log File
\item[Version 6.1.0]: All jobs in a DAG must specify the same Condor log file.
That Condor log file must be unique.  No other DAGs or Condor jobs can point
to that log file.
\item[Future Versions]: All jobs in a Dag must go to one log file, but
log file can be shared with other Dags and Condor jobs.
\end{description}

\begin{description}
\item[Feature]: Job UNDO
\item[Version 6.1.0]: All jobs must exit normally, else DAG will be aborted
\item[Future Versions]: A job can be ``undone'', or there is some
notion of a job instance.  Hence, a job that exits abnormally or is
cancelled by the user can be rerun such that the new run's log entry
is unique from the old run's log entry (in terms of recovery)
\end{description}

\begin{description}
\item[Feature]: Pre/Post Process
\item[Version 6.1.0]: Unsupported
\item[Future Versions]: A job can have a pre- and post-process script
specified, which are run before and after the job is submitted.  This can be
useful for performing tasks like compression or decompression or input or
output data.
\end{description}

%%%%%%%%%%%%%%%%%%%%%%%%%%%%%%%%%%%%%%%%%%%%%%%%%%%%%%%%%%%%%%%%%%%%%%

%%%%%%%%%%%%%%%%%%%%%%%%%%%%%%%%%%%%%%%%%%%%%%%%%%%%%%%%%%%%%%%%%%%%%%
%%%%%%%%%%%%%%%%%%%%%%%%%%%%%%%%%%%%%%%
\section{\label{sec:Stork}Stork Applications}
%%%%%%%%%%%%%%%%%%%%%%%%%%%%%%%%%%%%%%%
\index{Stork|(}

Today's scientific applications have huge data requirements,
which continue to increase drastically every year.
These data are generally accessed by many
users from all across the the globe.
This requires moving huge amounts of data
around wide area networks to complete the computation cycle,
which brings with
it the problem of efficient and reliable data placement.

Stork is a scheduler for data placement.
With Stork, \Term{data placement jobs}
have been elevated to the same level as Condor's computational jobs;
data placements are queued, managed, queried and
autonomously restarted upon error.
Stork understands the semantics and protocols of data placement.

The underlying data placement jobs are performed by Stork
\Term{modules}, typically installed in the Condor \File{libexec}
directory.  The module name is encoded from the data placement type
and functions.  
For example, the \File{stork.transfer.file-file} module transfers data
from the \File{file:/} (local filesystem) to the \File{file:/}
protocol.  The \File{stork.transfer.file-file} module is the only
module bundled with Condor/Stork.  Additionally, contributed modules
may be \htmladdnormallink{downloaded}
{http://www.cs.wisc.edu/condor/stork/download.html}
for these data transfer protocols:

\begin{table}[hbt]
\begin{tabular}{ l l }
%file:/		& (Stork Server) local file system \\
ftp://		& FTP File Transfer Protocol \\
http://		& HTTP Hypertext Transfer Protocol \\
gsiftp://	& Globus Grid FTP  \\ 
nest://		& Condor NeST  network storage appliance (see
\URL{http://www.cs.wisc.edu/condor/nest/}) \\
srb://		& SDSC  Storage Resource Broker (SRB) (see
\URL{http://www.sdsc.edu/srb/}) \\
srm://		& Storage Resource Manager (SRM) (see
\URL{http://sdm.lbl.gov/srm-wg/}) \\
csrm://		& Castor Storage Resource Manager (Castor SRM) (see
\URL{http://castor.web.cern.ch/castor/}) \\
unitree://    & NCSA UniTree (see
\URL{http://www.ncsa.uiuc.edu/Divisions/CC/HPDM/unitree/}) \\
%		diskrouter:// -> UW DiskRouter Tool
\end{tabular}
\end{table}

The Stork
module API is simple and extensible, enabling users to create 
and use their own modules.

Stork includes high level features for managing data transfers.
By configuration, the number of active jobs running
from a Stork server may be limited.
Stork includes built in fault tolerance,
with capabilities for retrying failed jobs,
together with the specification of alternate protocols.
Stork users also have access to a higher level job manager, 
Condor DAGMan (section \ref{sec:DAGMan}),
which can manage both Stork data placement jobs
and traditional Condor jobs at the same time.


%%%%%%%%%%%%%%%%%%%%%%%%%%%%%%%%%%%%%%%
\subsection{\label{sec:Stork-Job-Submission}Submitting Stork Jobs}
%%%%%%%%%%%%%%%%%%%%%%%%%%%%%%%%%%%%%%%
\index{Stork!submit description file}

As with Condor jobs, Stork jobs are specified with a
submit description file.
It is important to note the syntax of the submit description file
for a Stork job is different than that used by Condor jobs.
Specifically,
Stork submit description files are written in the
\htmladdnormallink{ClassAd}{http://www.cs.wisc.edu/condor/classad}
language.  
See the ClassAd Language Reference Manual for complete details.
Please note that while most of Condor uses ClassAds,
Stork utilizes the most recent version of this language,
which has evolved over time.
Stork defines keywords.
When present in the job submit file,
keywords define the function of the job.

Here is sample Stork job submit description file,
showing file syntax and keywords.
A job specifies a 1-to-1 mapping of a data source URL to
destination URL.

\footnotesize
\begin{verbatim}
// This is a comment line.
[
    dap_type = transfer;
    src_url = "file:/etc/termcap";
    dest_url = "file:/tmp/stork/file-termcap";
]
\end{verbatim}
\normalsize


This example shows the ClassAd pairs that form the
heart of a Stork job specification.
The minimum keywords required to specify a Stork job are:

\begin{description}
  \item[dap\_type] Currently, the data type is constrained to 
  \SubmitCmd{transfer}.

  \item[src\_url]  Specify the data protocol and URL of the source.

  \item[dest\_url]  Specify the data protocol and URL of the destination.
\end{description}

Additionally, the following keywords may be used in 
a Stork submit description file:

\begin{description}
    \item[x509proxy] Specifies the location of the
    X.509 proxy file for protocols
    that use GSI authentication, such as \SubmitCmd{gsiftp://}.
    The special value of \emph{"default"} (quotes are required) invokes
    GSI libraries to search for the user credential in the standard locations.

    \item[alt\_protocols] A comma separated list of
    alternative protocol pairs (for source and destination protocols),
    used in a round robin fashion
    when transfers fail.
    See section \ref{sec:Stork-Fault-Protection} for a further discussion
    and examples.


\end{description}

Stork places no restriction on the submit file name or extension, and will
accept any valid file name for a Stork submit description file.

Submit data placement jobs to Stork using the
\Stork{submit} tool.
For example, after creating the submit description file
\File{sample.stork} with an
editor, submit the data transfer job with the command:

\begin{verbatim}
stork_submit sample.stork
\end{verbatim}

Stork then returns the associated job id, which is used by other Stork
job control tools.

Only the first ClassAd (a record expression within brackets) within 
a Stork submit description file becomes a data placement job
upon submission.
Other ClassAds within the file are ignored.

%%%%%%%%%%%%%%%%%%%%%%%%%%%%%%%%%%%%%%%
\subsection{\label{sec:Stork-Job-Management}Managing Stork Jobs}
%%%%%%%%%%%%%%%%%%%%%%%%%%%%%%%%%%%%%%%
Stork provides a set of command-line user tools for job management, including
submitting, querying, and removing data placement jobs.

%%%%%%%%%%%%%%%%%%%%%%%%%%%%%%%%%%%%%%%
\subsubsection{\label{sec:stork-query}Querying Stork Jobs}
%%%%%%%%%%%%%%%%%%%%%%%%%%%%%%%%%%%%%%%

Use \Stork{status} to check the status of any active
or completed Stork job.
\Stork{status} takes a single argument: the job id.
For example, to check the status of the Stork job with job id 3:

\begin{verbatim}
stork_status 3
\end{verbatim}

Use \Stork{q} to query all active Stork jobs.
\Stork{q} does not report on completed Stork jobs.

For example, to check the status all active Stork jobs:

\begin{verbatim}
stork_q
\end{verbatim}

%%%%%%%%%%%%%%%%%%%%%%%%%%%%%%%%%%%%%%%
\subsubsection{\label{sec:stork-rm}Removing Stork Jobs}
%%%%%%%%%%%%%%%%%%%%%%%%%%%%%%%%%%%%%%%

Active jobs may be removed from the job queue with the 
\Stork{rm} tool.  
\Stork{rm} takes a single argument: the job id of the job to remove.
All jobs may
be removed, provided they have not completed.

For example, to remove the queued job with job id 4:

\begin{verbatim}
stork_rm 4
\end{verbatim}

%%%%%%%%%%%%%%%%%%%%%%%%%%%%%%%%%%%%%%%
\subsection{\label{sec:Stork-Fault-Protection}Fault Tolerance}
%%%%%%%%%%%%%%%%%%%%%%%%%%%%%%%%%%%%%%%

In an ideal world, all data transfers succeed on the first attempt.
However, data transfers do fail for various reasons.
Stork is designed with data transfer fault tolerance.
Based on configuration, Stork retries failed data transfer jobs
using specified protocols.

If a  transfer fails, Stork attempts the transfer again,
until the number of attempts reaches the limit,
as defined by the 
configuration variable
\Macro{STORK\_MAX\_RETRY} 
(section \ref{param:StorkMaxRetry}).  

For each attempt at transfer,
the transfer protocols to be used at both source and destination are defined.
These transfer protocols may vary,
when defined by an \SubmitCmd{alt\_protocols} entry in the
submit description file.
The location of the data at the source and destination
is unchanged by the \SubmitCmd{alt\_protocols} entry.
\SubmitCmd{alt\_protocols} defines an ordered list of alternative
translation protocols to be used.
Each entry in the list is a pair.
The first of the pair defines the protocol to be used at the source 
of the transfer.
The second of the pair defines the protocol to be used at the destination 
of the transfer.

The syntax is a comma-separated list of pairs.
A dash character separated the pairs.
The protocol name is given in all lower case letters,
without colons or slash characters.
Stork uses these strings to identify the protocol translation
and transfer module to be used. 

The initial translation protocol
(specified in the \SubmitCmd{src\_url} and \SubmitCmd{dest\_url} entries)
together with the list defined by  
an \SubmitCmd{alt\_protocols} entry form the ordered list
of protocols to be utilized in a round robin fashion.

For example, if \MacroNI{STORK\_MAX\_RETRY} has the value 4,
and the Stork job submit description file contains
\footnotesize
\begin{verbatim}
[
    dap_type = transfer;
    src_url = "gsiftp://serverA/dirA/fileA";
    dest_url = "http://serverB/dirB/fileB";
]
\end{verbatim}
\normalsize

then Stork will attempt up to 4 transfers,
with each using the same translation protocol.
\SubmitCmd{gsiftp://} is used at the source,
and \SubmitCmd{http://} is used at the destination.
The Stork job fails if it has not been completed after
4 attempts.

A second example shows the transfer protocols used for
each attempted transfer, 
when \SubmitCmd{alt\_protocols} is used.
For this example, assume that 
\MacroNI{STORK\_MAX\_RETRY} has the value 7.
\footnotesize
\begin{verbatim}
[
    dap_type = transfer;
    src_url = "gsiftp://no-such-server/dir/file";
    dest_url = "file:/dir/file";
    alt_protocols = "ftp-file, http-file";
]
\end{verbatim}
\normalsize

Stork attempts the following transfers, in the given order,
stopping when the transfer succeeds.
\begin{enumerate}
    \item from 
			\File{gsiftp://no-such-server/dir/file}
            to
			\File{file:/dir/file}
    \item from 
			\File{ftp://no-such-server/dir/file}
            to
			\File{file:/dir/file}
    \item from 
			\File{http://no-such-server/dir/file}
            to
			\File{file:/dir/file}
    \item from 
			\File{gsiftp://no-such-server/dir/file}
            to
			\File{file:/dir/file}
    \item from 
			\File{ftp://no-such-server/dir/file}
            to
			\File{file:/dir/file}
    \item from 
			\File{http://no-such-server/dir/file}
            to
			\File{file:/dir/file}
    \item from 
			\File{gsiftp://no-such-server/dir/file}
            to
			\File{file:/dir/file}
 \end{enumerate}

%%%%%%%%%%%%%%%%%%%%%%%%%%%%%%%%%%%%%%%
\subsection{\label{sec:Stork-Advanced}Running Stork Jobs Under DAGMan}
%%%%%%%%%%%%%%%%%%%%%%%%%%%%%%%%%%%%%%%
\index{Stork!jobs under DAGMan}

Condor DAGMan (section \ref{sec:DAGMan}) provides high level management
of both traditional CPU jobs and Stork data placement jobs. 
Using DAGMan, users can specify data placement
using the \Arg{DATA} keyword.
DAGMan can mix Stork data transfer jobs 
and Condor jobs.
This capability lends itself well to grid computing,
as data is often staged in (transferred)
before processing the data.
After processing, output is often staged out (transferred).

Here is a sample DAGMan input file
that stages in input files using Stork transfers,
processes the data as a Condor job,
and stages out the result using a Stork transfer.

\footnotesize
\begin{verbatim}
# Transfer input files using Stork
DATA INPUT1 transfer_input_data1.stork
DATA INPUT1 transfer_input_data2.stork

DATA INPUT2 transfer_data
#
# Process the data using Condor
JOB PROCESS process.condor
#
# Transfer output file using Stork
DATA RESULT transfer_result_data.stork
#
# Specify job dependencies
PARENT INPUT1 INPUT2 CHILD PROCESS
PARENT PROCESS CHILD RESULT
\end{verbatim}
\normalsize

\index{Stork|)}

%%%%%%%%%%%%%%%%%%%%%%%%%%%%%%%%%%%%%%%%%%%%%%%%%%%%%%%%%%%%%%%%%%%%%%

%%%%%%%%%%%%%%%%%%%%%%%%%%%%%%%%%%%%%%%%%%%%%%%%%%%%%%%%%%%%%%%%%%%%%%
\input{user-man/logview.tex}
%%%%%%%%%%%%%%%%%%%%%%%%%%%%%%%%%%%%%%%%%%%%%%%%%%%%%%%%%%%%%%%%%%%%%%


%%%%%%%%%%%%%%%%%%%%%%%%%%%%%%%%%%%%%%%%
\section{Special Environment Considerations}
%%%%%%%%%%%%%%%%%%%%%%%%%%%%%%%%%%%%%%%%

\subsection{AFS}

\index{file system!AFS}
\index{AFS!interaction with}
The Condor daemons do not run authenticated to AFS; they do not possess
AFS tokens.
Therefore, no child process of Condor will be AFS authenticated.
The implication of this is that you must set file permissions so
that your job can access any necessary files residing on an AFS volume
without relying on having your AFS permissions.

If a job you submit to Condor needs to access files residing in AFS,
you have the following choices:
\begin{enumerate}
\item Copy the needed files from AFS to either a local hard disk where 
Condor can access them using remote system calls (if
this is a standard universe job), or copy them to an NFS volume.
\item If you must keep the files on AFS, then set a host ACL
(using the AFS \Prog{fs setacl} command) on the subdirectory to
serve as the current working directory for the job.
If a standard universe job, then the host ACL needs
to give read/write permission to any process on the submit machine.
If vanilla universe job, then you need to set the ACL such that any host 
in the pool can access the files without being authenticated.
If you do not know how to use an AFS host ACL, ask the person at your 
site responsible for the AFS configuration.
\end{enumerate}

The Condor Team hopes to improve upon how Condor deals with AFS 
authentication in a subsequent release.

Please see section~\ref{sec:Condor-AFS-Users} on
page~\pageref{sec:Condor-AFS-Users} in the Administrators Manual for
further discussion of this problem.

\subsection{NFS Automounter}

\index{file system!NFS}
\index{NFS!interaction with}
If your current working directory when you run \Condor{submit}
\index{Condor commands!condor\_submit}
is accessed via an NFS automounter, Condor may have problems if the
automounter later decides to unmount the volume before your job has
completed.
This is because \Condor{submit} likely has stored the
dynamic mount point as the job's initial current working directory, and
this mount point could become automatically unmounted by the
automounter.

There is a simple work around: When submitting your job, use the 
\Arg{initialdir} command in your submit description file to point to
the stable access point.
For example,
suppose the NFS automounter is configured to mount a volume at mount point
\File{/a/myserver.company.com/vol1/johndoe}
whenever the directory \File{/home/johndoe} is accessed.
Adding the following line to the
submit description file solves the problem.
\begin{verbatim}
        initialdir = /home/johndoe
\end{verbatim}

\subsection{Condor Daemons That Do Not Run as root}

\index{Unix daemon!running as root}
\index{daemon!running as root}
Condor is normally installed such that the Condor daemons have root
permission.
This allows Condor to run the \condor{shadow} 
\index{Condor daemon!condor\_shadow}
\index{remote system call!condor\_shadow}
process and
your job with your UID and file access rights.
When Condor
is started as root, your Condor jobs can access whatever files you can.

However, it is possible that whomever installed Condor 
did not have root access, or
decided not to run the daemons as root.
That is unfortunate,
since Condor is designed to be run as the Unix user root.
To see if Condor is
running as root on a specific machine, enter the command
\begin{verbatim}
        condor_status -master -l <machine-name>
\end{verbatim}

where \verb@machine-name@ is the name of the specified machine.
This command displays a \condor{master} ClassAd; if the
attribute \AdAttr{RealUid} equals zero,
then the Condor daemons are indeed
running with root access.  If the
\AdAttr{RealUid} attribute is not zero, then the Condor daemons do not have
root access.

\Note The Unix program \Prog{ps}
is \emph{not} an effective
method of determining if Condor is running with root access.
When using \Prog{ps},
it may often appear that the daemons are
running as the condor user instead of root.
However, note that the \Prog{ps},
command shows the current \emph{effective} owner of the
process, not the \emph{real} owner.  (See the \Cmd{getuid}{2} and
\Cmd{geteuid}{2} Unix man pages for details.)  In Unix, a process
running under the real UID of root may switch its effective UID.
(See the \Cmd{seteuid}{2} man page.)
For security reasons, the daemons
only set the effective UID to root when absolutely necessary
(to perform a privileged operation).

If they are not running with root access, you need to make any/all files
and/or directories that your job will touch readable and/or writable by
the UID (user id) specified by the RealUid attribute.
Often this may
mean using the Unix command \verb@chmod 777@
on the directory where you submit your Condor job.

%%%%%%%%%%%%%%%%%%%%%%%%%%%%%%%%%%%%%%%%
\subsection{\label{sec:Job-Lease}
Job Leases}
%%%%%%%%%%%%%%%%%%%%%%%%%%%%%%%%%%%%%%%%
\index{job!lease}

A job lease specifies how long a given job will attempt to run
on a remote resource,
even if that resource loses contact with the submitting machine.
Similarly, it is the length of time the submitting machine will
spend trying to reconnect to the (now disconnected) execution host,
before the submitting machine gives up and tries to claim
another resource to run the job.
The goal aims at run only once semantics,
so that the \Condor{schedd} daemon does not allow the same job
to run on multiple sites simultaneously.

If the submitting machine is alive,
it periodically renews the job lease,
and all is well.
If the submitting machine is dead,
or the network goes down, the job lease will no longer be renewed.
Eventually the lease expires.
While the lease has not expired,
the execute host continues to try to run the job,
in the hope that the submit machine will come back to life
and reconnect.
If the job completes, the lease has not expired,yet the 
submitting machine is still dead,
the \Condor{starter} daemon will wait for a
\Condor{shadow} daemon to reconnect, 
before sending final information on the job,
and its output files.
Should the lease expire, the \Condor{startd} daemon
kills off the \Condor{starter} daemon and user job.

The user must set a value for \Attr{job\_lease\_duration}
to keep a job running in the case that the submit side no longer
renews the lease.
There is a tradeoff in setting the value of \Attr{job\_lease\_duration}. 
Too small a value,
and the job might get killed before the submitting machine has a
chance to recover.
Forward progress on the job will be lost.
Too large a value,
and execute resource will be tied up waiting for the job lease to expire.
The value 
should be chosen based on how long is the user willing to tie up
the execute machines, how quickly submit machines come  back up,
and how much work would be lost if the lease expires,
the job is killed, and the job must start over from its beginning.

\Attr{job\_lease\_duration}
is only valid for vanilla and java universe jobs.
Chirp I/O and streaming I/O (which uses Chirp I/O) may not
be used in conjunction with a defined \Attr{job\_lease\_duration}.

A current limitation
is that jobs with a defined \Attr{job\_lease\_duration} will not
reconnect if the jobs flock to a remote pool.


%%%%%%%%%%%%%%%%%%%%%%%%%%%%%%%%%%%%%%%%
\section{Potential Problems}
%%%%%%%%%%%%%%%%%%%%%%%%%%%%%%%%%%%%%%%%

\subsection{Renaming of argv[0]}

\index{argv[0]!Condor use of}
When Condor starts up your job, it renames argv[0] (which usually
contains the name of the program) to \condor{exec}.
This is
convenient when examining a machine's processes with the Unix
command \Prog{ps}; the process
is easily identified as a Condor job.  

Unfortunately, some programs read argv[0] expecting their own program
name and get confused if they find something unexpected like
\condor{exec}.

\index{Condor!user manual|)}
\index{user manual|)}
