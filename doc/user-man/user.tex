%%%%%%%%%%%%%%%%%%%%%%%%%%%%%%%%%%%%%%%%%%%%%%%%%%%%%%
\section{Welcome to Condor}  
%
% .... or alternatively called the 'warm fuzzies' section
% <smirk>  
% 
%
% Warning: much of what you are about to read was very 
% hastily written by a very tired Todd.... Good Luck.  
%%%%%%%%%%%%%%%%%%%%%%%%%%%%%%%%%%%%%%%%%%%%%%%%%%%%%

We are pleased to present Condor \VersionNotice! Condor is developed by
the Condor Team at the University of Wisconsin-Madison (UW-Madison), and
was first installed as a production system in the UW-Madison Computer
Sciences department nearly 10 years ago. This Condor pool has since
served as a major source of computing cycles to UW faculty and students.
For many, it has revolutionized the role computing plays in their
research. An increase in one, and sometimes even two, orders of
magnitude in the computing throughput of a research organization can
have a profound impact on its size, complexity, and scope. Over the
years, the Condor Team has established collaborations with scientists
from around the world and has provided them with access to surplus
cycles (one of whom has consumed 100 CPU years!). Today, our
department's pool consists of more than 350 desktop UNIX workstations.
On a typical day, our pool delivers more than 180 CPU days to UW
researchers. Additional Condor pools have been established over the
years across our campus and the world. Groups of researchers, engineers,
and scientists have used Condor to establish compute pools ranging in
size from a handful to hundreds of workstations. We hope that Condor
will help revolutionize your compute environment as well.


%%%%%%%%%%%%%%%%%%%%%%%%%%%%%%%%%%%%%%%%%%%%%%%%%%%%%%%
\section{What does Condor do?}
%%%%%%%%%%%%%%%%%%%%%%%%%%%%%%%%%%%%%%%%%%%%%%%%%%%%%%%


In a nutshell, Condor is a specialized batch system for managing
compute-intensive jobs.  Like most batch systems, Condor provides a
queueing mechanism, scheduling policy, priority scheme, and resource
classifications.  Users submit their compute jobs to Condor, Condor puts
the jobs in a queue, runs them, and then informs the user as to the
result.

Batch systems normally operate only with dedicated machines.  Often 
termed compute servers, these dedicated machines are typically owned by
one organization and dedicated to the sole purpose of running compute
jobs.  Condor can schedule jobs on dedicated machines.  But unlike traditional 
batch systems, Condor is also designed to effectively 
utilize non-dedicated machines to run jobs.  By being told to only
run compute jobs on machines which are currently not being used (no keyboard
activity, no load average, no active telnet users, etc), Condor can
effectively harness otherwise idle machines throughout the network.
This is important because often times the amount of
compute power represented by the aggregate total of all the non-dedicated 
desktop workstations sitting on people's desks throughout the
organization is far greater than the compute power of a dedicated
central resource.

Condor has several unique capabilities at its disposal which are geared 
towards effectively utilizing non-dedicated resources that are not owned or
managed by a centralized resource. These include transparent process
checkpoint and migration, remote system calls, and ClassAds.  Please be
certain to have read section~\ref{sec:what-is-condor} for a general 
discussion about these capabilities before reading any further.


%%%%%%%%%%%%%%%%%%%%%%%%%%%%%%%%%%%%%%%%%%%%%%%%%%%%%%%%
\section{Condor Matchmaking with ClassAds}
%%%%%%%%%%%%%%%%%%%%%%%%%%%%%%%%%%%%%%%%%%%%%%%%%%%%%%%%

Before you start learning about how to submit a job, it is important to
understand how Condor performs resource allocation. Understanding the
unique framework by which Condor matches submitted jobs with machines is
the key to getting the most from Condor's scheduling algorithm. 

Condor is unlike most other batch systems, which typically involve
submitting a job to one of several pre-defined job queues. These
environments typically become both constrained and complicated, as
system administrators scramble to add more queues in response to the
variety of user needs. Likewise, the user is asked to make compromises
and is left with the burden of not only keeping track of which queues
have which properties, but also deciding which queue would be the
optimal one to use for their jobs.

Instead, Condor simply acts as a matchmaker of ClassAds. Condor's
ClassAds are analogous to the classified advertising section of the
newspaper. Sellers advertise specifics about what they have to sell,
hoping to attract a buyer. Buyers may advertise specifics about what
they wish to purchase. Both buyers and sellers may have constraints that
need to be satisfied. For instance, perhaps the buyer is not willing to
spend more than X dollars, and the seller requires to receive a minimum
of Y dollars. Furthermore, both want to rank requests from the other in
such a fashion that is to their advantage. Certainly a seller would rank
a buyer offering \$50 dollars for a service higher than a different
buyer offering \$25 for the same service. In Condor, users submitting
jobs can be thought of as buyers of compute resources and machine owners
are the sellers. 

All machines in the Condor pool advertise their attributes, such as
available RAM memory, CPU type and speed, virtual memory size, current
load average, and many other static and dynamic properties, in a machine
ClassAd. The machine ClassAd also advertises under what conditions it is
willing to run a Condor job and what type of job it would prefer. These
policy attributes can reflect the individual terms and preferences by
which all the different owners have graciously allowed their machine to
be part of the Condor pool. For example, John Doe's machine may
advertise that it is only willing to run jobs at night and when there is
nobody typing at the keyboard. In addition, John Doe's machines may
advertise a preference (rank) for running jobs submitted by either John
Doe or one of his co-workers whenever possible. 

Likewise, when submitting a job, you can specify many different job
attributes, including whatever requirements and preferences for whatever
type of machine you'd like this job to use. For instance, perhaps you're
looking for the fastest floating-point machine available, i.e. you want to
rank matches based upon floating-point performance. Or perhaps you only
care that the machine has a minimum of 128 Meg of RAM. Or perhaps you'll
just take any machine you can get! These job attributes and requirements
are bundled up into a job ClassAd and ``published'' to Condor.

Condor then plays the role of a matchmaker by continuously reading
through all of the job ClassAds and all of the machine ClassAds and
matching and ranking job ads with machine ads, while making certain that all
requirements in both ads are satisfied. 

%%%%%
\subsection{Inspecting Machine ClassAds with \condor{status}}
%%%%%

Now would be a good time to try the \Condor{status} command to get a feel
for what a ClassAd actually looks like.
\Condor{status} displays summarizes and displays information from
ClassAds about the resources available in your pool. If you just type
\Condor{status} and hit enter, you will see a summary of all the machine
ClassAds similar to the following:
\begin{center}
\begin{verbatim}
Name       Arch     OpSys        State      Activity   LoadAv Mem  ActvtyTime

adriana.cs INTEL    SOLARIS251   Claimed    Busy       1.000  64    0+01:10:00
alfred.cs. INTEL    SOLARIS251   Claimed    Busy       1.000  64    0+00:40:00
amul.cs.wi SUN4u    SOLARIS251   Owner      Idle       1.000  128   0+06:20:04
anfrom.cs. SUN4x    SOLARIS251   Claimed    Busy       1.000  32    0+05:16:22
anthrax.cs INTEL    SOLARIS251   Claimed    Busy       0.285  64    0+00:00:00
astro.cs.w INTEL    SOLARIS251   Claimed    Busy       0.949  64    0+05:30:00
aura.cs.wi SUN4u    SOLARIS251   Owner      Idle       1.043  128   0+14:40:15
\end{verbatim}
\Dots 
\end{center}


\Condor{status} can summarize machine ads in a wide variety of ways.
For example, \Condor{status -available} shows only machines which are
willing to run jobs now, and \Condor{status -run} shows only machines
which are currently running jobs.  \Condor{status} can also display
other types of ads other than just machine ads; \Condor{status -help}
lists all the options, and/or refer to the \Condor{status} command
reference page located on page~\pageref{man-condor-status}.

To get a feel for what a typical machine ClassAd looks like in its
entirety, use the \Condor{status -l} command.
Figure~\ref{fig:CondorStatusL} shows the complete machine ad for
workstation alfred.cs.wisc.edu. Some of these attributes are used by
Condor itself for scheduling. Other attributes are simply informational.
But the important point is that \textit{any} of these attributes in the
machine ad can be utilized at job submission time as part of a request
or preference on what machine to use. Furthermore, additional attributes
can be easily added; for example, perhaps your site administrator has
added a physical location attribute to your machine ClassAds.

%
% figures for this section
%
% condor_status -l alfred
%
\begin{center}
\begin{figure}
\CondorVerySmall
\begin{verbatim}
MyType = "Machine"
TargetType = "Job"
Name = "alfred.cs.wisc.edu"
Machine = "alfred.cs.wisc.edu"
StartdIpAddr = "<128.105.83.11:32780>"
Arch = "INTEL"
OpSys = "SOLARIS251"
UidDomain = "cs.wisc.edu"
FileSystemDomain = "cs.wisc.edu"
State = "Unclaimed"
EnteredCurrentState = 892191963
Activity = "Idle"
EnteredCurrentActivity = 892191062
VirtualMemory = 185264
Disk = 35259
KFlops = 19992
Mips = 201
LoadAvg = 0.019531
CondorLoadAvg = 0.000000
KeyboardIdle = 5124
ConsoleIdle = 27592
Cpus = 1
Memory = 64
AFSCell = "cs.wisc.edu"
START = LoadAvg - CondorLoadAvg <= 0.300000 && KeyboardIdle > 15 * 60
Requirements = TRUE
Rank = Owner == "johndoe" || Owner == "friendofjohn" 
CurrentRank =  - 1.000000
LastHeardFrom = 892191963
\end{verbatim}
\normalsize
\caption{\label{fig:CondorStatusL}Sample output from \Condor{status -l alfred}}
\end{figure}
\end{center}


%%%%%%%%%%%%%%%%%%%%%%%%%%%%%%%%%%%%%%%%%%%%%%%%%%%%%%%%%%%%%
\section{Road-map for running jobs with Condor}
%%%%%%%%%%%%%%%%%%%%%%%%%%%%%%%%%%%%%%%%%%%%%%%%%%%%%%%%%%%%%

The road to effectively using Condor is short one.  The basics
are quickly and easily learned.  Unlike some other network-cluster
solutions, Condor typically does not require you to make any
changes to your program, even to do more advanced tasks such as
process checkpoint and migration. 

Using Condor can be broken down into the following steps:

\begin{description}

\item[Job Preparation.]First, you will need to prepare your job for
Condor. This involves preparing it to run as a background batch job,
deciding which Condor runtime environment (or \Term{Universe}) to use,
and possibly relinking your program with the Condor library via the
\Condor{compile} command. 

\item[Submit to Condor.]Next, you'll submit your program to Condor via
the \Condor{submit} command. With \Condor{submit} you'll tell Condor
information about the run, such as what executable to run, what
filenames to use for keyboard and screen (stdin and stdout) data, and
where to send email when the job completes. You can also tell Condor how
many times to run a program; many users may want to run the same program
multiple times with multiple different data files. Finally, you'll also
describe to Condor what type of machine you want to run your program. 

\item[Condor Runs the Job.]Once submitted, you'll monitor your job's
progress via the \Condor{q} and \Condor{status} commands, and/or
possibly modify the order in which Condor will run your jobs with
\Condor{prio}. If desired, Condor can even inform you every time your job
is checkpointed and/or migrated to a different machine. 

\item[Job Completion.]When your program completes, Condor will tell you
(via email if preferred) the exit status of your program and how much
CPU and wall clock time the program used. You can remove a job from the
queue prematurely with \Condor{rm}. 

\end{description}  % of Road Map steps


%%%%%%%%%%%%%%%%%%%%%%%%%%%%%%%%%%%%%%%%%%%%%%%%%%%%
\section{Job Preparation}
%%%%%%%%%%%%%%%%%%%%%%%%%%%%%%%%%%%%%%%%%%%%%%%%%%%

Before submitting your program to Condor, you must first make certain
your program is batch ready.  Next you'll need to decide upon a Condor
Universe, or runtime environment, for your job.

%%%%%%%%%%%%%%
\subsection{Batch Ready}
%%%%%%%%%%%%%%

Condor runs your program unattended and in the background. Make certain
that your program can do this before submitting it to Condor. Condor can
redirect console output (stdout and stderr) and keyboard (stdin) input
to/from files for you, so you may have to create file(s) that contain
the proper keystrokes needed for your file.

It is also very easy to quickly submit multiple runs of your program to
Condor. Perhaps you want to run the same program 500 times on 500
different input data sets. If so, you need to arrange your data files
accordingly so that each run can read its own input, and so one run's
output files do not clobber (overwrite) another run's files. For each
individual run, Condor allows you to easily customize that run's initial
working directory, stdin, stdout, stderr, command-line arguments, or
shell environment. Therefore, if your program directly opens its own
data files, hopefully it can read what filenames to use via either stdin
or the command-line. If your program opens a static filename every time,
you will likely need to make a separate subdirectory for each run to
store its data files into.

%%%%%%%%%%%%%%%%
\subsection{Choosing a Condor Universe}
%%%%%%%%%%%%%%%%

A Universe in Condor defines an execution environment. You can state
which Universe to use for each job in a submit-description file when the
job is submitted. Condor \VersionNotice\ supports three different
program Universes for user jobs:
\begin{itemize}
	\item Standard
	\item Vanilla
	\item PVM
\end{itemize}

If your program is a parallel application written for PVM, then you
would ask Condor for the PVM universe at submit time.  See
section~\ref{sec:PVM} for information on using Condor with PVM jobs.

Otherwise, you need to decide between Standard or Vanilla Universe.
In general, Standard Universe provides more services to your job than
Vanilla Universe and therefore Standard is usually preferable.  But 
Standard Universe also imposes some restrictions on
what your job can do.  Vanilla Universe has very few restrictions, and
can be used when either the Standard Universe's additional services are not
desired or when the job cannot abide by the Standard Universe's
restrictions.

%%%%%%%%%%%%%%%%%%%%%%%%%%%%%%%%%%%%%%%%%%%%%%%%%%%%%%%%%%%%%%%%%%%%%%
\subsubsection{\label{sec:standard-universe}Standard Universe}
%%%%%%%%%%%%%%%%%%%%%%%%%%%%%%%%%%%%%%%%%%%%%%%%%%%%%%%%%%%%%%%%%%%%%%

In the Standard Universe, which is the default, Condor will
automatically make checkpoints (take a snapshot of its current state) of
the job. So if a Standard Universe job is running on a machine and needs
to leave (perhaps because the owner of the machine returned), Condor
will checkpoint the job and then migrate it to some other idle machine.
Because the job was checkpointed, Condor will restart the job from the
checkpoint and therefore it can continue to run \underline{from where it
left off}. 

Furthermore, Standard Universe jobs can use Condor's \Term{remote system
calls} mechanism, which enables the program to access data files from
any machine in the Condor pool regardless of whether that machine is
sharing a file-system via NFS (or AFS) or if the user has an account
there. Even if your files are just sitting on your local hard-drive, or
in /tmp, Condor jobs can access them.  How it works is when your Condor
job start up on some remote machine, a corresponding \Condor{shadow}
process also starts up on the machine where you submitted the job.
As your job runs on the remote machine, Condor traps hundreds of operating system 
calls (such as calls to open, read, and
write files) and ships them over the network via a remote procedure call
to the \Condor{shadow} process.  The \Condor{shadow} executes the system
call on the submit machine and passes the result back over the network
to your Condor job.  The end result is everything appears to your job
like it is simply running on the submit machine, even as it bounces
around to different machines in the pool.

The transparent checkpoint/migration and remote system calls are highly
desirable services. However, all Standard Universe jobs must be
re-linked with the Condor libraries. Although this is a simple process,
after doing so there are a few restrictions on what the program can do:
\begin{enumerate}
	\item On some platforms, specifically HPUX and Digital Unix
(OSF/1), shared libraries are not supported; therefore on these
platforms applications must be statically linked (Note: shared library
checkpoint support is available on IRIX, Solaris, and LINUX). 
	\item Only single process jobs are supported, i.e. the fork(2), exec(2),
system(3) and similar calls are not implemented.
	\item Signals and signal handlers are supported, but Condor reserves the 
SIGUSR2 and SIGTSTP signals and does not permit their use by user code.
	\item Most interprocess communication (IPC) calls are not supported, i.e. the 
socket(2), send(2), recv(2), and similar calls are not implemented.
	\item All file operations must be idempotent --- read-only and write-only file 
accesses work correctly, but programs which both read and write to the 
same file may not.
	\item Each Condor job that has been checkpointed has an associated 
\Term{checkpoint file} which is approximately the size of the address space of the 
process. Disk space must be available to store the checkpoint file on the 
submitting machine (or on a Condor Checkpoint Server if your site administrator
has set one up).
\end{enumerate}

Although relinking a program for use in Condor's Standard Universe is
very easy to do and typically requires no changes to the program's
source code, sometimes users who wish to utilize Condor do not have
access to their program's source or object code. Without access to
either the source or object code, relinking for the Standard Universe is
impossible. This situation is typical with commercial applications,
which usually only provide a binary executable and only rarely provide
source or object code.

%%%%%%%%%%%%
\subsubsection{Vanilla Universe}
%%%%%%%%%%%%

The Vanilla Universe in Condor is for running any programs which cannot
be successfully re-linked for submission into the Standard Universe.
Shell scripts are another good reason to use the Vanilla Universe.
However, here's the down side: Vanilla jobs cannot checkpoint or use
remote system calls. So, for example, when a user returns to a
workstation running a Vanilla job, Condor can either suspend the job or
restart the job \underline{from the beginning} someplace else.
Furthermore, unlike Standard jobs, Vanilla jobs must rely on some
external mechanism in place (such as NFS, AFS, etc.) for accessing data
files from different machines because Remote System Calls are only
available in the Standard Universe.

%%%%%%%%%%%%%%
\subsection{Relinking for the Standard Universe}
%%%%%%%%%%%%%%

Relinking a program with the Condor libraries (\condor{rt0.o} and
\condor{syscall\_lib.a}) is a simple one-step process with Condor
\VersionNotice. To re-link a program with the Condor libraries for
submission into the Standard Universe, simply run \Condor{compile}. See
the command reference page for \Condor{compile} on
page~\pageref{man-condor-compile}.

Note that even once your job is re-linked, you can still run your program
outside of Condor directly from the shell prompt as usual.  When you do
this, the following message is printed to remind you that this 
binary is linked with the Condor libraries:
\begin{verbatim}
  WARNING: This binary has been linked for Condor.
  WARNING: Setting up to run outside of Condor...
\end{verbatim}

%%%%%%%%%%%%%%%%%%%%%%%%%%%%%%%%%%%%%%%%%%%%%%%%%%%%%%%%%%%%%%
\section{Submitting a Job to Condor}
%%%%%%%%%%%%%%%%%%%%%%%%%%%%%%%%%%%%%%%%%%%%%%%%%%%%%%%%%%%%%%

\Condor{submit} is the program for actually submitting jobs to Condor.
\Condor{submit} wants as its sole argument the name of a submit-description 
file which contains commands and keywords to direct the queuing of jobs.
In the submit-description file, you will tell Condor everything it needs
to know about the job.  Items such as the name of the executable to run,
the initial working directory, command-line arguments, etc., all go into
the submit-description file.  \Condor{submit} then creates a new job
ClassAd based upon this information and ships it along with the executable to run 
to the \Condor{schedd} daemon running on your machine.  At that point your job has
been submitted into Condor.

Now please read the \Condor{submit} manual page in the 
Command Reference chapter before you continue; it is on page~\pageref{man-condor-submit} and
contains a complete and full description of how to use \Condor{submit}.

%%%%%%%%%%%%%%%%%%%%
\subsection{Sample submit-description files}  
%%%%%%%%%%%%%%%%%%%%

Now that you have read about \Condor{submit} and have an idea of how it
works, we'll followup with a few additional examples of submit-description files.

\subsubsection{Example 1} 

Example 1 below about the simplest submit-description
file possible. It queues up one copy of the program ``foo'' for execution
by Condor. Condor will attempt to run the job on a machine which has the
same architecture and operating system as the machine from which it was
submitted. Since no input, output, and error commands were given, the
files stdin, stdout, and stderr will all refer to /dev/null. (The
program may produce output by explicitly opening a file and writing to
it.)
\begin{verbatim}
  ####################                                                    
  # 
  # Example 1                                                                       
  # Simple condor job description file                                    
  #                                                                       
  ####################                                                    
                                                                          
  Executable     = foo                                                    
  Queue    
\end{verbatim}

\subsubsection{Example 2}

Example 2 below queues 2 copies of program the program ``mathematica''. The
first copy will run in directory ``run\_1'', and the second will run in
directory ``run\_2''. In both cases the names of the files used for stdin,
stdout, and stderr will be test.data, loop.out, and loop.error,
but the actual files will be different as they are in different
directories. This is often a convenient way to organize your data if you
have a large group of condor jobs to run. The example file submits
``mathematica'' as a Vanilla Universe job, perhaps because the source
and/or object code to program ``mathematica'' was not available and
therefore the re-link step necessary for Standard Universe jobs could not
be performed. 
\begin{verbatim}
  ####################     
  #                       
  # Example 2: demonstrate use of multiple     
  # directories for data organization.      
  #                                        
  ####################                    
                                         
  Executable     = mathematica          
  Universe = vanilla                   
  input   = test.data                
  output  = loop.out                
  error   = loop.error             
                                  
  Initialdir     = run_1         
  Queue                         
                               
  Initialdir     = run_2      
  Queue                     
\end{verbatim}

\subsubsection{Example 3}

The submit-description file Example 3 below queues 150
runs of program ``foo'' which must have been compiled and linked for
Silicon Graphics workstations running IRIX 6.x. Condor will not attempt
to run the processes on machines which have less than 32 megabytes of
physical memory, and will run them on machines which have at least 64
megabytes if such machines are available. Stdin, stdout, and stderr will
refer to ``in.0'', ``out.0'', and ``err.0'' for the first run of this program
(process 0). Stdin, stdout, and stderr will refer to ``in.1'', ``out.1'',
and ``err.1'' for process 1, and so forth. A log file containing entries
about where/when Condor runs, checkpoints, and migrates processes in this
cluster will be written into file ``foo.log''.
\begin{verbatim}
      ####################                                                    
      #                                                                       
      # Example 3: Show off some fancy features including                            
      # use of pre-defined macros and logging.                                
      #                                                                       
      ####################                                                    
                                                                          
      Executable     = foo                                                    
      Requirements   = Memory >= 32 && OpSys == "IRIX6" && Arch =="SGI"     
      Rank		     = Memory >= 64
      Image_Size     = 28 Meg                                                 
                                                                          
      Error   = err.$(Process)                                                
      Input   = in.$(Process)                                                 
      Output  = out.$(Process)                                                
      Log = foo.log                                                                       
                                                                          
      Queue 150
\end{verbatim}

%%%%%%%%%%%%%%%%%
\subsection{More about Requirements and Rank}
%%%%%%%%%%%%%%%%%

There are a few more things you should know about the powerful
\Opt{Requirements} and \Opt{Rank} commands in the submit-description file.

First of all, both of them need to be valid Condor ClassAd expressions.
From the \Condor{submit} manual page and the above examples, you can see
that writing ClassAd expressions is quite intuitive (especially if you
are familiar with the programming language C).  However, there are some
pretty nifty expressions you can write with ClassAds if you care to read
more about them.  The complete lowdown on ClassAds and their expressions
can be found in section~\ref{classad-reference} on 
page~\pageref{classad-reference}.

All of the commands in the submit-description file are case insensitive, 
\underline{except} for the ClassAd attribute string values that appear in the
ClassAd expressions that you write!  ClassAds attribute names are
case insensitive, but ClassAd string
values are always \underline{case sensitive}.  If you accidently say
\begin{verbatim}
        requirements = arch == "alpha"
\end{verbatim}
instead of what you should have said, which is:
\begin{verbatim}
        requirements = arch == "ALPHA"
\end{verbatim}
you will not get what you want.

So now that you know ClassAd attributes are case-sensitive, how do you
know what the capitalization should be for an arbitrary attribute ?  For
that matter, how do you know what attributes you can use ?  The answer
is you can use any attribute that appears in either a machine or a job
ClassAd.  To view all of the machine ClassAd attributes, simply run \Condor{status -l}.  The \Arg{-l} argument to
\Condor{status} means to display the complete machine ClassAd.  Similarly
for job ClassAds, do a \Condor{q -l} command (Note: you'll have to submit some
jobs first before you can view a job ClassAd).  This
will show you all the available attributes you can play with, along with their
proper capitalization.  

To help you out with what these attributes all signify, below we list
descriptions for the attributes which will be common by default to every
machine ClassAd. Remember that because ClassAds are flexible, the
machine ads in your pool may be including additional attributes specific
to your site's installation/policies. 
\label{user-man-machad}
\begin{description}
%
\item[Activity] : String which describes Condor job activity on the machine.
Can have one of the following values:
	\begin{description}
	\item[``Idle''] : There is no job activity
	\item[``Busy''] : A job is busy running
	\item[``Suspended''] : A job is currently suspended
	\item[``Vacating''] : A job is currently checkpointing
	\item[``Killing''] : A job is currently being killed
	\item[``Benchmarking''] : The startd is running benchmarks
	\end{description}
%
\item[AFSCell] : If the machine is running AFS, this is a string
containing the AFS cell name.
%
\item[Arch] : String with the architecture of the machine.  Typically
one of the following: 
	\begin{description}
	\item[``INTEL''] : Intel CPU (Pentium, Pentium II, etc).
	\item[``ALPHA''] : Digital Alpha CPU
	\item[``SGI''] : Silicon Graphics MIPS CPU
	\item[``SUN4u''] : Sun UltraSparc CPU
	\item[``SUN4x''] : A Sun Sparc CPU other than an UltraSparc, i.e.
sun4m or sun4c CPU found in older Sparc workstations such as the Sparc~10, 
Sparc~20, IPC, IPX, etc.
	\item[``HPPA1''] :  Hewlett Packard PA-RISC 1.x CPU (i.e. PA-RISC    
                      7000 series CPU) based workstation
	\item[``HPPA2''] :  Hewlett Packard PA-RISC 2.x CPU (i.e. PA-RISC    
                      8000 series CPU) based workstation
	\end{description}
%
\item[ClockDay] : The day of the week, where 0 = Sunday, 1 = Monday, \Dots, 6 = Saturday. 
%
\item[ClockMin] : The number of minutes passed since midnight.
%
\item[CondorLoadAvg] : The load average generated by Condor (either
from remote jobs or running benchmarks).
%
\item[ConsoleIdle] : The number of seconds since activity on the system
console keyboard or console mouse has last been detected.
%
\item[Cpus] : Number of CPUs in this machine, i.e. 1 = single CPU machine, 2 = dual
CPUs, etc.
%
\item[CurrentRank] : A float which represents this machine owner's affinity
for running the Condor job which it is currently hosting.  If not
currently hosting a Condor job, CurrentRank is -1.0.
%
\item[Disk] : The amount of disk space on this machine available for
the job in kbytes ( e.g. 23000 = 23 megabytes ).  Specifically, this
is the amount of disk space available in the directory specified in
the Condor configuration files by the \Macro{EXECUTE} macro, minus any
space reserved with the \Macro{RESERVED\_DISK} macro.
%
\item[EnteredCurrentActivity] : Time at which the machine entered the 
current Activity (see \AdAttr{Activity} entry above).  Measured in the
number of seconds since the epoch (00:00:00 UTC, Jan 1, 1970).
%
\item[FileSystemDomain] : a domain name configured by the Condor 
administrator which describes a cluster of machines which all access 
the same networked filesystems usually via NFS or AFS.  
%
\item[KeyboardIdle] : The number of seconds since activity on any
keyboard or mouse associated with this machine has last been detected.
Unlike \AdAttr{ConsoleIdle}, \AdAttr{KeyboardIdle} also takes activity 
on pseudo-terminals into
account (i.e. virtual ``keyboard'' activity from telnet and rlogin
sessions as well).  Note that \AdAttr{KeyboardIdle} will always be equal to or
less than \AdAttr{ConsoleIdle}.
%
\item[KFlops] : Relative floating point performance as determined via a
linpack benchmark.
%
\item[LastHeardForm] : Time when the Condor Central Manager last
received a status update from this machine.  
Expressed as seconds since the epoch (integer value).
Note: This attribute is only inserted by the Central Manager once it
receives the ClassAd.
It is not present in the startd's copy of the ClassAd.
Therefore, you couldn't use this attribute in defining startd
expressions (which you wouldn't want to, anyway).
%
\item[LoadAvg] : A floating point number with the machine's current load
average.
%
\item[Machine] : A string with the machine's fully qualified hostname.
%
\item[Memory] : The amount of RAM in megabytes.
%
\item[Mips] : Relative integer performance as determined via a dhrystone
benchmark.
%
\item[MyType] : The ClassAd type; always set to the literal string ``Machine''.
%
\item[Name] : The name of this resource; typically the same value as
the \AdAttr{Machine} attribute, but could be customized by the site
administrator.
On SMP machines, the startd will divide the CPUs up into seperate
virtual machines, each with with a unique name.
These names will be of the form ``vm\#@full.hostname'', for example,
``vm1@vulture.cs.wisc.edu'', which signifies virtual machine 1 from
vulture.cs.wisc.edu. 
%
\item[OpSys] : String describing the operating system running on this
machine.  For Condor \VersionNotice\ typically one of the following:
	\begin{description}
	\item ``HPUX10'' (for HPUX 10.20)
	\item ``IRIX6''  (for IRIX 6.2, 6.3, or 6.4)
	\item ``LINUX''  (for LINUX 2.0.x kernel systems)
	\item ``LINUX-GLIBC''  (for LINUX systems, using GNU's libc)
	\item ``OSF1''	 (for Digital Unix 4.x)
	\item ``SOLARIS251''
	\item ``SOLARIS26''
	\end{description}
%
\item[Requirements] : A boolean which, when evaluated within the context
of the Machine ClassAd and a Job ClassAd, must evaluate to
TRUE before Condor will allow the job to use this machine.
%
\item[StartdIpAddr] : String with the IP and port address of the
\Condor{startd} daemon which is publishing this Machine ClassAd.
%
\item[State] : String which publishes the machine's Condor state, which
can be:
	\begin{description}
	\item[``Owner''] : The machine owner is using the machine, and
it is unavailable to Condor.
	\item[``Unclaimed''] : The machine is available to run Condor jobs,
but a good match (i.e. job to run here) is either not available or not 
yet found.
	\item[``Matched''] : The Condor Central Manager has found a good
match for this resource, but a Condor scheduler has not yet claimed it.
	\item[``Claimed''] : The machine is claimed by a remote
\Condor{schedd} and is probably running a job.
	\item[``Preempting''] : A Condor job is being preempted (possibly
via checkpointing) in order to clear the machine for either a higher
priority job or because the machine owner wants the machine back.
	\end{description}   % of State
%
\item[TargetType] : Describes what type of ClassAd to match with.
Always set to the string literal ``Job'', because Machine ClassAds
always want to be matched with Jobs, and vice-versa.
%
\item[UidDomain] : a domain name configured by the Condor 
administrator which describes a cluster of machines which all have 
the same "passwd" file entries, and therefore all have the same logins.
%
\item[VirtualMemory] : The amount of currently available virtual memory 
(swap space) expressed in kbytes.

\end{description}



%%%%%%%%%%%% 
\subsection{Heterogeneous submit: submit to a different architecture} 
%%%%%%%%%%%%

There are times when you would like to submit jobs across machine
architectures. For instance, let's say you have an Intel machine running
LINUX sitting on your desk. This is the machine where you do all your
work and where all your files are stored. But perhaps the majority of
machines in your pool are Sun SPARC machines running Solaris. You would
want to submit jobs directly from your LINUX box that would run on the
SPARC machines.

This is easily accomplished.  You will need, or course, to create your
executable on the same type of machine where you want your job to run ---
Condor will not convert machine instructions hetrogeneously for you! The
trick is simply what to specify for your \Opt{requirements} command in
your submit-description file.  By default, \Condor{submit} inserts
requirements that will make your job run on the same type of machine you
are submitting from.  To override this, simply state what you want.
Returning to our example, you would put the following into your
submit-description file:
\begin{verbatim}
        requirements = Arch == "SUN4x" && OpSys == "SOLARIS251"
\end{verbatim}
Just run \Condor{status} to display the Arch and OpSys values for any/all 
machines in the pool.

%%%%%%%%%%%%
% \subsection{Remote Submit}  % it is in the man page, why bother?
%%%%%%%%%%%%



%%%%%%%%%%%%%%%%%%%%%%%%%%%%%%%%%%%%%%%%%%
\section{Managing a Condor Job}
This section provides a brief summary of what can be done once jobs
are submitted. The basic mechanisms for monitoring a job are
introduced, but the commands are discussed briefly.
You are encouraged to
look at the man pages of the commands referred to (located in
Chapter~\ref{command-reference} beginning on
page~\pageref{command-reference}) for more information. 

When jobs are submitted, Condor will attempt to find resources
to run the jobs. 
A list of all those with jobs submitted
may be obtained through \Condor{status}
\index{Condor commands!condor\_status}
with the 
\Arg{-submitters} option. 
An example of this would yield output similar to:
\begin{verbatim}
%  condor_status -submitters

Name                 Machine      Running IdleJobs HeldJobs

ballard@cs.wisc.edu  bluebird.c         0       11        0
nice-user.condor@cs. cardinal.c         6      504        0
wright@cs.wisc.edu   finch.cs.w         1        1        0
jbasney@cs.wisc.edu  perdita.cs         0        0        5

                           RunningJobs           IdleJobs           HeldJobs

 ballard@cs.wisc.edu                 0                 11                  0
 jbasney@cs.wisc.edu                 0                  0                  5
nice-user.condor@cs.                 6                504                  0
  wright@cs.wisc.edu                 1                  1                  0

               Total                 7                516                  5
\end{verbatim}

\subsection{Checking on the progress of jobs}
At any time, you can check on the status of your jobs with the \Condor{q}
command.
\index{Condor commands!condor\_q}
This command displays the status of all queued jobs.
An example of the output from \Condor{q} is
\begin{verbatim}
%  condor_q

-- Submitter: froth.cs.wisc.edu : <128.105.73.44:33847> : froth.cs.wisc.edu
 ID      OWNER            SUBMITTED    CPU_USAGE ST PRI SIZE CMD               
 125.0   jbasney         4/10 15:35   0+00:00:00 I  -10 1.2  hello.remote      
 127.0   raman           4/11 15:35   0+00:00:00 R  0   1.4  hello             
 128.0   raman           4/11 15:35   0+00:02:33 I  0   1.4  hello             

3 jobs; 2 idle, 1 running, 0 held

\end{verbatim} 
This output contains many columns of information about the
queued jobs.
\index{status!of queued jobs}
The \verb@ST@ column (for status) shows the status of
current jobs in the queue. An \verb@R@ in the status column
means the the job is currently running.
An \verb@I@ stands for idle. The job is not running right
now, because it is waiting for a machine to become available. 
The status
\verb@H@ is the hold state. In the hold state,
the job will not be scheduled to
run until it is released (see condor\_hold and condor\_release man pages).
Older versions of Condor used a
\verb@U@ in the status column to stand for unexpanded.
In this state,
a job has never 
checkpointed and when it starts running, it will start running from the
beginning.
Newer versions of Condor do not use the \verb@U@ state.

The \verb@CPU_USAGE@ time reported for a job is the time that has been
committed to the job.  It is not updated for a job until
the job checkpoints. At that time, the job has made guaranteed forward 
progress.  Depending upon how the site administrator configured the pool,
several hours may pass between checkpoints, so do not worry if you do
not observe the \verb@CPU_USAGE@ entry changing by the hour.
Also note that this is actual CPU
time as reported by the operating system; it is not time as
measured by a wall clock.

Another useful method of tracking the progress of jobs is through the
user log.  If you have specified a \AdAttr{log} command in 
your submit file, the progress of the job may be followed by viewing the
log file.  Various events such as execution commencement, checkpoint, eviction 
and termination are logged in the file.
Also logged is the time at which the event occurred.

% Karen's note:  degraded performance where?
When your job begins to run, Condor starts up a \Condor{shadow} process
\index{condor\_shadow}
\index{remote system call!condor\_shadow}
on the submit machine.  The shadow process is the mechanism by which the
remotely executing jobs can access the environment from which it was
submitted, such as input and output files.  

It is normal for a machine which has submitted hundreds of jobs to have 
hundreds of shadows running on the machine.  Since the text segments of 
all these processes is the same, the load on the submit machine is usually 
not significant.  If, however, you notice degraded performance, you can limit 
the number of jobs that can run simultaneously through the 
\Macro{MAX\_JOBS\_RUNNING} configuration parameter.  Please talk to your 
system administrator for the necessary configuration change.

You can also find all the machines that are running your job through the
\Condor{status} command.
\index{Condor commands!condor\_status}
For example, to find all the machines that are
running jobs submitted by ``breach@cs.wisc.edu,'' type:
\begin{verbatim}
%  condor_status -constraint 'RemoteUser == "breach@cs.wisc.edu"'

Name       Arch     OpSys        State      Activity   LoadAv Mem  ActvtyTime

alfred.cs. INTEL    SOLARIS251   Claimed    Busy       0.980  64    0+07:10:02
biron.cs.w INTEL    SOLARIS251   Claimed    Busy       1.000  128   0+01:10:00
cambridge. INTEL    SOLARIS251   Claimed    Busy       0.988  64    0+00:15:00
falcons.cs INTEL    SOLARIS251   Claimed    Busy       0.996  32    0+02:05:03
happy.cs.w INTEL    SOLARIS251   Claimed    Busy       0.988  128   0+03:05:00
istat03.st INTEL    SOLARIS251   Claimed    Busy       0.883  64    0+06:45:01
istat04.st INTEL    SOLARIS251   Claimed    Busy       0.988  64    0+00:10:00
istat09.st INTEL    SOLARIS251   Claimed    Busy       0.301  64    0+03:45:00
...
\end{verbatim}
To find all the machines that are running any job at all, type:
\begin{verbatim}
%  condor_status -run

Name       Arch     OpSys        LoadAv RemoteUser           ClientMachine  

adriana.cs INTEL    SOLARIS251   0.980  hepcon@cs.wisc.edu   chevre.cs.wisc.
alfred.cs. INTEL    SOLARIS251   0.980  breach@cs.wisc.edu   neufchatel.cs.w
amul.cs.wi SUN4u    SOLARIS251   1.000  nice-user.condor@cs. chevre.cs.wisc.
anfrom.cs. SUN4x    SOLARIS251   1.023  ashoks@jules.ncsa.ui jules.ncsa.uiuc
anthrax.cs INTEL    SOLARIS251   0.285  hepcon@cs.wisc.edu   chevre.cs.wisc.
astro.cs.w INTEL    SOLARIS251   1.000  nice-user.condor@cs. chevre.cs.wisc.
aura.cs.wi SUN4u    SOLARIS251   0.996  nice-user.condor@cs. chevre.cs.wisc.
balder.cs. INTEL    SOLARIS251   1.000  nice-user.condor@cs. chevre.cs.wisc.
bamba.cs.w INTEL    SOLARIS251   1.574  dmarino@cs.wisc.edu  riola.cs.wisc.e
bardolph.c INTEL    SOLARIS251   1.000  nice-user.condor@cs. chevre.cs.wisc.
...
\end{verbatim}

\subsection{Removing a job from the queue}
A job can be removed from the queue at any time by using the \Condor{rm}
\index{Condor commands!condor\_rm}
command.  If the job that is being removed is currently running, the job
is killed without a checkpoint, and its queue entry is removed.  
The following example shows the queue of jobs before and after
a job is removed.
\begin{verbatim}
%  condor_q

-- Submitter: froth.cs.wisc.edu : <128.105.73.44:33847> : froth.cs.wisc.edu
 ID      OWNER            SUBMITTED    CPU_USAGE ST PRI SIZE CMD               
 125.0   jbasney         4/10 15:35   0+00:00:00 I  -10 1.2  hello.remote      
 132.0   raman           4/11 16:57   0+00:00:00 R  0   1.4  hello             

2 jobs; 1 idle, 1 running, 0 held

%  condor_rm 132.0
Job 132.0 removed.

%  condor_q

-- Submitter: froth.cs.wisc.edu : <128.105.73.44:33847> : froth.cs.wisc.edu
 ID      OWNER            SUBMITTED    CPU_USAGE ST PRI SIZE CMD               
 125.0   jbasney         4/10 15:35   0+00:00:00 I  -10 1.2  hello.remote      

1 jobs; 1 idle, 0 running, 0 held
\end{verbatim}

%%%%%%%%%%%%%%%%%%%%%%%%%%%%%%%%%%%%%%%%%%%%%%%%%%%%%%%%%%%%%%%%%%%%%%
\subsection{\label{sec:job-prio}Changing the priority of jobs}
%%%%%%%%%%%%%%%%%%%%%%%%%%%%%%%%%%%%%%%%%%%%%%%%%%%%%%%%%%%%%%%%%%%%%%

\index{job!priority}
\index{priority!of a job}
In addition to the priorities assigned to each user, Condor also provides
each user with the capability of assigning priorities to each submitted job.
These job priorities are local to each queue and range from -20 to +20, with
higher values meaning better priority.

The default priority of a job is 0, but can be changed using the \Condor{prio}
command.
\index{Condor commands!condor\_prio}
For example, to change the priority of a job to -15,
\begin{verbatim}
%  condor_q raman

-- Submitter: froth.cs.wisc.edu : <128.105.73.44:33847> : froth.cs.wisc.edu
 ID      OWNER            SUBMITTED    CPU_USAGE ST PRI SIZE CMD               
 126.0   raman           4/11 15:06   0+00:00:00 I  0   0.3  hello             

1 jobs; 1 idle, 0 running, 0 held

%  condor_prio -p -15 126.0

%  condor_q raman

-- Submitter: froth.cs.wisc.edu : <128.105.73.44:33847> : froth.cs.wisc.edu
 ID      OWNER            SUBMITTED    CPU_USAGE ST PRI SIZE CMD               
 126.0   raman           4/11 15:06   0+00:00:00 I  -15 0.3  hello             

1 jobs; 1 idle, 0 running, 0 held
\end{verbatim}

It is important to note that these \emph{job} priorities are completely 
different from the \emph{user} priorities assigned by Condor.  Job priorities
do not impact user priorities.  They are only a mechanism for the user to
identify the relative importance of jobs among all the jobs submitted by the
user to that specific queue.

\subsection{Why does the job not run?}
\index{job!analysis}
\index{job!not running}
Users sometimes find that their jobs do not run.  There are several reasons why
a specific job does not run.  These reasons include failed job or machine
constraints, bias due to preferences, insufficient priority, and the preemption
throttle that is implemented by the \Condor{negotiator} to prevent
thrashing.  Many of these reasons can be diagnosed by using the \Arg{-analyze}
option of \Condor{q}.
\index{Condor commands!condor\_q}
For example, the following job submitted by user
jbasney was found to have not run for several days.
\begin{verbatim}
% condor_q

-- Submitter: froth.cs.wisc.edu : <128.105.73.44:33847> : froth.cs.wisc.edu
 ID      OWNER            SUBMITTED    CPU_USAGE ST PRI SIZE CMD               
 125.0   jbasney         4/10 15:35   0+00:00:00 I  -10 1.2  hello.remote      

1 jobs; 1 idle, 0 running, 0 held
\end{verbatim}

Running \Condor{q}'s analyzer provided the following information:

\begin{verbatim}
%  condor_q 125.0 -analyze

-- Submitter: froth.cs.wisc.edu : <128.105.73.44:33847> : froth.cs.wisc.edu
---
125.000:  Run analysis summary.  Of 323 resource offers,
          323 do not satisfy the request's constraints
            0 resource offer constraints are not satisfied by this request
            0 are serving equal or higher priority customers
            0 are serving more preferred customers
            0 cannot preempt because preemption has been held
            0 are available to service your request

WARNING:  Be advised:
   No resources matched request's constraints
   Check the Requirements expression below:

Requirements = Arch == "INTEL" && OpSys == "IRIX6" && 
  Disk >= ExecutableSize && VirtualMemory >= ImageSize
\end{verbatim}

%%%%%%%%%%%%%%%%%%%
%condor_status -total lists the Arch/OS combinations in our pool:
%
%                     Machines Owner Claimed Unclaimed Matched Preempting
%
%           SGI/IRIX6       14     3       0        11       0          0
%          ALPHA/OSF1        8     6       1         1       0          0
%     SUN4u/SOLARIS26       84    38      46         0       0          0
%    SUN4u/SOLARIS251        8     0       1         7       0          0
%     SUN4x/SOLARIS26      104    47      56         1       0          0
%    SUN4x/SOLARIS251        1     0       1         0       0          0
%     INTEL/SOLARIS26      214    63     144         7       0          0
%       INTEL/WINNT40        6     0       0         6       0          0
%
%               Total      439   157     249        33       0          0
%
%So, one example of a platform that does not exist would be:
%
% requirements = Arch == "INTEL" && OpSys == "IRIX6"
%
%%%%%%%%%%%%%%%%%%%

For this job,
the \Attr{Requirements}
\index{ClassAd attribute!requirements}
expression specifies a platform that does not exist.
Therefore, the expression always evaluates to false.

While the analyzer can diagnose most common problems, there are some situations
that it cannot reliably detect due to the instantaneous and local nature of the
information it uses to detect the problem.  Thus, it may be that the analyzer
reports that resources are available to service the request, but the job still 
does not run.  In most of these situations, the delay is transient, and the
job will run during the next negotiation cycle.

If the problem persists and the analyzer is unable to detect the situation, it
may be that the job begins to run but immediately terminates due to some 
problem.  Viewing the job's error and log files
(specified in the submit command file) and Condor's \Macro{SHADOW\_LOG} file
may assist in tracking down the problem.  If the cause is still unclear, please
contact your system administrator.

\subsection{\label{sec:job-completion}Job Completion}
\index{job!completion}

When your Condor job completes(either through normal means or abnormal
termination by signal), Condor will remove it from the job queue (i.e.,
it will no longer appear in the output of \Condor{q}) and insert it into
the job history file.  You can examine the job history file with the
\Condor{history} command. If you specified a log file in your submit
description file, then the job exit status will be recorded there as well.

By default, Condor will send you an email message
when your job completes.  You can modify this behavior with the
\Condor{submit} ``notification'' command.
The message will include the exit status of your job (i.e., the
argument your job passed to the exit system call when it completed) or
notification that your job was killed by a signal.  It will also
include the following statistics (as appropriate) about your job:

\begin{description}

\item[Submitted at:] when the job was submitted with \Condor{submit}

\item[Completed at:] when the job completed

\item[Real Time:] elapsed time between when the job was submitted and
when it completed (days hours:minutes:seconds)

\item[Run Time:] total time the job was running (i.e., real time minus
queueing time)

\item[Committed Time:] total run time that contributed to job
completion (i.e., run time minus the run time that was lost because
the job was evicted without performing a checkpoint)

\item[Remote User Time:] total amount of committed time the job spent
executing in user mode

\item[Remote System Time:] total amount of committed time the job spent
executing in system mode 

\item[Total Remote Time:] total committed CPU time for the job

\item[Local User Time:] total amount of time this job's
\Condor{shadow} (remote system call server) spent executing in user
mode

\item[Local System Time:] total amount of time this job's
\Condor{shadow} spent executing in system mode

\item[Total Local Time:] total CPU usage for this job's \Condor{shadow}

\item[Leveraging Factor:] the ratio of total remote time to total
system time (a factor below 1.0 indicates that the job ran
inefficiently, spending more CPU time performing remote system calls
than actually executing on the remote machine)

\item[Virtual Image Size:] memory size of the job, computed when the
job checkpoints

\item[Checkpoints written:] number of successful checkpoints performed
by the job

\item[Checkpoint restarts:] number of times the job successfully
restarted from a checkpoint

\item[Network:] total network usage by the job for checkpointing and
remote system calls

\item[Buffer Configuration:] configuration of remote system call I/O
buffers

\item[Total I/O:] total file I/O detected by the remote system call
library

\item[I/O by File:] I/O statistics per file produced by the remote
system call library

\item[Remote System Calls:] listing of all remote system calls
performed (both Condor-specific and Unix system calls) with a count of
the number of times each was performed

\end{description}

%%%%%%%%%%%%%%%%%%%%%%%%%%%%%%%%%%%%%%%%%%

%%%%%%%%%%%%%%%%%%%%%%%%%%%%%%%%%%%%%%%%
\section{Priorities in Condor}
%%%%%%%%%%%%%%%%%%%%%%%%%%%%%%%%%%%%%%%%

Condor has two independent priority controls: \Term{job}
priorities and \Term{user} priorities.  

\subsection{Job Priority}

Job priorities allow you to assign a priority level to each of your jobs in order to
control their order of execution.  To do this, use the \Condor{prio}
command --- see the example in section~\ref{sec:job-prio}, or the
command reference page on page~\pageref{man-condor-prio}.  Job
priorities, however, do not impact user priorities in any fasion.  No matter what you
set a job's priority to be, it will not alter your user priority in
relation to other users.  Job priorities range from -20 to +20, with +20
being the best and -20 the worst.  

%%%%%%%%%%%%%%%%%%%%%%%%%%%%%%%%%%%%%%%%%%%%%%%%%%%%%%%%%%%%%%%%%%%%%%
\subsection{\label{sec:user-priority-explained}User priority}
%%%%%%%%%%%%%%%%%%%%%%%%%%%%%%%%%%%%%%%%%%%%%%%%%%%%%%%%%%%%%%%%%%%%%%

Machines are allocated to users based upon that user's priority. User
priorities in Condor can be examined with the \Condor{userprio}
command (see page~\pageref{man-condor-userprio}),
and Condor administrators can set and edit individual user priorities
with the same utility. A lower numerical user priority value means
higher priority, so a user with priority 5 will get more resources than
a user with priority 50.  

Condor continuously calculates the share of available machines that each
user should be allocated.    This share is inversely related to the ratio
between user priorities; for example, a user with a priority of 10 will
get twice as many machines as a user with a priority of 20. The priority
of each individual user changes according to the number of resources he
is using. Each user starts out with a priority of .5 (the best
priority allowed).  If the number of machines a user currently has is greater than his
priority, the priority will numerically increase (worsen) over time, and if it is less then
the priority, the priority will numerically decrease (improve) over time. 
The long-term result is fair-share access across all users.  The speed
at which Condor adjusts the priorities is controlled via an exponential
half-life value (parameter \Macro{PRIORITY\_HALFLIFE} which can be adjusted
by the site administrator) which has a
default of one day.  So if a user with a user priority of 100 is
utilizing 100 machines and then deletes all his/her jobs, one day later that user's
priority will  50, two days later the priority will be 25, etc. 

Condor enforces that each user gets his/her fair share of machines
according to user priority both when allocating machines which become
available and by priority preemption of currently allocated machines.
For instance, if a low priority user is utilizing all available machines
and suddenly a higher priority user submits jobs, Condor will
imediately checkpoint and vacate jobs belonging to the lower priority
user. This will free up machines that Condor will then give over to the
higher priority user. Condor will not starve the lower priority user; it
will preempt only enough jobs so that the higher priority user's fair
share can be realized (based upon the ratio between user priorities). To
prevent thrashing of the system due to priority preemption, the Condor 
site administrator can define a \Macro{PREEMPTION\_HOLD} expression in Condor's configuration.
The default expression that ships with Condor is configured to only preempt 
lower priority jobs that have run
for at least one hour. So in the previous example, in the worse case it
could take up to a maximum of one hour until the higher priority user
receives his fair share of machines. 

User priorities are keyed on ``username@domain'', for example
``johndoe@cs.wisc.edu''. (The domainname to use, if any, is also configured by
the Condor site administrator).  Thus, user priority and therefore resource
allocation is not impacted by which machine the user submits from or
even if the user submits jobs from multiple machines.

Finally, any job submitted to Condor can be specified as a ``nice'' job at 
the time the job is submitted (see page~\pageref{man-condor-submit-nice}).
Nice jobs will artificially have their numerical priority boosted by
over one million. This effectively means that nice jobs will only run on
machines that no other Condor job (i.e. non-niced job) wants. Similarly,
the Condor administrators could set the numerical priority of any
individual Condor user, such as a guest account, so that these guest
accounts would only use cycles not wanted by other users of the system.

%%%%%%%%%%%%%%%%%%%%%%%%%%%%%%%%%%%%%%%%%%%%%%%%%%%%%%%%%%%%%%%%%%%%%%

\newcommand{\func}[1]{\texttt{#1}}

Condor has a PVM submit Universe which allows the user to submit PVM jobs to
the Condor pool.  In this section, we will first
discuss the differences between running under normal PVM and running PVM under the Condor
environment.  Then we give some hints on how to write good PVM
programs to suit the Condor environment via an example program.  In the
end, we illustrate how to submit PVM jobs to Condor by examining a
sample Condor submit-description file which submits a PVM job.

Note that Condor-PVM is an optional Condor module.  To check and see if
it has been installed at your site, enter the following command:
\begin{verbatim}
        ls -l `condor_config_val PVMD`
\end{verbatim}
(notice the use of backticks in the above command).  If this shows the
file ``condor\_pvmd'' on your system, Condor-PVM is installed.  If not,
ask your site administrator to download Condor-PVM from
\Url{http://www.cs.wisc.edu/condor/condor-pvm} and install it.

\subsection{What does Condor-PVM do?}

Condor-PVM provides a framework to run parallel applications written to
PVM in Condor's opportunistic environment.  This means that you no
longer need a
set of dedicated machines to run PVM applications; Condor can be used to dynamically 
construct PVM virtual machines out of non-dedicated desktop machines on your network
which would have otherwise been idle.   In Condor-PVM, Condor acts as the
resource manager for the PVM daemon.  Whenever your PVM program asks
for nodes (machines), the request is re-mapped to Condor.  Condor then
finds a machine in the Condor pool via the usual mechanisms, and adds it
to the PVM virtual machine.  If a machine needs to leave the pool, your
PVM program is notified of that as well via the normal PVM mechanisms.

\subsection{The Master-Worker Paradigm}

There are several different parallel programming paradigms.  One of the
more common is the \Term{master-worker} (or \Term{pool of tasks})
arrangement.  In a master-worker program model, one node acts as the
controlling master for the parallel application and sends pieces work out to worker nodes.  The
worker node does some computation, and sends the result back to the
master node.  The master has a pool of work that needs to be
done, and simply assigns the next piece of work out to the next worker
that becomes available.  

Not all parallel programming paradigms lend themselves to an
opportunistic environment. In such an environment, any of the nodes
could be preempted and therefore disappear at any moment. The
master-worker model, on the other hand, is a model that can work well.
The idea is the master needs to keep track of which piece of work it
sends to each worker. If the master node is then informed that a worker
has disappeared, it puts the piece of work it assigned to that worker
back into the pool of tasks, and sends it out again to the next
available worker. If the master notices that the number of workers has
dropped below an acceptable level, it could request for more workers
(via \func{pvm\_addhosts()}). Or perhaps perhaps the master will request
a replacement node every single time it is notified that a worker has
gone away. The point is that in this paradigm, the number of workers is
not important (although more is better!) and changes in the size of
the virtual machine can be handled naturally.

Condor-PVM is designed to run PVM applications which follow the
master-worker paradigm.  Condor runs the master application on the
machine where the job was submitted and will not preempt it.  Workers
are pulled in from the Condor pool as they become available.

\subsection{Binary Compatibility}

Condor-PVM does not define a new API (application program interface);
programs can simply use the existing resource management PVM calls such
as \func{pvm\_addhosts()} and \func{pvm\_notify()}.  Because of this, some
master-worker PVM applications are ready to run under Condor-PVM with no
changes at all.  Regardless of using Condor-PVM or not, it is good
master-worker design to handle the case of a worker node disappearing,
and therefore many programmers have already constructed their master program
with all the necessary logic for fault-tolerance purposes.  

In fact, regular PVM and Condor-PVM are \underline{binary compatible}
with each other.  The same binary which runs under regular PVM will run
under Condor, and vice-versa.  There is no need to re-link for Condor-PVM.
This permits easy application development
(develop your PVM application interactively with the regular PVM console, XPVM,
etc) as well as binary sharing between Condor and some dedicated MPP systems.

\subsection{Runtime differences between Condor-PVM and regular PVM}

This release of the Condor-PVM is based on PVM 3.3.11.  The vast majority of the PVM
library functions under Condor maintain the same semantics as in
PVM 3.3.11, including messaging operations, group operations, and 
pvm\_catchout().

We summarize the changes and new features of PVM under running in the
Condor environment in the following list:

\begin{itemize}

\item Concept of machine class.  Under Condor-PVM, machines of
  different architectures attributes belong to different machine classes.  Machine
  classes are are numbered 0, 1, \Dots, etc.  A machine class can be
  specified by the user in the submit-description file when the job
  is submitted to Condor.

\item \func{pvm\_addhosts()}.  When the application
  needs to add a host machine, it should call \func{pvm\_addhosts()}
  with the first argument as a string that specifies the machine
  class.  For example, to specify class 0, a pointer to string ``0''
  should be used as the first argument.  Condor will find a machine
  that satisfies the requirements of class 0 and adds it to the PVM
  virtual machine.

  Furthermore, \func{pvm\_addhosts()} no longer blocks under Condor.  It
  will return immediately, before the hosts are actually added to the virtual
  machine.  After all, in a non-dedicated environment the amount of time it takes until
  a machine becomes available is not bound. The user should simply call 
  \func{pvm\_notify()} before calling
  \func{pvm\_addhosts()}, so that when a host is added later, the user
  will be notified via Condor in the usual PVM 
  fashion (via a PvmHostAdd notification message).
    
\item \func{pvm\_notify()}.  Under Condor, we added two additional 
  possible notification requests, \func{PvmHostSuspend} and
  \func{PvmHostResume}, to the function \func{pvm\_notify()}.  When a
  host is suspended (or resumed) by Condor, if the user has called
  \func{pvm\_notify()} with that host tid and with
  \func{PvmHostSuspend} (or \func{PvmHostResume}) as arguments, then
  the application will receive a notification for the corresponding
  event.  Note that after you receive one of these notifications, the notify
  is 'consumed' for that pid.  So if you set up a \func{PvmHostSuspend} 
  notification request for tid 4 and you get a \func{PvmHostSuspend}
  message for tid 4, you won't get a \func{PvmHostSuspend} message for
  that tid again unless you set up another notification request.

  The easiest way to handle this is the following:  When a worker
  node starts up, set up a notification for \func{PvmHostSuspend} on
  its tid.  When that node gets suspended, set up a \func{PvmHostResume}
  notification.  When it resumes, set up a \func{PvmHostSuspend}
  notification.


\item \func{pvm\_spawn()}.  If  the flag in \func{pvm\_spawn()} is 
  PvmTaskArch, then the string specifying the desired architecture
  class should be used.  Typically, if you are using only one class of
  machine in your virtual machine, specify ``0'' as the desired architecture.

  Furthermore, under Condor we currently only allow one
  PVM task to be spawned per node, since Condor's typical setup at most 
  sites will suspend or vacate
  a job if the load on its machine is higher than a specified
  threshold.

  A good fault-tolerant program will, of course, be able to deal with
  \func{pvm\_spawn()} failing.  This happens more often in opportunistic 
  environments like condor than in dedicated ones.

\end{itemize}

\subsection{A Sample PVM program for Condor-PVM}

Normal PVM applications assume dedicated machines.  However, when running a
PVM application under Condor, since Condor's environment is an
opportunistic environment, machines can be suspended and even removed
from the PVM virtual machine during the life-time of the PVM
application.  

Here, we include an extensively commented skeleton of a sample PVM
program \Prog{master\_sum.c}, which, we hope, will help you to
write PVM code that is better suited for a non-dedicated opportunistic
environment like Condor.

\CondorVerySmall
\begin{verbatim}
/* 
 * master_sum.c
 *
 * master program to perform parallel addition - takes a number n 
 * as input and returns the result of the sum 0..(n-1).  Addition 
 * is performed in parallel by k tasks, where k is also taken as 
 * input.  The numbers 0..(n-1) are stored in an array, and each 
 * worker adds a portion of the array, and returns the sum to the 
 * master.  The Master adds these sums and prints final sum.  
 *
 * To make the program fault-tolerant, the master has to monitor 
 * the tasks that exited without sending the result back.  The 
 * master creates some new tasks to do the work of those tasks 
 * that have exited. 
 */

#define NOTIFY_NUM 5  /* number of items to notify */

#define HOSTDELETE 12
#define HOSTSUSPEND 13
#define HOSTRESUME 14
#define TASKEXIT 15
#define HOSTADD 16
    
/* send the pertask and start number to the worker task i */
void send_data_to_worker(int i, int *tid, int *num, int pertask, 
            FILE *fp, int round)
{
     int status;
     int start_val;
     
     /* send the round number */
     pvm_initsend(PvmDataDefault); /* XDR format */
     pvm_pkint(&round, 1, 1);    /* number of numbers to add */
     status = pvm_send(tid[i], ROUND_TAG);

     pvm_initsend(PvmDataDefault); /* XDR format */
     pvm_pkint(&pertask, 1, 1);    /* number of numbers to add */
     status = pvm_send(tid[i], NUM_NUM_TAG);

     pvm_initsend(PvmDataDefault); /* XDR format */
     start_val = i * pertask; /* initial number for this task */
     pvm_pkint(&start_val, 1, 1);     /* the initial number */
     status = pvm_send(tid[i], START_NUM_TAG);   

     fprintf(fp, "Round %d: Send data %d to worker task %d, ``
           ``tid =%x. status %d \n", round, start_val, i, tid[i], status);
}

/* 
 * to see if more hosts are needed 
 * 1 = yes; 0 = no 
 */
int need_more_hosts(int i)
{
     int nhost, narch;
     char *hosts="0";  /* any host in arch class 0 */
     struct pvmhostinfo *hostp = (struct pvmhostinfo *) 
                     calloc (1, sizeof(struct pvmhostinfo));

     /* get the current configuration */
     pvm_config(&nhost, &narch, &hostp);
     
     if (nhost > i)
        return 0;
     else 
        return 1;
}

/* 
 * Add a new host until success, assuming that request for 
 * PvmAddHost notification has already been sent 
 */
void add_a_host(FILE *fp)
{
     int done = 0;
     int buf_id;
     int success = 0;
     int tid;
     int msg_len, msg_tag, msg_src;
     char *hosts="0";  /* any host in arch class 0 */
     int infos[1];

     while (done != 1) {
        /* 
        * add one host - no specific machine named 
        * add host will asynchronously, so we need
        * to receive the notification before go on.
        */
        pvm_addhosts(&hosts,1 , infos);
      
        /* receive hostadd notification from anyone */
        buf_id = pvm_recv(-1, HOSTADD);
      
        if (buf_id < 0) {
            fprintf(fp, "Error with buf_id = %d\n", buf_id);
            done = 0;
            continue;
        }
        done = 1;
     
        pvm_bufinfo(buf_id, &msg_len , &msg_tag, &msg_src);
        pvm_upkint(&tid, 1, 1);

        pvm_notify(PvmHostDelete, HOSTDELETE, 1, &tid);

        fprintf(fp, "Received HOSTADD: ");
        fprintf(fp, "Host %x added from %x\n", tid, msg_src);
        fflush(fp);
    }
}

/* 
 * Spawn a worker task until success.  
 * Return its tid, and the tid of its host. 
 */
void spawn_a_worker(int i, int* tid, int * host_tid, FILE *fp)
{
     int numt = 0;
     int status;

     while (numt == 0){
          /* spawn a worker on a host belonging to arch class 0 */
          numt = pvm_spawn ("worker_sum", NULL, PvmTaskArch, "0", 1, &tid[i]);

          fprintf(fp, "master spawned %d task tid[%d] = %x\n",numt,i,tid[i]);
          fflush(fp);
         
          /* if the spawn is successful */
          if (numt == 1) {
               /* notify when the task exits */
               status = pvm_notify(PvmTaskExit, TASKEXIT, 1, &tid[i]);
               
               fprintf(fp, "Notify status for exit = %d\n", status);
               
               if (pvm_pstat(tid[i]) != PvmOk) numt = 0;
          }
          
          if (numt != 1) {
               fprintf(fp, "!! Failed to spawn task[%d]\n", i);
               
               /* 
                * currently Condor-pvm allows only one task running on 
                * a host
                */
                while (need_more_hosts(i) == 1)
                    add_a_host(fp);
          }
     }
}


main()
{
    int n;                  /* will add <n> numbers n .. n-1 */
    int ntasks;             /* need <ntask> workers to do the addition. */
    int pertask;            /* numbers to add per task */
    int tid[MAX_TASKS];     /* tids of tasks */ 
    int deltid[MAX_TASKS];  /* tids monitored for deletion */
    int sum[MAX_TASKS];     /* hold the reported sum */
    int num[MAX_TASKS];     /* the initial numbers the workers should add */
    int host_tid[MAX_TASKS];/* the tids of the host that the *
                             * tasks <0..ntasks> are running on*/
    
    int i, numt, nhost, narch, status;
    int result;
    int mytid;    /* task id of master */
    int mypid;    /* process id of master */
    int buf_id;   /* id of recv buffer */
    int msg_leg, msg_tag, msg_src, msg_len;
    int int_val;  

    int infos[MAX_TASKS];
    char * hosts[MAX_TASKS];
    struct pvmhostinfo *hostp = (struct pvmhostinfo *) 
                    calloc (MAX_TASKS, sizeof(struct pvmhostinfo));

    FILE *fp;
    char outfile_name[100];

    char *codes[NOTIFY_NUM] = {"HostDelete", "HostSuspend", 
            "HostResume", "TaskExit", "HostAdd"};
    
    int count;   /* the number of times that while loops */
    int round_val;
    int correct = 0;
    int wrong = 0;

    mypid = getpid();

    sprintf(outfile_name, "out_sum.%d", mypid);
    fp = fopen(outfile_name, "w"); 

    /* redirect all children tasks' stdout to fp */
    pvm_catchout(stderr);  

    if (pvm_parent() == PvmNoParent){
        fprintf(fp, "I have no parent!\n");
        fflush(fp);
    }

    /* will add <n> numbers 0..(n-1) */
    fprintf(fp, "How many numbers? ");
    fflush(fp);
    scanf("%d", &n);
    fprintf(fp, "%d\n", n);
    fflush(fp);

    /* will spawn ntasks workers to perform addition */
    fprintf(fp, "How many tasks? ");
    fflush(fp);
    scanf("%d", &ntasks);
    fprintf(fp, "%d\n\n", ntasks);
    fflush(fp);

    /* will iterate count loops */
    fprintf(fp, "How many loops? ");
    fflush(fp);
    scanf("%d", &count);
    fprintf(fp, "%d\n", count);
    fflush(fp);

    /* set the hosts to be in arch class 0 */
    for (i = 0; i< ntasks; i++) hosts[i] = "0";

    /* numbers to be added by each worker */
    pertask = n/ntasks;

    /* get the master's TID */
    mytid = pvm_mytid();
    fprintf(fp, "mytid = %x; mypid = %d\n", mytid, mypid);

    /* get the current configuration */
    pvm_config(&nhost, &narch, &hostp);

    fprintf(fp, "current number of hosts = %d\n", nhost);
    fflush(fp);

    /* 
     * notify request for host addition, with tag HOSTADD, 
     * no tids to monitor.  
     *
     * -1 turns the notification request on;
     * 0 turns it off;
     * a positive integer n will generate at most n 
     * notifications.
     */     
    pvm_notify(PvmHostAdd, HOSTADD, -1, NULL);

    /* add more hosts - no specific machine named */
    i = ntasks - nhost;
    if (i > 0) {
        status = pvm_addhosts(hosts, i , infos);
      
        fprintf(fp, "master: addhost status = %d\n", status);
        fflush(fp);
    }
     
    /* if not enough hosts, loop and call pvm_addhosts */
    for (i = nhost; i < ntasks; i++) {
        /* receive notification from anyone, with HostAdd tag */
        buf_id = pvm_recv(-1, HOSTADD);

        if (buf_id < 0) {
           fprintf(fp, "Error with buf_id = %d\n", buf_id);
        } else {
           fprintf(fp, "Success with buf_id = %d\n", buf_id);
        }

        pvm_bufinfo(buf_id, &msg_len , &msg_tag, &msg_src);
        if (msg_tag==HOSTADD) {
            pvm_upkint(&int_val, 1, 1);

            fprintf(fp, "Received HOSTADD: ");
            fprintf(fp, "Host %x added from %x\n", int_val, msg_src);
           fflush(fp);
        } else {
           fprintf(fp, "Received unexpected message with tag: %d\n", msg_tag);
        }
    }

    /* get current configuration */
    pvm_config(&nhost, &narch, &hostp);

    /* notify all exceptional conditions about the hosts*/
    status = pvm_notify(PvmHostDelete, HOSTDELETE, ntasks, deltid);
    fprintf(fp, "Notify status for delete = %d\n", status);
     
    status = pvm_notify(PvmHostSuspend, HOSTSUSPEND, ntasks, deltid);
    fprintf(fp, "Notify status for suspend = %d\n", status);
     
    status = pvm_notify(PvmHostResume, HOSTRESUME, ntasks, deltid);
    fprintf(fp, "Notify status for resume = %d\n", status);

    /* spawn <ntasks> */
    for (i = 0; i < ntasks ; i++) {
        /* spawn the i-th task, with notifications. */
        spawn_a_worker(i, tid, host_tid, fp);
    }

    /* add the result <count> times */
    while (count > 0) {
        /* 
         * if array length was not perfectly divisible by ntasks, 
         *    some numbers are remaining. Add these yourself 
         */
        result = 0;
        for ( i = ntasks * pertask ; i < n ; i++)
           result += i;
     
        /* initialize the sum array with -1 */
        for (i = 0; i< ntasks; i++) 
            sum[i] = -1;
 
        /* send array partitions to each task */
        for (i = 0; i < ntasks ; i++) {
           send_data_to_worker(i, tid, num, pertask, fp, count);
        }

        /* 
        * Wait for results.  If a task exited without 
        * sending back the result, start another task to do
        * its job. 
        */
        for (i = 0; i< ntasks; ) {   
            buf_id = pvm_recv(-1, -1);
            pvm_bufinfo(buf_id, &msg_len , &msg_tag, &msg_src);
            fprintf(fp, "Receive: task %x returns mesg tag %d, ``
                ``buf_id = %d\n", msg_src, msg_tag, buf_id);
            fflush(fp);
           
            /* is a result returned by a worker */
            if(msg_tag == RESULT_TAG)  {
                int j;
                
                pvm_upkint(&round_val, 1, 1);
                fprintf(fp, "  round_val = %d\n", round_val);
                fflush(fp);
             
                if (round_val != count) continue;

                pvm_upkint(&int_val, 1, 1);
                for (j=0; (j<ntasks) && (tid[j] != msg_src); j++)
                    ;
                fprintf(fp, "  Data from task %d, tid = %x : %d\n", 
                    j, msg_src, int_val);
                fflush(fp);
                
                if (sum[j] == -1) {
                    sum[j] = int_val; /* store the sum */
                    i++;
                }
           } else if (msg_tag == TASKEXIT) {
                /* A task has exited. */
                /* Find out which task has exited. */ 
                int which_tid, j;          
                pvm_upkint(&which_tid, 1, 1);
                for (j=0; (j<ntasks) && (tid[j] != which_tid); j++)
                 ;
                fprintf(fp, "  from tid %x : task %d, tid =  %x, ``
                    ``exited.\n", 
                    msg_src, j, which_tid);
                fflush(fp);
                /* 
                 * If a task exited before sending back the message,
                 * create another task to do the same job.
                 */
                if (j < ntasks && sum[j] == -1) {
                     /* spawn the j-th task */
                     spawn_a_worker(j, tid, host_tid, fp);
                     
                     /* send unfinished work to the new task */
                     send_data_to_worker(j, tid, num, pertask, fp, count);
                }
            } else if (msg_tag == HOSTDELETE) {
                /* 
                * If a host has been deleted, check to see if 
                * the tasks running on it has been finished.  
                * If not, should create  new worker tasks to do 
                * the work on some other  hosts.
                */
                int which_tid, j;
                    
                /* get which host has been suspended/deleted */
                pvm_upkint(&which_tid, 1, 1);
                    
                fprintf(fp, "  from tid %x : %x %s\n", msg_src, which_tid, 
                    codes[msg_tag - HOSTDELETE]);
                fflush(fp);
                    
                /* 
                 * If the task on that host has not finished its
                 * work, then create new task to do the work.
                 */
                for (j = 0; j < ntasks; j++) {
                     if (host_tid[j] == which_tid && sum[j] == -1) {
                          fprintf(fp, "host_tid[%d] = %x, ``
                            ``need new task\n",
                              j, host_tid[j]);
                          fflush(fp);
                          
                          /* spawn the i-th task, with notifications. */
                          spawn_a_worker(j, tid, host_tid, fp);
                          
                          /* send the unfinished work to the new task */
                          send_data_to_worker(j,tid,num,pertask,fp,count);
                     }
                }
            } else {
                /* print out some other notifications or messages */
                int which_tid;
                pvm_upkint(&which_tid, 1, 1);
        
                fprintf(fp, "  from tid %x : %x %s\n", msg_src,
                        which_tid,   codes[msg_tag - HOSTDELETE]);
                fflush(fp);
            }
        }        
      
        /* add up the sum */
        for (i=0; i<ntasks; i++)
           result += sum[i];
          
        fprintf(fp, "Sum from  0 to %d is %d\n", n-1 , result);
        fflush(fp);
          
        /* check correctness */
        if (result == (n-1)*n/2) {
           correct++;
           fprintf(fp, "*** Result Correct! ***\n");
        } else {
           wrong++;
           fprintf(fp, "*** Result WRONG! ***\n");
        }

        fflush(fp);
        count--;
    }
     
     fprintf(fp, "correct = %d; wrong = %d\n", correct, wrong);
     fflush(fp);

     pvm_exit();
     exit(0);
}

\end{verbatim}
\normalsize

There is also a larger and far more robust application available
at \Url{http://www.cs.wisc.edu/condor/condor-pvm/}.

\subsection{Sample PVM submit file}
\label{submit}

Like submitting jobs in any other universe,
to submit a PVM job, the user needs to specify the requirements and
options in the submit-desciption file and run \Condor{submit}.  Figure~\ref{pvm_submit} on
page~\pageref{pvm_submit} is an example of
a submit-description file for a PVM job.  This job has a master PVM program called
\func{master\_pvm}.

\begin{figure}[hbt]
\CondorSmall
\begin{verbatim}
###########################################################
# sample_submit
# Sample submit file for PVM jobs. 
###########################################################

# The job is a PVM universe job.
universe = PVM  

# The executable of the master PVM program is ``master_pvm''.
executable = master_pvm

In = "in_sum"
Out = "stdout_sum"
Err = "err_sum"

###################  Architecture class 0  ##################

Requirements = (Arch == "INTEL") && (OpSys == "SOLARIS251") 

# We want at least 2 machines in class 0 before starting the 
# program.  We can use up to 4 machines.
machine_count = 2..4  
queue

###################  Architecture class 1  ##################

Requirements = (Arch == "SUN4x") && (OpSys == "SOLARIS251") 

# We need at least 1 machine in class 1 before starting the 
# executable.  We can use up to 3 to start with.
machine_count = 1..3
queue

###############################################################
# note: the program will not be started until the least 
#       requirements in all classes are satisfied.
###############################################################
\end{verbatim}
\normalsize

\label{pvm_submit}
\caption{A sample submit file for PVM jobs.}
\end{figure}

In this sample submit file, the command \func{universe = PVM}
specifies that the jobs should be submitted into PVM universe.

The command \func{executable = master\_pvm} tells Condor that the PVM
master program is \func{master\_sum}.  This program will be started on
the submitting machine.  The workers should be spawned by this master
program during execution.

This submit file also tells Condor that the PVM virtual machine is
consisted of two different classes of machine architectures.  Class
0 contains machines with INTEL architecture running SOLARIS251; class
1 contains machines with SUN4x (SPARC) architecture running SOLARIS251.

By using \func{machine\_count = <min>..<max>}, the submit file tells
Condor that before the PVM program, there should be at least \verb@<min>@
number of machines of the current class.  It also asks Condor to give
it as many as \verb@<max>@ machines.  During the execution of the program,
the application can get more machines of each of the class by calling
\func{pvm\_addhosts()} with a string specifying the desired architecture
class.  (See the sample program in this section for details.)

The \func{queue} command should be inserted after the specifications of
each class.







%%%%%%%%%%%%%%%%%%%%%%%%%%%%%%%%%%%%%%%%%%%%%%%%%%%%%%%%%%%%%%%%%%%%%%

%%%%%%%%%%%%%%%%%%%%%%%%%%%%%%%%%%%%%%%%%%%%%%%%%%%%%%%%%%%%%%%%%%%%%%
\section{Interjob Dependencies: DAGMan Meta-Scheduler}
\label{sec:DAGMan}

The Directed Acyclic Graph Manager (DAGMan) is a meta-scheduler for Condor
jobs.  DAGMan is responsible for submitting batch jobs in a predefined order
and processing the results. A configuration file is defined prior to execution
of DAGMan in which the jobs, their \textit{CondorConfigFile}, and job
dependencies are declared.

The importance of such a tool lies in the fact that the user is able to define
the execution order of a number of Condor Jobs. Just as Condor schedules
condor jobs, DAGMan schedules a system of jobs. In essence, it defines a
problem. Solving a problem may require multiple condor jobs that need data
from each other. This is best represented using a Directed Acyclic Graph
(DAG), which represents the flow of control from one node to another (i.e.,
from one condor job to another) through arrows.

From the point of view of the user, the scheduler is initialized with the
order of execution of jobs, and then started. DAGMan is responsible for all
scheduling, recovery and reporting activities of the submitted system of jobs.

The following sections explain the use of DAGMan in full detail.  However, if
the user only wants the bare essentials, please read
section~\ref{dagman:essentials} to get started more quickly.

\subsection{DAG Input File}

For Unix users, a useful analogy might be to think of the DAGMan input file as
a makefile, and DAGMan itself as the make executable.  However, DAGMan differs
from make.  Instead of looking at file modification timestamps, DAGMan reads
the Condor log file generated by each Condor job to find out which jobs are
unsubmitted, submitted, or complete.  DAGMan also makes a guarantee that a DAG
is recoverable, even if the machine running DAGMan goes down during execution.

\subsubsection{Description}
\label{dagman:dagdesc}

Job dependencies are defined prior to execution of the DAGMan program, using a
DAG input file.  An example input configuration file name is \File{diamond.dag}.
The input file is read completely, and the DAG data structure is constructed
in memory before the first job is submitted.  With the exception of the
\textit{CondorCommandFile} (see below), the input file is case insensitive.

Throughout the input file, comments can be placed.  Legal comments exist on a
single line which immediately starts with a `\texttt{\#}' character, followed
by any characters up to the newline `\texttt{$\backslash$n}'.

It is interesting to note that the DAGMan input file does not contain any
specifics about the individual jobs. Each condor job by itself is handled as
if DAGMan was not present (this includes compiling and linking of the
job). The executable and the input/output parameters for each job are
contained in the CondorCommandFile.  The DAG file merely describes the
relationship between the different condor jobs using the semantics just
described.

\begin{description}

\item[Signature]

The first line of a DAG input file is the signature, which precisely
identifies which DAG file format follows.  As of this writing, only one DAG
format exists, and thus only one signature is possible.

\begin{verbatim}
  ### DAGMan 6.1.0
\end{verbatim}

This line must appear as it is written here, character for character.
Anything different will be rejected by DAGMan.  Having a precise signature tag
will enable future versions of DAGMan to remain backward compatible with older
DAG input file formats.

\item[Job Section]

The Job Section of the input DAG file declares all the jobs that will appear
in the DAG.  Each job is described by a single line called a Job Entry.  The
following syntax is used:

\begin{verbatim}
	JOB <JobName> <CondorCommandFile>
\end{verbatim}

The \texttt{JOB} keyword (shown here in upper case only for clarity) declares
this line will map a \textit{JobName} to a Condor Command File.  The
\textit{JobName} is used by DAGMan to uniquely identify jobs throughout the
input file and to name them in output messages.  The
\textit{CondorCommandFile} is the input file used by \Condor{submit} to run
the individual condor job.  Because the Unix file system is case sensitive,
the case of the \textit{CondorCommandFile} is preserved.

The JobName can be any string that contains no white space.  The JobName is
not case sensitive, so ``JobA'' is equivalent to ``joba''.  An example
\textit{CondorCommandFile} name is \File{a.condor}.  Some important
restrictions are placed on the contents of the \textit{CondorCommandFile},
which will be discussed later.

The user can also have the option of declaring a job as being already
completed in the DAG input file. This may be useful in situations where the
user wishes to verify results, but does not need the entire job dependency
graph to be executed. This is done by adding the word "DONE" to the end of the
Job declaration line.

\begin{verbatim}
	JOB <JobName> <CondorCommandFile> DONE
\end{verbatim}

\item[Dependency Section]

The dependency section of the DAG input file follows the Job Section and
describes the dependencies between the jobs listed in the Job Section.  The
notion of a ``parent'' and ``child'' job is introduced here.  A parent job
produces output which is required by one or more child jobs.  None of the
children can run until the parent successfully terminates.  A child job is one
whose input is taken from one or more parent jobs.  The child job cannot run
until all of its parents have successfully terminated.

A single line in the input file can specify the dependencies from one or more
parents to one or more children.

\begin{verbatim}
	PARENT <ParentJobName>* CHILD <ChildJobName>*
\end{verbatim}

The \texttt{PARENT} keyword is followed by one or more
\textit{ParentJobName}s.  Those are followed by the \texttt{CHILD} keyword,
which is followed by one or more \textit{ChildJobName}s.  Each child job
depends on each and every parent job on this line.  So the line
``\texttt{PARENT p1 p2 CHILD c1 c2}'' would produce four dependencies.

\end{description}

\subsubsection{Example}

The following \File{diamond.dag} DAG input file shown below is illustrated in
Figure~\ref{fig:dagman-diamond}.

\begin{verbatim}
  ### DAGMan 6.1.0
  # Filename: diamond.dag
  #
  Job  A  A.condor 
  Job  B  B.condor 
  Job  C  C.condor	
  Job  D  D.condor
  PARENT A CHILD B C
  PARENT B C CHILD D
\end{verbatim}

\begin{figure}[hbt]
\centering
\includegraphics{user-man/dagman-diamond.eps}
\caption{\label{fig:dagman-diamond}Diamond DAG}
\end{figure}

With \File{diamond.dag}, job A must execute first, because all other jobs
directly or indirectly depend on it.  After job A successfully completes, both
job B and C are eligible to run.  In fact, they will be submitted at the same
time and hopefully Condor will find two remote hosts that can run them in
parallel.  Since job D depends on both B and C, it must wait for both to
complete successfully before it can be submitted.

\subsection{Execution}

\subsubsection{Preparing Jobs}
\label{dagman:prepjob}

Each individual job in a DAG is free to be a unique executable, with a unique
\textit{CondorCommandFile}.  The DAG can contain a mixture of standard and
vanilla jobs, or even other meta-scheduler jobs, like DAGMan.  On the other
hand, the jobs in the DAG could all use the same executable, or even the same
\textit{CondorCommandFile}.  Anything between both extremes is possible.
However, two limits are imposed.

First, each \textit{CondorCommandFile} must submit a cluster of size one.
There cannot be multiple \texttt{queue} lines.  The reasoning is long winded,
so a brief summary will be attempted.  If multi-job clusters were allowed,
DAGMan would have to parse the \textit{CondorCommandFile} to find out how many
jobs belong to that cluster.  Otherwise, DAGMan would not know for sure if a
cluster had terminated based on seeing the event from one job of that
cluster.  This restriction may be lifted in future DAGMan version, depending
on the design and implementation issues.

Second, all \textit{CondorCommandFile}s of a DAG must specify the same log.
In order for DAGMan to follow the order of events correctly, all events from
all jobs in the DAG must be sent to the same log file.  This restriction will
be loosened in later versions (see section~\ref{dagman:version}).

For this example, we will write a single \textit{CondorCommandFile} to be used
by all three jobs in the DAG.  Thus, each job will run the same executable.
This example is very artificial, because normally separate jobs would need
output for their child jobs to go to unique output and error files.
Otherwise, the jobs would be clobbering each other's output.  However, since
we are sending output and error to \File{/dev/null}, sharing the
\textit{CondorCommandFile} is OK.

\begin{verbatim}
  # Filename: diamond_job.condor
  #
  executable   = /path/diamond.exe
  output       = /dev/null
  error        = /dev/null
  log          = diamond_condor.log
  universe     = vanilla
  notification = NEVER
  queue
\end{verbatim}

Note that notification is set to \texttt{NEVER}.  This is recommended if you
prefer not to have Condor send you e-mail for every job in a large DAG.

\subsubsection{Writing the DAG File}
\label{dagman:writedag}

The DAG file names the jobs, associates jobs with their
\textit{CondorCommandFile}, and declares job dependencies.  For our artificial
DIAMOND example, all three jobs will use the same diamond\_job.condor file
written earlier.  However, a more typical DAG file would have unique
\textit{CondorCommandFile} for every job.

\begin{verbatim}
  ### DAGMan 6.1.0
  # Filename: diamond.dag
  # DIAMOND DAG File for DAGMan
  #
  Job  A  diamond_job.condor
  Job  B  diamond_job.condor
  Job  C  diamond_job.condor
  Job  D  diamond_job.condor
  PARENT A CHILD B C
  PARENT B C CHILD D
\end{verbatim}

This DAG file will be the input file for the \Condor{dagman} program.

\subsubsection{Submitting the DAG to Condor}
\label{dagman:submitdag}

In order to guarantee recoverability, the DAGMan program itself is run as a
Condor job.  However, DAGMan is not submitted as a standard universe or
vanilla universe job.  Instead, it is run as a meta-scheduler.  Standard and
vanilla universe jobs are usually submitted to the local schedd, which
schedules them for execution on some remote machine in the pool that is idle.
A meta-scheduler is also submitted to the local schedd, but runs on the local
schedd.  The meta-scheduler then submits jobs, according to its design, to the
same local schedd, just as if the user submitted them manually.  In fact, the
local schedd does not know the difference between DAGMan submitting a job, and
the user who originally submitted DAGMan, and could have submitted the DAG
jobs manually.

A DAG is submitted using the \Condor{submit\_dag} script.  For example, to
submit the \File{diamond.dag} DAG to Condor, simply type
``\Condor{submit\_dag} \File{diamond.dag}''.  This script will generate the
\File{diamond.dag.condor} \textit{CondorCommandFile} for the DAG, and submit
it to Condor.

If the user prefers to edit the \File{diamond.dag.condor} file before it is
submitted to Condor (for example, to change the pre-chosen filenames), she can
issue ``\Condor{submit\_dag} -n \File{diamond.dag}'', which specifies that
\File{diamond.dag.condor} is generated, but not submitted to Condor.  To run
the DAG, issue the command \Condor{submit} diamond.dag.condor.

\subsection{Removal}

After submitting a DAG, the user may change her mind and wish to remove the
entire DAG, plus any jobs submitted by that DAG which happen to currently be
running.  DAG removal is easily accomplished by issuing a \Condor{rm} on the
DAGMan job itself.  The schedd sends a special signal to the meta-scheduler,
telling it to remove any of its condor jobs (using \Condor{rm}) that are
currently running.

However, if the machine is scheduled to go down, and the schedd receives a
shutdown command from the master, the schedd will send a running DAGMan job a
similar shutdown, which instructs DAGMan to clean up memory and exit.
However, in this case, DAGMan does not remove its submitted jobs, but rather
expects them to persistently exist in the Condor queue after restart.

The important thing to remember is that DAGMan will not explicitly run
\Condor{rm} on its jobs except as a result of the user running \Condor{rm} on
the DAGMan job.

\subsection{Recovery}

The Condor system offers the benefit of recoverability, in that if any host
crashes, Condor jobs that were running can be recovered, either by continuing
from the last checkpoint, or rerunning from scratch.  In any event, Condor
guarantees that once a job is successfully submitted, the Condor system will
not loose it.

DAGMan makes the same guarantee about the DAG as a whole.  If the machine
running DAGMan goes down or crashes, upon restart DAGMan will be restarted,
and the state of the DAG jobs will be recovered from the log file
(\File{diamond.dag.condor.log} from our example before).  DAGMan knows to
recover a DAG (as apposed to starting a new one) because it will detect the
existance of a lock file that was not removed from the last run.  If DAGMan
successfully finishes a DAG, the lock file is removed, so that the next run
will not go into recover mode.  The lock file is specified via command-line
argument to DAGMan in the \textit{CondorCommandFile}.  Refer to
section~\ref{dagman:submitdag}.

\subsection{Essentials}
\label{dagman:essentials}

This section is written for those users looking for the boiled down,
absolutely essential steps to successfully submit a DAG.

\begin{description}

\item[Prepare Jobs] Each job in the DAG must have its own
\textit{CondorCommandFile}.  Each \textit{CondorCommandFile} can only submit
one job.  Multi-job clusters (multiple \texttt{queue} lines) are not
supported.  The \texttt{log=} for all \textit{CondorCommandFile}s must point
to the same Condor log file, otherwise, DAGMan will not see all the Condor log
entries for every job in the DAG.  Refer to section~\ref{dagman:prepjob} for
details on how to prepare jobs.

\item[Write DAG File] Write the DAG file, so that JOB entries refer to the
\textit{CondorCommandFile}s you wrote in the previous step.  Refer to
section~\ref{dagman:writedag} to learn about writing a DAG file.

\item[Submit the DAG] Finally, you submit the DAG written in the previous step
using the \Condor{submit\_dag} script.  Refer to
section~\ref{dagman:submitdag}.

\end{description}


\subsection{Version Summary}
\label{dagman:version}

This section addresses the features and limitations that exist in the current
version of DAGMan, and how they may change in future versions.

This first public release of DAGMan was written and tested in the Condor 6.1.0
environment.  It is shipped separate from the main Condor system as a
contribution program.  As such, it is not as rigorously tested as the core
components of Condor.  A reasonable effort has been made to test large DAGS
(on the order of 5000 jobs) on Solaris x86 and Sparc.  However, the DAGMan is
not arrogant enough to claim itself bug free.  Users are encouraged to send
e-mail to \Email{condor-admin@cs.wisc.edu}.

The following feature summary compares the current version with possible
versions of DAGMan still to come.

\begin{description}
\item[Feature] : Command Socket
\item[Version 6.1.0] : Unsupported
\item[Future Versions] : A general purpose command socket will be used to
direct Dagman while it's running.  Commands like CANCEL\_JOB X or DELETE\_ALL
would be supported, as well as notification messages like JOB\_SUBMIT or
JOB\_TERMINATE, etc.  Eventually, a Java Gui would graphically represent the
Dag's state, and offer buttons and dials for graphic Dag manipulation.
\end{description}

\begin{description}
\item[Feature]: DAG removal
\item[Version 6.1.0]: Supported via \Condor{rm} of the DAG.
\item[Future Versions]: Supported by a command socket such as DELETE\_ALL
\end{description}

\begin{description}
\item[Feature]: Condor Log File
\item[Version 6.1.0]: All jobs in a DAG must specify the same Condor log file.
That Condor log file must be unique.  No other DAGs or Condor jobs can point
to that log file.
\item[Future Versions]: All jobs in a Dag must go to one log file, but
log file can be shared with other Dags and Condor jobs.
\end{description}

\begin{description}
\item[Feature]: Job UNDO
\item[Version 6.1.0]: All jobs must exit normally, else DAG will be aborted
\item[Future Versions]: A job can be ``undone'', or there is some
notion of a job instance.  Hence, a job that exits abnormally or is
cancelled by the user can be rerun such that the new run's log entry
is unique from the old run's log entry (in terms of recovery)
\end{description}

\begin{description}
\item[Feature]: Pre/Post Process
\item[Version 6.1.0]: Unsupported
\item[Future Versions]: A job can have a pre- and post-process script
specified, which are run before and after the job is submitted.  This can be
useful for performing tasks like compression or decompression or input or
output data.
\end{description}

%%%%%%%%%%%%%%%%%%%%%%%%%%%%%%%%%%%%%%%%%%%%%%%%%%%%%%%%%%%%%%%%%%%%%%

%%%%%%%%%%%%%%%%%%%%%%%%%%%%%%%%%%%%%%%%%%%%%%%%%%%%%%%%%%%%%%%%%%%%%%
\section{\label{sec:Vacate-Explained}
More about how Condor vacates a job}
%%%%%%%%%%%%%%%%%%%%%%%%%%%%%%%%%%%%%%%%%%%%%%%%%%%%%%%%%%%%%%%%%%%%%%

When Condor needs to vacate a job from a machine for whatever reason, it
sends the job an asynchronous signal specified in the ``KillSig''
attribute of the job's classad.  The value of this attribute can be specified by
the user at submit time by placing the \Arg{kill\_sig} command in the
\Condor{submit} submit-description command file.  

If a program wanted to do some special work each time
Condor kicks them off a machine, all it would need to do is setup a
signal handler for some trappable signal as a ``cleanup'' signal.  When
submitting this job, specify this cleanup signal to use with
\Arg{kill\_sig}.  However, whatever cleanup work the job does had better be quick
--- if the job takes too long to go away after Condor tells it to do so, Condor
follows up with a SIGKILL signal which immediatly terminates the
process.

A job that linked with the Condor libraries via the \Condor{compile}
command and subsequently submitted into the Standard Universe 
will checkpoint and exit upon receit of a SIGTSTP signal.  Thus, SIGTSTP is
the default value for KillSig when submitting into the Standard
Universe.  However, the user's code can checkpoint itself at any time
by calling one of the following functions exported by the Condor libraries:
\begin{description}
\item[ckpt()] Will perform a checkpoint and then return
\item[ckpt\_and\_exit()] Will checkpoint and exit; Condor will then
restart the process again later, potentially on a different machine
\end{description}

For jobs submitted into the Vanilla Universe, the default value for
KillSig is SIGTERM, which is the usual method to nicely terminate a
program in Unix.

%%%%%%%%%%%%%%%%%%%%%%%%%%%%%%%%%%%%%%%%
\section{Special Environment Considerations}
%%%%%%%%%%%%%%%%%%%%%%%%%%%%%%%%%%%%%%%%

\subsection{AFS}

The Condor daemons do not run authenticated to AFS; they do not possess
an AFS token, and therefore no child process of Condor will be AFS
authenticated either. This means that you must set file permissions so
that your job can access any necessary files residing on an AFS volume
during its run without relying on having your AFS permissions.

So, if a job you submit to Condor needs to access files residing in AFS,
you have the following choices:
\begin{enumerate}
\item Copy the files needed off of AFS to either a local hard disk where Condor 
can access them via remote system calls (if
this is a Standard Universe job), or copy them to an NFS volume.
\item If you must keep the files on AFS, then you need to set a host ACL
(using the AFS ``fs setacl'' command) on the subdirectory which will
serve as the current working directory for the job.  If the job is a
Standard Universe job, then the host ACL needs to give read/write permission
to any process on the submit machine.  If the job is a Vanilla Universe
job, then you need to set the ACL such that any host in the pool can
access the files without being authenticated.  If you do not know how to
use an AFS host ACL, please ask whomever at your site is responsible for the
AFS configuration.
\end{enumerate}

How Condor deals with AFS authentication is something the Condor Team
hopes to improve in a subsequent release.

Please also see section~\ref{sec:Condor-AFS-Users} on
page~\pageref{sec:Condor-AFS-Users} in the Administrators Manual for
more discussion about this problem.

\subsection{NFS Automounter}

If your current working directory when you run \Condor{submit} is
accessed via an NFS automounter, Condor may have problems if the
automounter later decides to unmount the volume before your job has
completed.  This is because \Condor{submit} likely has stored the
dynamic mount point as the job's initial current working directory, and
this mount point could become automatically unmounted by the
automounter.

There is a simple workaround: When submitting your job, use the 
\Arg{initialdir} command in your submit-description file to point to
the stable access point.  For example,
say the NFS automounter is configured to mount a volume at mount point
/a/myserver.company.com/vol1/johndoe whenever the directory /home/johndoe is
accessed.  In this example case, simply add the following line to your \Condor{submit}
submit-description file:
\begin{verbatim}
        initialdir = /home/johndoe
\end{verbatim}

\subsection{Condor Daemons running as Non-root}

Condor is normally installed such that the Condor daemons have root
permission.  This allows Condor to run the \condor{shadow} process and
your job process with your UID and file access rights.  When Condor
is started as root, your Condor jobs can access whatever files you can.

However, it is possible that whomever installed Condor decided not to
run the daemons as root, or did not have root access.  That's a shame, 
since Condor is really designed to be run as root.  To see if Condor is
running as root on a given machine, enter the following command:
\begin{verbatim}
        condor_status -master -l <machine-name>
\end{verbatim}

where \Opt{machine-name} is the name of the machine you want to
inspect.  This command will display a \condor{master} ClassAd; if the
attribute ``RealUid'' equals zero, then the Condor daemons are indeed
running with root access and you can skip this section.  If the
``RealUid'' attribute is not zero, then the Condor daemons do not have
root access, and you should read on.

Please realize that using ``ps'' is \underline{not} an effective
method of determining if Condor is running with root access.  When
using the "ps" command, it may often appear that the daemons are
running as the condor user instead of root.  However, note that the
``ps'' command shows the current \emph{effective} owner of the
process, not the \emph{real} owner.  (See the \Cmd{getuid}{2} and
\Cmd{geteuid}{2} Unix man pages for details.)  In Unix, a process
running under the real uid of root may switch its effective uid.  (See
the \Cmd{seteuid}{2} man page.)  For security reasons, the daemons
only set effective uid to root when absolutely necessary (to perform a
privileged operation).



If they are not running with root access, you need to make any/all files
and/or directories that your job will touch readable and/or writable by
the UID (user id) specified by the RealUid attribute.  Often this may
mean doing a ``chmod 777'' on the directory where you submit your Condor
job.

%%%%%%%%%%%%%%%%%%%%%%%%%%%%%%%%%%%%%%%%
\section{Potential Problems}
%%%%%%%%%%%%%%%%%%%%%%%%%%%%%%%%%%%%%%%%

\subsection{Renaming of argv[0]}

When Condor starts up your job, it renames argv[0] (which usually
contains the name of the program) to ``\condor{exec}''.  This is
conveinent when examining a machine's processes with ``ps''; the process
is easily identified as a Condor job.  

Unfortunately, some programs read argv[0] expecting their own program
name and get confused if they find something unexpected like
\condor{exec}.
