%%%%%%%%%%%%%%%%%%%%%%%%%%%%%%%%%%%%%%%%%%%%%%%%%%%%%%
\section{Welcome to HTCondor}  
%
% .... or alternatively called the 'warm fuzzies' section
% <smirk>  
% 
%
% Warning: much of what you are about to read was very 
% hastily written by a very tired Todd.... Good Luck.  
%%%%%%%%%%%%%%%%%%%%%%%%%%%%%%%%%%%%%%%%%%%%%%%%%%%%%

\label{sec:usermanual}
\index{HTCondor!user manual|(}
\index{user manual|(}
Presenting HTCondor \VersionNotice! HTCondor is developed by
the Center for High Throughput Computing at the University of Wisconsin-Madison (UW-Madison), and
was first installed as a production system in the UW-Madison Computer
Sciences department more than 15 years ago. This HTCondor pool has since
served as a major source of computing cycles to UW faculty and students.
For many, it has revolutionized the role computing plays in their
research. An increase of one, and sometimes even two, orders of
magnitude in the computing throughput of a research organization can
have a profound impact on its size, complexity, and scope. Over the
years, the Center for High Throughput Computing has established collaborations with scientists
from around the world, and it has provided them with access to surplus
cycles (one scientist has consumed 100 CPU years!). Today, our
department's pool consists of more than 700 desktop Unix workstations
and more than 100 Windows machines.
On a typical day, our pool delivers more than 500 CPU days to UW
researchers. Additional HTCondor pools have been established over the
years across our campus and the world. Groups of researchers, engineers,
and scientists have used HTCondor to establish compute pools ranging in
size from a handful to hundreds of workstations. We hope that HTCondor
will help revolutionize your compute environment as well.


%%%%%%%%%%%%%%%%%%%%%%%%%%%%%%%%%%%%%%%%%%%%%%%%%%%%%%%
\section{Introduction}
%%%%%%%%%%%%%%%%%%%%%%%%%%%%%%%%%%%%%%%%%%%%%%%%%%%%%%%


In a nutshell, HTCondor is a specialized batch system 
\index{batch system}
for managing compute-intensive jobs.
Like most batch systems, HTCondor provides a
queuing mechanism, scheduling policy, priority scheme, and resource
classifications.  Users submit their compute jobs to HTCondor, HTCondor puts
the jobs in a queue, runs them, and then informs the user as to the
result.

Batch systems normally operate only with dedicated machines.  Often 
termed compute servers, these dedicated machines are typically owned by
one organization and dedicated to the sole purpose of running compute
jobs.  HTCondor can schedule jobs on dedicated machines.  But unlike traditional 
batch systems, HTCondor is also designed to effectively 
utilize non-dedicated machines to run jobs.  By being told to only
run compute jobs on machines which are currently not being used (no keyboard
activity, low load average, etc.), HTCondor can
effectively harness otherwise idle machines throughout a pool of machines.
This is important because often times the amount of
compute power represented by the aggregate total of all the non-dedicated 
desktop workstations sitting on people's desks throughout the
organization is far greater than the compute power of a dedicated
central resource.

HTCondor has several unique capabilities at its disposal which are geared 
toward effectively utilizing non-dedicated resources that are not owned or
managed by a centralized resource. These include transparent process
checkpoint and migration, remote system calls, and ClassAds.
Read section~\ref{sec:what-is-condor} for a general 
discussion of these features before reading any further.


%%%%%%%%%%%%%%%%%%%%%%%%%%%%%%%%%%%%%%%%%%%%%%%%%%%%%%%%
\section{Matchmaking with ClassAds}
\label{sec:matchmaking-with-classads}
%%%%%%%%%%%%%%%%%%%%%%%%%%%%%%%%%%%%%%%%%%%%%%%%%%%%%%%%

Before you learn about how to submit a job, it is important to
understand how HTCondor allocates resources. 
\index{HTCondor!resource allocation}
Understanding the
unique framework by which HTCondor matches submitted jobs with machines is
the key to getting the most from HTCondor's scheduling algorithm. 

HTCondor simplifies job submission by acting as a matchmaker of ClassAds.
HTCondor's ClassAds
\index{ClassAd}
are analogous to the classified advertising section of the
newspaper. Sellers advertise specifics about what they have to sell,
hoping to attract a buyer. Buyers may advertise specifics about what
they wish to purchase. Both buyers and sellers list constraints that
need to be satisfied.
For instance, a buyer has a maximum spending limit, 
and a seller requires a minimum purchase price.
Furthermore, both want to rank requests to their own advantage.
Certainly a seller would rank
one offer of \$50 dollars higher than a different
offer of \$25.
In HTCondor, users submitting
jobs can be thought of as buyers of compute resources and machine owners
are sellers. 

All machines in a HTCondor pool advertise their attributes,
\index{ClassAd!attributes}
such as
available memory, CPU type and speed, virtual memory size, current
load average, along with other static and dynamic properties.
This machine ClassAd
\index{ClassAd!machine}
also advertises under what conditions it is
willing to run a HTCondor job and what type of job it would prefer. These
policy attributes can reflect the individual terms and preferences by
which all the different owners have graciously allowed their machine to
be part of the HTCondor pool. 
You may
advertise that your machine is only willing to run jobs at night
and when there is no keyboard activity on your machine.
In addition, you may
advertise a preference (rank) for running jobs submitted by you
or one of your co-workers. 

Likewise, when submitting a job, you specify a ClassAd with
your requirements and preferences.
The ClassAd
\index{ClassAd!job}
includes the
type of machine you  wish to use. For instance, perhaps you are
looking for the fastest floating point performance available.
You want HTCondor to rank available machines
based upon floating point performance. Or, perhaps you
care only that the machine has a minimum of 128 Mbytes of RAM.
Or, perhaps you will
take any machine you can get! These job attributes and requirements
are bundled up into a job ClassAd.

HTCondor plays the role of a matchmaker by continuously reading
all the job ClassAds and all the machine ClassAds, 
matching and ranking job ads with machine ads.
HTCondor makes certain that all
requirements in both ClassAds are satisfied. 

%%%%%
\subsection{Inspecting Machine ClassAds with \condor{status}}
%%%%%

\index{HTCondor commands!condor\_status}
Once HTCondor is installed,
you will get a feel for what
a machine ClassAd does by trying
the \Condor{status} command.
Try the \Condor{status} command to get
a summary of information from
ClassAds about the resources available in your pool.
Type \Condor{status} and hit enter to see a summary 
similar to the following:
%\small       too big
%\tiny        too small
\footnotesize
\begin{verbatim}
Name               OpSys      Arch   State     Activity LoadAv Mem   ActvtyTime

amul.cs.wisc.edu   LINUX      INTEL  Claimed   Busy     0.990  1896  0+00:07:04
slot1@amundsen.cs. LINUX      INTEL  Owner     Idle     0.000  1456  0+00:21:58
slot2@amundsen.cs. LINUX      INTEL  Owner     Idle     0.110  1456  0+00:21:59
angus.cs.wisc.edu  LINUX      INTEL  Claimed   Busy     0.940   873  0+00:02:54
anhai.cs.wisc.edu  LINUX      INTEL  Claimed   Busy     1.400  1896  0+00:03:03
apollo.cs.wisc.edu LINUX      INTEL  Unclaimed Idle     1.000  3032  0+00:00:04
arragon.cs.wisc.ed LINUX      INTEL  Claimed   Busy     0.980   873  0+00:04:29
bamba.cs.wisc.edu  LINUX      INTEL  Owner     Idle     0.040  3032 15+20:10:19
\end{verbatim}
\normalsize
\Dots 


The \Condor{status} command has options that summarize machine ads 
in a variety of ways.
For example,
\begin{description}
\item[\Condor{status -available}] shows only machines which are
willing to run jobs now. 
\item[\Condor{status -run}] shows only machines
which are currently running jobs.  
\item[\Condor{status -long}] lists the machine ClassAds for all machines
in the pool.
\end{description}

Refer to the \Condor{status} command 
reference page located on page~\pageref{man-condor-status}
for a complete description of the \Condor{status} command.

The following shows a portion of a machine ClassAd
\index{ClassAd!machine example}
\index{machine ClassAd}
for a single machine: turunmaa.cs.wisc.edu. Some of the listed
attributes are used by
HTCondor for scheduling. Other attributes are for information purposes.
An important point is that \emph{any} of the attributes in a
machine ClassAd can be utilized at job submission time as part of a request
or preference on what machine to use. Additional attributes
can be easily added. For example, your site administrator can
add a physical location attribute to your machine ClassAds.

% condor_status -long turunmaa.cs.wisc.edu

\footnotesize
\begin{verbatim}
Machine = "turunmaa.cs.wisc.edu"
FileSystemDomain = "cs.wisc.edu"
Name = "turunmaa.cs.wisc.edu"
CondorPlatform = "$CondorPlatform: x86_rhap_5 $"
Cpus = 1
IsValidCheckpointPlatform = ( ( ( TARGET.JobUniverse == 1 ) == false ) || 
 ( ( MY.CheckpointPlatform =!= undefined ) && 
 ( ( TARGET.LastCheckpointPlatform =?= MY.CheckpointPlatform ) || 
 ( TARGET.NumCkpts == 0 ) ) ) )
CondorVersion = "$CondorVersion: 7.6.3 Aug 18 2011 BuildID: 361356 $"
Requirements = ( START ) && ( IsValidCheckpointPlatform )
EnteredCurrentActivity = 1316094896
MyAddress = "<128.105.175.125:58026>"
EnteredCurrentState = 1316094896
Memory = 1897
CkptServer = "pitcher.cs.wisc.edu"
OpSys = "LINUX"
State = "Owner"
START = true
Arch = "INTEL"
Mips = 2634
Activity = "Idle"
StartdIpAddr = "<128.105.175.125:58026>"
TargetType = "Job"
LoadAvg = 0.210000
CheckpointPlatform = "LINUX INTEL 2.6.x normal 0x40000000"
Disk = 92309744
VirtualMemory = 2069476
TotalSlots = 1
UidDomain = "cs.wisc.edu"
MyType = "Machine"
\end{verbatim}
\normalsize


%%%%%%%%%%%%%%%%%%%%%%%%%%%%%%%%%%%%%%%%%%%%%%%%%%%%%%%%%%%%%
\section{Road-map for Running Jobs}
%%%%%%%%%%%%%%%%%%%%%%%%%%%%%%%%%%%%%%%%%%%%%%%%%%%%%%%%%%%%%

\index{job!preparation}
The road to using HTCondor effectively is a short one.  The basics
are quickly and easily learned.

Here are all the steps needed to run a job using HTCondor.
\begin{description}

\item[Code Preparation.]
A job run under HTCondor must be able to 
run as a background batch job.
\index{job!batch ready}
HTCondor runs the program unattended and in the background. 
A program that runs in the background will not be able
to do interactive input and output.
HTCondor can redirect console output (stdout and stderr)
and keyboard input (stdin)
to and from files for you.
Create any needed files that contain
the proper keystrokes needed for program input.
Make certain the program will run correctly with the files.

\item[The HTCondor Universe.]
HTCondor has several 
runtime environments (called a \Term{universe}) from which to choose.
Of the universes, two are likely choices when learning
to submit a job to HTCondor: the standard universe and the vanilla universe.
The standard universe allows a job running under HTCondor to
handle system calls by returning them to the machine where the
job was submitted.
The standard universe also provides the mechanisms necessary
to take a checkpoint and migrate a partially completed job,
should the machine on which the job is executing become
unavailable.
To use the standard universe, it is necessary to
relink the program with the HTCondor library using the
\Condor{compile} command.
The manual page for \Condor{compile} on page~\pageref{man-condor-compile} has details.

The vanilla universe provides a way to run jobs that cannot be
relinked.
There is no way to take a checkpoint or migrate a job executed
under the vanilla universe.
For access to input and output files, jobs must either use a shared
file system, or use HTCondor's File Transfer mechanism.

Choose a universe under which to run the HTCondor program,
and re-link the program if necessary.

\item[Submit description file.]
Controlling the details of a job submission is a
submit description file.
The file contains information
about the job such as what executable to run, the
files to use for keyboard and screen data,
the platform type required to run the program, and
where to send e-mail when the job completes.
You can also tell HTCondor how many times to run a program;
it is simple to run the same program
multiple times with multiple data sets.

Write a submit description file to go with the job, using
the examples provided in section~\ref{sec:sample-submit-files}
for guidance.

\item[Submit the Job.]Submit the program to HTCondor with
the \Condor{submit} command.
\index{HTCondor commands!condor\_submit}

\end{description}

Once submitted, HTCondor does the rest toward running
the job.
Monitor the job's progress with the \Condor{q}
\index{HTCondor commands!condor\_q}
and \Condor{status} commands.
\index{HTCondor commands!condor\_status}
You may modify the order in which HTCondor will run your jobs with
\Condor{prio}. If desired, HTCondor can even inform you in a log file 
every time your job is checkpointed and/or migrated to a different machine. 

When your program completes, HTCondor will tell you
(by e-mail, if preferred) the exit status of your program and various
statistics about its performances, including time used and I/O performed.
If you are using a log file for the job (which is recommended) the exit
status will be recorded in the log file.
You can remove a job from the
queue prematurely with \Condor{rm}. 
\index{HTCondor commands!condor\_rm}


%%%%%%%%%%%%%%%%%%%%%%%%%%%%%%%%%%%%%%%%%%%%%%%%
\subsection{\label{sec:Choosing-Universe}
Choosing an HTCondor Universe}
%%%%%%%%%%%%%%%%%%%%%%%%%%%%%%%%%%%%%%%%%%%%%%%%

A \Term{universe} in HTCondor
\index{universe}
\index{HTCondor!universe}
defines an execution environment. 
HTCondor \VersionNotice\ supports several different
universes for user jobs:
\begin{itemize}
	\item Standard
	\item Vanilla
	\item Grid
	\item Java
	\item Scheduler
	\item Local
 	\item Parallel
 	\item VM
\end{itemize}

The \SubmitCmd{universe} under which a job runs
is specified in the submit description file.
If a universe is not specified,
the default is vanilla,
unless your HTCondor administrator has changed the default.
However, we strongly encourage you to specify the universe,
since the default can be changed by your HTCondor administrator,
and the default that ships with HTCondor has changed.

\index{universe!standard}
The standard universe provides migration and reliability, but has some
restrictions on the programs that can be run. 
\index{universe!vanilla}
The vanilla universe provides fewer services, but has very few
restrictions.
\index{universe!Grid}
The grid universe allows users to submit 
jobs using HTCondor's interface.
These jobs are submitted for execution on grid resources.
\index{universe!java}
\index{Java}
\index{Java Virtual Machine}
\index{JVM}
The java universe allows users to run jobs written for the
Java Virtual Machine (JVM).
The scheduler universe allows users to submit lightweight jobs
to be spawned by the program known as a daemon on the submit host itself.
\index{universe!parallel}
The parallel universe is for programs that require multiple machines
for one job.
See section~\ref{sec:Parallel} for more about the Parallel universe.
%\index{universe!Local}
%The local universe . . .
\index{universe!vm}
The vm universe allows users to run jobs where the job is
no longer a simple executable, but a disk image, facilitating
the execution of a virtual machine.

%%%%%%%%%%%%%%%%%%%%%%%%%%%%%%%%%%%%%%%%%%%%%%%%%%%%%%%%%%%%%%%%%%%%%%
\subsubsection{\label{sec:standard-universe}Standard Universe}
%%%%%%%%%%%%%%%%%%%%%%%%%%%%%%%%%%%%%%%%%%%%%%%%%%%%%%%%%%%%%%%%%%%%%%

\index{universe!standard}
In the standard universe, HTCondor provides \Term{checkpointing} and
\Term{remote system calls}.  These features make a job more reliable
and allow it uniform access to resources from anywhere in the pool.
To prepare a program as a standard universe job, it must be relinked
with \Condor{compile}.  Most programs can be prepared as a standard
universe job, but there are a few restrictions.

\index{checkpoint}
\index{checkpoint image}
HTCondor checkpoints a job at regular intervals.
A \Term{checkpoint image} is essentially a snapshot of the current
state of a job. 
If a job must be migrated from one machine to another,
HTCondor makes a checkpoint image, copies the image to the new machine,
and restarts the job continuing the job from where it left off.
If a machine should
crash or fail while it is running a job, HTCondor can restart the job on
a new machine using the most recent checkpoint image.
In this way, jobs
can run for months or years even in the face of occasional computer failures.

\index{remote system call}
\index{shadow}
Remote system calls make a job perceive that it is executing on its home
machine, even though the job may execute on many different machines over its
lifetime.
When a job runs on a remote machine, a second process, called
a \Condor{shadow} runs on the machine where the job was submitted.
\index{condor\_shadow}
\index{agents!condor\_shadow}
\index{HTCondor daemon!condor\_shadow}
\index{remote system call!condor\_shadow}
When the job attempts a system call, the \Condor{shadow} performs
the system call instead and sends the results to the remote
machine.
For example, if a job attempts to open a file that is
stored on the submitting machine,
the \Condor{shadow} will find the file,
and send the data to the machine where
the job is running.

To convert your program into a standard universe job, you must use
\Condor{compile} to relink it with the HTCondor libraries.
Put \Condor{compile} in front of your usual link command.
You do not need to modify the program's source code,
but you do need access to the unlinked object files.
A commercial program that is packaged as a single executable file cannot be
converted into a standard universe job.

For example, if you would have linked the job by executing:
\begin{verbatim}
% cc main.o tools.o -o program
\end{verbatim}

Then, relink the job for HTCondor with:
\begin{verbatim}
% condor_compile cc main.o tools.o -o program
\end{verbatim}

There are a few restrictions on standard universe jobs:


\begin{enumerate}

\index{Unix!fork}
\index{Unix!exec}
\index{Unix!system}
\item Multi-process jobs are not allowed.  This includes system calls such as
\Syscall{fork}, \Syscall{exec}, and \Syscall{system}.

\index{Unix!pipe}
\index{Unix!semaphore}
\index{Unix!shared memory}
\item Interprocess communication is not allowed.  This includes pipes, semaphores, and shared memory.

\index{Unix!socket}
\index{network}
\item Network communication must be brief.  A job \emph{may} make network
connections using system calls such as \Syscall{socket}, but a network
connection left open for long periods will delay checkpointing and migration.

\index{signal}
\index{signal!SIGUSR2}
\index{signal!SIGTSTP}
\item Sending or receiving the SIGUSR2 or SIGTSTP signals is not allowed.
HTCondor reserves these signals for its own use.  Sending or receiving all
other signals \emph{is} allowed.

\index{Unix!alarm}
\index{Unix!timer}
\index{Unix!sleep}
\item Alarms, timers, and sleeping are not allowed.  This includes system
calls such as \Syscall{alarm}, \Syscall{getitimer}, and \Syscall{sleep}.

\index{thread!kernel-level}
\index{thread!user-level}
\item Multiple kernel-level threads are not allowed.  However,
multiple user-level threads \emph{are} allowed.

\index{file!memory-mapped}
\index{Unix!mmap}
\item Memory mapped files are not allowed.  This includes system calls such
as \Syscall{mmap} and \Syscall{munmap}.

\index{file!locking}
\index{Unix!flock}
\index{Unix!lockf}
\item File locks are allowed, but not retained between checkpoints.

\index{file!read only}
\index{file!write only}
\item All files must be opened read-only or write-only.  A file opened
for both reading and writing will cause trouble if a job must be rolled back
to an old checkpoint image.  For compatibility reasons, a file opened
for both reading and writing will result in a warning but not an error.

\item A fair amount of disk space must be available on the submitting machine
for storing a job's checkpoint images.  A checkpoint image is approximately
equal to the virtual memory consumed by a job while it runs.  If disk space
is short, a special \Term{checkpoint server} can be designated for storing
all the checkpoint images for a pool.

\index{linking!dynamic}
\index{linking!static}
\item On Linux, the job must be statically linked. 
\Condor{compile} does this by default.

\index{Unix!large files} 
\item Reading to or writing from files larger than 2 GBytes is only supported
when the submit side \Condor{shadow} and the standard universe user job
application itself are both 64-bit executables.

\end{enumerate}






%%%%%%%%%%%%
\subsubsection{Vanilla Universe}
%%%%%%%%%%%%

\index{universe!vanilla}
The vanilla universe in HTCondor is intended
for programs which cannot
be successfully re-linked.
Shell scripts are another case where the vanilla universe
is useful.
Unfortunately, jobs run under the vanilla universe cannot checkpoint or use
remote system calls. 
This has unfortunate consequences for a job that is partially
completed 
when the remote machine running a job must be returned
to its owner.
HTCondor has only two choices.  It can suspend the job, hoping to
complete it at a later time,
or it can give up and restart the job \emph{from the beginning} 
on another machine in the pool.

Since HTCondor's remote system call features cannot be used with the
vanilla universe, access to the job's input and output files becomes a
concern.
One option is for HTCondor to rely on a shared file system, such as NFS
or AFS. 
Alternatively, HTCondor has a mechanism for transferring files on behalf
of the user.
In this case, HTCondor will transfer any files needed by a job to the
execution site, run the job, and transfer the output back to the
submitting machine.

Under Unix, HTCondor presumes a shared file system for vanilla jobs. 
However, if a shared file system is unavailable, a user can enable the
HTCondor File Transfer mechanism.
On Windows platforms, the default is to use the File Transfer
mechanism.
For details on running a job with a shared file system, see
section~\ref{sec:shared-fs} on page~\pageref{sec:shared-fs}.
For details on using the HTCondor File Transfer mechanism, see 
section~\ref{sec:file-transfer} on page~\pageref{sec:file-transfer}.


%%%%%%%%%%%%
\subsubsection{Grid Universe}
%%%%%%%%%%%%

\index{universe!Grid}
The Grid universe in HTCondor is intended to provide the standard
HTCondor interface to users who wish to start jobs
intended for remote management systems.
Section~\ref{sec:GridUniverse} on page~\pageref{sec:GridUniverse}
has details on using the Grid universe.
The manual page for \Condor{submit}
on page~\pageref{man-condor-submit}
has detailed descriptions of
the grid-related attributes.

%%%%%%%%%%%%
\subsubsection{Java Universe}
%%%%%%%%%%%%

\index{universe!Java}

A program submitted to the Java universe may run on any sort of machine
with a JVM regardless of its location, owner, or JVM version.  HTCondor
will take care of all the details such as finding the JVM binary and
setting the classpath.

%%%%%%%%%%%%
\subsubsection{Scheduler Universe}
%%%%%%%%%%%%

\index{universe!scheduler}
\index{scheduler universe}

The scheduler universe allows users to submit lightweight jobs
to be run immediately, alongside the \Condor{schedd} daemon on the submit host
itself.
Scheduler universe jobs are not matched with a remote machine,
and will never be preempted.
The job's requirements expression is evaluated against the \Condor{schedd}'s
ClassAd.

Originally intended for meta-schedulers such as \Condor{dagman},
the scheduler universe can also be
used to manage jobs of any sort that must run on the submit host.

However, unlike the local universe, the scheduler
universe does not use a \Condor{starter} daemon to manage the job, and thus
offers limited features and policy support.  The local universe
is a better choice for most jobs which must run on the submit host, as
it offers a richer set of job management features, and is more
consistent with other universes such as the vanilla universe.
The scheduler universe may be retired in the future, in
favor of the newer local universe.


%%%%%%%%%%%%%%%%%%%%%%%%%%%%%%%%%%%%%%%%%%%%%%%%%%%%%%%%%%%%%%%%%%%%%%
\subsubsection{\label{sec:local-universe}Local Universe}
%%%%%%%%%%%%%%%%%%%%%%%%%%%%%%%%%%%%%%%%%%%%%%%%%%%%%%%%%%%%%%%%%%%%%%

\index{universe!local}
\index{local universe}
The local universe allows an HTCondor job to be submitted and
executed with different assumptions for the execution conditions
of the job.
The job does not wait to be matched with a machine.
It instead executes right away, on the machine where the job
is submitted.
The job will never be preempted.
The job's requirements expression is evaluated against the \Condor{schedd}'s
ClassAd.

%%%%%%%%%%%%
\subsubsection{Parallel Universe}
%%%%%%%%%%%%
\index{universe!parallel}
\index{parallel universe}
The parallel universe allows parallel programs, such as MPI jobs,
to be run within the opportunistic HTCondor environment.
Please see section~\ref{sec:Parallel} for more details.

%%%%%%%%%%%%
\subsubsection{VM Universe}
%%%%%%%%%%%%
\index{universe!vm}
\index{vm universe}
HTCondor facilitates the execution of VMware and Xen
virtual machines with the vm universe.

Please see section~\ref{sec:vmuniverse} for details.


%%%%%%%%%%%%%%%%%%%%%%%%%%%%%%%%%%%%%%%%%%%%%%%%%%%%%%%%%%%%%%
\section{Submitting a Job}
%%%%%%%%%%%%%%%%%%%%%%%%%%%%%%%%%%%%%%%%%%%%%%%%%%%%%%%%%%%%%%

\index{job!submitting}
A job is submitted for execution to HTCondor using the
\Condor{submit} command.
\index{HTCondor commands!condor\_submit}
\Condor{submit} takes as an argument the name of a
file called a submit description file.
\index{submit description file}
\index{file!submit description}
This file contains commands and keywords to direct the queuing of jobs.
In the submit description file, HTCondor finds everything it needs
to know about the job.  Items such as the name of the executable to run,
the initial working directory, and command-line arguments to the
program all go into
the submit description file.  \Condor{submit} creates a job
ClassAd based upon the information,
and HTCondor
works toward running the job.

The contents of a submit file
\index{submit description file!contents of}
can save time for HTCondor users.
It is easy to submit multiple runs of a program to
HTCondor. To run the same program 500 times on 500
different input data sets, arrange your data files
accordingly so that each run reads its own input, and each run
writes its own output.
Each individual run may have its own initial
working directory, stdin, stdout, stderr, command-line arguments, and
shell environment.
A program that directly opens its own
files will read the file names to use either from stdin
or from the command line. 
A program that opens a static filename every time
will need to use a separate subdirectory for the output of each run.

The \Condor{submit} manual page 
is on page~\pageref{man-condor-submit} and
contains a complete and full description of how to use \Condor{submit}.
It also includes descriptions of all the commands that may be placed
into a submit description file.
In addition, the index lists entries for each command under the
heading of Submit Commands.

%%%%%%%%%%%%%%%%%%%%
\subsection{\label{sec:sample-submit-files}Sample submit description files}  
%%%%%%%%%%%%%%%%%%%%

In addition to the examples of submit description files given
in the 
\Condor{submit} manual page, here are a few more.
\index{submit description file!examples|(}

\subsubsection{Example 1} 

Example 1 is one of the simplest submit description
files possible. It queues up one copy of the program \Prog{foo}
(which had been created by \Condor{compile})
for execution by HTCondor.
Since no platform is specified, HTCondor will use its default,
which is to run the job on a machine which has the
same architecture and operating system as the machine from which it was
submitted. 
No 
\AdAttr{input},
\AdAttr{output}, and
\AdAttr{error}
commands are given in the submit
description file, so the
files \File{stdin}, \File{stdout}, and \File{stderr} will all refer to 
\File{/dev/null}.
The program may produce output by explicitly opening a file and writing to
it.
A log file, \File{foo.log}, will also be produced that contains events
the job had during its lifetime inside of HTCondor.
When the job finishes, its exit conditions will be noted in the log file.
It is recommended that you always have a log file so you know what
happened to your jobs.
\begin{verbatim}
  ####################                                                    
  # 
  # Example 1                                                            
  # Simple condor job description file                                    
  #                                                                       
  ####################                                                    
                                                                          
  Executable   = foo                                                    
  Universe     = standard                                                    
  Log          = foo.log                                                    
  Queue    
\end{verbatim}

\subsubsection{Example 2}

Example 2 queues two copies of the program \Prog{mathematica}. The
first copy will run in directory \File{run\_1}, and the second will run in
directory \File{run\_2}. For both queued copies, 
\File{stdin} will be \File{test.data},
\File{stdout} will be \File{loop.out}, and
\File{stderr} will be \File{loop.error}.
There will be two sets of files written,
as the files are each written to their own directories.
This is a convenient way to organize data if you
have a large group of HTCondor jobs to run. The example file 
shows program submission of
\Prog{mathematica} as a vanilla universe job.
This may be necessary if the source
and/or object code to \Prog{mathematica} is not available.

The \SubmitCmd{request\_memory} command is included to insure
that the \Prog{mathematica} jobs match with and then execute on
pool machines that provide at least 1 GByte of memory.

\begin{verbatim}
  ####################     
  #                       
  # Example 2: demonstrate use of multiple     
  # directories for data organization.      
  #                                        
  ####################                    
                                         
  executable     = mathematica          
  universe       = vanilla                   
  input          = test.data                
  output         = loop.out                
  error          = loop.error             
  log            = loop.log                                                    
  request_memory = 1 GB
                                  
  initialdir     = run_1         
  queue                         
                               
  initialdir     = run_2      
  queue                     
\end{verbatim}

\subsubsection{Example 3}

The submit description file for Example 3 queues 150
\index{running multiple programs}
runs of program \Prog{foo} which has been compiled and linked for
LINUX running on a 32-bit Intel processor.
This job requires HTCondor to run the program on machines which have
greater than 32 Mbytes of physical memory, and expresses a
preference to run the program on machines with more than 64 Mbytes.
It also advises HTCondor that this standard universe job will
use up to 28000 Kbytes of memory when running.
Each of the 150 runs of the program is given its own process number,
starting with process number 0.
So, files 
\File{stdin}, \File{stdout}, and \File{stderr} will
refer to \File{in.0}, \File{out.0}, and \File{err.0} for the first run
of the program,
\File{in.1}, \File{out.1},
and \File{err.1} for the second run of the program, and so forth.
A log file containing entries
about when and where HTCondor runs, checkpoints, and migrates processes for
all the 150 queued programs
will be written into the single file \File{foo.log}.
\begin{verbatim}
  ####################                    
  #
  # Example 3: Show off some fancy features including
  # use of pre-defined macros and logging.
  #
  ####################                                                    

  Executable     = foo                                                    
  Universe       = standard                                                    
  requirements   = OpSys == "LINUX" && Arch =="INTEL"     
  rank           = Memory >= 64
  image_size     = 28000
  request_memory = 32

  error   = err.$(Process)                                                
  input   = in.$(Process)                                                 
  output  = out.$(Process)                                                
  log     = foo.log

  queue 150
\end{verbatim}

\index{submit description file!examples|)}

%%%%%%%%%%%%%%%%%
\subsection{\label{sec:user-man-req-and-rank}About Requirements and Rank}
%%%%%%%%%%%%%%%%%

The 
\AdAttr{requirements} and \AdAttr{rank} commands in the submit description file
are powerful and flexible. 
\index{submit commands!requirements}
\index{requirements attribute}
\index{rank attribute}
\index{ClassAd attribute!requirements}
\index{ClassAd attribute!rank}
Using them effectively requires care, and this section presents
those details.

Both \AdAttr{requirements} and \AdAttr{rank} need to be specified 
as valid HTCondor ClassAd expressions, however, default values are set by the
\Condor{submit} program if these are not defined in the submit description file.
From the \Condor{submit} manual page and the above examples, you see
that writing ClassAd expressions is intuitive, especially if you
are familiar with the programming language C.  There are some
pretty nifty expressions you can write with ClassAds.
A complete description of ClassAds and their expressions
can be found in section~\ref{sec:classad-reference} on 
page~\pageref{sec:classad-reference}.

All of the commands in the submit description file are case insensitive, 
\emph{except} for the ClassAd attribute string values.
ClassAd attribute names are
case insensitive, but ClassAd string
values are \emph{case preserving}.

Note that the comparison operators
(\verb@<@, \verb@>@, \verb@<=@, \verb@>=@, and \verb@==@)
compare strings
case insensitively.  The special comparison operators 
\verb@=?=@ and \verb@=!=@
compare strings case sensitively.

A  \SubmitCmd{requirements} or \SubmitCmd{rank} command in
the submit description file may utilize attributes
that appear in a machine or a job ClassAd.
Within the submit description file (for a job) the
prefix \verb@MY.@ (on a ClassAd attribute name)
causes a reference to the job ClassAd attribute,
and the prefix \verb@TARGET.@ causes a reference to 
a potential machine or matched machine ClassAd attribute.

The \Condor{status} command displays
\index{HTCondor commands!condor\_status}
statistics about machines within the pool.
The \Opt{-l} option displays the
machine ClassAd attributes for all machines in the HTCondor pool.
The job ClassAds, if there are jobs in the queue, can be seen
with the \Condor{q -l} command.
This shows all the defined attributes for current jobs in the queue.

A list of defined ClassAd attributes for job ClassAds
is given in the unnumbered Appendix on 
page~\pageref{sec:Job-ClassAd-Attributes}.
A list of defined ClassAd attributes for machine ClassAds
is given in the unnumbered Appendix on 
page~\pageref{sec:Machine-ClassAd-Attributes}.


\subsubsection{\label{rank-examples}Rank Expression Examples}

\index{rank attribute!examples}
\index{ClassAd attribute!rank examples}
\index{submit commands!rank}
When considering the match between a job and a machine, rank is used
to choose a match from among all machines that satisfy the job's
requirements and are available to the user, after accounting for
the user's priority and the machine's rank of the job.
The rank expressions, simple or complex, define a numerical value
that expresses preferences.

The job's \Attr{Rank} expression evaluates to one of three values.
It can be UNDEFINED, ERROR, or a floating point value.
If \Attr{Rank} evaluates to a floating point value,
the best match will be the one with the largest, positive value.
If no \Attr{Rank} is given 
in the submit description file,
then HTCondor substitutes a default value of 0.0 when considering
machines to match.
If the job's \Attr{Rank} of a given machine evaluates
to UNDEFINED or ERROR,
this same value of 0.0 is used.
Therefore, the machine is still considered for a match,
but has no ranking above any other.

A boolean expression evaluates to the numerical value of 1.0
if true, and 0.0 if false.

The following \Attr{Rank} expressions provide examples to
follow.

For a job that desires the machine with the most available memory:
\begin{verbatim}
   Rank = memory
\end{verbatim}

For a job that prefers to run on a friend's machine
on Saturdays and Sundays:
\begin{verbatim}
   Rank = ( (clockday == 0) || (clockday == 6) )
          && (machine == "friend.cs.wisc.edu")
\end{verbatim}

For a job that prefers to run on one of three specific machines:
\begin{verbatim}
   Rank = (machine == "friend1.cs.wisc.edu") ||
          (machine == "friend2.cs.wisc.edu") ||
          (machine == "friend3.cs.wisc.edu")
\end{verbatim}

For a job that wants the machine with the best floating point
performance (on Linpack benchmarks):
\begin{verbatim}
   Rank = kflops
\end{verbatim}
This particular example highlights a difficulty with \Attr{Rank} expression
evaluation as currently defined.
While all machines have floating point processing ability,
not all machines will have the \Attr{kflops} attribute defined.
For machines where this attribute is not defined,
\Attr{Rank} will evaluate to the value UNDEFINED, and
HTCondor will use a default rank of the machine of 0.0.
The \Attr{Rank} attribute will only rank machines where
the attribute is defined.
Therefore, the machine with the highest floating point
performance may not be the one given the highest rank.

So, it is wise when writing a \Attr{Rank} expression to check
if the expression's evaluation will lead to the expected
resulting ranking of machines.
This can be accomplished using the \Condor{status} command with the
\Arg{-constraint} argument.  This allows the user to see a list of
machines that fit a constraint.
To see which machines in the pool have \Attr{kflops} defined,
use
\begin{verbatim}
condor_status -constraint kflops
\end{verbatim}
Alternatively, to see a list of machines where 
\AdAttr{kflops} is not defined, use
\begin{verbatim}
condor_status -constraint "kflops=?=undefined"
\end{verbatim}

For a job that prefers specific machines in a specific order:
\begin{verbatim}
   Rank = ((machine == "friend1.cs.wisc.edu")*3) +
          ((machine == "friend2.cs.wisc.edu")*2) +
           (machine == "friend3.cs.wisc.edu")
\end{verbatim}
If the machine being ranked is \Expr{friend1.cs.wisc.edu}, then the
expression
\begin{verbatim}
   (machine == "friend1.cs.wisc.edu")
\end{verbatim}
is true, and gives the value 1.0.
The expressions
\begin{verbatim}
   (machine == "friend2.cs.wisc.edu")
\end{verbatim}
and
\begin{verbatim}
   (machine == "friend3.cs.wisc.edu")
\end{verbatim}
are false, and give the value 0.0.
Therefore, \Attr{Rank} evaluates to the value 3.0.
In this way, machine \Expr{friend1.cs.wisc.edu} is ranked higher than
machine \Expr{friend2.cs.wisc.edu},
machine \Expr{friend2.cs.wisc.edu}
is ranked higher than 
machine \Expr{friend3.cs.wisc.edu},
and all three of these machines are ranked higher than others.

%%%%%%%%%%%% 

%%%%%%%%%%%% 
\subsection{\label{sec:shared-fs}
Submitting Jobs Using a Shared File System} 
%%%%%%%%%%%%
\index{job!submission using a shared file system}
\index{shared file system!submission of jobs}

If vanilla, java, or parallel universe
jobs are submitted without using the File Transfer mechanism, 
Condor must use a shared file system to access input and output
files. 
In this case, the job \emph{must} be able to access the data files
from any machine on which it could potentially run.

As an example, suppose a job is submitted from blackbird.cs.wisc.edu,
and the job requires a particular data file called
\File{/u/p/s/psilord/data.txt}.  If the job were to run on
cardinal.cs.wisc.edu, the file \File{/u/p/s/psilord/data.txt} must be
available through either NFS or AFS for the job to run correctly.

Condor allows users to ensure their jobs have access to the right
shared files by using the \AdAttr{FileSystemDomain} and
\AdAttr{UidDomain} machine ClassAd attributes.
These attributes specify which machines have access to the same shared
file systems.
All machines that mount the same shared directories in the same
locations are considered to belong to the same file system domain.
Similarly, all machines that share the same user information (in
particular, the same UID, which is important for file systems like
NFS) are considered part of the same UID domain.

The default configuration for Condor places each machine
in its own UID domain and file system domain, using the full host name of the
machine as the name of the domains.
So, if a pool \emph{does} have access to a shared file system,
the pool administrator \emph{must} correctly configure Condor 
such that all
the machines mounting the same files have the same
\AdAttr{FileSystemDomain} configuration.
Similarly, all machines that share common user information must be
configured to have the same \AdAttr{UidDomain} configuration.

When a job relies on a shared file system,
Condor uses the
\AdAttr{requirements} expression to ensure that the job runs
on a machine in the
correct \AdAttr{UidDomain} and \AdAttr{FileSystemDomain}.
In this case, the default \AdAttr{requirements} expression specifies
that the job must run on a machine with the same \AdAttr{UidDomain}
and \AdAttr{FileSystemDomain} as the machine from which the job
is submitted.
This default is almost always correct.
However, in a pool spanning multiple \AdAttr{UidDomain}s and/or
\AdAttr{FileSystemDomain}s, the user may need to specify a different
\AdAttr{requirements} expression to have the job run on the correct
machines.

For example, imagine a pool made up of both desktop workstations and a
dedicated compute cluster.
Most of the pool, including the compute cluster, has access to a
shared file system, but some of the desktop machines do not.
In this case, the administrators would probably define the
\AdAttr{FileSystemDomain} to be \File{cs.wisc.edu} for all the machines
that mounted the shared files, and to the full host name for each
machine that did not. An example is \File{jimi.cs.wisc.edu}.

In this example,
a user wants to submit vanilla universe jobs from her own desktop
machine (jimi.cs.wisc.edu) which does not mount the shared file system
(and is therefore in its own file system domain, in its own world).
But, she wants the jobs to be able to run on more than just her own
machine (in particular, the compute cluster), so she puts the program
and input files onto the shared file system.
When she submits the jobs, she needs to tell Condor to send them to
machines that have access to that shared data, so she specifies a
different \AdAttr{requirements} expression than the default:
\begin{verbatim}
   Requirements = TARGET.UidDomain == "cs.wisc.edu" && \
                  TARGET.FileSystemDomain == "cs.wisc.edu"
\end{verbatim}

\Warn If there is \emph{no} shared file system, or the Condor pool
administrator does not configure the \AdAttr{FileSystemDomain}
setting correctly (the default is that each machine in a pool is in
its own file system and UID domain), a user submits a job that cannot
use remote system calls (for example, a vanilla universe job), and the
user does not enable Condor's File Transfer mechanism, the job will
\emph{only} run on the machine from which it was submitted.


%%%%%%%%%%%% 
\subsection{\label{sec:file-transfer}
Submitting Jobs Without a Shared File System:
Condor's File Transfer Mechanism} 
%%%%%%%%%%%%

\index{job!submission without a shared file system}
\index{shared file system!submission of jobs without one}
\index{file transfer mechanism}
\index{transferring files}

Condor works well without a shared file system.
The Condor file transfer mechanism permits the user to select which files are
transferred and under which circumstances.
Condor can transfer any files needed by a job from
the machine where the job was submitted into a
remote scratch directory on the machine where the
job is to be executed.
Condor executes the job
and transfers output back to the submitting machine.
The user specifies which files and directories to transfer,
and at what point the output files should be copied back to the
submitting machine.
This specification is done within the job's submit description file.

%%%%%%%%%%%% 
\subsubsection{Default Behavior across Condor Universes and Platforms}
%%%%%%%%%%%%

The default behavior of the file transfer mechanism
varies across the different Condor universes,
and it differs between Unix and Windows machines.

For jobs submitted under the \SubmitCmd{standard} universe,
the existence of a shared file system is not relevant.
Access to files (input and output) is handled through Condor's
remote system call mechanism.
The executable and checkpoint files are transferred automatically, when
needed. 
Therefore, the user does not need to change the submit description
file if there is no shared file system,
as the file transfer mechanism is not utilized.

For the \SubmitCmd{vanilla}, \SubmitCmd{java}, and \SubmitCmd{parallel}
universes, access to input files and the executable
through a shared file system is presumed as a default 
on jobs submitted from Unix machines.
If there is no shared file system, then Condor's file transfer
mechanism must be explicitly enabled.
When submitting a job from a Windows machine,
Condor presumes the opposite: no access to a shared file system.
It instead enables the file transfer mechanism by default.
Submission of a job might need to specify which files to
transfer, and/or when to transfer the output files back.

For the grid universe,
jobs are to be executed on remote machines, so there would never
be a shared file system between machines.
See section~\ref{sec:Condor-G} for more details.

For the scheduler universe,
Condor is only using the machine from which the job is submitted.
Therefore, the existence of a shared file system is not relevant.


%%%%%%%%%%%% 
\subsubsection{Specifying If and When to Transfer Files
\label{sec:file-transfer-if-when}}
%%%%%%%%%%%%

To enable the file transfer mechanism, place two commands
in the job's submit description file:
\SubmitCmd{should\_transfer\_files} and \SubmitCmd{when\_to\_transfer\_output}.
\index{submit commands!should\_transfer\_files}
\index{submit commands!when\_to\_transfer\_output}
In the common case, they will be set as:

\begin{verbatim}
  should_transfer_files = YES
  when_to_transfer_output = ON_EXIT
\end{verbatim}

Setting the \SubmitCmd{should\_transfer\_files} command explicitly
enables or disables the file transfer mechanism.
The command takes on one of three possible values:
\begin{enumerate}

\item \verb@YES@: Condor transfers both the executable and the file
defined by the \SubmitCmd{input} command from the machine where the job is
submitted to the remote machine where the job is to be executed.
The file defined by the \SubmitCmd{output} command as well as any files
created by the execution of the job are transferred back to the machine
where the job was submitted.
When they are transferred and the directory location of the files
is determined by the command \SubmitCmd{when\_to\_transfer\_output}.

\item \verb@IF_NEEDED@: Condor transfers files if the job is
matched with and to be executed on a machine in a
different \Attr{FileSystemDomain} than the
one the submit machine belongs to, the same as if 
\verb@should_transfer_files = YES@.
If the job is matched with a machine in the local \Attr{FileSystemDomain},
Condor will not transfer files and relies
on the shared file system.

\item \verb@NO@: Condor's file transfer mechanism is disabled. 

\end{enumerate}

The \SubmitCmd{when\_to\_transfer\_output} command tells Condor when output
files are to be transferred back to the submit machine.
The command takes on one of two possible values:

\begin{enumerate}
\item \verb@ON_EXIT@: Condor transfers the file defined by the
\SubmitCmd{output} command,
 as well as any other files in the remote scratch directory created by the job,
back to the submit machine only when the job exits on its own.

\item \verb@ON_EXIT_OR_EVICT@: Condor behaves the same as described
for the value \verb@ON_EXIT@ when the job exits on its own.
However, if, and each time the job is evicted from a machine,
\emph{files are transferred back at eviction time}.  The files that
are transferred back at eviction time may include intermediate files
that are not part of the final output of the job.  Before the job
starts running again, all of the files that were stored when the job
was last evicted are copied to the job's new remote scratch
directory.

The purpose of saving files at eviction time is to allow the job to
resume from where it left off.
This is similar to using the checkpoint feature of the standard universe,
but just specifying \verb@ON_EXIT_OR_EVICT@ is not enough to make a job 
capable of producing or utilizing checkpoints.
The job must be designed to save and restore its state
using the files that are saved at eviction time.

The files that are transferred back at eviction time are not stored in
the location where the job's final output will be written when the job exits.
Condor manages these files automatically,
so usually the only reason for a user to worry about them 
is to make sure that there is enough space to store them.
The files are stored on the submit machine in a temporary directory within the
directory defined by the configuration variable \MacroNI{SPOOL}. 
The directory is named using the \Attr{ClusterId} and \Attr{ProcId} job
ClassAd attributes.  The directory name takes the form:
\begin{verbatim}
   <X mod 10000>/<Y mod 10000>/cluster<X>.proc<Y>.subproc0
\end{verbatim}
where \verb@<X>@ is the value of \Attr{ClusterId}, and 
\verb@<Y>@ is the value of \Attr{ProcId}. 
As an example, if job 735.0 is evicted, it will produce the directory
\begin{verbatim}
   $(SPOOL)/735/0/cluster735.proc0.subproc0
\end{verbatim}

\end{enumerate}

There is no default value for \SubmitCmd{when\_to\_transfer\_output}.
If using the file transfer mechanism, 
this command must be defined.
However, if \SubmitCmd{when\_to\_transfer\_output} is specified in the submit
description file,
but \SubmitCmd{should\_transfer\_files} is not, Condor assumes a
value of \verb@YES@ for \SubmitCmd{should\_transfer\_files}.

\Note The combination of:
\begin{verbatim}
  should_transfer_files = IF_NEEDED
  when_to_transfer_output = ON_EXIT_OR_EVICT
\end{verbatim}
would produce undefined file access semantics.
Therefore, this combination is prohibited by \Condor{submit}.

When submitting from a Windows platform,
the file transfer mechanism is enabled by default.
If the two commands \SubmitCmd{when\_to\_transfer\_output} and
\SubmitCmd{should\_transfer\_files} are \emph{not} in the job's
submit description file, then Condor uses the values:

\begin{verbatim}
  should_transfer_files = YES
  when_to_transfer_output = ON_EXIT
\end{verbatim}


%%%%%%%%%%%% 
\subsubsection{Specifying What Files to Transfer}
%%%%%%%%%%%%

% transfers before execution
If the file transfer mechanism is enabled,
Condor will transfer the following files before the job
is run on a remote machine.
\begin{enumerate}
  \item the executable, as defined with the \SubmitCmd{executable} command
  \item the input, as defined with the \SubmitCmd{input} command
  \item any jar files, for the \SubmitCmd{java} universe,
  as defined with the \SubmitCmd{jar\_files} command
\end{enumerate}
If the job requires other input files,
the submit description file should utilize the
\SubmitCmd{transfer\_input\_files} command.
This comma-separated list specifies any other files or directories that Condor is to
transfer to the remote scratch directory,
to set up the execution environment for the job before it is run.
These files are placed in the same directory as the job's executable.
For example:

\begin{verbatim}
  should_transfer_files = YES
  when_to_transfer_output = ON_EXIT
  transfer_input_files = file1,file2 
\end{verbatim}
This example explicitly enables the file transfer mechanism,
and it transfers the executable, the file specified by the \SubmitCmd{input}
command, any jar files specified by the \SubmitCmd{jar\_files} command,
and files \File{file1} and \File{file2}.

% transfers back after execution
If the file transfer mechanism is enabled,
Condor will transfer the following files from the execute machine
back to the submit machine after the job exits.
\begin{enumerate}
  \item the output file, as defined with the \SubmitCmd{output} command
  \item the error file, as defined with the \SubmitCmd{error} command
  \item any files created by the job in the remote scratch directory;
this only occurs for jobs other than \SubmitCmd{grid}
universe, and for Condor-C \SubmitCmd{grid} universe jobs;
directories created by the job within the remote scratch directory
are ignored for this automatic detection of files to be transferred.
\end{enumerate}

A path given for \SubmitCmd{output} and \SubmitCmd{error} commands represents
a path on the submit machine.  If no path is specified, the directory
specified with \SubmitCmd{initialdir} is used, and if that is not specified,
the directory from which the job was submitted is used.
At the time the job is submitted, zero-length files are created
on the submit machine, at the given path for the files defined by the  
\SubmitCmd{output} and \SubmitCmd{error} commands.
This permits job submission failure, if these files cannot be written by
Condor.

To \emph{restrict} the output files 
or permit entire directory contents to be transferred,
specify the exact list with  \SubmitCmd{transfer\_output\_files}.
Delimit the list of file names, directory names, or paths with commas.
When this list is defined, and any of the files or directories
do not exist as the job exits,
Condor considers this an error, and places the job on hold.
When this list is defined, automatic detection of output files created by
the job is disabled.
Paths specified in this list refer to locations on the execute
machine.  
The naming and placement of files and directories relies on the
term \Term{base name}.  
By example, the path \File{a/b/c} has the base name \File{c}.
It is the file name or directory name with all directories
leading up to that name stripped off.
On the submit machine, the transferred files or directories
are named using only the base name.
Therefore, each output file or directory must have a different name,
even if they originate from different paths.

For \SubmitCmd{grid} universe jobs other than than Condor-C grid jobs,
files to be transferred 
(other than standard output and standard error)
must be specified using \SubmitCmd{transfer\_output\_files}
in the submit description file, because automatic detection of new files
created by the job does not take place.

Here are examples to promote understanding of what files and
directories are transferred, and how they are named after transfer.
Assume that the job produces the following structure within the
remote scratch directory:
\begin{verbatim}
      o1
      o2
      d1 (directory)
          o3
          o4 
\end{verbatim}

If the submit description file sets
\begin{verbatim}
   transfer_output_files = o1,o2,d1
\end{verbatim}
then transferred back to the submit machine will be
\begin{verbatim}
      o1
      o2
      d1 (directory)
          o3
          o4 
\end{verbatim}
Note that the directory \File{d1} and all its contents are specified,
and therefore transferred.  
If the directory \File{d1} is not created by the job before exit,
then the job is placed on hold. 
If the directory \File{d1} is created by the job before exit,
but is empty, this is not an error.

If, instead, the submit description file sets
\begin{verbatim}
   transfer_output_files = o1,o2,d1/o3
\end{verbatim}
then transferred back to the submit machine will be
\begin{verbatim}
      o1
      o2
      o3
\end{verbatim}
Note that only the base name is used in the naming and placement
of the file specified with \File{d1/o3}.


%%%%%%%%%%%%
\subsubsection{File Paths for File Transfer}
%%%%%%%%%%%%

% Note: it might be nice to get the initialdir entry in
% the index to refer to something in here.

% Note: a Windows-based example would be good, too.

The file transfer mechanism specifies file names and/or paths on
both the file system of the submit machine and on the
file system of the execute machine.
Care must be taken to know which machine, submit or execute,
is utilizing the file name and/or path. 

Files in the \SubmitCmd{transfer\_input\_files} command
are specified as they are accessed on the submit machine.
The job, as it executes, accesses files as they are
found on the execute machine.

There are three ways to specify files and paths
for \SubmitCmd{transfer\_input\_files}:
\begin{enumerate}
\item Relative to the current working directory as the job is submitted,
if the submit command \SubmitCmd{initialdir} is not specified.
\item Relative to the initial directory, if the submit command 
\SubmitCmd{initialdir} is specified.
\item Absolute.
\end{enumerate}

Before executing the program, Condor copies the
executable, an input file as specified
by the submit command \SubmitCmd{input},
along with any input files specified 
by \SubmitCmd{transfer\_input\_files}.
All these files are placed into
a remote scratch directory on the execute machine,
in which the program runs.
Therefore,
the executing program must access input files relative to its
working directory.
Because all files and directories listed for transfer are placed into a single,
flat directory,
inputs must be uniquely named to
avoid collision when transferred.
A collision causes the last file in the list to
overwrite the earlier one.

Both relative and absolute paths may be used in
\SubmitCmd{transfer\_output\_files}.  Relative paths are relative to
the job's remote scratch directory on the execute machine.
When the files and directories are copied back to the submit machine, they
are placed in the job's initial working directory as the base name of
the original path.  An alternate name or path may be specified by using
\SubmitCmd{transfer\_output\_remaps}.

A job may create files outside the remote scratch directory
but within the file system of the execute machine,
in a directory such as \File{/tmp},
if this directory is guaranteed to exist and be
accessible on all possible execute machines.
However,
Condor will not automatically
transfer such files back after execution completes, nor will it clean
up these files.

Here are several examples to illustrate the use of file transfer.
The program executable is called \Prog{my\_program},
and it uses three command-line arguments as it executes: 
two input file names and an output file name.
The program executable and the submit description file 
for this job are located in directory
\File{/scratch/test}. 

Here is the directory tree as it exists on the submit machine,
for all the examples:
\begin{verbatim}
/scratch/test (directory)
      my_program.condor (the submit description file)
      my_program (the executable)
      files (directory)
          logs2 (directory)
          in1 (file)
          in2 (file)
      logs (directory)
\end{verbatim}

%--------------------------
\begin{description}
\item[Example 1]

This first example explicitly transfers input files.
These input files to be transferred
are specified relative to the directory where the job is submitted.
An output file specified in the \SubmitCmd{arguments} command, \File{out1},
is created when the job is executed.
It will be transferred back into the directory \File{/scratch/test}.

\footnotesize
\begin{verbatim}
# file name:  my_program.condor
# Condor submit description file for my_program
Executable      = my_program
Universe        = vanilla
Error           = logs/err.$(cluster)
Output          = logs/out.$(cluster)
Log             = logs/log.$(cluster)

should_transfer_files = YES
when_to_transfer_output = ON_EXIT
transfer_input_files = files/in1,files/in2

Arguments       = in1 in2 out1
Queue
\end{verbatim}
\normalsize

The log file is written on the submit machine, and is not involved
with the file transfer mechanism.
%--------------------------
\item[Example 2]

This second example is identical to Example 1,
except that absolute paths to the input files are specified,
instead of relative paths to the input files.

\footnotesize
\begin{verbatim}
# file name:  my_program.condor
# Condor submit description file for my_program
Executable      = my_program
Universe        = vanilla
Error           = logs/err.$(cluster)
Output          = logs/out.$(cluster)
Log             = logs/log.$(cluster)

should_transfer_files = YES
when_to_transfer_output = ON_EXIT
transfer_input_files = /scratch/test/files/in1,/scratch/test/files/in2

Arguments       = in1 in2 out1
Queue
\end{verbatim}
\normalsize

%--------------------------
\item[Example 3]

This third example illustrates the use of the 
submit command \SubmitCmd{initialdir}, and its effect
on the paths used for the various files.
The expected location of the 
executable is not affected by the 
\SubmitCmd{initialdir} command.
All other files
(specified by \SubmitCmd{input}, \SubmitCmd{output}, \SubmitCmd{error},
\SubmitCmd{transfer\_input\_files},
as well as files modified or created by the job
and automatically transferred back)
are located relative to the specified \SubmitCmd{initialdir}.
Therefore, the output file, \File{out1},
will be placed in the \verb@files@ directory.
Note that the \File{logs2} directory
exists to make this example work correctly.

\footnotesize
\begin{verbatim}
# file name:  my_program.condor
# Condor submit description file for my_program
Executable      = my_program
Universe        = vanilla
Error           = logs2/err.$(cluster)
Output          = logs2/out.$(cluster)
Log             = logs2/log.$(cluster)

initialdir      = files

should_transfer_files = YES
when_to_transfer_output = ON_EXIT
transfer_input_files = in1,in2

Arguments       = in1 in2 out1
Queue
\end{verbatim}
\normalsize

%--------------------------
\item[Example 4 -- Illustrates an Error]

This example illustrates a job that will fail.
The files specified using the
\SubmitCmd{transfer\_input\_files} command work
correctly (see Example 1).
However,
relative paths to files in the
\SubmitCmd{arguments} command
cause the executing program to fail.
The file system on the submission side may utilize
relative paths to files,
however those files are placed into the single,
flat, remote scratch directory on the execute machine.

\footnotesize
\begin{verbatim}
# file name:  my_program.condor
# Condor submit description file for my_program
Executable      = my_program
Universe        = vanilla
Error           = logs/err.$(cluster)
Output          = logs/out.$(cluster)
Log             = logs/log.$(cluster)

should_transfer_files = YES
when_to_transfer_output = ON_EXIT
transfer_input_files = files/in1,files/in2

Arguments       = files/in1 files/in2 files/out1
Queue
\end{verbatim}
\normalsize

This example fails with the following error:
\footnotesize
\begin{verbatim}
err: files/out1: No such file or directory.
\end{verbatim}
\normalsize

%--------------------------
\item[Example 5 -- Illustrates an Error]

As with Example 4,
this example illustrates a job that will fail.
The executing program's use of 
absolute paths cannot work.

\footnotesize
\begin{verbatim}
# file name:  my_program.condor
# Condor submit description file for my_program
Executable      = my_program
Universe        = vanilla
Error           = logs/err.$(cluster)
Output          = logs/out.$(cluster)
Log             = logs/log.$(cluster)

should_transfer_files = YES
when_to_transfer_output = ON_EXIT
transfer_input_files = /scratch/test/files/in1, /scratch/test/files/in2

Arguments = /scratch/test/files/in1 /scratch/test/files/in2 /scratch/test/files/out1
Queue
\end{verbatim}
\normalsize

The job fails with the following error:
\footnotesize
\begin{verbatim}
err: /scratch/test/files/out1: No such file or directory.
\end{verbatim}
\normalsize

%--------------------------
\item[Example 6]

This example illustrates a case
where the executing program creates an output file in a directory
other than within the remote scratch directory that the 
program executes within.
The file creation may or may not cause an error,
depending on the existence and permissions
of the directories on the remote file system.

The output file \File{/tmp/out1} is transferred back to the job's
initial working directory as \File{/scratch/test/out1}.

\footnotesize
\begin{verbatim}
# file name:  my_program.condor
# Condor submit description file for my_program
Executable      = my_program
Universe        = vanilla
Error           = logs/err.$(cluster)
Output          = logs/out.$(cluster)
Log             = logs/log.$(cluster)

should_transfer_files = YES
when_to_transfer_output = ON_EXIT
transfer_input_files = files/in1,files/in2
transfer_output_files = /tmp/out1

Arguments       = in1 in2 /tmp/out1
Queue
\end{verbatim}
\normalsize

\end{description}

%%%%%%%%%%%%
\subsubsection{Behavior for Error Cases}
%%%%%%%%%%%%
This section describes Condor's behavior for some error cases
in dealing with the transfer of files.
\begin{description}
\item[Disk Full on Execute Machine]
  When transferring any files from the submit machine to the remote scratch
  directory,
  if the disk is full on the execute machine,
  then the job is place on hold.
\item[Error Creating Zero-Length Files on Submit Machine]
  As a job is submitted, Condor creates zero-length files as placeholders
  on the submit machine for the files defined by 
  \SubmitCmd{output} and \SubmitCmd{error}.
  If these files cannot be created, then job submission fails.

  This job submission failure avoids having the job run to completion,
  only to be unable to transfer the job's output due to permission errors.
\item[Error When Transferring Files from Execute Machine to Submit Machine]
  When a job exits, or potentially when a job is evicted from an execute
  machine, one or more files may be transferred from the execute machine
  back to the machine on which the job was submitted.

  During transfer, if any of the following three similar types of errors occur,
  the job is put on hold as the error occurs.
  \begin{enumerate}
  \item If the file cannot be opened on the submit machine, for example
    because the system is out of inodes.
  \item If the file cannot be written on the submit machine, for example
    because the permissions do not permit it.
  \item If the write of the file on the submit machine fails, for example
    because the system is out of disk space.
  \end{enumerate}
\end{description}

%%%%%%%%%%%%
\subsubsection{File Transfer Using a URL \label{sec:file-transfer-by-URL}}
%%%%%%%%%%%%
\index{file transfer mechanism!input file specified by URL}
\index{file transfer mechanism!output file(s) specified by URL}
\index{URL file transfer}

Instead of file transfer that goes only between the submit machine
and the execute machine,
Condor has the ability to transfer files from a location specified
by a URL for a job's input file,
or from the execute machine to a location specified by a URL
for a job's output file(s).
This capability requires administrative set up, 
as described in section~\ref{sec:URL-transfer}.

The transfer of an input file is restricted to
vanilla and vm universe jobs only.
Condor's file transfer mechanism must be enabled.
Therefore, the submit description file for the job will define both
\SubmitCmd{should\_transfer\_files} and \SubmitCmd{when\_to\_transfer\_output}.
In addition, the URL for any files specified with a URL are
given in the \SubmitCmd{transfer\_input\_files} command.
An example portion of the submit description file for a job
that has a single file specified with a URL:

\footnotesize
\begin{verbatim}
should_transfer_files = YES
when_to_transfer_output = ON_EXIT
transfer_input_files = http://www.full.url/path/to/filename
\end{verbatim}
\normalsize

The destination file is given by the file name within the URL. 

For the transfer of the entire contents of the output sandbox,
which are all files that the job creates or modifies,
Condor's file transfer mechanism must be enabled.
In this sample portion of the submit description file,
the first two commands explicitly enable file transfer,
and the added \SubmitCmd{output\_destination} command provides
both the protocol to be used and the destination of the transfer.
\footnotesize
\begin{verbatim}
should_transfer_files = YES
when_to_transfer_output = ON_EXIT
output_destination = urltype://path/to/destination/directory
\end{verbatim}
\normalsize
Note that with this feature, no files are transferred back to the 
submit machine.  
This does not interfere with the streaming of output. 

If only a subset of the output sandbox should be transferred,
the subset is specified by further adding a submit command of the form:
\footnotesize
\begin{verbatim}
transfer_output_files = file1, file2
\end{verbatim}
\normalsize

%%%%%%%%%%%% 
\subsubsection{Requirements and Rank for File Transfer}
%%%%%%%%%%%%

\index{submit commands!requirements}
The \Attr{requirements} expression for a job must depend
on the \verb@should_transfer_files@ command.
The job must specify the correct logic to ensure that the job is matched
with a resource that meets the file transfer needs.
If no \Attr{requirements} expression is in the submit description file,
or if the expression specified does not refer to the
attributes listed below, \Condor{submit} adds an
appropriate clause to the \Attr{requirements} expression for the job.
\Condor{submit} appends these clauses with a logical AND, \verb@&&@,
to ensure that the proper conditions are met.
Here are the default clauses corresponding to the different values of
\verb@should_transfer_files@:

\begin{enumerate}

\item 
\verb@should_transfer_files = YES@ results in the addition of
the clause \verb@(HasFileTransfer)@.
  If the job is always going to transfer files, it is required to 
  match with a machine that has the capability to transfer files.

\item 
\verb@should_transfer_files = NO@ results in the addition of
  \verb@(TARGET.FileSystemDomain == MY.FileSystemDomain)@.
  In addition, Condor automatically adds the
  \Attr{FileSystemDomain} attribute to the job ClassAd, with whatever
  string is defined for the \Condor{schedd} to which the job is
  submitted.
  If the job is not using the file transfer mechanism, Condor assumes
  it will need a shared file system, and therefore, a machine in the
  same \Attr{FileSystemDomain} as the submit machine.

\item \verb@should_transfer_files = IF_NEEDED@ results in the addition of
\footnotesize
\begin{verbatim}
  (HasFileTransfer || (TARGET.FileSystemDomain == MY.FileSystemDomain))
\end{verbatim}
\normalsize
  If Condor will optionally transfer files, it must require
  that the machine is \emph{either} capable of transferring files
  \emph{or} in the same file system domain.

\end{enumerate}

To ensure that the job is matched to a machine with enough local disk
space to hold all the transferred files, Condor automatically adds the
\Attr{DiskUsage} job attribute.
This attribute includes the total
size of the job's executable and all input files to be transferred.
Condor then adds an additional clause to the \Attr{Requirements}
expression that states that the remote machine must have at least
enough available disk space to hold all these files:
\begin{verbatim}
  && (Disk >= DiskUsage)
\end{verbatim}

\index{submit commands!rank}
If \verb@should_transfer_files = IF_NEEDED@ and the job prefers
to run on a machine in the local file system domain
over transferring files,
but is still willing to allow the job to run remotely and transfer files,
the \Attr{Rank} expression works well.  Use:

\footnotesize
\begin{verbatim}
rank = (TARGET.FileSystemDomain == MY.FileSystemDomain)
\end{verbatim}
\normalsize

The \Attr{Rank} expression is a floating point value,
so if other items are considered in ranking the possible machines this job
may run on, add the items:

\footnotesize
\begin{verbatim}
Rank = kflops + (TARGET.FileSystemDomain == MY.FileSystemDomain)
\end{verbatim}
\normalsize

The value of \Attr{kflops} can vary widely among machines,
so this \Attr{Rank} expression will likely not do as it intends.
To place emphasis on the job running in the same file system domain,
but still consider floating point speed among the machines 
in the file system domain,
weight the part of the expression that is matching the file system domains.
For example: 

\footnotesize
\begin{verbatim}
Rank = kflops + (10000 * (TARGET.FileSystemDomain == MY.FileSystemDomain))
\end{verbatim}
\normalsize

%%%%%%%%%%%% 

%%%%%%%%%%%% 
\subsection{Environment Variables}
%%%%%%%%%%%% 

\index{environment variables}
\index{execution environment}
The environment under which a job executes often contains
information that is potentially useful to the job.
HTCondor allows a user to both set and reference environment
variables for a job or job cluster.

Within a submit description file, the user may define environment
variables for the job's environment by using the 
\Opt{environment} command.
See within the \Condor{submit} manual page at
section~\ref{man-condor-submit-environment} for more details about this command.

The submitter's entire environment can be copied into the job
ClassAd for the job at job submission.
The \SubmitCmd{getenv} command within the submit description file
does this,
as described at section~\ref{man-condor-submit-getenv}.

If the environment is set with the \SubmitCmd{environment} command \emph{and}
\SubmitCmd{getenv} is also set to true, values specified with
\SubmitCmd{environment} override values in the submitter's environment,
regardless of the order of the \SubmitCmd{environment} and \SubmitCmd{getenv}
commands.

Commands within the submit description file may reference the
environment variables of the submitter as a job is submitted.
Submit description file commands use \verb@$ENV(EnvironmentVariableName)@
to reference the value of an environment variable.

HTCondor sets several additional environment variables for each executing
job that may be useful for the job to reference.

\begin{itemize}
\item \Env{\_CONDOR\_SCRATCH\_DIR}
\index{\_CONDOR\_SCRATCH\_DIR environment variable}
\index{environment variables!\_CONDOR\_SCRATCH\_DIR}
 gives the directory
where the job may place temporary data files. 
This directory is unique for every job that is run,
and its contents are deleted by HTCondor
when the job stops running on a machine, no matter how the job completes.

\item \Env{\_CONDOR\_SLOT}
\index{\_CONDOR\_SLOT environment variable}
\index{environment variables!\_CONDOR\_SLOT}
gives the name of the slot (for SMP machines), on which the job is run.
On machines with only a single slot, the value of this variable will be
\verb@1@, just like the \AdAttr{SlotID} attribute in the machine's
ClassAd.
This setting is available in all universes.
See section~\ref{sec:Configuring-SMP} for more details about SMP
machines and their configuration.

\item \Env{CONDOR\_VM}
\index{CONDOR\_VM environment variable}
\index{environment variables!CONDOR\_VM}
equivalent to \Env{\_CONDOR\_SLOT} described above, except that it is
only available in the standard universe.
\Note As of HTCondor version 6.9.3, this environment variable is no longer
used.
It will only be defined if the \Macro{ALLOW\_VM\_CRUFT} configuration
variable is set to \Expr{True}.

\item \Env{X509\_USER\_PROXY}
\index{X509\_USER\_PROXY environment variable}
\index{environment variables!X509\_USER\_PROXY}
gives the full path to the X.509 user proxy file if one is
associated with the job.  Typically, a user will specify
\SubmitCmd{x509userproxy} in the submit description file.
This setting is currently available in the
local, java, and vanilla universes.

\item \Env{\_CONDOR\_JOB\_AD}
\index{\_CONDOR\_JOB\_AD environment variable}
\index{environment variables!\_CONDOR\_JOB\_AD}
is the path to a file in the job's scratch directory which contains
the job ad for the currently running job.  The job ad is current
as of the start of the job, but is not updated during the running
of the job.  The job may read attributes and their values out of
this file as it runs, but any changes will not be acted on in any
way by HTCondor.  The format is the same as the output of the
\Condor{q}  \Opt{-l} command.  This environment variable may be particularly
useful in a USER\_JOB\_WRAPPER.

\item \Env{\_CONDOR\_MACHINE\_AD}
\index{\_CONDOR\_MACHINE\_AD environment variable}
\index{environment variables!\_CONDOR\_MACHINE\_AD}
is the path to a file in the job's scratch directory which contains
the machine ad for the slot the currently running job is using.  
The machine ad is current as of the start of the job, but is not updated during the running
of the job.  The format is the same as the output of the
\Condor{status}  \Opt{-l} command.

\item \Env{\_CONDOR\_JOB\_IWD}
\index{\_CONDOR\_JOB\_IWD environment variable}
\index{environment variables!\_CONDOR\_JOB\_IWD}
is the path to the initial working directory the job was born with.

\item \Env{\_CONDOR\_WRAPPER\_ERROR\_FILE}
\index{\_CONDOR\_WRAPPER\_ERROR\_FILE environment variable}
\index{environment variables!\_CONDOR\_WRAPPER\_ERROR\_FILE}
is only set when the administrator has installed a USER\_JOB\_WRAPPER.
If this file exists, HTCondor assumes that the job wrapper has failed
and copies the contents of the file to the StarterLog for the administrator
to debug the problem.

\end{itemize}



%%%%%%%%%%%% 
\subsection{Heterogeneous Submit: Execution on Differing Architectures} 
%%%%%%%%%%%%

\index{job!heterogeneous submit}
\index{running a job!on a different architecture}
\index{heterogeneous pool!submitting a job to}
If executables are available for the different platforms of machines
in the HTCondor pool,
HTCondor can be allowed the choice of a larger number of machines
when allocating a machine for a job.
Modifications to the submit description file allow this choice
of platforms.

A simplified example is a cross submission.
An executable is available for one platform, but
the submission is done from a different platform.
Given the correct executable, the \AdAttr{requirements} command in
the submit description file specifies the target architecture.
For example, an executable compiled for a 32-bit Intel processor
running  Windows Vista, submitted
from an Intel architecture running Linux would add the 
\AdAttr{requirement}
\begin{verbatim}
  requirements = Arch == "INTEL" && OpSys == "WINDOWS"
\end{verbatim}
Without this \AdAttr{requirement}, \Condor{submit}
will assume that the program is to be executed on
a machine with the same platform as the machine where the job
is submitted.

Cross submission works for all universes except \Expr{scheduler} and
\Expr{local}.
See section~\ref{sec:Grid-Matchmaking} for how matchmaking works in the
\Expr{grid} universe.
The burden is on the user to both obtain and specify
the correct executable for the target architecture.
To list the architecture and operating systems of the machines
in a pool, run \Condor{status}.

%%%%%%%%%%%% 
\subsubsection{Vanilla Universe Example for Execution on Differing Architectures} 
%%%%%%%%%%%%

A more complex example of a heterogeneous submission
occurs when a job may be executed on
many different architectures to gain full
use of a diverse architecture and operating system pool.
If the executables are available for the different architectures,
then a modification to the submit description file
will allow HTCondor to choose an executable after an
available machine is chosen.

A special-purpose Machine Ad substitution macro can be used in
string
attributes in the submit description file.
The macro has the form
\begin{verbatim}
  $$(MachineAdAttribute)
\end{verbatim}
The \$\$() informs HTCondor to substitute the requested 
\AdAttr{MachineAdAttribute} 
from the machine where the job will be executed.

An example of the heterogeneous job submission
has executables available for two platforms:
RHEL 3 on both 32-bit and 64-bit Intel processors.
This example uses \Prog{povray}
to render images using a popular free rendering engine.

The substitution macro chooses a specific executable after
a platform for running the job is chosen.
These executables must therefore be named based on the
machine attributes that describe a platform.
The executables named \begin{verbatim}
  povray.LINUX.INTEL
  povray.LINUX.X86_64
\end{verbatim}
will work correctly for the macro
\begin{verbatim}
  povray.$$(OpSys).$$(Arch)
\end{verbatim}

The executables or links to executables with this name
are placed into the initial working directory so that they may be
found by HTCondor. 
A submit description file that queues three jobs for this example:

\begin{verbatim}
  ####################
  #
  # Example of heterogeneous submission
  #
  ####################

  universe     = vanilla
  Executable   = povray.$$(OpSys).$$(Arch)
  Log          = povray.log
  Output       = povray.out.$(Process)
  Error        = povray.err.$(Process)

  Requirements = (Arch == "INTEL" && OpSys == "LINUX") || \
                 (Arch == "X86_64" && OpSys =="LINUX") 

  Arguments    = +W1024 +H768 +Iimage1.pov
  Queue 

  Arguments    = +W1024 +H768 +Iimage2.pov
  Queue 

  Arguments    = +W1024 +H768 +Iimage3.pov
  Queue 
\end{verbatim}

These jobs are submitted to the vanilla universe
to assure that once a job is started on a specific platform,
it will finish running on that platform.
Switching platforms in the middle of job execution cannot
work correctly.

There are two common errors made with the substitution macro.
The first is the use of a non-existent \AdAttr{MachineAdAttribute}.
If the specified \AdAttr{MachineAdAttribute} does not
exist in the machine's ClassAd, then HTCondor will place
the job in the held state until the problem is resolved.

The second common error occurs due to an incomplete job set up.
For example, the submit description file given above specifies
three available executables.
If one is missing, HTCondor reports back that an
executable is missing when it happens to match the
job with a resource that requires the missing binary.

%%%%%%%%%%%% 
\subsubsection{Standard Universe Example for Execution on Differing Architectures} 
%%%%%%%%%%%%

Jobs submitted to the standard universe may produce checkpoints.
A checkpoint can then be used to start up and continue execution
of a partially completed job.
For a partially completed job, the checkpoint and the job are specific
to a platform.
If migrated to a different machine, correct execution requires that
the platform must remain the same.

In previous versions of HTCondor, the author of the heterogeneous
submission file would need to write extra policy expressions in the
\AdAttr{requirements} expression to force HTCondor to choose the
same type of platform when continuing a checkpointed job.
However, since it is needed in the common case, this
additional policy is now automatically added
to the \AdAttr{requirements} expression.
The additional expression is added
provided the user does not use
\AdAttr{CkptArch} in the \AdAttr{requirements} expression.
HTCondor will remain backward compatible for those users who have explicitly
specified \AdAttr{CkptRequirements}--implying use of \AdAttr{CkptArch},
in their \AdAttr{requirements} expression.

The expression added when the attribute \AdAttr{CkptArch} is not specified 
will default to

\footnotesize
\begin{verbatim}
  # Added by HTCondor
  CkptRequirements = ((CkptArch == Arch) || (CkptArch =?= UNDEFINED)) && \
                      ((CkptOpSys == OpSys) || (CkptOpSys =?= UNDEFINED))

  Requirements = (<user specified policy>) && $(CkptRequirements)
\end{verbatim}
\normalsize

The behavior of the \AdAttr{CkptRequirements} expressions and its addition to
\AdAttr{requirements} is as follows.
The \AdAttr{CkptRequirements} expression guarantees correct operation
in the two possible cases for a job.
In the first case, the job has not produced a checkpoint.
The ClassAd attributes \Attr{CkptArch} and \Attr{CkptOpSys}
will be undefined, and therefore the meta operator (\verb@=?=@)
evaluates to true.
In the second case, the job has produced a checkpoint.
The Machine ClassAd is restricted to require further execution
only on a machine of the same platform.
The attributes \Attr{CkptArch} and \Attr{CkptOpSys}
will be defined, ensuring that the platform chosen for further
execution will be the same as the one used just before the
checkpoint.

Note that this restriction of platforms also applies to platforms where
the executables are binary compatible.

The complete submit description file for this example:

\begin{verbatim}
  ####################
  #
  # Example of heterogeneous submission
  #
  ####################

  universe     = standard
  Executable   = povray.$$(OpSys).$$(Arch)
  Log          = povray.log
  Output       = povray.out.$(Process)
  Error        = povray.err.$(Process)

  # HTCondor automatically adds the correct expressions to insure that the
  # checkpointed jobs will restart on the correct platform types.
  Requirements = ( (Arch == "INTEL" && OpSys == "LINUX") || \
                 (Arch == "X86_64" && OpSys == "LINUX") )

  Arguments    = +W1024 +H768 +Iimage1.pov
  Queue 

  Arguments    = +W1024 +H768 +Iimage2.pov
  Queue 

  Arguments    = +W1024 +H768 +Iimage3.pov
  Queue 
\end{verbatim}


%%%%%%%%%%%% 
\subsubsection{Vanilla Universe Example for Execution on Differing Operating Systems} 
%%%%%%%%%%%%

The addition of several related OpSys attributes assists in selection of specific operating systems and versions in heterogeneous pools.


\begin{verbatim}
  ####################
  #
  # Example of submission targeting RedHat platforms in a heterogeneous Linux pool
  #
  ####################

  universe     = vanilla
  Executable   = /bin/date
  Log          = distro.log
  Output       = distro.out
  Error        = distro.err

  Requirements = (OpSysName == "RedHat")

  Queue
\end{verbatim}


\begin{verbatim}
  ####################
  #
  # Example of submission targeting RedHat 6 platforms in a heterogeneous Linux pool
  #
  ####################

  universe     = vanilla
  Executable   = /bin/date
  Log          = distro.log
  Output       = distro.out
  Error        = distro.err

  Requirements = ( OpSysName == "RedHat" && OpSysMajorVersion == 6)

  Queue
\end{verbatim}


Here is a more compact way to specify a RedHat 6 platform.

\begin{verbatim}
  ####################
  #
  # Example of submission targeting RedHat 6 platforms in a heterogeneous Linux pool
  #
  ####################

  universe     = vanilla
  Executable   = /bin/date
  Log          = distro.log
  Output       = distro.out
  Error        = distro.err

  Requirements = ( OpSysAndVer == "RedHat6")

  Queue
\end{verbatim}

%%%%%%%%%%%%%%%%%%%%%%%%%%%%%%%%%%%%%%%%%%
%%%%%%%%%%%%%%%%%%%%%%%%%%%%%%%%%%%%%%%%%%%%%%%%%%%%%%%%%%%%%%%%%%%%%%
\subsection{\label{sec:Submit-Interactive}Interactive Jobs}
%%%%%%%%%%%%%%%%%%%%%%%%%%%%%%%%%%%%%%%%%%%%%%%%%%%%%%%%%%%%%%%%%%%%%%
\index{job!interactive}
\index{interactive jobs}

An \Term{interactive job} is a Condor job that is provisioned and
scheduled like any other vanilla universe Condor job 
onto an execute machine within the pool.
The result of a running interactive job is a shell prompt 
issued on the execute machine where the job runs.
The user that submitted the interactive job may then use the
shell as desired,
perhaps to interactively run an instance of what is to become
a Condor job.
This might aid in checking that the set up and execution
environment are correct,
or it might provide information on the RAM or disk space needed.
This job (shell) continues until the user logs out or any other
policy implementation causes the job to stop running.
A useful feature of the interactive job is that the users and jobs
are accounted for within Condor's scheduling and priority system.

Neither the submit nor the execute host for interactive
jobs may be on Windows platforms. 

The current working directory of the shell will be the
initial working directory of the running job.
The shell type will be the default for the user that submits
the job.
At the shell prompt, X11 forwarding is enabled.

Each interactive job will have a job ClassAd attribute of 
\begin{verbatim}
  InteractiveJob = True
\end{verbatim}

Submission of an interactive job specifies the option \Opt{-interactive}
on the \Condor{submit} command line.

A submit description file may be specified for this interactive job.
Within this submit description file, 
a specification of these 5 commands will be either ignored or altered:
\begin{enumerate}
\item \SubmitCmd{executable}
\item \SubmitCmd{transfer\_executable}
\item \SubmitCmd{arguments}
\item \SubmitCmd{universe}.  The interactive job is a vanilla universe job. 
\item \SubmitCmd{queue <n>}.  In this case the value of \SubmitCmd{<n>} is
ignored; exactly one interactive job is queued.
\end{enumerate}
The submit description file may specify anything else needed for
the interactive job, such as files to transfer.

If \emph{no} submit description file is specified for the job,
a default one is utilized as identified by the value of the
configuration variable \Macro{INTERACTIVE\_SUBMIT\_FILE}.

Here are examples of situations where interactive jobs may be
of benefit.
\begin{itemize}
\item An application that cannot be batch processed might be run
as an interactive job.
Where input or output cannot be captured in a file and the
executable may not be modified,
the interactive nature of the job may still be run on a pool
machine, and within the purview of Condor.
\item A pool machine with specialized hardware that requires
interactive handling can be scheduled with an interactive
job that utilizes the hardware.
\item The debugging and set up of complex jobs or environments
may benefit from an interactive session.
This interactive session provides the opportunity to run scripts 
or applications, 
and as errors are identified, 
they can be corrected on the spot.
\item Development may have an interactive nature,
and proceed more quickly when done on a pool machine.
It may also be that the development platforms required
reside within Condor's purview as execute hosts. 
\end{itemize}

%%%%%%%%%%%%%%%%%%%%%%%%%%%%%%%%%%%%%%%%%%

%%%%%%%%%%%%%%%%%%%%%%%%%%%%%%%%%%%%%%%%%%
\section{Managing a Job}
This section provides a brief summary of what can be done once jobs
are submitted. The basic mechanisms for monitoring a job are
introduced, but the commands are discussed briefly.
You are encouraged to
look at the man pages of the commands referred to (located in
Chapter~\ref{command-reference} beginning on
page~\pageref{command-reference}) for more information. 

When jobs are submitted, Condor will attempt to find resources
to run the jobs. 
A list of all those with jobs submitted
may be obtained through \Condor{status}
\index{Condor commands!condor\_status}
with the 
\Arg{-submitters} option. 
An example of this would yield output similar to:
\begin{verbatim}
%  condor_status -submitters

Name                 Machine      Running IdleJobs HeldJobs

ballard@cs.wisc.edu  bluebird.c         0       11        0
nice-user.condor@cs. cardinal.c         6      504        0
wright@cs.wisc.edu   finch.cs.w         1        1        0
jbasney@cs.wisc.edu  perdita.cs         0        0        5

                           RunningJobs           IdleJobs           HeldJobs

 ballard@cs.wisc.edu                 0                 11                  0
 jbasney@cs.wisc.edu                 0                  0                  5
nice-user.condor@cs.                 6                504                  0
  wright@cs.wisc.edu                 1                  1                  0

               Total                 7                516                  5
\end{verbatim}

\subsection{Checking on the progress of jobs}
At any time, you can check on the status of your jobs with the \Condor{q}
command.
\index{Condor commands!condor\_q}
This command displays the status of all queued jobs.
An example of the output from \Condor{q} is
\begin{verbatim}
%  condor_q

-- Submitter: froth.cs.wisc.edu : <128.105.73.44:33847> : froth.cs.wisc.edu
 ID      OWNER            SUBMITTED    CPU_USAGE ST PRI SIZE CMD               
 125.0   jbasney         4/10 15:35   0+00:00:00 I  -10 1.2  hello.remote      
 127.0   raman           4/11 15:35   0+00:00:00 R  0   1.4  hello             
 128.0   raman           4/11 15:35   0+00:02:33 I  0   1.4  hello             

3 jobs; 2 idle, 1 running, 0 held

\end{verbatim} 
This output contains many columns of information about the
queued jobs.
\index{status!of queued jobs}
The \verb@ST@ column (for status) shows the status of
current jobs in the queue. An \verb@R@ in the status column
means the the job is currently running.
An \verb@I@ stands for idle. The job is not running right
now, because it is waiting for a machine to become available. 
The status
\verb@H@ is the hold state. In the hold state,
the job will not be scheduled to
run until it is released (see condor\_hold and condor\_release man pages).
Older versions of Condor used a
\verb@U@ in the status column to stand for unexpanded.
In this state,
a job has never 
checkpointed and when it starts running, it will start running from the
beginning.
Newer versions of Condor do not use the \verb@U@ state.

The \verb@CPU_USAGE@ time reported for a job is the time that has been
committed to the job.  It is not updated for a job until
the job checkpoints. At that time, the job has made guaranteed forward 
progress.  Depending upon how the site administrator configured the pool,
several hours may pass between checkpoints, so do not worry if you do
not observe the \verb@CPU_USAGE@ entry changing by the hour.
Also note that this is actual CPU
time as reported by the operating system; it is not time as
measured by a wall clock.

Another useful method of tracking the progress of jobs is through the
user log.  If you have specified a \AdAttr{log} command in 
your submit file, the progress of the job may be followed by viewing the
log file.  Various events such as execution commencement, checkpoint, eviction 
and termination are logged in the file.
Also logged is the time at which the event occurred.

% Karen's note:  degraded performance where?
When your job begins to run, Condor starts up a \Condor{shadow} process
\index{condor\_shadow}
\index{remote system call!condor\_shadow}
on the submit machine.  The shadow process is the mechanism by which the
remotely executing jobs can access the environment from which it was
submitted, such as input and output files.  

It is normal for a machine which has submitted hundreds of jobs to have 
hundreds of shadows running on the machine.  Since the text segments of 
all these processes is the same, the load on the submit machine is usually 
not significant.  If, however, you notice degraded performance, you can limit 
the number of jobs that can run simultaneously through the 
\Macro{MAX\_JOBS\_RUNNING} configuration parameter.  Please talk to your 
system administrator for the necessary configuration change.

You can also find all the machines that are running your job through the
\Condor{status} command.
\index{Condor commands!condor\_status}
For example, to find all the machines that are
running jobs submitted by ``breach@cs.wisc.edu,'' type:
\begin{verbatim}
%  condor_status -constraint 'RemoteUser == "breach@cs.wisc.edu"'

Name       Arch     OpSys        State      Activity   LoadAv Mem  ActvtyTime

alfred.cs. INTEL    SOLARIS251   Claimed    Busy       0.980  64    0+07:10:02
biron.cs.w INTEL    SOLARIS251   Claimed    Busy       1.000  128   0+01:10:00
cambridge. INTEL    SOLARIS251   Claimed    Busy       0.988  64    0+00:15:00
falcons.cs INTEL    SOLARIS251   Claimed    Busy       0.996  32    0+02:05:03
happy.cs.w INTEL    SOLARIS251   Claimed    Busy       0.988  128   0+03:05:00
istat03.st INTEL    SOLARIS251   Claimed    Busy       0.883  64    0+06:45:01
istat04.st INTEL    SOLARIS251   Claimed    Busy       0.988  64    0+00:10:00
istat09.st INTEL    SOLARIS251   Claimed    Busy       0.301  64    0+03:45:00
...
\end{verbatim}
To find all the machines that are running any job at all, type:
\begin{verbatim}
%  condor_status -run

Name       Arch     OpSys        LoadAv RemoteUser           ClientMachine  

adriana.cs INTEL    SOLARIS251   0.980  hepcon@cs.wisc.edu   chevre.cs.wisc.
alfred.cs. INTEL    SOLARIS251   0.980  breach@cs.wisc.edu   neufchatel.cs.w
amul.cs.wi SUN4u    SOLARIS251   1.000  nice-user.condor@cs. chevre.cs.wisc.
anfrom.cs. SUN4x    SOLARIS251   1.023  ashoks@jules.ncsa.ui jules.ncsa.uiuc
anthrax.cs INTEL    SOLARIS251   0.285  hepcon@cs.wisc.edu   chevre.cs.wisc.
astro.cs.w INTEL    SOLARIS251   1.000  nice-user.condor@cs. chevre.cs.wisc.
aura.cs.wi SUN4u    SOLARIS251   0.996  nice-user.condor@cs. chevre.cs.wisc.
balder.cs. INTEL    SOLARIS251   1.000  nice-user.condor@cs. chevre.cs.wisc.
bamba.cs.w INTEL    SOLARIS251   1.574  dmarino@cs.wisc.edu  riola.cs.wisc.e
bardolph.c INTEL    SOLARIS251   1.000  nice-user.condor@cs. chevre.cs.wisc.
...
\end{verbatim}

\subsection{Removing a job from the queue}
A job can be removed from the queue at any time by using the \Condor{rm}
\index{Condor commands!condor\_rm}
command.  If the job that is being removed is currently running, the job
is killed without a checkpoint, and its queue entry is removed.  
The following example shows the queue of jobs before and after
a job is removed.
\begin{verbatim}
%  condor_q

-- Submitter: froth.cs.wisc.edu : <128.105.73.44:33847> : froth.cs.wisc.edu
 ID      OWNER            SUBMITTED    CPU_USAGE ST PRI SIZE CMD               
 125.0   jbasney         4/10 15:35   0+00:00:00 I  -10 1.2  hello.remote      
 132.0   raman           4/11 16:57   0+00:00:00 R  0   1.4  hello             

2 jobs; 1 idle, 1 running, 0 held

%  condor_rm 132.0
Job 132.0 removed.

%  condor_q

-- Submitter: froth.cs.wisc.edu : <128.105.73.44:33847> : froth.cs.wisc.edu
 ID      OWNER            SUBMITTED    CPU_USAGE ST PRI SIZE CMD               
 125.0   jbasney         4/10 15:35   0+00:00:00 I  -10 1.2  hello.remote      

1 jobs; 1 idle, 0 running, 0 held
\end{verbatim}

%%%%%%%%%%%%%%%%%%%%%%%%%%%%%%%%%%%%%%%%%%%%%%%%%%%%%%%%%%%%%%%%%%%%%%
\subsection{\label{sec:job-prio}Changing the priority of jobs}
%%%%%%%%%%%%%%%%%%%%%%%%%%%%%%%%%%%%%%%%%%%%%%%%%%%%%%%%%%%%%%%%%%%%%%

\index{job!priority}
\index{priority!of a job}
In addition to the priorities assigned to each user, Condor also provides
each user with the capability of assigning priorities to each submitted job.
These job priorities are local to each queue and range from -20 to +20, with
higher values meaning better priority.

The default priority of a job is 0, but can be changed using the \Condor{prio}
command.
\index{Condor commands!condor\_prio}
For example, to change the priority of a job to -15,
\begin{verbatim}
%  condor_q raman

-- Submitter: froth.cs.wisc.edu : <128.105.73.44:33847> : froth.cs.wisc.edu
 ID      OWNER            SUBMITTED    CPU_USAGE ST PRI SIZE CMD               
 126.0   raman           4/11 15:06   0+00:00:00 I  0   0.3  hello             

1 jobs; 1 idle, 0 running, 0 held

%  condor_prio -p -15 126.0

%  condor_q raman

-- Submitter: froth.cs.wisc.edu : <128.105.73.44:33847> : froth.cs.wisc.edu
 ID      OWNER            SUBMITTED    CPU_USAGE ST PRI SIZE CMD               
 126.0   raman           4/11 15:06   0+00:00:00 I  -15 0.3  hello             

1 jobs; 1 idle, 0 running, 0 held
\end{verbatim}

It is important to note that these \emph{job} priorities are completely 
different from the \emph{user} priorities assigned by Condor.  Job priorities
do not impact user priorities.  They are only a mechanism for the user to
identify the relative importance of jobs among all the jobs submitted by the
user to that specific queue.

\subsection{Why does the job not run?}
\index{job!analysis}
\index{job!not running}
Users sometimes find that their jobs do not run.  There are several reasons why
a specific job does not run.  These reasons include failed job or machine
constraints, bias due to preferences, insufficient priority, and the preemption
throttle that is implemented by the \Condor{negotiator} to prevent
thrashing.  Many of these reasons can be diagnosed by using the \Arg{-analyze}
option of \Condor{q}.
\index{Condor commands!condor\_q}
For example, the following job submitted by user
jbasney was found to have not run for several days.
\begin{verbatim}
% condor_q

-- Submitter: froth.cs.wisc.edu : <128.105.73.44:33847> : froth.cs.wisc.edu
 ID      OWNER            SUBMITTED    CPU_USAGE ST PRI SIZE CMD               
 125.0   jbasney         4/10 15:35   0+00:00:00 I  -10 1.2  hello.remote      

1 jobs; 1 idle, 0 running, 0 held
\end{verbatim}

Running \Condor{q}'s analyzer provided the following information:

\begin{verbatim}
%  condor_q 125.0 -analyze

-- Submitter: froth.cs.wisc.edu : <128.105.73.44:33847> : froth.cs.wisc.edu
---
125.000:  Run analysis summary.  Of 323 resource offers,
          323 do not satisfy the request's constraints
            0 resource offer constraints are not satisfied by this request
            0 are serving equal or higher priority customers
            0 are serving more preferred customers
            0 cannot preempt because preemption has been held
            0 are available to service your request

WARNING:  Be advised:
   No resources matched request's constraints
   Check the Requirements expression below:

Requirements = Arch == "INTEL" && OpSys == "IRIX6" && 
  Disk >= ExecutableSize && VirtualMemory >= ImageSize
\end{verbatim}

%%%%%%%%%%%%%%%%%%%
%condor_status -total lists the Arch/OS combinations in our pool:
%
%                     Machines Owner Claimed Unclaimed Matched Preempting
%
%           SGI/IRIX6       14     3       0        11       0          0
%          ALPHA/OSF1        8     6       1         1       0          0
%     SUN4u/SOLARIS26       84    38      46         0       0          0
%    SUN4u/SOLARIS251        8     0       1         7       0          0
%     SUN4x/SOLARIS26      104    47      56         1       0          0
%    SUN4x/SOLARIS251        1     0       1         0       0          0
%     INTEL/SOLARIS26      214    63     144         7       0          0
%       INTEL/WINNT40        6     0       0         6       0          0
%
%               Total      439   157     249        33       0          0
%
%So, one example of a platform that does not exist would be:
%
% requirements = Arch == "INTEL" && OpSys == "IRIX6"
%
%%%%%%%%%%%%%%%%%%%

For this job,
the \Attr{Requirements}
\index{ClassAd attribute!requirements}
expression specifies a platform that does not exist.
Therefore, the expression always evaluates to false.

While the analyzer can diagnose most common problems, there are some situations
that it cannot reliably detect due to the instantaneous and local nature of the
information it uses to detect the problem.  Thus, it may be that the analyzer
reports that resources are available to service the request, but the job still 
does not run.  In most of these situations, the delay is transient, and the
job will run during the next negotiation cycle.

If the problem persists and the analyzer is unable to detect the situation, it
may be that the job begins to run but immediately terminates due to some 
problem.  Viewing the job's error and log files
(specified in the submit command file) and Condor's \Macro{SHADOW\_LOG} file
may assist in tracking down the problem.  If the cause is still unclear, please
contact your system administrator.

\subsection{\label{sec:job-completion}Job Completion}
\index{job!completion}

When your Condor job completes(either through normal means or abnormal
termination by signal), Condor will remove it from the job queue (i.e.,
it will no longer appear in the output of \Condor{q}) and insert it into
the job history file.  You can examine the job history file with the
\Condor{history} command. If you specified a log file in your submit
description file, then the job exit status will be recorded there as well.

By default, Condor will send you an email message
when your job completes.  You can modify this behavior with the
\Condor{submit} ``notification'' command.
The message will include the exit status of your job (i.e., the
argument your job passed to the exit system call when it completed) or
notification that your job was killed by a signal.  It will also
include the following statistics (as appropriate) about your job:

\begin{description}

\item[Submitted at:] when the job was submitted with \Condor{submit}

\item[Completed at:] when the job completed

\item[Real Time:] elapsed time between when the job was submitted and
when it completed (days hours:minutes:seconds)

\item[Run Time:] total time the job was running (i.e., real time minus
queueing time)

\item[Committed Time:] total run time that contributed to job
completion (i.e., run time minus the run time that was lost because
the job was evicted without performing a checkpoint)

\item[Remote User Time:] total amount of committed time the job spent
executing in user mode

\item[Remote System Time:] total amount of committed time the job spent
executing in system mode 

\item[Total Remote Time:] total committed CPU time for the job

\item[Local User Time:] total amount of time this job's
\Condor{shadow} (remote system call server) spent executing in user
mode

\item[Local System Time:] total amount of time this job's
\Condor{shadow} spent executing in system mode

\item[Total Local Time:] total CPU usage for this job's \Condor{shadow}

\item[Leveraging Factor:] the ratio of total remote time to total
system time (a factor below 1.0 indicates that the job ran
inefficiently, spending more CPU time performing remote system calls
than actually executing on the remote machine)

\item[Virtual Image Size:] memory size of the job, computed when the
job checkpoints

\item[Checkpoints written:] number of successful checkpoints performed
by the job

\item[Checkpoint restarts:] number of times the job successfully
restarted from a checkpoint

\item[Network:] total network usage by the job for checkpointing and
remote system calls

\item[Buffer Configuration:] configuration of remote system call I/O
buffers

\item[Total I/O:] total file I/O detected by the remote system call
library

\item[I/O by File:] I/O statistics per file produced by the remote
system call library

\item[Remote System Calls:] listing of all remote system calls
performed (both Condor-specific and Unix system calls) with a count of
the number of times each was performed

\end{description}

%%%%%%%%%%%%%%%%%%%%%%%%%%%%%%%%%%%%%%%%%%

%%%%%%%%%%%%%%%%%%%%%%%%%%%%%%%%%%%%%%%%
\section{\label{sec:Priorities}Priorities and Preemption}
%%%%%%%%%%%%%%%%%%%%%%%%%%%%%%%%%%%%%%%%

HTCondor has two independent priority controls: \Term{job}
priorities and \Term{user} priorities.  

\subsection{Job Priority}

\index{job!priority}
\index{priority!of a job}
Job priorities allow the assignment of a priority level to
each submitted HTCondor job in order to
control the order of their execution.
\index{HTCondor commands!condor\_prio}
To set a job priority, use the \Condor{prio} command;
see the example in section~\ref{sec:job-prio}, or the
command reference page on page~\pageref{man-condor-prio}.
Job priorities do not impact user priorities in any fashion.
A job priority can be any integer, and higher values are \emph{better}.

%%%%%%%%%%%%%%%%%%%%%%%%%%%%%%%%%%%%%%%%%%%%%%%%%%%%%%%%%%%%%%%%%%%%%%
\subsection{\label{sec:user-priority-explained}User priority}
%%%%%%%%%%%%%%%%%%%%%%%%%%%%%%%%%%%%%%%%%%%%%%%%%%%%%%%%%%%%%%%%%%%%%%

\index{preemption!priority}
\index{user!priority}
\index{priority!of a user}
Machines are allocated to users based upon a user's priority.
A lower numerical value for user priority means higher priority,
so a user with priority 5 will get more resources than
a user with priority 50.
User priorities in HTCondor can be examined with the \Condor{userprio}
command (see page~\pageref{man-condor-userprio}).
\index{HTCondor commands!condor\_userprio}
HTCondor administrators can set and change individual user priorities
with the same utility.

HTCondor continuously calculates the share of available machines that each
user should be allocated.    This share is inversely related to the ratio
between user priorities.
For example, a user with a priority of 10 will get twice as many
machines as a user with a priority of 20.
The priority of each individual user changes according to
the number of resources the individual is using.
Each user starts out with the best possible priority: 0.5.
If the number of machines a user currently has is greater than 
the user priority,
the user priority will worsen by numerically increasing over time.
If the number of machines is less then the priority,
the priority will improve by numerically decreasing over time. 
The long-term result is fair-share access across all users.
The speed at which HTCondor adjusts the priorities is
controlled with the configuration variable \Macro{PRIORITY\_HALFLIFE},
an exponential half-life value.
The default is one day.
If a user that has user priority of 100 and is
utilizing 100 machines removes all his/her jobs,
one day later that user's
priority will be 50, and two days later the priority will be 25.

HTCondor enforces that each user gets his/her fair share of machines
according to user priority both when allocating machines which become
available and by priority preemption of currently allocated machines.
For instance, if a low priority user is utilizing all available machines
and suddenly a higher priority user submits jobs, HTCondor will
immediately take a checkpoint and vacate jobs belonging to the lower priority
user. This will free up machines that HTCondor will then give over to the
higher priority user. HTCondor will not starve the lower priority user; it
will preempt only enough jobs so that the higher priority user's fair
share can be realized (based upon the ratio between user priorities). To
prevent thrashing of the system due to priority preemption, the HTCondor 
site administrator can define a \Macro{PREEMPTION\_REQUIREMENTS} expression in HTCondor's configuration.
The default expression that ships with HTCondor is configured to only preempt 
lower priority jobs that have run
for at least one hour. So in the previous example, in the worse case it
could take up to a maximum of one hour until the higher priority user
receives a fair share of machines.
For a general discussion of
limiting preemption,
please see
section \ref{sec:Disabling Preemption} of the Administrator's manual.

User priorities are keyed on \Expr{<username>@<domain>}, for example
\Expr{johndoe@cs.wisc.edu}. The domain name to use, if any, is configured by
the HTCondor site administrator.  Thus, user priority and therefore resource
allocation is not impacted by which machine the user submits from or
even if the user submits jobs from multiple machines.

\index{nice job}
\index{priority!nice job}
An extra feature is the ability to submit a job as
a \Term{nice} job (see page~\pageref{man-condor-submit-nice}).
Nice jobs artificially boost the user priority 
by ten million just for the nice job.
This effectively means that nice jobs will only run on
machines that no other HTCondor job (that is, non-niced job) wants.
In a similar fashion, an HTCondor administrator could set
the user priority of any specific HTCondor user very high.
If done, for example, with a guest account,
the guest could only use cycles not wanted by other users of the system.


%%%%%%%%%%%%%%%%%%%%%%%%%%%%%%%%%%%%%%%%%%%%%%%%%%%%%%%%%%%%%%%%%%%%%%
\subsection{\label{sec:Vacate-Explained}
Details About How HTCondor Jobs Vacate Machines}
%%%%%%%%%%%%%%%%%%%%%%%%%%%%%%%%%%%%%%%%%%%%%%%%%%%%%%%%%%%%%%%%%%%%%%

\index{vacate}
\index{preemption!vacate}
When HTCondor needs a job to vacate a machine for whatever reason, it
sends the job an asynchronous signal specified in the \AdAttr{KillSig}
attribute of the job's ClassAd.
The value of this attribute can be specified by
the user at submit time by placing the \Opt{kill\_sig} option in the
HTCondor submit description file.  

If a program wanted to do some special work when required
to vacate a machine, the program may set up a
signal handler to use a trappable signal as an indication
to clean up.
When submitting this job, this clean up signal is specified to be used with
\Opt{kill\_sig}.
Note that the clean up work needs to be quick.
If the job takes too long to go away, HTCondor
follows up with a SIGKILL signal which immediately terminates the
process.

\index{HTCondor commands!condor\_compile}
A job that is linked using \Condor{compile}
and is subsequently submitted into the standard universe, 
will checkpoint and exit upon receipt of a SIGTSTP signal.
Thus, SIGTSTP is
the default value for \AdAttr{KillSig} when submitting to the standard
universe.
The user's code may still checkpoint itself at any time
by calling one of the following functions exported by the HTCondor libraries:
\begin{description}
\item[\Procedure{ckpt()}] Performs a checkpoint and then returns.
\item[\Procedure{ckpt\_and\_exit()}] Checkpoints and exits; HTCondor will then
restart the process again later, potentially on a different machine.
\end{description}

For jobs submitted into the vanilla universe, the default value for
\AdAttr{KillSig} is SIGTERM,
the usual method to nicely terminate a Unix program.

%%%%%%%%%%%%%%%%%%%%%%%%%%%%%%%%%%%%%%%%%%%%%%%%%%%%%%%%%%%%%%%%%%%%%%
%%%%%%%%%%%%%%%%%%%%%%%%%%%%%%%%%%%%%%%%%%%%%%%%%%%%%%%%%%%%%%%%%%%%%%
\section{\label{sec:java-install}Java Support Installation}
%%%%%%%%%%%%%%%%%%%%%%%%%%%%%%%%%%%%%%%%%%%%%%%%%%%%%%%%%%%%%%%%%%%%%%

\index{installation!Java}
\index{Java}

Compiled Java programs may be executed (under HTCondor) on
any
execution site with a
\index{Java Virtual Machine}
\index{JVM}
Java Virtual Machine (JVM).
To do this,
HTCondor must be informed of some details of the
JVM installation.

Begin by installing a Java distribution according to the vendor's
instructions.
We have successfully used the Sun Java Developer's Kit,
but any distribution should suffice.
Your machine may have
been delivered with a JVM already installed -- installed code
is frequently found in \File{/usr/bin/java}.

HTCondor's configuration includes the location of the installed
JVM.
Edit the configuration file.
Modify the \Macro{JAVA} entry to point to the JVM binary,
typically \File{/usr/bin/java}.
Restart the \Condor{startd} daemon on that host.  For example,

\begin{verbatim}
% condor_restart -startd bluejay
\end{verbatim}

The \Condor{startd} daemon takes a few moments to exercise the Java
capabilities of the \Condor{starter}, query its properties,
and then advertise the machine
to the pool as Java-capable.
If the set up succeeded, then \Condor{status} will
tell you the host is now Java-capable by printing the Java
vendor and the version number:

\begin{verbatim}
% condor_status -java bluejay
\end{verbatim}

After a suitable amount of time, if this command does not give any output,
then the \Condor{starter}  is having difficulty executing the JVM.
The exact cause of the problem depends on the details of the
JVM, the local installation, and a variety of other factors.
We can offer only limited advice on these matters,
but here is an approach to solving the problem.

To reproduce the test that the \Condor{starter} is attempting,
try running the Java \Condor{starter} directly.  To find
where the \Condor{starter} is installed, run this command:

\begin{verbatim}
% condor_config_val STARTER
\end{verbatim}

This command prints out the path to the \Condor{starter},
perhaps something like this:

\begin{verbatim}
/usr/condor/sbin/condor_starter
\end{verbatim}

Use this path to execute the \Condor{starter} directly
with the \Arg{-classad} argument.
This tells the starter to run its tests and display its properties.

\begin{verbatim}
/usr/condor/sbin/condor_starter -classad
\end{verbatim}

This command will display a short list of cryptic properties, such as:

\begin{verbatim}
IsDaemonCore = True
HasFileTransfer = True
HasMPI = True
CondorVersion = "$CondorVersion: 7.1.0 Mar 26 2008 BuildID: 80210 $"
\end{verbatim}

If the Java configuration is correct, there will also
be a short list of Java properties, such as:

\begin{verbatim}
JavaVendor = "Sun Microsystems Inc."
JavaVersion = "1.2.2"
JavaMFlops = 9.279696
HasJava = True
\end{verbatim}

If the Java installation is incorrect, then any error
messages from the shell or Java will be printed
on the error stream instead.

The Sun JVM sets a value of 64 Mbytes for the Java Maxheap Argument,
which HTCondor uses.
This value is often too small for the application.
The administrator can change this value through configuration by setting
a different value for \Macro{JAVA\_EXTRA\_ARGUMENTS}.

\footnotesize
\begin{verbatim}
JAVA_EXTRA_ARGUMENTS = -Xmx1024m
\end{verbatim}
\normalsize
Note that if a specific job sets the value in the submit description
file, using the submit command \SubmitCmd{java\_vm\_args},
this job's value takes precedence over a configured value.



%%%%%%%%%%%%%%%%%%%%%%%%%%%%%%%%%%%%%%%%%%%%%%%%%%%%%%%%%%%%%%%%%%%%%%

%%%%%%%%%%%%%%%%%%%%%%%%%%%%%%%%%%%%%%%%%%%%%%%%%%%%%%%%%%%%%%%%%%%%%%
%%%%%%%%%%%%%%%%%%%%%%%%%%%%%%%%%%%%%%%%%%%%%%%%%%%%%%%%%%%%%%%%%%%%%%
\section{\label{sec:Parallel}Parallel Applications}
%%%%%%%%%%%%%%%%%%%%%%%%%%%%%%%%%%%%%%%%%%%%%%%%%%%%%%%%%%%%%%%%%%%%%%
\index{Parallel|(}

Condor's Parallel universe is a mechanism to support a wide variety of
parallel programming environments, including most implementations of
MPI.  This universe also supports jobs which need to be co-scheduled,
that is, jobs where more than one process must be running at the same
time to be correct.  It supersedes the older MPI universe, which
eventually will be removed.


%%%%%%%%%%%%%%%%%%%%%%%%%%%%%%%%%%%%%%%%%%%%%%%%%%%%%%%%%%%%%%%%%%%
\subsection{\label{sec:parallel-setup}Prerequisites to running parallel jobs}

Condor must be configured such that resources (machines) running
parallel jobs are dedicated.  \index{scheduling!dedicated} Note that
``dedicated'' has a very specific meaning in Condor: Dedicated
machines never vacate their running condor jobs should the machine's
interactive owner return.  Once the dedicated scheduler claims a
dedicated machine for use, it will try to use that machine to satisfy
the requirements of the queue of parallel universe or MPI universe
jobs.  If the dedicated scheduler cannot use a machine for a
configurable amount of time, it will release its claim on the machine,
making it available again for the opportunistic scheduler.

Since Condor does not ordinarily run this way, (Condor usually uses
opportunistic scheduling), dedicated machines must be specially
configured.  Section~\ref{sec:Config-Dedicated-Jobs} of
Administrator's Manual describes the necessary configuration and
provides detailed examples.

To simplify the dedicated scheduling of resources, a single machine
becomes the scheduler of dedicated resources.  This leads to a further
restriction that jobs submitted to execute under the parallel universe
must be submitted from the machine running as the dedicated scheduler.

%%%%%%%%%%%%%%%%%%%%%%%%%%%%%%%%%%%%%%%%%%%%%%%%%%%%%%%%%%%%%%%%%%%
\subsection{\label{sec:parallel-submit}Parallel Job Submission}

Once Condor resources are correctly configured, jobs may be submitted.
Each Condor job requires a submit description file.  Here is a simple
submit description file for a parallel job.

\begin{verbatim}
#############################################
##   submit description file for parallel program
#############################################
universe = parallel
executable = /bin/sleep
arguments = 30
machine_count = 8
queue 
\end{verbatim}

This job specifies the \Attr{universe} as \Attr{parallel}, letting
Condor know that dedicated resources are required.  The
\Attr{machine\_count} command identifies the number of machines
required by the job. 

When this job is submitted, the dedicated scheduler allocates eight
machines with the same architecture and operating system as the submit
machine.  It waits until all eight machines are available before
starting the job.  When all the machines are ready, it runs the
/bin/sleep command, with the argument 30 on all eight machines
simultaneously (more or less).  The first machine selected is treated
specially -- when that job exits, Condor shuts down all the other
nodes, even if they haven't finished running yet.

This simple example does not specify an input or output,
meaning that the computation completed is useless,
since both input comes from and the output goes to \File{/dev/null}.
A more complex example of a submit description file
utilizes other features.
\begin{verbatim}
######################################
## Parallel example submit description file
######################################
universe = parallel
executable = /bin/cat
log = logfile
input = infile.$(NODE)
output = outfile.$(NODE)
error = errfile.$(NODE)
machine_count = 4
queue
\end{verbatim}

The specification of the input, output, and error files utilize a
predefined macro \index{macro!predefined} See the \Condor{submit}
manual page on page~\pageref{man-condor-submit} for further
description of predefined macros.  The \MacroU{NODE} macro is given a
unique value as programs are assigned to machines.  The
\MacroUNI{NODE} value is fixed for the entire length of the job.  It
can therefore be used to identify individual aspects of the
computation.  In this example, it is used to give unique names to
input and output files.

If your site does NOT have a shared file system across all the nodes
where your parallel computation will execute, you can use Condor's
file transfer mechanism.  You can find out more details about these
settings by reading the \Condor{submit} man page or
section~\ref{sec:file-transfer} on page~\pageref{sec:file-transfer}.
Assuming your job only reads input from STDIN, here is an example
submit file for a site without a shared file system:

\begin{verbatim}
######################################
## Parallel example submit description file
## without using a shared file system
######################################
universe = parallel
executable = /bin/cat
log = logfile
input = infile.$(NODE)
output = outfile.$(NODE)
error = errfile.$(NODE)
machine_count = 4
should_transfer_files = yes
when_to_transfer_output = on_exit
queue
\end{verbatim}

The submission to Condor requires exactly four machines,
and queues four programs.
Each of these programs requires an input file (correctly
named) and produces an output file.

\subsection{\label{sec:parallel-mpi-submit}Submitting MPI jobs with the parallel universe}

The above examples simplistically show how to co-schedule otherwise
sequential executables in parallel.  To run MPI jobs in the parallel
universe, a bit more framework is needed.  Condor provides this
framework in the form of user visible and modifiable scripts, to allow
flexibility for the different kinds of parallel systems it can
support.  The Condor parallel universe works somewhat like a SIMD
(Single Instruction, Multiple Data) machine -- there is one named
executable which is run on all the machines in parallel, but this one
machine may have different inputs and outputs.  If different
executables are needed to run on different nodes, the submit file
should contain a script, which knows which node it is running on, and
forks an appropriate executable.

Most MPI implementations require two system-wide prerequisites.
First, the ability to run a command on a remote machine without being
prompted for a password.  Usually, ssh is used for this, but the
specific command used is configurable.  Second, an ASCII file with the
list of machines that can be ssh'd to, as per above.

So, to run MPI application in the parallel universe, we run a script
on each node we submit to.  This script generates ssh keys, to enable
password-less remote execution, start an sshd daemon, and send the
names and rank (node number) back to the submit directory.  Thus, for
each Condor job submitted, the scripts set up an ad-hoc MPI
environment, which is torn down at the end of the job run.  This ssh
script is a common requirement for running MPI jobs, so we have
factored it out into a common script, which is called from each of the
MPI-specific scripts.  After the ssh script has been started, the
MPI-specific script runs, starts the rest of the MPI job by looking at
its arguments, and waits for the MPI job to finish.  Condor provides
the ssh script, and example MPI scripts for both LAM and MPICH.  The
former is named ``lamscript'', and the latter ``mp1script''.  The
first argument to each script is the name of the real MPI executable,
and any subsequent arguments are arguments to that executable.  Other
implementations should be easy to add, by modifying the given
examples.  Note that because the actual MPI executable (i.e. the
output of mpicc) is not the named executable in the submit script, it
must be accessible either via a network file system, or by condor file
transfer.

The sshd.sh script requires several configuration file settings.
\Macro{CONDOR\_SSHD} should be an absolute path to an implementation of
sshd.  sshd.sh has been tested with openssh version 3.9, but should
work with more recent versions.  \Macro{CONDOR\_SSH\_KEYGEN} should
point to the corresponding ssh-keygen executable.

The LAM and MPICH scripts each have their own idiosyncrasies.  In the
mp1script, the PATH to the mpich installation must be set.  Look
for the shell variable MPDIR, and set it to the proper value.  This
directory should contain the mpich mpirun command.

For LAM, there is a similar path setting, but called LAMDIR in the
lamscript shell script.  In addition, this path must be part of the
path set in the user's .cshrc script.  (As of this writing, lam doesn't
work if the user's login shell is the Bourne or compatible shell).

\begin{verbatim}
######################################
## Example submit description file
## for MPICH 1 MPI
## works with MPICH 1.2.4, 1.2.5 and 1.2.6
######################################
universe = parallel
executable = mp1script
arguments = my_mpich_linked_executable arg1 arg2
machine_count = 4
should_transfer_files = yes
when_to_transfer_output = on_exit
transfer_input_files = my_mpich_linked_executable
queue
\end{verbatim}

\begin{verbatim}
######################################
## Example submit description file
## for LAM MPI
######################################
universe = parallel
executable = lamscript
arguments = my_lam_linked_executable arg1 arg2
machine_count = 4
should_transfer_files = yes
when_to_transfer_output = on_exit
transfer_input_files = my_lam_linked_executable
queue
\end{verbatim}

\subsection{\label{sec:parallel-multi-proc}Submitting parallel jobs with multiple requirements}
Different nodes for a parallel job can have different machine
requirements.  For example, often the first node, sometimes called the
head node, needs to run on a specific machine.  This can be also
useful for debugging.  Condor accommodates this by supporting multiple
\Attr{queue} statements in the submit file, much like with the other
universes.  For example:

\begin{verbatim}
######################################
## Example submit description file
## with multiple procs
######################################
universe = parallel
executable = example
machine_count = 1
requirements = ( machine == "machine1")
queue

requirements = ( machine =!= "machine1")
machine_count = 3
queue
\end{verbatim}

The dedicated scheduler will allocate four machines (nodes) total across
two procs for this job.  The first proc has one node,
 and will run on the machine named machine1.  The 
other three nodes, in the second proc, will run on other machines.  
Like in the other condor universes, the second requirements command 
overwrites the first, but the other commands are inherited from the 
first proc.

When submitting jobs with multiple requirements, it is
best to write the requirements to be mutually exclusive,
or to have the most selective requirement first in the submit file.
This is because the scheduler tries to match jobs to machine in
submit file order.  If the requirements are not mutually exclusive,
it can happen that the scheduler may unable to schedule the job, even
if all needed resources are available.

\index{Parallel|)}

%%%%%%%%%%%%%%%%%%%%%%%%%%%%%%%%%%%%%%%%%%%%%%%%%%%%%%%%%%%%%%%%%%%%%%

%%%%%%%%%%%%%%%%%%%%%%%%%%%%%%%%%%%%%%%%%%%%%%%%%%%%%%%%%%%%%%%%%%%%%%
\section{Interjob Dependencies: DAGMan Meta-Scheduler}
\label{sec:DAGMan}

The Directed Acyclic Graph Manager (DAGMan) is a meta-scheduler for Condor
jobs.  DAGMan is responsible for submitting batch jobs in a predefined order
and processing the results. A configuration file is defined prior to execution
of DAGMan in which the jobs, their \textit{CondorConfigFile}, and job
dependencies are declared.

The importance of such a tool lies in the fact that the user is able to define
the execution order of a number of Condor Jobs. Just as Condor schedules
condor jobs, DAGMan schedules a system of jobs. In essence, it defines a
problem. Solving a problem may require multiple condor jobs that need data
from each other. This is best represented using a Directed Acyclic Graph
(DAG), which represents the flow of control from one node to another (i.e.,
from one condor job to another) through arrows.

From the point of view of the user, the scheduler is initialized with the
order of execution of jobs, and then started. DAGMan is responsible for all
scheduling, recovery and reporting activities of the submitted system of jobs.

The following sections explain the use of DAGMan in full detail.  However, if
the user only wants the bare essentials, please read
section~\ref{dagman:essentials} to get started more quickly.

\subsection{DAG Input File}

For Unix users, a useful analogy might be to think of the DAGMan input file as
a makefile, and DAGMan itself as the make executable.  However, DAGMan differs
from make.  Instead of looking at file modification timestamps, DAGMan reads
the Condor log file generated by each Condor job to find out which jobs are
unsubmitted, submitted, or complete.  DAGMan also makes a guarantee that a DAG
is recoverable, even if the machine running DAGMan goes down during execution.

\subsubsection{Description}
\label{dagman:dagdesc}

Job dependencies are defined prior to execution of the DAGMan program, using a
DAG input file.  An example input configuration file name is \File{diamond.dag}.
The input file is read completely, and the DAG data structure is constructed
in memory before the first job is submitted.  With the exception of the
\textit{CondorCommandFile} (see below), the input file is case insensitive.

Throughout the input file, comments can be placed.  Legal comments exist on a
single line which immediately starts with a `\texttt{\#}' character, followed
by any characters up to the newline `\texttt{$\backslash$n}'.

It is interesting to note that the DAGMan input file does not contain any
specifics about the individual jobs. Each condor job by itself is handled as
if DAGMan was not present (this includes compiling and linking of the
job). The executable and the input/output parameters for each job are
contained in the CondorCommandFile.  The DAG file merely describes the
relationship between the different condor jobs using the semantics just
described.

\begin{description}

\item[Signature]

The first line of a DAG input file is the signature, which precisely
identifies which DAG file format follows.  As of this writing, only one DAG
format exists, and thus only one signature is possible.

\begin{verbatim}
  ### DAGMan 6.1.0
\end{verbatim}

This line must appear as it is written here, character for character.
Anything different will be rejected by DAGMan.  Having a precise signature tag
will enable future versions of DAGMan to remain backward compatible with older
DAG input file formats.

\item[Job Section]

The Job Section of the input DAG file declares all the jobs that will appear
in the DAG.  Each job is described by a single line called a Job Entry.  The
following syntax is used:

\begin{verbatim}
	JOB <JobName> <CondorCommandFile>
\end{verbatim}

The \texttt{JOB} keyword (shown here in upper case only for clarity) declares
this line will map a \textit{JobName} to a Condor Command File.  The
\textit{JobName} is used by DAGMan to uniquely identify jobs throughout the
input file and to name them in output messages.  The
\textit{CondorCommandFile} is the input file used by \Condor{submit} to run
the individual condor job.  Because the Unix file system is case sensitive,
the case of the \textit{CondorCommandFile} is preserved.

The JobName can be any string that contains no white space.  The JobName is
not case sensitive, so ``JobA'' is equivalent to ``joba''.  An example
\textit{CondorCommandFile} name is \File{a.condor}.  Some important
restrictions are placed on the contents of the \textit{CondorCommandFile},
which will be discussed later.

The user can also have the option of declaring a job as being already
completed in the DAG input file. This may be useful in situations where the
user wishes to verify results, but does not need the entire job dependency
graph to be executed. This is done by adding the word "DONE" to the end of the
Job declaration line.

\begin{verbatim}
	JOB <JobName> <CondorCommandFile> DONE
\end{verbatim}

\item[Dependency Section]

The dependency section of the DAG input file follows the Job Section and
describes the dependencies between the jobs listed in the Job Section.  The
notion of a ``parent'' and ``child'' job is introduced here.  A parent job
produces output which is required by one or more child jobs.  None of the
children can run until the parent successfully terminates.  A child job is one
whose input is taken from one or more parent jobs.  The child job cannot run
until all of its parents have successfully terminated.

A single line in the input file can specify the dependencies from one or more
parents to one or more children.

\begin{verbatim}
	PARENT <ParentJobName>* CHILD <ChildJobName>*
\end{verbatim}

The \texttt{PARENT} keyword is followed by one or more
\textit{ParentJobName}s.  Those are followed by the \texttt{CHILD} keyword,
which is followed by one or more \textit{ChildJobName}s.  Each child job
depends on each and every parent job on this line.  So the line
``\texttt{PARENT p1 p2 CHILD c1 c2}'' would produce four dependencies.

\end{description}

\subsubsection{Example}

The following \File{diamond.dag} DAG input file shown below is illustrated in
Figure~\ref{fig:dagman-diamond}.

\begin{verbatim}
  ### DAGMan 6.1.0
  # Filename: diamond.dag
  #
  Job  A  A.condor 
  Job  B  B.condor 
  Job  C  C.condor	
  Job  D  D.condor
  PARENT A CHILD B C
  PARENT B C CHILD D
\end{verbatim}

\begin{figure}[hbt]
\centering
\includegraphics{user-man/dagman-diamond.eps}
\caption{\label{fig:dagman-diamond}Diamond DAG}
\end{figure}

With \File{diamond.dag}, job A must execute first, because all other jobs
directly or indirectly depend on it.  After job A successfully completes, both
job B and C are eligible to run.  In fact, they will be submitted at the same
time and hopefully Condor will find two remote hosts that can run them in
parallel.  Since job D depends on both B and C, it must wait for both to
complete successfully before it can be submitted.

\subsection{Execution}

\subsubsection{Preparing Jobs}
\label{dagman:prepjob}

Each individual job in a DAG is free to be a unique executable, with a unique
\textit{CondorCommandFile}.  The DAG can contain a mixture of standard and
vanilla jobs, or even other meta-scheduler jobs, like DAGMan.  On the other
hand, the jobs in the DAG could all use the same executable, or even the same
\textit{CondorCommandFile}.  Anything between both extremes is possible.
However, two limits are imposed.

First, each \textit{CondorCommandFile} must submit a cluster of size one.
There cannot be multiple \texttt{queue} lines.  The reasoning is long winded,
so a brief summary will be attempted.  If multi-job clusters were allowed,
DAGMan would have to parse the \textit{CondorCommandFile} to find out how many
jobs belong to that cluster.  Otherwise, DAGMan would not know for sure if a
cluster had terminated based on seeing the event from one job of that
cluster.  This restriction may be lifted in future DAGMan version, depending
on the design and implementation issues.

Second, all \textit{CondorCommandFile}s of a DAG must specify the same log.
In order for DAGMan to follow the order of events correctly, all events from
all jobs in the DAG must be sent to the same log file.  This restriction will
be loosened in later versions (see section~\ref{dagman:version}).

For this example, we will write a single \textit{CondorCommandFile} to be used
by all three jobs in the DAG.  Thus, each job will run the same executable.
This example is very artificial, because normally separate jobs would need
output for their child jobs to go to unique output and error files.
Otherwise, the jobs would be clobbering each other's output.  However, since
we are sending output and error to \File{/dev/null}, sharing the
\textit{CondorCommandFile} is OK.

\begin{verbatim}
  # Filename: diamond_job.condor
  #
  executable   = /path/diamond.exe
  output       = /dev/null
  error        = /dev/null
  log          = diamond_condor.log
  universe     = vanilla
  notification = NEVER
  queue
\end{verbatim}

Note that notification is set to \texttt{NEVER}.  This is recommended if you
prefer not to have Condor send you e-mail for every job in a large DAG.

\subsubsection{Writing the DAG File}
\label{dagman:writedag}

The DAG file names the jobs, associates jobs with their
\textit{CondorCommandFile}, and declares job dependencies.  For our artificial
DIAMOND example, all three jobs will use the same diamond\_job.condor file
written earlier.  However, a more typical DAG file would have unique
\textit{CondorCommandFile} for every job.

\begin{verbatim}
  ### DAGMan 6.1.0
  # Filename: diamond.dag
  # DIAMOND DAG File for DAGMan
  #
  Job  A  diamond_job.condor
  Job  B  diamond_job.condor
  Job  C  diamond_job.condor
  Job  D  diamond_job.condor
  PARENT A CHILD B C
  PARENT B C CHILD D
\end{verbatim}

This DAG file will be the input file for the \Condor{dagman} program.

\subsubsection{Submitting the DAG to Condor}
\label{dagman:submitdag}

In order to guarantee recoverability, the DAGMan program itself is run as a
Condor job.  However, DAGMan is not submitted as a standard universe or
vanilla universe job.  Instead, it is run as a meta-scheduler.  Standard and
vanilla universe jobs are usually submitted to the local schedd, which
schedules them for execution on some remote machine in the pool that is idle.
A meta-scheduler is also submitted to the local schedd, but runs on the local
schedd.  The meta-scheduler then submits jobs, according to its design, to the
same local schedd, just as if the user submitted them manually.  In fact, the
local schedd does not know the difference between DAGMan submitting a job, and
the user who originally submitted DAGMan, and could have submitted the DAG
jobs manually.

A DAG is submitted using the \Condor{submit\_dag} script.  For example, to
submit the \File{diamond.dag} DAG to Condor, simply type
``\Condor{submit\_dag} \File{diamond.dag}''.  This script will generate the
\File{diamond.dag.condor} \textit{CondorCommandFile} for the DAG, and submit
it to Condor.

If the user prefers to edit the \File{diamond.dag.condor} file before it is
submitted to Condor (for example, to change the pre-chosen filenames), she can
issue ``\Condor{submit\_dag} -n \File{diamond.dag}'', which specifies that
\File{diamond.dag.condor} is generated, but not submitted to Condor.  To run
the DAG, issue the command \Condor{submit} diamond.dag.condor.

\subsection{Removal}

After submitting a DAG, the user may change her mind and wish to remove the
entire DAG, plus any jobs submitted by that DAG which happen to currently be
running.  DAG removal is easily accomplished by issuing a \Condor{rm} on the
DAGMan job itself.  The schedd sends a special signal to the meta-scheduler,
telling it to remove any of its condor jobs (using \Condor{rm}) that are
currently running.

However, if the machine is scheduled to go down, and the schedd receives a
shutdown command from the master, the schedd will send a running DAGMan job a
similar shutdown, which instructs DAGMan to clean up memory and exit.
However, in this case, DAGMan does not remove its submitted jobs, but rather
expects them to persistently exist in the Condor queue after restart.

The important thing to remember is that DAGMan will not explicitly run
\Condor{rm} on its jobs except as a result of the user running \Condor{rm} on
the DAGMan job.

\subsection{Recovery}

The Condor system offers the benefit of recoverability, in that if any host
crashes, Condor jobs that were running can be recovered, either by continuing
from the last checkpoint, or rerunning from scratch.  In any event, Condor
guarantees that once a job is successfully submitted, the Condor system will
not loose it.

DAGMan makes the same guarantee about the DAG as a whole.  If the machine
running DAGMan goes down or crashes, upon restart DAGMan will be restarted,
and the state of the DAG jobs will be recovered from the log file
(\File{diamond.dag.condor.log} from our example before).  DAGMan knows to
recover a DAG (as apposed to starting a new one) because it will detect the
existance of a lock file that was not removed from the last run.  If DAGMan
successfully finishes a DAG, the lock file is removed, so that the next run
will not go into recover mode.  The lock file is specified via command-line
argument to DAGMan in the \textit{CondorCommandFile}.  Refer to
section~\ref{dagman:submitdag}.

\subsection{Essentials}
\label{dagman:essentials}

This section is written for those users looking for the boiled down,
absolutely essential steps to successfully submit a DAG.

\begin{description}

\item[Prepare Jobs] Each job in the DAG must have its own
\textit{CondorCommandFile}.  Each \textit{CondorCommandFile} can only submit
one job.  Multi-job clusters (multiple \texttt{queue} lines) are not
supported.  The \texttt{log=} for all \textit{CondorCommandFile}s must point
to the same Condor log file, otherwise, DAGMan will not see all the Condor log
entries for every job in the DAG.  Refer to section~\ref{dagman:prepjob} for
details on how to prepare jobs.

\item[Write DAG File] Write the DAG file, so that JOB entries refer to the
\textit{CondorCommandFile}s you wrote in the previous step.  Refer to
section~\ref{dagman:writedag} to learn about writing a DAG file.

\item[Submit the DAG] Finally, you submit the DAG written in the previous step
using the \Condor{submit\_dag} script.  Refer to
section~\ref{dagman:submitdag}.

\end{description}


\subsection{Version Summary}
\label{dagman:version}

This section addresses the features and limitations that exist in the current
version of DAGMan, and how they may change in future versions.

This first public release of DAGMan was written and tested in the Condor 6.1.0
environment.  It is shipped separate from the main Condor system as a
contribution program.  As such, it is not as rigorously tested as the core
components of Condor.  A reasonable effort has been made to test large DAGS
(on the order of 5000 jobs) on Solaris x86 and Sparc.  However, the DAGMan is
not arrogant enough to claim itself bug free.  Users are encouraged to send
e-mail to \Email{condor-admin@cs.wisc.edu}.

The following feature summary compares the current version with possible
versions of DAGMan still to come.

\begin{description}
\item[Feature] : Command Socket
\item[Version 6.1.0] : Unsupported
\item[Future Versions] : A general purpose command socket will be used to
direct Dagman while it's running.  Commands like CANCEL\_JOB X or DELETE\_ALL
would be supported, as well as notification messages like JOB\_SUBMIT or
JOB\_TERMINATE, etc.  Eventually, a Java Gui would graphically represent the
Dag's state, and offer buttons and dials for graphic Dag manipulation.
\end{description}

\begin{description}
\item[Feature]: DAG removal
\item[Version 6.1.0]: Supported via \Condor{rm} of the DAG.
\item[Future Versions]: Supported by a command socket such as DELETE\_ALL
\end{description}

\begin{description}
\item[Feature]: Condor Log File
\item[Version 6.1.0]: All jobs in a DAG must specify the same Condor log file.
That Condor log file must be unique.  No other DAGs or Condor jobs can point
to that log file.
\item[Future Versions]: All jobs in a Dag must go to one log file, but
log file can be shared with other Dags and Condor jobs.
\end{description}

\begin{description}
\item[Feature]: Job UNDO
\item[Version 6.1.0]: All jobs must exit normally, else DAG will be aborted
\item[Future Versions]: A job can be ``undone'', or there is some
notion of a job instance.  Hence, a job that exits abnormally or is
cancelled by the user can be rerun such that the new run's log entry
is unique from the old run's log entry (in terms of recovery)
\end{description}

\begin{description}
\item[Feature]: Pre/Post Process
\item[Version 6.1.0]: Unsupported
\item[Future Versions]: A job can have a pre- and post-process script
specified, which are run before and after the job is submitted.  This can be
useful for performing tasks like compression or decompression or input or
output data.
\end{description}

%%%%%%%%%%%%%%%%%%%%%%%%%%%%%%%%%%%%%%%%%%%%%%%%%%%%%%%%%%%%%%%%%%%%%%

%%%%%%%%%%%%%%%%%%%%%%%%%%%%%%%%%%%%%%%%%%%%%%%%%%%%%%%%%%%%%%%%%%%%%%
%%%%%%%%%%%%%%%%%%%%%%%%%%%%%%%%%%%%%%%
\section{\label{sec:vmuniverse}Virtual Machine Applications}
%%%%%%%%%%%%%%%%%%%%%%%%%%%%%%%%%%%%%%%
\index{virtual machine universe|(}
\index{universe!vm}
\index{vm universe}

The \SubmitCmd{vm} universe facilitates an HTCondor job
that matches and then lands a disk image on an execute machine
within an HTCondor pool.
This disk image is intended to be a virtual machine.
In this manner, the virtual machine is the job to be executed.

This section describes this type of HTCondor job.
See section~\ref{sec:Config-VMs}
for details of configuration variables.

%%%%%%%%%%%%%%%%%%%%%%%%%%%%%%%%%%%%%%%
\subsection{\label{sec:vm-submitfile}The Submit Description File}
%%%%%%%%%%%%%%%%%%%%%%%%%%%%%%%%%%%%%%%

Different than all other universe jobs,
the \SubmitCmd{vm} universe job specifies a disk image,
not an executable.
Therefore, the submit commands \SubmitCmd{input}, \SubmitCmd{output},
and \SubmitCmd{error} do not apply.
If specified, \Condor{submit} rejects the job with an error.
The \SubmitCmd{executable} command changes definition within a
\SubmitCmd{vm} universe job.
It no longer specifies an executable file, but instead
provides a string that identifies the job for tools such
as \Condor{q}.
Other commands specific to the type of virtual machine software
identify the disk image.

VMware, Xen, and KVM virtual machine software are supported.
As these differ from each other, the submit description file
specifies one of
\begin{verbatim}
  vm_type = vmware
\end{verbatim}
or
\begin{verbatim}
  vm_type = xen
\end{verbatim}
or
\begin{verbatim}
  vm_type = kvm
\end{verbatim}

The job is required to specify its memory needs 
for the disk image with \SubmitCmd{vm\_memory},
which is given in Mbytes.
HTCondor uses this number to assure a match with a machine
that can provide the needed memory space.

Virtual machine networking is enabled with the command
\begin{verbatim}
  vm_networking = true
\end{verbatim}
And, when networking is enabled, a definition of
\SubmitCmd{vm\_networking\_type} as \SubmitCmd{bridge}
matches the job only with a machine that is configured to use
bridge networking.
A definition of
\SubmitCmd{vm\_networking\_type} as \SubmitCmd{nat}
matches the job only with a machine that is configured to use
NAT networking.
When no definition of
\SubmitCmd{vm\_networking\_type} is given,
HTCondor may
match the job with a machine that enables networking,
and further, the choice of bridge or NAT networking
is determined by the machine's configuration.

Modified disk images are transferred back to the machine from which
the job was submitted as the \SubmitCmd{vm} universe job completes.
Job completion for a \SubmitCmd{vm} universe job occurs when 
the virtual machine is shut down, and HTCondor notices 
(as the result of a periodic check on the state of the virtual machine).
Should the job not want any files transferred back (modified or not),
for example because the job explicitly transferred its own files,
the submit command to prevent the transfer is
\begin{verbatim}
  vm_no_output_vm = true
\end{verbatim}

The required disk image must be identified for a virtual machine.
This \SubmitCmd{vm\_disk} command specifies a list of comma-separated files.
Each disk file is specified by colon-separated fields.
The first field is the path and file name of the disk file.
The second field specifies the device.
The third field specifies permissions, and the optional 
fourth specifies the format.
Here is an example that identifies a single file:
\footnotesize
\begin{verbatim}
  vm_disk = /var/lib/libvirt/images/swap.img:sda2:w:raw
\end{verbatim}
\normalsize

Setting values in the submit description file for some commands
have consequences for the virtual machine description file.
These commands are
\begin{itemize}
  \item \SubmitCmd{vm\_memory}
  \item \SubmitCmd{vm\_macaddr}
  \item \SubmitCmd{vm\_networking}
  \item \SubmitCmd{vm\_networking\_type}
  \item \SubmitCmd{vm\_disk}
\end{itemize}
For VMware virtual machines,
setting values for these commands causes HTCondor to modify the
\File{.vmx} file, overwriting existing values.
For KVM and Xen virtual machines,
HTCondor uses these values when it produces the description file.

For Xen and KVM jobs, if any files need to be transferred from the submit machine
to the machine where the \SubmitCmd{vm} universe job will execute, 
HTCondor must be explicitly told to do so with the 
standard file transfer attributes:
\footnotesize
\begin{verbatim}
  should_transfer_files = YES
  when_to_transfer_output = ON_EXIT
  transfer_input_files = /myxen/diskfile.img,/myxen/swap.img
\end{verbatim}
\normalsize
Any and all needed files on a system without a shared file
system (between the submit machine and the machine where the
job will execute) must be listed.

Further commands specify information that is specific to the
virtual machine type targeted.

%%%%%%%%%%%%%%%%%%%%%%%%%%%%%%%%%%%%%%%
\subsubsection{\label{sec:vm-VMwaresubmitfile}VMware-Specific Submit Commands}
%%%%%%%%%%%%%%%%%%%%%%%%%%%%%%%%%%%%%%%
\index{vm universe!submit commands specific to VMware}

Specific to VMware, the submit description file command
\SubmitCmd{vmware\_dir} gives the path and directory
(on the machine from which the job is submitted)
to where VMware-specific files and applications reside.
One example of a VMware-specific application is the VMDK files,
which form a virtual hard drive (disk image) for the virtual machine.
VMX files containing the primary configuration for the virtual
machine would also be in this directory.

HTCondor must be told whether or not the contents of the \SubmitCmd{vmware\_dir}
directory must be transferred to the machine where the job is
to be executed.
This required information is given with the submit command
\SubmitCmd{vmware\_should\_transfer\_files}.
With a value of \Expr{True},
HTCondor does transfer the contents of the directory.
With a value of \Expr{False},
HTCondor does not transfer the contents of the directory,
and instead presumes that access to this directory is
available through a shared file system.

By default, HTCondor uses a snapshot disk for new and modified files.
They may also be utilized for checkpoints.
The snapshot disk is initially quite small,
growing only as new files are created or files are modified.
When \SubmitCmd{vmware\_should\_transfer\_files} is \Expr{True},
a job may specify that a snapshot disk is \emph{not} to be
used with the command
\begin{verbatim}
  vmware_snapshot_disk = False
\end{verbatim}
In this case, HTCondor will utilize original disk files in producing
checkpoints. 
Note that \Condor{submit} issues an error message and does not
submit the job if both \SubmitCmd{vmware\_should\_transfer\_files}
and \SubmitCmd{vmware\_snapshot\_disk} are \Expr{False}.

Because \Prog{VMware Player} does not support snapshots, 
machines using \Prog{VMware Player} may only run \SubmitCmd{vm} jobs
that set \SubmitCmd{vmware\_snapshot\_disk} to \Expr{False}.
These jobs will also set
\SubmitCmd{vmware\_should\_transfer\_files} to \Expr{True}.
A job using \Prog{VMware Player} will go on hold if it attempts
to use a snapshot.
The pool administrator should have configured the pool
such that machines will not start jobs they can not run.

Note that if snapshot disks are requested and file transfer is not
being used, the \SubmitCmd{vmware\_dir} setting given in 
the submit description file
should not contain any symbolic link path components,
as described on the
\URL{https://htcondor-wiki.cs.wisc.edu/index.cgi/wiki?p=HowToAdminRecipes}
page under the answer to why VMware jobs with symbolic links fail.

Here is a sample submit description file for a VMware virtual machine:
\begin{verbatim}
universe                     = vm
executable                   = vmware_sample_job
log                          = simple.vm.log.txt
vm_type                      = vmware
vm_memory                    = 64
vmware_dir                   = C:\condor-test
vmware_should_transfer_files = True
queue
\end{verbatim}
This sample uses the \SubmitCmd{vmware\_dir} command to identify
the location of the disk image to be executed as an HTCondor job.
The contents of this directory are transferred to the machine assigned
to execute the HTCondor job.

%%%%%%%%%%%%%%%%%%%%%%%%%%%%%%%%%%%%%%%
\subsubsection{\label{sec:vm-Xensubmitfile}Xen-Specific Submit Commands}
%%%%%%%%%%%%%%%%%%%%%%%%%%%%%%%%%%%%%%%
\index{vm universe!submit commands specific to Xen}

% xen_kernel description
A Xen \SubmitCmd{vm} universe job requires specification of the
guest kernel. 
The \SubmitCmd{xen\_kernel} command accomplishes this, 
utilizing one of the following definitions.
\begin{enumerate}
\item \SubmitCmd{xen\_kernel = included} implies that the kernel
  is to be found in disk image given by the definition of the single file
  specified in \SubmitCmd{vm\_disk}. 

\item \SubmitCmd{xen\_kernel = path-to-kernel} gives a full path and
  file name of the required kernel.  If this kernel must be transferred
  to machine on which the \SubmitCmd{vm} universe job will execute,
  it must also be included in the \SubmitCmd{xen\_transfer\_files} command. 

  This form of the \SubmitCmd{xen\_kernel} command also requires further
  definition of the \SubmitCmd{xen\_root} command.
  \SubmitCmd{xen\_root} defines the device containing files needed by
  \Login{root}.

\end{enumerate}

%%%%%%%%%%%%%%%%%%%%%%%%%%%%%%%%%%%%%%%
\subsection{\label{sec:vm-checkpoints}Checkpoints}
%%%%%%%%%%%%%%%%%%%%%%%%%%%%%%%%%%%%%%%
\index{vm universe!checkpoints}

Creating a checkpoint is straightforward for a virtual machine,
as a checkpoint is a set of files that represent
a snapshot of both disk image and memory.
The checkpoint is created and all files are transferred back
to the \MacroUNI{SPOOL} directory on the machine from which
the job was submitted.
The submit command to create checkpoints is
\begin{verbatim}
  vm_checkpoint = true
\end{verbatim}
Without this command, no checkpoints are created (by default).
With the command, a checkpoint is created any time the \SubmitCmd{vm}
universe jobs is evicted from the machine upon which it is executing.
This occurs as a result of the machine configuration indicating
that it will no longer execute this job.

\SubmitCmd{vm} universe jobs can \emph{not} use a checkpoint server.

Periodic creation of checkpoints is not supported at this time.

Enabling both networking and checkpointing for a \SubmitCmd{vm}
universe job can cause networking problems when the job restarts,
particularly if the job migrates to a different machine.
\Condor{submit} will normally reject such jobs.
To enable both, then add the command
\begin{verbatim}
  when_to_transfer_output = ON_EXIT_OR_EVICT
\end{verbatim}

Take care with respect to the use of network connections within
the virtual machine and their interaction with checkpoints.
Open network connections at the time of the checkpoint will likely
be lost when the checkpoint is subsequently used to resume execution
of the virtual machine.
This occurs whether or not the execution resumes
on the same machine or a different one within the HTCondor pool.   

%%%%%%%%%%%%%%%%%%%%%%%%%%%%%%%%%%%%%%%
\subsection{\label{sec:vm-disk-image-details}Disk Images}
%%%%%%%%%%%%%%%%%%%%%%%%%%%%%%%%%%%%%%%

%%%%%%%%%%%%%%%%%%%%%%%%%%%%%%%%%%%%%%%
\subsubsection{\label{sec:vm-disk-image-details-vmware}
VMware on Windows and Linux}
%%%%%%%%%%%%%%%%%%%%%%%%%%%%%%%%%%%%%%%

Following the platform-specific
guest OS installation instructions found at
\URL{http://partnerweb.vmware.com/GOSIG/home.html},
creates a VMware disk image.

%%%%%%%%%%%%%%%%%%%%%%%%%%%%%%%%%%%%%%%
\subsubsection{\label{sec:vm-disk-image-details-xen}Xen and KVM}
%%%%%%%%%%%%%%%%%%%%%%%%%%%%%%%%%%%%%%%
While the following web page contains instructions specific to
Fedora on how to create a virtual guest image,
it should provide a good starting point for 
other platforms as well.

\URL{http://fedoraproject.org/wiki/Virtualization\_Quick\_Start}

%%%%%%%%%%%%%%%%%%%%%%%%%%%%%%%%%%%%%%%
\subsection{\label{sec:vm-job-completion-details}Job Completion in the vm Universe}
%%%%%%%%%%%%%%%%%%%%%%%%%%%%%%%%%%%%%%%

Job completion for a \SubmitCmd{vm} universe job occurs when 
the virtual machine is shut down, and HTCondor notices 
(as the result of a periodic check on the state of the virtual machine).
This is different from jobs executed under the environment of other 
universes.

Shut down of a virtual machine occurs from within the virtual
machine environment.
A script, executed with the proper authorization level,
is the likely source of the shut down commands.

Under a Windows 2000, Windows XP, or Vista virtual machine,
an administrator issues the command
\begin{verbatim}
  shutdown -s -t 01
\end{verbatim}

Under a Linux virtual machine,
the \Login{root} user executes
\begin{verbatim}
  /sbin/poweroff
\end{verbatim}
The command \verb@/sbin/halt@ will not completely
shut down some Linux distributions, and instead
causes the job to hang.

Since the successful completion of the \SubmitCmd{vm} universe job
requires the successful shut down of the virtual machine,
it is good advice to try the shut down procedure outside of
HTCondor, before a \SubmitCmd{vm} universe job is submitted.


\index{virtual machine universe|)}

%%%%%%%%%%%%%%%%%%%%%%%%%%%%%%%%%%%%%%%%%%%%%%%%%%%%%%%%%%%%%%%%%%%%%%

%%%%%%%%%%%%%%%%%%%%%%%%%%%%%%%%%%%%%%%%%%%%%%%%%%%%%%%%%%%%%%%%%%%%%%
%%%%%%%%%%%%%%%%%%%%%%%%%%%%%%%%%%%%%%%%%%%%%%%%%%%%%%%%%%%%%%%%%%%%%%
\section{Time Scheduling for Job Execution}
\label{sec:Job-Executetime-Scheduling}
%%%%%%%%%%%%%%%%%%%%%%%%%%%%%%%%%%%%%%%%%%%%%%%%%%%%%%%%%%%%%%%%%%%%%%
\index{scheduling jobs!to execute at a specific time}
\index{job execution!at a specific time}

Jobs may be scheduled to begin execution at a specified time in the future
with HTCondor's job deferral functionality.
All specifications are in a job's submit description file.
Job deferral functionality is expanded to provide for the
periodic execution of a job, known as the CronTab scheduling.

%%%%%%%%%%%%%%%%%%%%%%%%%%%%%%%%%%%%%%%%%%%
\subsection{Job Deferral}
\label{sec:JobDeferral}
%%%%%%%%%%%%%%%%%%%%%%%%%%%%%%%%%%%%%%%%%%%
\index{job deferral time}
\index{deferral time!of a job}

Job deferral allows the specification of
the exact date and time at which a job is to begin executing.
HTCondor attempts to match the job to an execution machine
just like any other job,
however, the job will wait until the exact time to begin execution.
A user can define the job to allow some flexibility in the execution of jobs
that miss their execution time.

%%%%%%%%%%%%%%%%%%%%%%%%%%%%%%%%%%%%%%%%%%%
\subsubsection{Deferred Execution Time}
\label{sec:JobDeferral-DeferralTime}
%%%%%%%%%%%%%%%%%%%%%%%%%%%%%%%%%%%%%%%%%%%
\index{deferral time!of a job}
\index{ClassAd job attribute!DeferralTime}

A job's deferral time is the exact time that HTCondor should attempt
to execute the job.
The deferral time attribute is defined as an expression
that evaluates to a Unix Epoch timestamp
(the number of seconds elapsed since 00:00:00 on January 1, 1970,
Coordinated Universal Time).
This is the time that HTCondor will begin to execute the job.

After a job is matched and all of its files have been transferred
to an execution machine,
HTCondor checks to see if the job's ClassAd contains a deferral time.
If it does,
HTCondor calculates the number of seconds between the execution
machine's current system time and the job's deferral time.
If the deferral time is in the future,
the job waits to begin execution.
While a job waits,
its job ClassAd attribute \AdAttr{JobStatus} indicates the job
is in the Running state.
As the deferral time arrives, the job begins to execute.
If a job misses its execution time,
that is, if the deferral time is in the past,
the job is evicted from the execution machine and put on hold in the queue.

The specification of a deferral time does not interfere
with HTCondor's behavior.
For example, if a job is waiting to begin execution
when a \Condor{hold} command is issued,
the job is removed from the execution machine and is put on hold.
If a job is waiting to begin execution when 
a \Condor{suspend} command is issued,
the job continues to wait.
When the deferral time arrives,
HTCondor begins execution for the job,
but immediately suspends it.

The deferral time is specified in the job's submit description file
with the command \SubmitCmd{deferral\_time}.

%%%%%%%%%%%%%%%%%%%%%%%%%%%%%%%%%%%%%%%%%%%
\subsubsection{Deferral Window}
\label{sec:JobDeferral-DeferralWindow}
%%%%%%%%%%%%%%%%%%%%%%%%%%%%%%%%%%%%%%%%%%%
\index{ClassAd job attribute!DeferralWindow}
\index{submit commands!deferral\_window}

If a job arrives at its execution machine
after the deferral time has passed,
the job is evicted from the machine and put on hold in the job queue.
This may occur, for example,
because the transfer of needed files took too long
due to a slow network connection.
A deferral window permits the execution of a job
that misses its deferral time by specifying a window of
time within which the job may begin.

The deferral window 
is the number of seconds after the deferral time,
within which the job may begin.
When a job arrives too late,
HTCondor calculates the difference in seconds
between the execution machine's current time
and the job's deferral time.
If this difference is less than or equal to the deferral window,
the job immediately begins execution.
If this difference is greater than the deferral window,
the job is evicted from the execution machine
and is put on hold in the job queue.

The deferral window is specified in the job's submit description file
with the command \SubmitCmd{deferral\_window}.

%%%%%%%%%%%%%%%%%%%%%%%%%%%%%%%%%%%%%%%%%%%
\subsubsection{Preparation Time}
\label{sec:JobDeferral-PrepTime}
%%%%%%%%%%%%%%%%%%%%%%%%%%%%%%%%%%%%%%%%%%%
\index{ClassAd job attribute!DeferralPrepTime}

When a job defines a deferral time far in the future and then 
is matched to an execution machine,
potential computation cycles are lost because the deferred job
has claimed the machine, but is not actually executing. 
Other jobs could execute during the interval when the job 
waits for its deferral time.
To make use of the wasted time,
\index{submit commands!deferral\_prep\_time}
a job defines a \SubmitCmd{deferral\_prep\_time}
with an integer expression that evaluates to a
number of seconds.
At this number of seconds before the deferral time,
the job may be matched with a machine.

%%%%%%%%%%%%%%%%%%%%%%%%%%%%%%%%%%%%%%%%%%%
\subsubsection{Usage Examples}
\label{sec:JobDeferral-Examples}
%%%%%%%%%%%%%%%%%%%%%%%%%%%%%%%%%%%%%%%%%%%

\index{submit commands!deferral\_time}
Here are examples of how the job deferral time,
deferral window, and the preparation time may be used.

The job's submit description file specifies that
the job is to begin execution 
on January 1st, 2006 at 12:00 pm:

\begin{verbatim} 
   deferral_time = 1136138400
\end{verbatim} 

The Unix \Prog{date} program may be used to calculate
a Unix epoch time.
The syntax of the command to do this depends on the options provided
within that flavor of Unix.  In some, it appears as
\begin{verbatim} 
%  date --date "MM/DD/YYYY HH:MM:SS" +%s
\end{verbatim} 
and in others, it appears as 
\begin{verbatim} 
%  date -d "YYYY-MM-DD HH:MM:SS" +%s
\end{verbatim} 

\verb@MM@ is a 2-digit month number,
\verb@DD@ is a 2-digit day of the month number, and
\verb@YYYY@ is a 4-digit year.
\verb@HH@ is the 2-digit hour of the day,
\verb@MM@ is the 2-digit minute of the hour, and
\verb@SS@ are the 2-digit seconds within the minute.
The characters \verb@+%s@ tell the \Prog{date} program
to give the output as a Unix epoch time.

The job always waits 60 seconds before
beginning execution:

\begin{verbatim} 
   deferral_time = (CurrentTime + 60)
\end{verbatim}

In this example, assume that the deferral time is 45 seconds
in the past as the job is available.
The job begins execution, because 75 seconds remain in the
deferral window:

\begin{verbatim} 
   deferral_window = 120
\end{verbatim}

In this example, a job is scheduled to execute
far in the future,
on January 1st, 2010 at 12:00 pm. 
The \SubmitCmd{deferral\_prep\_time} attribute delays the job 
from being matched until 60 seconds before the job is to begin execution. 

\begin{verbatim}
   deferral_time      = 1262368800
   deferral_prep_time = 60
\end{verbatim}

%%%%%%%%%%%%%%%%%%%%%%%%%%%%%%%%%%%%%%%%%%%
\subsubsection{Limitations}
\label{sec:JobDeferral-Limitations}
%%%%%%%%%%%%%%%%%%%%%%%%%%%%%%%%%%%%%%%%%%%
There are some limitations to HTCondor's job deferral feature.

\begin{itemize}
\item Job deferral is not available for scheduler universe jobs.
% no referring to daemons in the user's manual!
% Scheduler universe jobs are not executed under the control 
% of the \Condor{starter} daemon, 
% which is needed to defer the job until the correct execution time. 
A scheduler universe job defining the \AdAttr{deferral\_time}
produces a fatal error when submitted.

\item The time that the job begins to execute 
is based on the execution machine's system clock, 
and not the submission machine's system clock. 
Be mindful of the ramifications when
the two clocks show dramatically different times.

\item A job's \AdAttr{JobStatus} attribute is always in the Running state 
when job deferral is used.
There is currently no way to distinguish between a job that is 
executing and a job that is waiting for its deferral time. 

\end{itemize}

%%%%%%%%%%%%%%%%%%%%%%%%%%%%%%%%%%%%%%%%%%%
\subsection{CronTab Scheduling}
\label{sec:CronTab}
%%%%%%%%%%%%%%%%%%%%%%%%%%%%%%%%%%%%%%%%%%%
\index{CronTab job scheduling}
\index{job scheduling!periodic}
\index{scheduling jobs!to execute periodically}

HTCondor's CronTab scheduling functionality allows jobs to be 
scheduled to execute periodically. 
A job's execution schedule is defined by commands within
the submit description file.
The notation is much like that used by the Unix \Prog{cron} daemon. 
As such, HTCondor developers are fond of referring to CronTab
\index{Crondor}
scheduling as \Term{Crondor}.
The scheduling of jobs using HTCondor's CronTab feature 
calculates and utilizes
the \Attr{DeferralTime} ClassAd attribute. 

Also, unlike the Unix \Prog{cron} daemon, 
HTCondor never runs more than one instance of a job at the same time. 

The capability for repetitive or periodic execution of the job is 
enabled by specifying an \SubmitCmd{on\_exit\_remove}
command for the job,
such that the job does not leave the queue until desired.

%%%%%%%%%%%%%%%%%%%%%%%%%%%%%%%%%%%%%%%%%%%
\subsubsection{Semantics for CronTab Specification}
\label{sec:CronTab-Semantics}
%%%%%%%%%%%%%%%%%%%%%%%%%%%%%%%%%%%%%%%%%%%

A job's execution schedule is defined by a set of specifications
within the submit description file.
HTCondor uses these to calculate a \Attr{DeferralTime} for the job.

Table \ref{tab:CronTab-Attributes} 
lists the submit commands and acceptable values for these commands.
At least one of these must be defined 
in order for HTCondor to calculate a \Attr{DeferralTime} for the job.
Once one CronTab value is defined, 
the default for all the others uses 
all the values in the allowed values ranges.

\index{submit commands!cron\_minute}
\index{submit commands!cron\_hour}
\index{submit commands!cron\_day\_of\_month}
\index{submit commands!cron\_month}
\index{submit commands!cron\_day\_of\_week}

\begin{table}
   \begin{center}
   \begin{tabular}{ll}
   Submit Command & Allowed Values \\
   \hline
   \SubmitCmd{cron\_minute} & 0 - 59 \\
   \SubmitCmd{cron\_hour} & 0 - 23 \\
   \SubmitCmd{cron\_day\_of\_month} & 1 - 31 \\
   \SubmitCmd{cron\_month} & 1 - 12 \\
   \SubmitCmd{cron\_day\_of\_week} & 0 - 7 (Sunday is 0 or 7)\\
   \end{tabular}
   \end{center}
   \caption{The list of submit commands and their value ranges.}
   \label{tab:CronTab-Attributes}
\end{table}

The day of a job's execution can be specified 
by both the \SubmitCmd{cron\_day\_of\_month} 
and the \SubmitCmd{cron\_day\_of\_week} attributes. 
The day will be the logical or of both.

The semantics allow more than one value to be specified 
by using the \verb@*@ operator,
ranges, lists, and steps (strides) within ranges.

\begin{description}
   \item[The asterisk operator]
   The \verb@*@ (asterisk) operator specifies that all of the 
   allowed values are used for scheduling.
   For example,
   \begin{verbatim}
      cron_month = *
   \end{verbatim}
   becomes any and all of the list of possible months:
   (1,2,3,4,5,6,7,8,9,10,11,12).
   Thus, a job runs any month in the year.

   \item[Ranges]
   A range creates a set of integers from all the allowed values between two
   integers separated by a hyphen. The specified range is inclusive, and the
   integer to the left of the hyphen must be less than the right hand integer.
   For example,
   \begin{verbatim}
      cron_hour = 0-4
   \end{verbatim}
   represents the set of
   hours from 12:00 am (midnight) to 4:00 am, or (0,1,2,3,4).
   
   \item[Lists]
   A list is the union of the values or ranges separated by commas. Multiple
   entries of the same value are ignored. 
   For example,
   \begin{verbatim}
      cron_minute = 15,20,25,30
      cron_hour   = 0-3,9-12,15
   \end{verbatim}
   where this \SubmitCmd{cron\_minute} example represents (15,20,25,30)
   and \SubmitCmd{cron\_hour} represents (0,1,2,3,9,10,11,12,15).
      
   \item[Steps]
   Steps select specific numbers from a range, based on an interval.
   A step is specified by appending a range or the asterisk
   operator with a slash character (\verb@/@),
   followed by an integer value.
   For example,
   \begin{verbatim}
      cron_minute = 10-30/5
      cron_hour = */3
   \end{verbatim}
   where this \SubmitCmd{cron\_minute} example specifies
   every five minutes within the specified range 
   to represent (10,15,20,25,30),
   and \SubmitCmd{cron\_hour} specifies every three hours of the day
   to represent (0,3,6,9,12,15,18,21).
   

\end{description}

%%%%%%%%%%%%%%%%%%%%%%%%%%%%%%%%%%%%%%%%%%%
\subsubsection{Preparation Time and Execution Window}
\label{sec:CronTab-PrepTime}
%%%%%%%%%%%%%%%%%%%%%%%%%%%%%%%%%%%%%%%%%%%

The \SubmitCmd{cron\_prep\_time} command
is analogous to the deferral time's \SubmitCmd{deferral\_prep\_time} command. 
It specifies the number of seconds before the deferral time
that the job is to be matched and sent to the execution machine. 
This permits HTCondor to
make necessary preparations before the deferral time occurs. 

Consider the submit description file example that includes 
\begin{verbatim}
   cron_minute = 0
   cron_hour = *
   cron_prep_time = 300
\end{verbatim}
The job is scheduled to begin execution at the top of every hour.
Note that the setting of \SubmitCmd{cron\_hour} in this example
is not required, as the default value will be \verb@*@, 
specifying any and every hour of the day.
The job will be matched and sent to an execution machine 
no more than five minutes before the next deferral time. 
For example, if a job is submitted at 9:30am, then the 
next deferral time will be calculated to be 10:00am.
HTCondor may attempt to match the job to a machine and send the job
once it is 9:55am.

As the CronTab scheduling calculates and uses deferral time,
jobs may also make use of the deferral window.
The submit command \SubmitCmd{cron\_window} is analogous to
the submit command \SubmitCmd{deferral\_window}.
Consider the submit description file example that includes 
\begin{verbatim}
   cron_minute = 0
   cron_hour = *
   cron_window = 360
\end{verbatim}
As the previous example, the job is scheduled to begin execution
at the top of every hour.
Yet with no preparation time, the job is likely to miss
its deferral time.
The 6-minute window allows the job to begin execution,
as long as it arrives and can begin within 6 minutes of
the deferral time,
as seen by the time kept on the execution machine.

%%%%%%%%%%%%%%%%%%%%%%%%%%%%%%%%%%%%%%%%%%%
\subsubsection{Scheduling}
\label{sec:crontab-scheduling}
%%%%%%%%%%%%%%%%%%%%%%%%%%%%%%%%%%%%%%%%%%%

When a job using the CronTab functionality is submitted to HTCondor, 
use of at least one of the submit description file commands
beginning with \SubmitCmd{cron\_} causes HTCondor
to calculate and set a deferral time for when the job should run. 
A deferral time is determined based on the current time 
rounded later in time to the next minute. 
The deferral time is the job's \AdAttr{DeferralTime} attribute. 
A new deferral time is calculated when the job 
first enters the job queue, when 
the job is re-queued, or when the job is released from the hold state. 
New deferral times for \emph{all} jobs in the job queue 
using the CronTab functionality are recalculated 
when a \Condor{reconfig} or a \Condor{restart} command that
affects the job queue is issued.

A job's deferral time is not always the same time that a job 
will receive a match and be sent to the execution machine. 
This is because HTCondor operates on the job queue
at times that are independent of job events,
such as when job execution completes.
Therefore,
HTCondor may operate on the job queue just after 
a job's deferral time states that it is to begin execution. 
HTCondor attempts to start a job when the 
following pseudo-code boolean expression evaluates to \Expr{True}:

\footnotesize
\begin{verbatim}
   ( CurrentTime + SCHEDD_INTERVAL ) >= ( DeferralTime - CronPrepTime )
\end{verbatim}
\normalsize

If the \Attr{CurrentTime} plus the number of seconds 
until the next time HTCondor checks 
the job queue is greater than or equal to the time that the job 
should be submitted to the execution machine, 
then the job is to be matched and sent now.

Jobs using the CronTab functionality are not automatically 
re-queued by HTCondor after their execution is complete. 
The submit description file for a job
must specify an appropriate \SubmitCmd{on\_exit\_remove} 
command to ensure that a job remains in the queue. 
This job maintains its original \Attr{ClusterId} and \Attr{ProcId}.

%%%%%%%%%%%%%%%%%%%%%%%%%%%%%%%%%%%%%%%%%%%
\subsubsection{Usage Examples}
\label{sec:crontab-examples}
%%%%%%%%%%%%%%%%%%%%%%%%%%%%%%%%%%%%%%%%%%%

Here are some examples of the submit commands
necessary to schedule jobs to run at multifarious times. 
Please note that it is not necessary to 
explicitly define each attribute; the default value is \verb@*@.

Run 23 minutes after every two hours, every day of the week:

\begin{verbatim}
   on_exit_remove = false
   cron_minute = 23
   cron_hour = 0-23/2
   cron_day_of_month = *
   cron_month = *
   cron_day_of_week = *
\end{verbatim}

Run at 10:30pm on each of May 10th to May 20th, as well as every 
remaining Monday within the month of May:

\begin{verbatim}
   on_exit_remove = false
   cron_minute = 30
   cron_hour = 20
   cron_day_of_month = 10-20
   cron_month = 5
   cron_day_of_week = 2
\end{verbatim}

Run every 10 minutes and every 6 minutes before noon 
on January 18th with a 2-minute preparation time:

\begin{verbatim}
   on_exit_remove = false
   cron_minute = */10,*/6
   cron_hour = 0-11
   cron_day_of_month = 18
   cron_month = 1
   cron_day_of_week = *
   cron_prep_time = 120
\end{verbatim}

%%%%%%%%%%%%%%%%%%%%%%%%%%%%%%%%%%%%%%%%%%%
\subsubsection{Limitations}
\label{sec:Crontab-Limitations}
%%%%%%%%%%%%%%%%%%%%%%%%%%%%%%%%%%%%%%%%%%%
The use of the CronTab functionality has all of the same 
limitations of deferral times,
because the mechanism is based upon deferral times.

\begin{itemize}
\item It is impossible to schedule vanilla 
and standard universe jobs 
at intervals that are smaller than the
interval at which HTCondor evaluates jobs.
This interval is determined by 
the configuration variable \Macro{SCHEDD\_INTERVAL}. 
As a vanilla or standard universe job completes execution 
and is placed back into the job queue, 
it may not be placed in the idle state in time.
This problem does not afflict local universe jobs.

\item HTCondor cannot guarantee that a job will be
matched in order to make its scheduled deferral time.
A job must be matched with an execution machine just as
any other HTCondor job; 
if HTCondor is unable to find a match, 
then the job will miss its chance for executing
and must wait for the next execution time 
specified by the CronTab schedule.

\end{itemize}

%%%%%%%%%%%%%%%%%%%%%%%%%%%%%%%%%%%%%%%%%%%%%%%%%%%%%%%%%%%%%%%%%%%%%%

%%%%%%%%%%%%%%%%%%%%%%%%%%%%%%%%%%%%%%%%%%%%%%%%%%%%%%%%%%%%%%%%%%%%%%
%%%%%%%%%%%%%%%%%%%%%%%%%%%%%%%%%%%%%%%%
\section{\label{sec:Stork}Stork Applications}
%%%%%%%%%%%%%%%%%%%%%%%%%%%%%%%%%%%%%%%
\index{Stork|(}

Today's scientific applications have huge data requirements,
which continue to increase drastically every year.
These data are generally accessed by many
users from all across the the globe.
This requires moving huge amounts of data
around wide area networks to complete the computation cycle,
which brings with
it the problem of efficient and reliable data placement.

Stork is a scheduler for data placement.
With Stork, \Term{data placement jobs}
have been elevated to the same level as Condor's computational jobs;
data placements are queued, managed, queried and
autonomously restarted upon error.
Stork understands the semantics and protocols of data placement.

The underlying data placement jobs are performed by Stork
\Term{modules}, typically installed in the Condor \File{libexec}
directory.  The module name is encoded from the data placement type
and functions.  
For example, the \File{stork.transfer.file-file} module transfers data
from the \File{file:/} (local filesystem) to the \File{file:/}
protocol.  The \File{stork.transfer.file-file} module is the only
module bundled with Condor/Stork.  Additionally, contributed modules
may be \htmladdnormallink{downloaded}
{http://www.cs.wisc.edu/condor/stork/download.html}
for these data transfer protocols:

\begin{table}[hbt]
\begin{tabular}{ l l }
%file:/		& (Stork Server) local file system \\
ftp://		& FTP File Transfer Protocol \\
http://		& HTTP Hypertext Transfer Protocol \\
gsiftp://	& Globus Grid FTP  \\ 
nest://		& Condor NeST  network storage appliance (see
\URL{http://www.cs.wisc.edu/condor/nest/}) \\
srb://		& SDSC  Storage Resource Broker (SRB) (see
\URL{http://www.sdsc.edu/srb/}) \\
srm://		& Storage Resource Manager (SRM) (see
\URL{http://sdm.lbl.gov/srm-wg/}) \\
csrm://		& Castor Storage Resource Manager (Castor SRM) (see
\URL{http://castor.web.cern.ch/castor/}) \\
unitree://    & NCSA UniTree (see
\URL{http://www.ncsa.uiuc.edu/Divisions/CC/HPDM/unitree/}) \\
%		diskrouter:// -> UW DiskRouter Tool
\end{tabular}
\end{table}

The Stork
module API is simple and extensible, enabling users to create 
and use their own modules.

Stork includes high level features for managing data transfers.
By configuration, the number of active jobs running
from a Stork server may be limited.
Stork includes built in fault tolerance,
with capabilities for retrying failed jobs,
together with the specification of alternate protocols.
Stork users also have access to a higher level job manager, 
Condor DAGMan (section \ref{sec:DAGMan}),
which can manage both Stork data placement jobs
and traditional Condor jobs at the same time.


%%%%%%%%%%%%%%%%%%%%%%%%%%%%%%%%%%%%%%%
\subsection{\label{sec:Stork-Job-Submission}Submitting Stork Jobs}
%%%%%%%%%%%%%%%%%%%%%%%%%%%%%%%%%%%%%%%
\index{Stork!submit description file}

As with Condor jobs, Stork jobs are specified with a
submit description file.
It is important to note the syntax of the submit description file
for a Stork job is different than that used by Condor jobs.
Specifically,
Stork submit description files are written in the
\htmladdnormallink{ClassAd}{http://www.cs.wisc.edu/condor/classad}
language.  
See the ClassAd Language Reference Manual for complete details.
Please note that while most of Condor uses ClassAds,
Stork utilizes the most recent version of this language,
which has evolved over time.
Stork defines keywords.
When present in the job submit file,
keywords define the function of the job.

Here is sample Stork job submit description file,
showing file syntax and keywords.
A job specifies a 1-to-1 mapping of a data source URL to
destination URL.

\footnotesize
\begin{verbatim}
// This is a comment line.
[
    dap_type = transfer;
    src_url = "file:/etc/termcap";
    dest_url = "file:/tmp/stork/file-termcap";
]
\end{verbatim}
\normalsize


This example shows the ClassAd pairs that form the
heart of a Stork job specification.
The minimum keywords required to specify a Stork job are:

\begin{description}
  \item[dap\_type] Currently, the data type is constrained to 
  \SubmitCmd{transfer}.

  \item[src\_url]  Specify the data protocol and URL of the source.

  \item[dest\_url]  Specify the data protocol and URL of the destination.
\end{description}

Additionally, the following keywords may be used in 
a Stork submit description file:

\begin{description}
    \item[x509proxy] Specifies the location of the
    X.509 proxy file for protocols
    that use GSI authentication, such as \SubmitCmd{gsiftp://}.
    The special value of \emph{"default"} (quotes are required) invokes
    GSI libraries to search for the user credential in the standard locations.

    \item[alt\_protocols] A comma separated list of
    alternative protocol pairs (for source and destination protocols),
    used in a round robin fashion
    when transfers fail.
    See section \ref{sec:Stork-Fault-Protection} for a further discussion
    and examples.


\end{description}

Stork places no restriction on the submit file name or extension, and will
accept any valid file name for a Stork submit description file.

Submit data placement jobs to Stork using the
\Stork{submit} tool.
For example, after creating the submit description file
\File{sample.stork} with an
editor, submit the data transfer job with the command:

\begin{verbatim}
stork_submit sample.stork
\end{verbatim}

Stork then returns the associated job id, which is used by other Stork
job control tools.

Only the first ClassAd (a record expression within brackets) within 
a Stork submit description file becomes a data placement job
upon submission.
Other ClassAds within the file are ignored.

%%%%%%%%%%%%%%%%%%%%%%%%%%%%%%%%%%%%%%%
\subsection{\label{sec:Stork-Job-Management}Managing Stork Jobs}
%%%%%%%%%%%%%%%%%%%%%%%%%%%%%%%%%%%%%%%
Stork provides a set of command-line user tools for job management, including
submitting, querying, and removing data placement jobs.

%%%%%%%%%%%%%%%%%%%%%%%%%%%%%%%%%%%%%%%
\subsubsection{\label{sec:stork-query}Querying Stork Jobs}
%%%%%%%%%%%%%%%%%%%%%%%%%%%%%%%%%%%%%%%

Use \Stork{status} to check the status of any active
or completed Stork job.
\Stork{status} takes a single argument: the job id.
For example, to check the status of the Stork job with job id 3:

\begin{verbatim}
stork_status 3
\end{verbatim}

Use \Stork{q} to query all active Stork jobs.
\Stork{q} does not report on completed Stork jobs.

For example, to check the status all active Stork jobs:

\begin{verbatim}
stork_q
\end{verbatim}

%%%%%%%%%%%%%%%%%%%%%%%%%%%%%%%%%%%%%%%
\subsubsection{\label{sec:stork-rm}Removing Stork Jobs}
%%%%%%%%%%%%%%%%%%%%%%%%%%%%%%%%%%%%%%%

Active jobs may be removed from the job queue with the 
\Stork{rm} tool.  
\Stork{rm} takes a single argument: the job id of the job to remove.
All jobs may
be removed, provided they have not completed.

For example, to remove the queued job with job id 4:

\begin{verbatim}
stork_rm 4
\end{verbatim}

%%%%%%%%%%%%%%%%%%%%%%%%%%%%%%%%%%%%%%%
\subsection{\label{sec:Stork-Fault-Protection}Fault Tolerance}
%%%%%%%%%%%%%%%%%%%%%%%%%%%%%%%%%%%%%%%

In an ideal world, all data transfers succeed on the first attempt.
However, data transfers do fail for various reasons.
Stork is designed with data transfer fault tolerance.
Based on configuration, Stork retries failed data transfer jobs
using specified protocols.

If a  transfer fails, Stork attempts the transfer again,
until the number of attempts reaches the limit,
as defined by the 
configuration variable
\Macro{STORK\_MAX\_RETRY} 
(section \ref{param:StorkMaxRetry}).  

For each attempt at transfer,
the transfer protocols to be used at both source and destination are defined.
These transfer protocols may vary,
when defined by an \SubmitCmd{alt\_protocols} entry in the
submit description file.
The location of the data at the source and destination
is unchanged by the \SubmitCmd{alt\_protocols} entry.
\SubmitCmd{alt\_protocols} defines an ordered list of alternative
translation protocols to be used.
Each entry in the list is a pair.
The first of the pair defines the protocol to be used at the source 
of the transfer.
The second of the pair defines the protocol to be used at the destination 
of the transfer.

The syntax is a comma-separated list of pairs.
A dash character separated the pairs.
The protocol name is given in all lower case letters,
without colons or slash characters.
Stork uses these strings to identify the protocol translation
and transfer module to be used. 

The initial translation protocol
(specified in the \SubmitCmd{src\_url} and \SubmitCmd{dest\_url} entries)
together with the list defined by  
an \SubmitCmd{alt\_protocols} entry form the ordered list
of protocols to be utilized in a round robin fashion.

For example, if \MacroNI{STORK\_MAX\_RETRY} has the value 4,
and the Stork job submit description file contains
\footnotesize
\begin{verbatim}
[
    dap_type = transfer;
    src_url = "gsiftp://serverA/dirA/fileA";
    dest_url = "http://serverB/dirB/fileB";
]
\end{verbatim}
\normalsize

then Stork will attempt up to 4 transfers,
with each using the same translation protocol.
\SubmitCmd{gsiftp://} is used at the source,
and \SubmitCmd{http://} is used at the destination.
The Stork job fails if it has not been completed after
4 attempts.

A second example shows the transfer protocols used for
each attempted transfer, 
when \SubmitCmd{alt\_protocols} is used.
For this example, assume that 
\MacroNI{STORK\_MAX\_RETRY} has the value 7.
\footnotesize
\begin{verbatim}
[
    dap_type = transfer;
    src_url = "gsiftp://no-such-server/dir/file";
    dest_url = "file:/dir/file";
    alt_protocols = "ftp-file, http-file";
]
\end{verbatim}
\normalsize

Stork attempts the following transfers, in the given order,
stopping when the transfer succeeds.
\begin{enumerate}
    \item from 
			\File{gsiftp://no-such-server/dir/file}
            to
			\File{file:/dir/file}
    \item from 
			\File{ftp://no-such-server/dir/file}
            to
			\File{file:/dir/file}
    \item from 
			\File{http://no-such-server/dir/file}
            to
			\File{file:/dir/file}
    \item from 
			\File{gsiftp://no-such-server/dir/file}
            to
			\File{file:/dir/file}
    \item from 
			\File{ftp://no-such-server/dir/file}
            to
			\File{file:/dir/file}
    \item from 
			\File{http://no-such-server/dir/file}
            to
			\File{file:/dir/file}
    \item from 
			\File{gsiftp://no-such-server/dir/file}
            to
			\File{file:/dir/file}
 \end{enumerate}

%%%%%%%%%%%%%%%%%%%%%%%%%%%%%%%%%%%%%%%
\subsection{\label{sec:Stork-Advanced}Running Stork Jobs Under DAGMan}
%%%%%%%%%%%%%%%%%%%%%%%%%%%%%%%%%%%%%%%
\index{Stork!jobs under DAGMan}

Condor DAGMan (section \ref{sec:DAGMan}) provides high level management
of both traditional CPU jobs and Stork data placement jobs. 
Using DAGMan, users can specify data placement
using the \Arg{DATA} keyword.
DAGMan can mix Stork data transfer jobs 
and Condor jobs.
This capability lends itself well to grid computing,
as data is often staged in (transferred)
before processing the data.
After processing, output is often staged out (transferred).

Here is a sample DAGMan input file
that stages in input files using Stork transfers,
processes the data as a Condor job,
and stages out the result using a Stork transfer.

\footnotesize
\begin{verbatim}
# Transfer input files using Stork
DATA INPUT1 transfer_input_data1.stork
DATA INPUT1 transfer_input_data2.stork

DATA INPUT2 transfer_data
#
# Process the data using Condor
JOB PROCESS process.condor
#
# Transfer output file using Stork
DATA RESULT transfer_result_data.stork
#
# Specify job dependencies
PARENT INPUT1 INPUT2 CHILD PROCESS
PARENT PROCESS CHILD RESULT
\end{verbatim}
\normalsize

\index{Stork|)}

%%%%%%%%%%%%%%%%%%%%%%%%%%%%%%%%%%%%%%%%%%%%%%%%%%%%%%%%%%%%%%%%%%%%%%


%%%%%%%%%%%%%%%%%%%%%%%%%%%%%%%%%%%%%%%%
\section{Special Environment Considerations}
%%%%%%%%%%%%%%%%%%%%%%%%%%%%%%%%%%%%%%%%

%%%%%%%%%%%%%%%%%%%%%%%%%%%%%%%%%%%%%%%%
\subsection{AFS}

\index{file system!AFS}
\index{AFS!interaction with}
The HTCondor daemons do not run authenticated to AFS; they do not possess
AFS tokens.
Therefore, no child process of HTCondor will be AFS authenticated.
The implication of this is that you must set file permissions so
that your job can access any necessary files residing on an AFS volume
without relying on having your AFS permissions.

If a job you submit to HTCondor needs to access files residing in AFS,
you have the following choices:
\begin{enumerate}
\item Copy the needed files from AFS to either a local hard disk where 
HTCondor can access them using remote system calls (if
this is a standard universe job), or copy them to an NFS volume.
\item If the files must be kept on AFS, then set a host ACL
(using the AFS \Prog{fs setacl} command) on the subdirectory to
serve as the current working directory for the job.
If this is a standard universe job, then the host ACL needs
to give read/write permission to any process on the submit machine.
If this is a vanilla universe job, then set the ACL such that any host 
in the pool can access the files without being authenticated.
If you do not know how to use an AFS host ACL, ask the person at your 
site responsible for the AFS configuration.
\end{enumerate}

The Center for High Throughput Computing hopes to improve upon how 
HTCondor deals with AFS 
authentication in a subsequent release.

Please see section~\ref{sec:Condor-AFS-Users} on
page~\pageref{sec:Condor-AFS-Users} in the Administrators Manual for
further discussion of this problem.

%%%%%%%%%%%%%%%%%%%%%%%%%%%%%%%%%%%%%%%%
\subsection{NFS}

\index{file system!NFS}
\index{NFS!interaction with}
If the current working directory when a job is submitted
is accessed via an NFS automounter, HTCondor may have problems if the
automounter later decides to unmount the volume before the job has
completed.
This is because \Condor{submit} likely has stored the
dynamic mount point as the job's initial current working directory, and
this mount point could become automatically unmounted by the
automounter.

There is a simple work around.
When submitting the job,
use the submit command \SubmitCmd{initialdir} to point to
the stable access point.
For example,
suppose the NFS automounter is configured to mount a volume at mount point
\File{/a/myserver.company.com/vol1/johndoe}
whenever the directory \File{/home/johndoe} is accessed.
Adding the following line to the
submit description file solves the problem.
\begin{verbatim}
  initialdir = /home/johndoe
\end{verbatim}

\index{NFS!cache flush on submit machine}
\index{ClassAd job attribute!IwdFlushNFSCache}
%As of HTCondor version 7.4.0, 
HTCondor attempts to flush the NFS cache on a submit machine in order to
refresh a job's initial working directory.
This allows files written by the job into an NFS mounted 
initial working directory to be immediately visible on the submit machine.
Since the flush operation can require multiple round trips
to the NFS server, it is expensive.
Therefore, a job may disable the flushing by setting
\begin{verbatim}
  +IwdFlushNFSCache = False
\end{verbatim}
in the job's submit description file.
See page~\pageref{IwdFlushNFSCache-job-attribute} for a definition
of the job ClassAd attribute.

%%%%%%%%%%%%%%%%%%%%%%%%%%%%%%%%%%%%%%%%
\subsection{HTCondor Daemons That Do Not Run as root}

\index{running as root}
\index{daemon!running as root}
HTCondor is normally installed such that the HTCondor daemons have root
permission.
This allows HTCondor to run the \Condor{shadow} 
\index{HTCondor daemon!condor\_shadow}
\index{remote system call!condor\_shadow}
daemon and
the job with the submitting user's UID and file access rights.
When HTCondor
is started as root, HTCondor jobs can access whatever files the
user that submits the jobs can.

However, it is possible that the HTCondor installation 
does not have root access, or
has decided not to run the daemons as root.
That is unfortunate,
since HTCondor is designed to be run as root.
To see if HTCondor is
running as root on a specific machine, use the command
\begin{verbatim}
  condor_status -master -l <machine-name>
\end{verbatim}

where \verb@<machine-name>@ is the name of the specified machine.
This command displays the full \condor{master} ClassAd; if the
attribute \AdAttr{RealUid} equals zero,
then the HTCondor daemons are indeed
running with root access.  If the
\AdAttr{RealUid} attribute is not zero, then the HTCondor daemons do not have
root access.

\Note The Unix program \Prog{ps}
is \emph{not} an effective
method of determining if HTCondor is running with root access.
When using \Prog{ps},
it may often appear that the daemons are
running as the condor user instead of root.
However, note that the \Prog{ps}
command shows the current \emph{effective} owner of the
process, not the \emph{real} owner.  (See the \Cmd{getuid}{2} and
\Cmd{geteuid}{2} Unix man pages for details.)  In Unix, a process
running under the real UID of root may switch its effective UID.
(See the \Cmd{seteuid}{2} man page.)
For security reasons, the daemons
only set the effective UID to root when absolutely necessary,
as it will be to perform a privileged operation.

If daemons are not running with root access, 
make any and all files
and/or directories that the job will touch readable and/or writable by
the UID (user id) specified by the \Attr{RealUid} attribute.
Often this may
mean using the Unix command \verb@chmod 777@
on the directory from which the HTCondor job is submitted.

%%%%%%%%%%%%%%%%%%%%%%%%%%%%%%%%%%%%%%%%
\subsection{\label{sec:Job-Lease}
Job Leases}
%%%%%%%%%%%%%%%%%%%%%%%%%%%%%%%%%%%%%%%%
\index{job lease}

A \Term{job lease} specifies how long a given job will attempt to run
on a remote resource,
even if that resource loses contact with the submitting machine.
Similarly, it is the length of time the submitting machine will
spend trying to reconnect to the (now disconnected) execution host,
before the submitting machine gives up and tries to claim
another resource to run the job.
The goal aims at run only once semantics,
so that the \Condor{schedd} daemon does not allow the same job
to run on multiple sites simultaneously.

If the submitting machine is alive,
it periodically renews the job lease,
and all is well.
If the submitting machine is dead,
or the network goes down, the job lease will no longer be renewed.
Eventually the lease expires.
While the lease has not expired,
the execute host continues to try to run the job,
in the hope that the submit machine will come back to life
and reconnect.
If the job completes and the lease has not expired, yet the 
submitting machine is still dead,
the \Condor{starter} daemon will wait for a
\Condor{shadow} daemon to reconnect, 
before sending final information on the job,
and its output files.
Should the lease expire, the \Condor{startd} daemon
kills off the \Condor{starter} daemon and user job.

\index{ClassAd job attribute!JobLeaseDuration}
\index{JobLeaseDuration!job ClassAd attribute}
A default value equal to 20 minutes exists for a job's
ClassAd attribute \Attr{JobLeaseDuration}, 
or this attribute may be set in the submit description file,
using \SubmitCmd{job\_lease\_duration},
to keep a job running in the case that the submit side no longer
renews the lease.
There is a trade off in setting the value of \SubmitCmd{job\_lease\_duration}. 
Too small a value,
and the job might get killed before the submitting machine has a
chance to recover.
Forward progress on the job will be lost.
Too large a value,
and an execute resource will be tied up waiting for the job lease to expire.
The value should be chosen based on how long the user is willing to tie up
the execute machines, how quickly submit machines come  back up,
and how much work would be lost if the lease expires,
the job is killed, and the job must start over from its beginning.

As a special case, a submit description file setting of
\begin{verbatim}
 job_lease_duration = 0
\end{verbatim}
as well as utilizing submission other than \Condor{submit}
that do not set \Attr{JobLeaseDuration}
(such as using the web services interface)
results in the corresponding job ClassAd attribute to be explicitly
undefined.
This has the further effect of changing the duration of a claim lease,
the amount of time that the execution machine waits before
dropping a claim due to missing keep alive messages.

%%%%%%%%%%%%%%%%%%%%%%%%%%%%%%%%%%%%%%%%
\section{Potential Problems}
%%%%%%%%%%%%%%%%%%%%%%%%%%%%%%%%%%%%%%%%

\subsection{\label{sec:renaming-argv}Renaming of argv[0]}

\index{argv[0]!HTCondor use of}
When HTCondor starts up your job, it renames argv[0] (which usually
contains the name of the program) to \condor{exec}.
This is
convenient when examining a machine's processes with the Unix
command \Prog{ps}; the process
is easily identified as an HTCondor job.  

Unfortunately, some programs read argv[0] expecting their own program
name and get confused if they find something unexpected like
\condor{exec}.

\index{HTCondor!user manual|)}
\index{user manual|)}
