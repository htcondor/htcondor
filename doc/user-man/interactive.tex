%%%%%%%%%%%%%%%%%%%%%%%%%%%%%%%%%%%%%%%%%%%%%%%%%%%%%%%%%%%%%%%%%%%%%%
\subsection{\label{sec:Submit-Interactive}Interactive Jobs}
%%%%%%%%%%%%%%%%%%%%%%%%%%%%%%%%%%%%%%%%%%%%%%%%%%%%%%%%%%%%%%%%%%%%%%
\index{job!interactive}
\index{interactive jobs}

An \Term{interactive job} is a Condor job that is provisioned and
scheduled like any other vanilla universe Condor job 
onto an execute machine within the pool.
The result of a running interactive job is a shell prompt 
issued on the execute machine where the job runs.

This job (shell) continues until the user logs out or any other
policy implementation causes the job to stop running.

The current working directory of the shell will be the
initial working directory of the running job.
The shell type will be the default for the user that submits
the job.

Each interactive job will have a job ClassAd attribute of 
\begin{verbatim}
  InteractiveJob = True
\end{verbatim}

Submission of an interactive job specifies the option \Opt{-interactive}
on the \Condor{submit} command line.

A submit description file may be specified for this interactive job.
Within this submit description file, 
a specification of these 4 commands will be either ignored or altered:
\begin{enumerate}
\item \SubmitCmd{executable}
\item \SubmitCmd{transfer\_executable}
\item \SubmitCmd{universe}
\item \SubmitCmd{arguments}
\item \SubmitCmd{queue <n>}.  In this case the value of \Expr{<n>} is
ignored, exactly one interactive job is queued.
\end{enumerate}

If \emph{no} submit description file is specified for the job,
a default one is utilized as identified by the value of the
configuration variable \Macro{INTERACTIVE\_SUBMIT\_FILE}.

\MoreTodo
