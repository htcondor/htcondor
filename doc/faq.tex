\index{Condor!FAQ|(}
\index{Condor!Frequently Asked Questions|(}
\index{FAQ|(}
\index{Frequently Asked Questions|(}
This section is the where you can find quick answers to some commonly
asked questions that are asked about Condor.
\index{Condor!FAQ|)}
\index{Condor!Frequently Asked Questions|)}
\index{FAQ|)}
\index{Frequently Asked Questions|)}

\section{Obtaining Condor Distributions}

\subsection{Where can I download Condor?}

Condor can be downloaded from \Url{http://www.cs.wisc.edu/condor/downloads/} or from
\Url{http://www.bo.infn.it/condor-mirror/downloads/}.

\subsection{If I try to download Condor, it sends me back to the downloads page!}

If you are try to download Condor with a "junk-busting" web-proxy, try disabling it.
We follow the path you take through our download menus, and sometimes these proxies strip off necessary information.

\subsection{Do you have any website mirrors?}

Yes.  We have our main site at \Url{http://www.cs.wisc.edu/condor/}
which is located in Madison, Wisconsin, USA.  We also operate a mirror
site at the Istituto Nazionale di Fisica Nucleare in Bologna, Italy
at \Url{http://www.bo.infn.it/condor-mirror/}.  Currently we have the 
binary distrubations of Condor and the Condor Manual available on the
mirror site.

\subsection{What platforms do you support?}

See Section~\ref{sec:Availability}, on page~\pageref{sec:Availability}.

\subsection{Do you distribute source code?}

At this time we do \Bold{not} distribute source code.
Even so, if you need the source code, email us at \Email{condor-admin@cs.wisc.edu} and tell us why.

\subsection{What is Personal Condor?}

Personal Condor is a term used for a specific style of installing Condor.
A Personal Condor is a one-machine pool.  When you use flocking to bind
them together, its simular to a normal Condor pool, except that you are restricted
to "standard" jobs, and you are running additional \condor{collector} and \condor{negotiator} 
daemons.

\subsection{Do you support submitting to Globus?}

Yes, we do.  At this time, however, we have not released it (yet!)



\section{Setting up Condor}


\subsection{How do I get more than one job to run on my SMP machine?}

Starting with Condor version 6.1, Condor will recognize a SMP machine and advertise each CPU of the
machine seperately.  For more details, see section~\ref{sec:Configuring-SMP} on page~\pageref{sec:Configuring-SMP}.

Support for SMP machines in Condor Version 6.0 is available as a contrib module from the download site.

\subsection{How do I configure Condor to only run my jobs at certain times of the day?}

\Todo



\section{Running Condor Jobs}


\subsection{I'm at the University of Wisconsin--Madison, Computer Science department, and I am having problems!}

First, please read the webpage \Url{http://www.cs.wisc.edu/condor/uwcs.html} before proceeding!
Specifically, your home directory is in AFS, which causes Condor some specific problems.
The above URL will contain the way to fix things here at UWCS.

\subsection{I'm getting a lot of email from Condor, should I just delete all of it?}

Generally you shouldn't ignore all of the mail Condor sends.  Often, Condor will
be sending legitimate email.  To reduce the email that a job will send you, see

\Todo


\subsection{Why do my vanilla jobs only run on the machine where I submitted them from?}
Consider the following:
\begin {enumerate}
\item{Did you set the special requirements expressions for
your vanilla jobs?}

See Section~\ref{sec:Shared-Filesystem-Config-File-Entries}, on page~\pageref{sec:Shared-Filesystem-Config-File-Entries}.

\item{Did you submit the job from a shared filesystem?}

See Section~\ref{sec:Shared-Filesystem-Config-File-Entries}, on page~\pageref{sec:Shared-Filesystem-Config-File-Entries}.

\item{Is Condor running as root?}

See Section~\ref{sec:Non-Root}, on page~\pageref{sec:Non-Root}.

\end{enumerate}
\subsection{How do I configure my pool to run vanilla jobs or to use a "clipped" port of Condor?}

\Todo

\subsection{Why won't my jobs run?}

Common problems that are often reported to us are that no jobs are running, or
something like:

\begin{verbatim}
> I have submitted 100 jobs to my pool, and only 18 apper to be running,
> and there are plenty of resources available.  What should I do to
> investigate the reason why this happens?
\end{verbatim}

Start by using the following steps to solve this problem:

\begin{enumerate}
\item Try \condor{q} -analyze 

\item Look at the UserLog file (whatever you specify "log = XXX" in the
submit file).  See if the jobs are starting to run and then exit right
away, or if they never even start to run.

\item Look at the SchedLog on the submit machine after it negotiates
for this user.  Often if a user doesn't have enough priority to get
more machines, and if so, the SchedLog will have a message with something
like "lost priority, no more jobs".

\item If the jobs are getting matched, they might be dying right away due
to permission problems or something like that.  Check the ShadowLog on
the submit machine for warnings or errors.

\item Look at the NegotiatorLog during the negotiation for the user.
Look for similar messages about priority, "no more machines", or
something like this.  

\end{enumerate}

\section{Troubleshooting}


\subsection{What happens if the central manager crashes?} 

If the central manager crashes, all running jobs will continue to
run. All queued jobs will simply remain in the queues until the central
manager comes up and begins matchmaking again.  Nothing special needs
to be done when the cenrtal manager is brought back online. 


\section{Other questions}


\subsection{Is Condor Y2K compliant?}

Yes.

\subsection{Is there a Condor mailing list?}

Yes.  We run an extremely low traffic mailing list that announces new versions of Condor.
To subscribe email
\Email{majordomo@cs.wisc.edu}
with the message body of
\begin{verbatim}
subscribe condor-world
\end{verbatim}

\subsection{My question isn't in the FAQ!}

If you have any questions that are not listed here in the FAQ, feel free to contact us at \Email{condor-admin@cs.wisc.edu}.
