\begin{ManPage}{\label{man-condor-bypass}\Condor{bypass}}{1}
{Generate code for bypassing standard system calls.}

\Synopsis \SynProg{\Condor{bypass}}
\oOpt{-local}
\oOpt{-remote}
\oOpt{-syscall}
\oOpt{-dynamic}
\oOpt{-static}
\oOpt{-plain}
\oOptArg{-dynamiclibrary}{library-name}
\oOptArg{-staticlibrary}{library-name}

\index{Condor commands!condor\_bypass}
\index{condor\_bypass command}

\Description

\Condor{bypass} is used to instrument a program with new code that replaces
operating system calls.

\Condor{bypass} is a standalone programming tool that is used to build
the Condor system.  This tool is {\em not} needed to use the Condor system,
but is made available for others to take advantage of some of the machinery
inside Condor.

\Condor{bypass} reads an interface description (ID) file on its standard input
and generates several C++ files that implement the ID.  Instructions for
writing ID files are not included here -- consult the Condor manual instead.

The files generated are:

\begin{itemize}
\item[\File{bypass\_client.C}] This code should be compiled and linked with your application.
\item[\File{bypass\_extract.o}] When using static library linkage, this code should be linked with your application.
\item[\File{bypass\_server.C}] When building a client/server pair, this code should be compiled to create the server.
\item[\File{bypass\_global.h}] This header file is used by both the client and server modules.
\end{itemize}

\begin{Options}
\OptItem{\Opt{-local}} { Create code only for a standalone client. (The default) }
\OptItem{\Opt{-remote}} { Create code for both a client and server. }
\OptItem{\Opt{-syscall}} { Set the default linkage to \Keyword{syscall}. (The default) }
\OptItem{\Opt{-dynamic}} { Set the default linkage to \Keyword{dynamic}. }
\OptItem{\Opt{-static}} { Set the default linkage to \Keyword{static}. }
\OptItem{\Opt{-plain}} { Set the default linkage to \Keyword{plain}. }
\OptItem{\OptArg{-dynamiclibrary}{library}} { Set the dynamic library name. (Default is \File{/usr/lib/libc.so}.) }
\OptItem{\OptArg{-staticlibrary}{library}} { Set the static library name.  (Default is \File{/usr/lib/libc.a}.) }
\end{Options}

\Examples

Read \File{info.bypass} and produce code for a standalone client with \Keyword{syscall} linkage:
\begin{verbatim}
condor_bypass -local < info.bypass
\end{verbatim}

Read \File{remote.bypass} and produce code for a client/server pair with \Keyword{dynamic} linkage on /File{/lib/libc.so.6}:
\begin{verbatim}
condor_bypass -remote -dynamic -dynamiclibrary /lib/libc.so.6 < remote.bypass
\end{verbatim}

\SeeAlso
Condor User Manual

\end{ManPage}

