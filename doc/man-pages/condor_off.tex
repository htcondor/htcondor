\begin{ManPage}{\label{man-condor-off}\Condor{off}}{1}
{Shutdown condor daemons}
\Synopsis \SynProg{\Condor{off}}
\oOpt{general-options}
\oOpt{targets}
\oOpt{subsystem}

\SynProg{\Condor{off}}
% because of graceful, need to do this by hand
\oOpt{-help $|$ -version $|$ -pool hostname $|$ -graceful $|$ -fast}
\oOpt{hostname $|$ "\SinfulAny" $|$ -name hostname $|$ -addr "\SinfulAny" ...}
\oOpt{-master $|$ -startd $|$ -schedd $|$ -collector $|$ -negotiator $|$ -kbdd}
\oOpt{hostname ...}

\index{Condor commands!condor\_off}
\index{condor\_off command}

\Description 

\Condor{off} shuts down all of the condor daemons running on a given
machine.  It does this cleanly without a loss of work done by any jobs
currently running on this machine, or jobs that are running on other machines
that have been submitted from this machine.  The only daemon that remains
running is the \Condor{master}, which can handle both local and remote
requests to restart the other condor daeomns if need be.  To restart
condor running on a machine, see the \Condor{on} command.

\begin{Options}
	\OptItem{\Opt{-graceful}}{Gracefully shutdown daemons (the default)}
	\OptItem{\Opt{-fast}}{Quickly shutdown daemons}
	\ToolArgsDesc
\end{Options}

\ExitStatus

\Condor{off} will exit with a status value of 0 (zero) upon success,
and it will exit with the value 1 (one) upon failure.

\end{ManPage}
