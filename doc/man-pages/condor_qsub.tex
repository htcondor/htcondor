\begin{ManPage}{\label{man-condor-qsub}\Condor{qsub}}{1}
{Queue jobs that use PBS/SGE-style submission}
\Synopsis \SynProg{\Condor{qsub}}
[\verb@--@\textbf{version}]

\SynProg{\Condor{qsub}}
\oOpt{Specific options}
\oOpt{Directory options}
\oOpt{Environmental options}
\oOpt{File options}
\oOpt{Notification options}
\oOpt{Resource options}
\oOpt{Status options}
\oOpt{Submission options}

\Description

The primary purpose of this emulation is to allow PBS/Torque-style and 
SGE-style submission of dependent jobs without the need to specify the full
dependency graph at submission time. This implementation is neither as efficient
as HTCondor's DAGMan, nor as functional as SGE's qsub/qalter. It merely serves 
as a minimal adapter to be able to use software original written to interact 
with PBS/Torque/SGE in an HTCondor pool. In Version 0.2, many additional 
features have been added.

In general \Condor{qsub} behaves just like qsub. However, only a fraction of the
original functionality is available (about 1/3). The following list of options only
describes the differences in the behavior of PBS/Torque/SGE's qsub and this 
emulation. Qsub options not listed here are not supported.

For a full listing of options, please see
\begin{description}
\item[POSIX]: \URL{http://pubs.opengroup.org/onlinepubs/9699919799/utilities/qsub.html}
\item[SGE]: \URL{http://gridscheduler.sourceforge.net/htmlman/htmlman1/qsub.html}
\item[PBS/Torque]: \URL{http://docs.adaptivecomputing.com/torque/4-1-3/Content/topics/commands/qsub.htm}
\end{description}

\Condor{qsub} is a shell script that submits jobs to HTCondor for execution, but 
uses the syntax of Torque/PBS or SGE/Open Grid Scheduler. \Condor{qsub} accepts
both command line options and arguments or a single file that contains all of the 
commands. This works in the reverse direction of \SubmitCmd{grid} universe's 
\SubmitCmd{batch} grid type (see section~\ref{sec:batch} on 
page~\pageref{sec:batch}), which takes HTCondor jobs submitted with HTCondor
syntax and submits them to PBS, SGE, or LSF.

Some concepts present in PBS/SGE do not apply to HTCondor and so these options are
not enabled.

\begin{Options}
\OptItem{\OptArg{-a}{date\_time}}{(Submission option)
  Specifies a deferred execution date and time (see 
  We follow the PBS/Torque implementation of this option. The \Arg{date\_time} 
  argument is a string in the form \Arg{[[[[CC]YY]MM]DD]hhmm[.SS]}.  
  \Arg{CC}, \Arg{YY}, \Arg{MM}, \Arg{DD}, and \Arg{SS} are optional. SGE does not
  permit \Arg{MM} or \Arg{DD} to be optional, although PBS does; 
  we follow the PBS style.}
\OptItem{\OptArg{-A}{account\_string}}{(Status option)
  Allows you to use group accounting (see section~\ref{sec:group-accounting} on 
  page~\pageref{sec:group-accounting}) and have the string \Arg{account\_string} 
  be the accounting group associated with this job. Unlike SGE, we do not give a
  default group of \"sge\".}
\OptItem{\OptArg{-b}{y|n}}{(Submission option)
 Tells \Condor{qsub} whether the command given to \Condor{qsub} should be treated
 as a binary or as a script. This option follows the SGE version 
}
\OptItem{\OptArg{-c}{checkpoint\_options}}{(Submission option)  
  Standard universe only. By default, HTCondor uses checkpointing and this option
  controls the how HTCondor checkpoints.  Currently, these options from POSIX,
  SGE, and PBS are the only ones that work.
  \begin{description}
  \item[n/N] Disable checkpointing.
  \item[s/S] Disable periodic checkpointing. A job will only checkpoint when it
    is evicted.
  \end{description}
  We hope to implement options more in the future.
}
\OptItem{\Opt{---condor-keep-files}}{(\Condor{qsub} Specific option)
  \Condor{qsub} specific. \Condor{qsub} creates some temporary files (generated
  HTCondor submit files, sentinel jobs) and using this option will cause these 
  files to remain around after \Condor{qsub} finishes. Mostly important for
  debugging.}
\OptItem{\Opt{-cwd}}{(Directory option)
  If this option is present, the initial directory in which the job will run 
  will be set to the current 
  directory from where the job was submitted from. It sets 
  \SubmitCmd{initialdir} for \Condor{submit} to the current directory.
}
\OptItem{\OptArg{-d}{path} or \OptArg{-wd}{path}}{(Directory option)
  Specifies the initial directory to be the to the value given by
  \Arg{path}. It sets \SubmitCmd{initialdir} for \Condor{submit} to \Arg{path}.
  See section~\ref{man-condor-submit-initialdir} on 
  page~\pageref{man-condor-submit-initialdir} for further explanation of 
  \SubmitCmd{initialdir}.}
\OptItem{\OptArg{-e}{filename}}{(File option)
  Specifies the file which should contain the standard error stream. See 
  section~\ref{man-condor-submit-error} on 
  page~\pageref{man-condor-submit-error}.}
\OptItem{\OptArg{---f}{qsub\_file}}{(\Condor{qsub} Specific option)
  \Condor{qsub} specific. \Condor{qsub} nominally accepts command line arguments. 
  However, submissions to \Shell{qsub} are often done in a batch
  file with commands like \Env{\#PBS} or \Env{\#SGE}. When this option is used,
  \Condor{qsub} will parse the batch file listed as \Arg{qsub\_file}. This 
  command may be replaced with \Opt{-@} in the future to emulate SGE's 
  implementation of dealing with these kinds of files.}
\OptItem{\Opt{-h}}{(Status option)
  Job submitted with this option gets placed into the hold state at submission
  time.}
\OptItem{\Opt{---help}}{(\Condor{qsub} Specific option)
  \Condor{qsub} will print out a help message. Please refer to this manual page
  as it is more complete and up to date than the help message. We hope to
  synchronize them soon.}
\OptItem{\OptArg{-hold\_jid}{<jid>}}{(Status option)
  If given, the job will be submitted in the hold state. Along with the actual 
  job, a `sentinel' job will be submitted to HTCondor's local universe. This 
  sentinel then watches the specified job and releases the submitted job whenever
  the job has completed. The sentinel observes HTCondor's queue and history
  to detect a job exiting with code 100 and not start the dependent job in this 
  case. If a cluster id of an array job is given, the dependent job will only be 
  released after all individual jobs of a cluster have completed.}
\OptItem{\OptArg{-i}{[[hostname]:]file,...}}{(File option)
  Specifies the file which should contain the standard input stream. See 
  section~\ref{man-condor-submit-input} on 
  page~\pageref{man-condor-submit-input}.
}
\OptItem{\OptArg{-j}{characters}}{(File option)
  Determines if the standard error stream and the standard output stream of the
  job will be merged into one stream. For HTCondor, this means merging them into
  one file that will eventually be placed back into the submitting directory.}
\OptItem{\OptArg{-l}{resource\_spec}}{(Resource option)
  Specifies various requirements for the job, e.g.~amount of RAM, number of CPUs.
  We currently support only PBS style resource requests, though we hope to have 
  SGE options soon.
  \Arg{resource\_spec} is a comma separated list of the form:
%\begin{verbatim}
\Shell{resource\_name[=[value]][,resource\_name[=[value]],\ldots]}
%\end{verbatim}
where \Arg{resource\_name} and \Arg{value} take the following values with the 
following meanings.
\begin{tabular*}{5in}{|c|p{1in}|p{3in}|} \hline
\Arg{resource\_name} & \Arg{value} & Description \\ \hline
arch  & string & Should use HTCondor's values listed in section~\ref{Arch-machine-attribute} \\ \hline
file  & size & Amount of disk space requested. \\ \hline
host  & string & Name of the host machine on which the job should run. For HTCondor,
this means that no other machines may run this job. \\ \hline
mem   & size & Amount of memory requested.\\ \hline
nodes & 
%\begin{verbatim}
%\ShortExpr{\{<node\_count> | <hostname>\} 
%[:ppn=<ppn>][:gpus=<gpu>]\\
%[:<property>[:<property>]\ldots] \\
%[+ \ldots]}
%\end{verbatim}
\Shell{\{<node\_count> | <hostname>\} [:ppn=<ppn>] [:gpus=<gpu>] [:<property> [:<property>] \ldots] [+ \ldots]}
%\{<node_count> | <hostname>\} [:ppn=<ppn>][:gpus=<gpu>][:<property>[:<property>]\ldots] [+ \ldots]}
& Number and/or properties of nodes to be used. We are working on having 
arbitrary properties be supported. For exmaples, please see
%\URL{http://docs.adaptivecomputing.com/torque/4-1-3/Content/topics/2-jobs/requestingRes.htm#qsub}
\parbox{2in}{ 
\URL{http://docs.adaptivecomputing.com/torque/4-1-3/Content/topics/2-jobs/requestingRes.htm\#qsub}
}
\\ \hline
opsys & string & Specify which OS should be used. See 
section~\ref{OpSys-machine-attribute} on page~\pageref{OpSys-machine-attribute} 
for permitted options. \\ \hline
procs & procs=<integer>  & Number of CPUs requested \\ \hline  
\end{tabular*}
For the value size, it should be in the format of an integer. PBS/Torque 
defaults to having the value represent the number of bytes; HTCondor defaults 
to using kilobytes. Both recognize the appending of kb, mb, gb, and tb. We do 
not currently support the use of words (w, kw, mw, gw, and tw).
}
\OptItem{\OptArg{-m}{a|e|n}}{(Notification option)
Set notification level. See section~\ref{man-condor-submit-notification} on 
page~\pageref{man-condor-submit-notification} for how \Condor{submit} handles 
notification. When an e-mail is sent to the owners of an HTCondor job depends on 
the settings. These settings are:
\begin{description}
\item[a] Mail is sent when the job terminates abnormally.
\item[e] Mail is sent when the job terminates.
\item[n] Mail is not sent.
\end{description}
There are two additional options that may become supported in the future:
\begin{description}
\item[b] UNSUPPORTED. Mail is sent when the job commences execution.
\item[s] UNSUPPORTED. Mail is sent when the job is suspended.
\end{description}
We hope to implement these options.
}
\OptItem{\OptArg{-M}{e-mail address}}{(Notification option)
Sets the e-mail address to use when HTCondor sends e-mail. See 
section~\ref{man-condor-submit-notify-user} on 
page~\pageref{man-condor-submit-notify-user}
for further details.}
\OptItem{\OptArg{-o}{filename}}{(File option)
  Specifies the file which should contain the standard error stream. See 
  section~\ref{man-condor-submit-output} on 
page~\pageref{man-condor-submit-output}.}
\OptItem{\OptArg{-p}{integer}}{(Status option)
Sets the priority, with 0 being the default. Jobs with higher numerical priority will
run before jobs with lower numerical priority.
See section~\ref{man-condor-submit-priority} on 
page~\pageref{man-condor-submit-priority}
for more information.}
\OptItem{\Opt{---print}}{(\Condor{qsub} Specific option) 
\Condor{qsub} specific. Prints to \File{stdout} the contents of the HTCondor submit
file that \Condor{qsub} generates.}
\OptItem{\OptArg{-r}{y|n}}{(Status option)
Specifies whether or not a job is re-runnable. Normally, HTCondor tries to rerun jobs,
so this tries as best as possible ensure that this does not occur. \Arg{y} is the 
default value.}
\OptItem{\OptArg{-S}{shell}}{(Submission option)
Specifies the path to the shell that should interpret the job.}
\OptItem{\OptArg{-t}{<start>[-<stop>[:<step>]]}}{(Submission option)
Specifies that this array jobs a number of essentially identical jobs should be
submitted. Currently, only the SGE style syntax is supported. \Arg{start}, 
\Arg{stop}, and \Arg{step} are all integers. \Arg{start} is the start index of the 
jobs, \Arg{stop} is the end index (inclusive) of the jobs, and \Arg{step} is the 
step size of the indices.
\Warn 
Currently, trying to use more than more than one processor or 
node in a job will not work with this option.}
\OptItem{\Opt{---test}}{(\Condor{qsub} Specific option)
\Condor{qsub} specific. It makes \Condor{qsub} not submit the job, but \Condor{qsub}
will still generate an HTCondor submit file.}
\OptItem{\OptArg{-v}{variable list}}{(Environmental option)
\Arg{variable list} is a comma separated string with the format of
\Env{variable[=value][,variable[=value],...]}. Since \Env{[=value]} can be optional
and HTCondor does not accept a variable without a value, \Condor{qsub} appends 
\Env{=True} to the variable listed. Note that this option does not handle white space
well as they must be contained by quotes. Also, all single and double quotes need to
be escaped by backslashes.
}
\OptItem{\Opt{-V}}{(Environmental option)
  If given, HTCondor will copy all of the user's current shell environment variables
for submission by having 'getenv = True' added to the HTCondor submit file. See 
section~\ref{man-condor-submit-getenv} on page~\pageref{man-condor-submit-getenv} 
for more details.}
\OptItem{\OptArg{-W}{attr\_name=attr\_value[,attr\_name=attr\_value\ldots]}}{
(File option)
PBS/Torque supports a number of attributes. However, \Condor{qsub} only supports
the names \Arg{stagein} and \Arg{stageout} for \Arg{attr\_name}.
The format of \Arg{attr\_value} for \Arg{stagein} and \Arg{stageout} is
\Env{local\_file@hostname:remote\_file[,\ldots]} and we strip it to 
\Env{remote\_file[,\ldots]}. HTCondor's file transfer mechanism is then used if needed.}
\OptItem{\Opt{---version}}{(\Condor{qsub} Specific option)
  Print version information of \Condor{qsub} and exit.}
\end{Options}

\end{ManPage}


















