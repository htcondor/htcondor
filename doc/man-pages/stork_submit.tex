\begin{ManPage}{\label{man-stork-submit}\Stork{submit}}{1}
{submit a Stork job}
\Synopsis \SynProg{\Stork{submit}}
\ToolArgsBase

\SynProg{\Stork{submit}}
\oOpt{-debug}
\oOpt{-stdin}
\Storkname
\oOptArg{-lognotes}{"prose"}
\Arg{submit-descpription-file}


\index{Condor commands!stork\_submit}
\index{Stork commands!stork\_submit}
\index{stork\_submit command}

\Description 

\Stork{submit} is used to submit a Stork data placement job.
Upon job submission, an integer identifier is assigned to the
submission, and printed to the standard output.  The returned job id is
required for managing the job via the \Stork{status} and \Stork{rm} tools.

The name of the Stork submit description file is the single,
required, command-line argument.  Stork places no constraints on the submit
description file name. See the Condor User Manual, \ref{sec:Stork}, for a
complete description of Stork submit description file syntax and keywords.

\begin{Options}
	\ToolArgsBaseDesc
	\OptItem{\Opt{-debug}}{Show extra debugging information.}
	\OptItem{\Opt{-stdin}}{Read commands from \File{stdin} instead of from a file.}
	\StorknameDesc
	\OptItem{\OptArg{-lognotes}{"prose"}}{The string given within the quote marks is appended to the \Arg{submit-description-file} before the job is submitted}
\end{Options}

\ExitStatus

\Stork{submit} will exit with a status value of 0 (zero) upon success,
and it will exit with the value 1 (one) upon failure.

\end{ManPage}
