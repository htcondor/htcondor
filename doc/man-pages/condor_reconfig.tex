\begin{ManPage}{\label{man-condor-reconfig}\Condor{reconfig}}{1}
{Reconfigure condor daemons}
\Synopsis \SynProg{\Condor{reconfig}}
\ToolArgsBase
\oOptnm{-full}
\ToolArgsLocate

\index{Condor commands!condor\_reconfig}
\index{condor\_reconfig command}

\Description 

\Condor{reconfig} reconfigures all of the condor daemons in accordance with 
the current
status of the condor configuration file(s).  
Once reconfiguration is complete, the daemons will behave according to
the policies stated in the configuration file(s).
The main exception is with the DAEMON\_LIST variable, which will only be
updated if the \Condor{restart} command is used.  
There are a few other configuration settings that can only be changed
if the Condor daemons are restarted.
Whenever this is the case, it will be mentioned in
section~\ref{sec:Configuring-Condor} on
page~\pageref{sec:Configuring-Condor} which lists all of the settings
used to configure Condor. 
In general, \Condor{reconfig} should be used when making changes to
the configuration files, since it is faster and more efficient than
restarting the daemons.

\begin{Options}
    \ToolArgsBaseDesc
    \OptItem{\Opt{-full}}{Perform a full reconfig.  In addition to
    re-reading the configuration files, a full reconfig will clear
    cached DNS information in the daemons.  It should only be used if
    you need this functionality.}
    \ToolArgsLocateDesc
\end{Options}

\ExitStatus

\Condor{reconfig} will exit with a status value of 0 (zero) upon success,
and it will exit with the value 1 (one) upon failure.

\end{ManPage}
