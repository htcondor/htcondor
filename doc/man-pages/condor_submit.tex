\begin{ManPage}{\Condor{submit}}{1}{queue jobs for execution on remote machines}
\label{man-condor-submit}
\Synopsis \SynProg{\Condor{submit}}
\oOpt{-v}
\oOptArg{-n}{schedd\_name}
\oOptArg{-r}{schedd\_name}
\Arg{cmdfile}

\Description
\Condor{submit} is the program for submitting jobs to Condor. \Condor{submit} reads a submit-description file which contains
commands that direct the queuing of jobs. A description file may contain specifications for the queuing of many condor jobs.
All jobs queued by a single invocation of \Condor{submit} must share the same executable, and are referred to as a "job
cluster". It is advantageous to submit multiple jobs as a single cluster because only one copy of the checkpoint file is
needed to represent all jobs in a cluster until they begin execution. Detailed instructions for creating description files follows
below after a brief introduction to Condor Universes. 

SUBMIT DESCRIPTION FILE 

Each condor job description file describes one cluster of jobs to be placed in the condor execution pool. All jobs in a cluster
must share the same executable, but they may have different input and output files, and different program arguments. The
submit-description file is then used as the only command-line argument to \Condor{submit}. The submit-description file must
contain one or more of the following commands: 

executable= <name> 

The name of the executable file for this job cluster. Only one executable command may be present in a description file. If
submitting into the Standard Universe, then the named executable must have been relinked with the Condor libraries (such
as via the \Condor{compile} command). If submitting into the Vanilla Universe, then the named executable need not be
relinked and can be any process which can run in the background (shell scripts work fine as well). 

requirements = <boolean\_expression> 

This is a list of requirements which a remote machine must meet to execute any jobs in the job cluster. The requirements
command is a boolean expression using C like operators. In order for your job to run on a given machine, this requirements
expression must evaluate to true on the given machine. , e.g. to require that whatever machine executes your program has
a least 64 megs of RAM and has a MIPS performance rating greater than 45, use: 

requirements = Memory >= 64 \&\& MIPS > 45. 

Only one requirements command may be present in a description file. Unless otherwise specified, \Condor{submit} by default
appends to the requirements expression a clause that states Arch and OpSys should equal the Arch and OpSys values of
the machine where the job was submitted. Furthermore, if Universe is set to Vanilla, then by default a clause that
FilesystemDomain equals the submit machine's FilsystemDomain is also appended. You can view the requirements of a job
which has already been submitted (along with everything else about the job) with the command "condor\_q -l"; see the
command reference page for \Condor{q} and \Condor{globalq}. 

The following system-macros may be used in the requirements boolean expression (note that the keywords in this list are
case sensitive!): 

     Arch - the architecture of the machine. See "Defined Arch/OpSys Values" section below for default Arch value
     definitions. 
     Clock\_Day - the day of the week, where 0 = Sunday, 1 = Monday, ..., 6 = Saturday. 
     Clock\_Min - the number of minutes passed since midnight. 
     ConsoleIdle - the number of seconds since a user has pressed a key or moved the mouse on the physical console
     of the machine (keypress activity on terminals, telnet/rlogin sessions are not considered for ConsoleIdle). Not
     available on all platforms. 
     Disk - the amount of free disk space in 1k blocks (i.e. 6000 = 6megabytes ) on the machine's root partition. 
     FilesystemDomain - a domain name configured by the Condor administrator which describes a cluster of machines
     which all access the same networked filesystems (usually via NFS, AFS, or DFS). 
     Flavor - the result of running the command "uname -m"; typically the model number/type of the machine 
     KFLOPS - the result of a linpack benchmark that rates the machine's floating-point performance speed in KFLOPS (
     a KFLOP = one thousand FLOPS ). 
     Machine - the unqualified name of the machine 
     Memory - the amount of RAM memory in megabytes 
     MIPS - the result of a dhrystone benchmark that rates the machine's integer computation performance speed in
     MIPS. 
     Nest - a string specified and defined by the Condor administrator, typically used to define the machine's physical
     location (such as "Rm1340-Engineering"). 
     OpSys - the operating system of the machine. See "Defined Arch/OpSys Values" section below for default OpSys
     value definitions. 
     Subnet - the IP subnet address of the machine 
     UidDomain - a domain name configured by the Condor administrator which describes a cluster of machines which
     all have the same "passwd" file entries, and therefore all have the same logins. 
     VirtualMemory - the amount of virtual memory (RAM + swap space) available on the machine 

preferences = <boolean expression> 

This is a list of characteristics which would be preferable in a remote machine executing jobs in this cluster. The condor
software will attempt to supply remote machines which meet the preferences, but if no such machines are available it will
then supply machines which meet only the requirements. Only one preferences command may be present in a description
file. The format for the boolean expression and the macros permissable in the expression are identical to the "requirements"
statement detailed above. 

image\_size = <size> 

This command tells Condor the maximum virtual image size to which you believe your program will grow during its execution.
Condor will then execute your job only on machines which have enough resources, (such as virtual memory), to support
executing your job. If you do not specify the image size of your job in the description file, Condor will automatically make a
(reasonably accurate) estimate about its size and adjust this estimate as your program runs. If the image size of your job is
underestimated, it may crash due to inability to acquire more address space, e.g. malloc(). If the image size is
overestimated, Condor may have difficulty finding machines which have the required resources. 

input = <pathname> 

output = <pathname> 

error = <pathname> 

Condor assumes that its jobs are long-running, and that the user will not wait at the terminal for their completion. Because
of this, the standard files which normally access the terminal, (stdin, stdout, and stderr), must refer to files. Thus, the filename
specified with "input" should contain any keyboard input the program requires (i.e. stdin). The "output" filename will capture
any information the program would normally write to the screen (i.e. stdout). The "error" filename will capture any error
messages the program would normally write to the screen (i.e. stderr). If any of these commands are not specified, the
default value of /dev/null is used. 

arguments = <argument\_list> 

List of arguments to be supplied to the program on the command line. 

environment = <parameter\_list> 

List of environment variables of the form "<parameter> = <value>". Multiple environment variables can be specified by
seperating them with a semicolon (" ; "). These environment variables will be placed into the job's environment before
execution. The length of all characters specified in the environment is currently limited to 255 characters. See sample
submit files below. 

Furthermore, there are two Condor-specific parameters supported: LOG and CONDOR\_CORESIZE. 

Use LOG to specify a filename where Condor will write a log file of what is happening with this job cluster. For example,
Condor will log into this file when and where the job begins running, when the job is checkpointed and/or migrated, when the
job completes, etc. Most users find specifying a LOG file to be very handy; its use is recommended. 

Should the user's program abort and product a core file, CONDOR\_CORESIZE specifies the maximum size of the core file
which the user wishes to keep. If CONDOR\_CORESIZE is not specified in the command file, the user's shell environment is
checked to see if the variable is set. If CONDOR\_CORESIZE is not set by either the command file or a shell environment
variable, the user's resource limit "coredumpsize" is used (except on HP-UX). 

priority = <prio> 

Condor job priorities range from -20 to +20, with 0 being the default. Jobs with higher numerical priority will run before jobs
with lower numerical priority. Note that this priority is on a per user basis; setting the priority will determine the order in which
your own jobs are executed, but will have no effect on whether or not your jobs will run ahead of another user's jobs. 

notification = <when> 

Owners of condor jobs are notified by mail when certain events occur. If <when> is set to "ALWAYS", the owner will be
notified whenever the job is checkpointed, and when it completes. If <when> is set to "COMPLETE" (the default), the owner
will be notified when the job terminates. If <when> is set to "ERROR", the owner will only be notified if the job terminates
abnormally. Finally, if <when> is set to "NEVER", the owner will not be mailed, regardless what happens to the job. 

initialdir = <directory-path> 

Initialdir should specify a path to a directory. This path will be prepended to all relative file pathnames. In other words,
initialdir specifies the current-working directory for the Condor job. If not specified, \Condor{submit} will automatically insert
the user's current-working directory at the time \Condor{submit} was run as the value for initialdir. 

universe = < standard | vanilla > 

Specify which Condor Universe to use when running this job, either "standard" or "vanilla". There can only be one Universe
statement per submit-description file. Any job specified to run in the "standard" universe must be relinked with the Condor
libraries; see above for a description of Standard Universe -vs- Vanilla Universe. If the Universe is not specified, the
default is "standard". [ Note: shell scripts should be submitted as Vanilla ]. 

queue 

Place one copy of the job in the condor queue. If desired, new input, output, error, initialdir, and arguments commands may
be issued between queue commands. 

[ Macros ] 

Parameterless macros in the form of \$(macro\_name) may be inserted anywhere in condor description files. Macros can be
defined by lines in the form of <macro\_name> = <string>. Two pre-defined macros are supplied by the description file
parser. The \$(Cluster) macro supplies the number of the job cluster, and the \$(Process) macro supplies the number of the
job. These macros are intended to aid in the specification of input and output files for clusters with lots of jobs, and may be
used to supply a condor process with its own cluster and process numbers on the command line. 

[ Comments ] 

Blank lines and lines beginning with a '\#' are ignored by the submit-description file parser. 

Arch/OpSys DEFINED DEFAULT VALUES 

Below are default Arch and OpSys values that could be specified in the Requirements command [ note that these are just
default values; the Condor administrator could have customized these ]. These Arch/OpSys combinations define a unique
platform described by the "Description" column. Note that a Condor Standard Universe job cannot checkpoint and then
subsequently restart on a different platform. 


Arch     OpSys        Description                                      

ALPHA    OSF1 (same   Digital ALPHA based workstation running OSF1     
         as Digital   (Digital Unix) version 3.2 or above              
         Unix)                                                         

HPPA1    HPUX9        Hewlett Packard PA-RISC 1.x CPU (i.e. PA-RISC    
                      7000 series CPU) based-workstation running HPUX  
                      9.x                                              

HPPA1    HPUX10       Hewlett Packard PA-RISC 1.x CPU (i.e. PA-RISC    
                      7000 series CPU) based-workstation running HPUX  
                      10.20 or above.                                  

HPPA2    HPUX10       Hewlett Packard PA-RISC 2.x CPU (i.e. PA-RISC    
                      8000 series CPU) based-workstation running HPUX  
                      10.20 or above.                                  

INTEL    SOLARIS251   Intel x86 running Solaris 2.5.1 (SunOS 5.5.1)    
                      or above                                         

INTEL    LINUX        Intel x86 running Linux 2 (Kernel 2.0.x,         
                      libc.so.5.3.12 or libc.so.5.4.7 or               
                      libc.so.5.4.20 or newer, and an ELF capable      
                      system).  See URL                                
                      http://www.cs.wisc.edu/condor/linux for          
                      up-to-the minute Linux notes.                    

SGI      IRIX53       Silicon Graphics workstation running IRIX 5.3.   

SGI      IRIX62       Silicon Graphics workstation running IRIX 6.2    
                      or above.                                        

SUN4c    SUNOS4       Sun SPARC sun4c (i.e. a Sparc 1, Sparc 2, IPC,   
                      IPX, etc) based workstation running SunOS 4.1.x  

SUN4m    SUNOS4       Sun SPARC sun4m (i.e. a Sparc 10, Sparc 20,      
                      etc) based workstations running SunOS 4.1.x      

SUN4u    SOLARIS251   Sun ULTRASPARC based workstation running         
                      Solaris 2.5.1 (SunOS 5.5.1) or above             

SUN4x    SOLARIS251   Sun SPARC sun4c -or- sun4m based workstation     
                      running Solaris 2.5.1 (SunOS 5.5.1) or above     

SUN4x    SOLARIS24    Sun SPARC sun4c -or- sun4m based workstation     
                      running Solaris 2.4 (SunOS 5.4).





BUGS 

For security reasons, Condor will refuse to run any jobs submitted by user root (UID = 0)or by a user whose default group is
group wheel (GID = 0). Jobs submitted by user root or a user with a default group of wheel will appear to sit forever in the
queue in an unexpanded state. 

All pathnames and argument lists specified in the submit-description file must be less than 256 characters in length;
otherwise, \Condor{submit} gives a warning message but the jobs will not execute properly. 

On some platforms, some FORTRAN jobs may produce a few error messages into stderr at startup time, but should then run
normally if there are no errors. If errors occur, and the program is compiled to produce core dumps, the core will be returned
to the initiating machine. 

Somewhat understandably, behavior gets bizzare if the user makes the silly mistake of requesting multiple Condor jobs to
write to the same file, and/or if the user alters any files that need to be accessed by a Condor job which is still in the queue
(i.e. compressing of data or output files before a Condor job has compelted is a common mistake). 

\begin{Options}
    \OptItem{\Opt{-v}}{Verbose output - display the created job class-ad}
    \OptItem{\OptArg{-n}{schedd\_name}}{Submit to the specified schedd. This option is used when there is more than one schedd running on the submitting machine}
    \OptItem{\OptArg{-r}{schedd\_name}}{Submit to a remote schedd. The jobs will be submitted to the schedd on the specified remote host, and their owner will be set to "nobody".}
\end{Options}

\end{ManPage}
