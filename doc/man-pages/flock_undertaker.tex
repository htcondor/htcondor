\begin{ManPage}{\label{man-flock-undertaker}\Prog{flock\_undertaker}}{1}
{determines whether a process has exited.}

\Synopsis \SynProg{\Prog{flock\_undertaker}}
\oOptArg{--file}{file}
\oOpt{--block}

\index{Deployment commands!flock\_undertaker}
\index{flock\_undertaker}

\Description
\Prog{flock\_undertaker} can examine an artifact file created by
\Prog{flock\_midwife} and determine whether the program started by
the \Prog{midwife} has exited.  It does this by attempting to aqcuire
a file lock.  Warning, this will not work on NFS unless the separate
file lock server is running.

\begin{Options}
  \OptItem{\Opt{--block}}{
    If the process has not exited, block until it does.
  }
  \OptItem{\OptArg{--file}{file}}{
    The name of the \Prog{flock\_midwife} created artifact file.
    Defaults to \File{lock.file}.
  }
\end{Options}

\ExitStatus 
\Prog{flock\_undertaker} will exit with a status of 0 (zero) if the
monitored process has exited, with a status of 1 (one) if the
monitored process has definitely not exited, with a status of 2 if it
is uncertain whether the process has exited (this is generally due to
a failure by the \Prog{flock\_midwife}), or with any other value
for program failure.

\SeeAlso
\Prog{uniq\_pid\_undertaker} (on page~\pageref{man-uniq-pid-undertaker}),
\Prog{flock\_midwife} (on page~\pageref{man-flock-midwife}).

\end{ManPage}
