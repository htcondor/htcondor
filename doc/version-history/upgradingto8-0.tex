%%%%%%%%%%%%%%%%%%%%%%%%%%%%%%%%%%%%%%%%%%%%%%%%%%%%%%%%%%%%%%%%%%%%%%
\section{\label{sec:to-8.0}Upgrading from the 7.8 series to the 8.0 series of HTCondor}
%%%%%%%%%%%%%%%%%%%%%%%%%%%%%%%%%%%%%%%%%%%%%%%%%%%%%%%%%%%%%%%%%%%%%%

\index{upgrading!items to be aware of}
While upgrading from the 7.8 series of HTCondor to the 8.0 series 
will bring many
new features and improvements introduced in the 7.9 series of HTCondor,
it will
also introduce changes that administrators of sites running from an older
HTCondor version should be aware of when planning an upgrade.  
Here is a list of items that administrators should be aware of.

\begin{itemize}

\item \Condor{dagman} jobs that are on hold when an upgrade occurs will have
problems running when they are released after the upgrade. This is because of
the new feature introduced in HTCondor version 7.9.0, where \Condor{dagman}
uses a single log file to monitor events.  The suggested workaround is for the
DAG file to set the config option \Macro{DAGMAN\_ALWAYS\_USE\_NODE\_LOG} to
\Macro{False} in a DAG config file, before running \Condor{release} on the held
DAG job.  \Condor{dagman} jobs that have a chance to write out a rescue
DAG---those that are \Condor{rm}'d or that have failed---are expected to
successfully restart after the upgrade. Administrators can be proactive by
checking that no \Condor{dagman} jobs are on hold when the upgrade occurs, and
encourage users to use the halt file feature and release their \Condor{dagman}
jobs before the upgrade.  Running \Condor{rm} on a held \Condor{dagman} job is
not advised in this case, unless the user is abandoning the DAG job.

\end{itemize}

