%%%%%%%%%%%%%%%%%%%%%%%%%%%%%%%%%%%%%%%%%%%%%%%%%%%%%%%%%%%%%%%%%%%%%%
\section{\label{sec:History-Intro}Introduction to HTCondor Versions}
%%%%%%%%%%%%%%%%%%%%%%%%%%%%%%%%%%%%%%%%%%%%%%%%%%%%%%%%%%%%%%%%%%%%%%

This chapter provides descriptions of what features have been added or
bugs fixed for each version of HTCondor.
The first section describes the HTCondor version numbering scheme, what
the numbers mean, and what the different \Term{release series} are.
The rest of the sections each describe a specific release series, and
all the HTCondor versions found in that series.

%%%%%%%%%%%%%%%%%%%%%%%%%%%%%%%%%%%%%%%%%%%%%%%%%%%%%%%%%%%%%%%%%%%%%%
\subsection{\label{sec:Version-Number-Scheme}
HTCondor Version Number Scheme}
%%%%%%%%%%%%%%%%%%%%%%%%%%%%%%%%%%%%%%%%%%%%%%%%%%%%%%%%%%%%%%%%%%%%%%

Starting with version 6.0.1, HTCondor adopted a new, hopefully easy to
understand version numbering scheme.
It reflects the fact that HTCondor is both a production system and a
research project.
The numbering scheme was primarily taken from the Linux kernel's
version numbering, so if you are familiar with that, it should seem
quite natural.

There will usually be two HTCondor versions available at any given time,
the \Term{stable} version, and the \Term{development} version.
Gone are the days of ``patch level 3'', ``beta2'', or any other random
words in the version string.
All versions of HTCondor now have exactly three numbers, separated by
``.''   

\begin{itemize}

\item The first number represents the major version number, and will
change very infrequently.

\item \emph{The thing that determines whether a version of HTCondor is
\Term{stable} or \Term{development} is the second digit.
Even numbers represent stable versions, while odd numbers represent
development versions.}

\item The final digit represents the minor version number, which
defines a particular version in a given release series.

\end{itemize}


%%%%%%%%%%%%%%%%%%%%%%%%%%%%%%%%%%%%%%%%%%%%%%%%%%%%%%%%%%%%%%%%%%%%%%
\subsection{\label{sec:Stable-Series}The Stable Release Series}
%%%%%%%%%%%%%%%%%%%%%%%%%%%%%%%%%%%%%%%%%%%%%%%%%%%%%%%%%%%%%%%%%%%%%%

People expecting the stable, production HTCondor system should download
the stable version, denoted with an even number in the second digit of
the version string.
Most people are encouraged to use this version.  
We will only offer our paid support for versions of HTCondor from the
stable release series.

\emph{On the stable series, new minor version releases will only
be made for bug fixes and to support new platforms.}
No new features will be added to the stable series.
People are encouraged to install new stable versions of HTCondor when
they appear, since they probably fix bugs you care about.
Hopefully, there will not be many minor version releases for any given
stable series.


%%%%%%%%%%%%%%%%%%%%%%%%%%%%%%%%%%%%%%%%%%%%%%%%%%%%%%%%%%%%%%%%%%%%%%
\subsection{\label{sec:Developement-Series}
The Development Release Series}
%%%%%%%%%%%%%%%%%%%%%%%%%%%%%%%%%%%%%%%%%%%%%%%%%%%%%%%%%%%%%%%%%%%%%%

Only people who are interested in the latest research, new features
that haven't been fully tested, etc, should download the development
version, denoted with an odd number in the second digit of the version
string.  
We will make a best effort to ensure that the development series will
work, but we make no guarantees.

On the development series, new minor version releases will probably
happen frequently.
People should not feel compelled to install new minor versions unless
they know they want features or bug fixes from the newer development
version.

\emph{Most sites will probably never want to install a development
version of HTCondor for any reason.}
Only if you know what you are doing (and like pain), or were
explicitly instructed to do so by someone on the HTCondor Team, should
you install a development version at your site.

After the feature set of the development series is satisfactory to the
HTCondor Team, we will put a code freeze in place, and from that point
forward, only bug fixes will be made to that development series.
When we have fully tested this version, we will release a new stable
series, resetting the minor version number, and start work on a new
development release from there.

%%%%%%%%%%%%%%%%%%%%%%%%%%%%%%%%%%%%%%%%%%%%%%%%%%%%%%%%%%%%%%%%%%%%%%
% The rest of this file just inputs other files which contain sections
% describing each release series in detail.
%%%%%%%%%%%%%%%%%%%%%%%%%%%%%%%%%%%%%%%%%%%%%%%%%%%%%%%%%%%%%%%%%%%%%%

% upgrade instructions are in the Pool Management section
%%%%%%%%%%%%%%%%%%%%%%%%%%%%%%%%%%%%%%%%%%%%%%%%%%%%%%%%%%%%%%%%%%%%%%
\section{\label{sec:gotchas}Upgrading from the 7.6 series to the 7.8 series of HTCondor}
%%%%%%%%%%%%%%%%%%%%%%%%%%%%%%%%%%%%%%%%%%%%%%%%%%%%%%%%%%%%%%%%%%%%%%

\index{upgrading!items to be aware of}
While upgrading from the 7.6 series of HTCondor to the 7.8 series will bring many
new features and improvements introduced in the 7.7 series of HTCondor, it will
also introduce changes that administrators of sites running from an older
HTCondor version should be aware of when planning an upgrade.  Here is a list of
items that administrators should be aware of.

\begin{itemize}

\item In the grid universe, the Amazon grid-type is gone and has been replaced
	with the EC2 grid-type.  Also, support for grid-type gt4 (Web Services
	GRAM) has been removed.

\item Default job submit options related to file transfers have changed. 
Across all platforms, defaults are now
\begin{verbatim}
  should_transfer_files = IF_NEEDED
  when_to_transfer_output = ON_EXIT
\end{verbatim}
See section~\ref{sec:file-transfer-if-when} for details.

\item  On Linux and Mac OS, common utility code is now contained in a set of
shared libraries. In the Linux native packages, most of these libraries
are placed under \File{/usr/lib[64]/condor} and the RUNPATH attribute is set in
the binaries to search there for them.
In the tarball packages, these libraries are placed under \File{lib} and
\File{lib/condor}, and the RUNPATH attribute is set in the binaries to search
for them under the relative paths \File{../lib} and \File{../lib/condor}.
This means that if you move or copy an HTCondor binary from a tarball
package to a different location, you must do one of the following:
\begin{itemize}
	\item Move or copy the corresponding \File{lib/} directory with it, or
  \item Make a symlink in the new location pointing back to the original \File{lib/}
  directory, or
  \item Set environment variable \Env{LD\_LIBRARY\_PATH} to point to the original \File{lib/} and \File{lib/condor/}
  directories
\end{itemize}
One of the new shared libraries, \File{libcondor\_utils\_7\_8\_0}, has no \File{.so}
versioning. Instead, the HTCondor version is included in the library name.
This means that an HTCondor binary must always be matched with the
\File{libcondor\_utils} library from the same HTCondor release.


\item  The \Condor{hdfs} service is no longer included within the HTCondor
	release.  Instead, the HTCondor + HDFS integration previously bundled with
	version 7.6 is available in version 7.8 as a \Term{Contribution Module}.
	Contribution Modules are optional packages that add functionality to
	HTCondor, but are provided and maintained outside of the core code base.  See
	the HTCondor Wiki at
	\URL{https://condor-wiki.cs.wisc.edu/index.cgi/wiki?p=ContribModules}.

\item Previous to version 7.8, by default the \Condor{master} would restart any
	individual daemon under its control if it notices that the file
	modification time of the binary for that daemon has changed.  Now the
	\Condor{master} will only monitor the file modification time of the
	\Condor{master} binary itself.  See section~\ref{sec:Pool-Upgrade}.  Also,
	see \MacroNI{MASTER\_NEW\_BINARY\_RESTART} on
	page~\pageref{param:MasterNewBinaryRestart}.

\item In DAGMan, if you have a PRE and a POST script on a node, the default now
	is that the POST script is run even if the PRE script failed.   This change
	could impact unaware workflows such that POST scripts might erroneously
	report the node as succeeded. You can get the old behavior by setting
	\MacroNI{DAGMAN\_ALWAYS\_RUN\_POST} to False.  In addition, you can no
	longer directly submit a rescue DAG file with \Condor{submit\_dag} unless
	\MacroNI{DAGMAN\_WRITE\_PARTIAL\_RESCUE} is set to False (not normally
	recommended).  See section~\ref{sec:DAGMan}.

\item The \MacroNI{KILL} expression cannot be used to grant more time to a job
	than offered by \Macro{MachineMaxVacateTime}. In HTCondor v7.8 and above, it
	is anticipated that most sites will simply use a default value of False for
	\MacroNI{KILL} and set \MacroNI{MachineMaxVacateTime} to control how long
	to wait.  See page~\pageref{param:MachineMaxVacateTime} for more
	information.


\end{itemize}


%%%      PLEASE RUN A SPELL CHECKER BEFORE COMMITTING YOUR CHANGES!
%%%      PLEASE RUN A SPELL CHECKER BEFORE COMMITTING YOUR CHANGES!
%%%      PLEASE RUN A SPELL CHECKER BEFORE COMMITTING YOUR CHANGES!
%%%      PLEASE RUN A SPELL CHECKER BEFORE COMMITTING YOUR CHANGES!
%%%      PLEASE RUN A SPELL CHECKER BEFORE COMMITTING YOUR CHANGES!

%%%%%%%%%%%%%%%%%%%%%%%%%%%%%%%%%%%%%%%%%%%%%%%%%%%%%%%%%%%%%%%%%%%%%%
\section{\label{sec:History-7-9}Development Release Series 7.9}
%%%%%%%%%%%%%%%%%%%%%%%%%%%%%%%%%%%%%%%%%%%%%%%%%%%%%%%%%%%%%%%%%%%%%%

This is the development release series of HTCondor.
The details of each version are described below.

%%%%%%%%%%%%%%%%%%%%%%%%%%%%%%%%%%%%%%%%%%%%%%%%%%%%%%%%%%%%%%%%%%%%%%
\subsection*{\label{sec:New-7-9-3}Version 7.9.3}
%%%%%%%%%%%%%%%%%%%%%%%%%%%%%%%%%%%%%%%%%%%%%%%%%%%%%%%%%%%%%%%%%%%%%%

\noindent Release Notes:

\begin{itemize}

\item HTCondor version 7.9.3 not yet released.
%\item HTCondor version 7.9.3 released on Month Date, 2012.

\end{itemize}


\noindent New Features:

\begin{itemize}

\item When NEGOTIATOR\_CONSIDER\_PREEMPTION is false, the \Condor{negotiator}
now fetches machine ads more quickly from the collector by skipping most
attributes of the busy machines.  This can make negotiation much faster in
a very large pool of mostly claimed machines.
\Ticket{3366}

\item Exposed the quantization level for windowed statistics to
configuration with the parameter \Macro{STATISTICS\_WINDOW\_QUANTUM}.
\Ticket{3288}

\item \emph{This item is tentative, and must be reviewed or updated
before release of 7.9.3.}
\Condor{submit} now supports first-class syntax for specifying
accounting groups, via the 'group' and 'group\_user' commands.  The
separator character between group name and user name on internal
representations is now configurable with \Macro{GROUP\_SEPARATOR}.
Refer to section~\ref{sec:group-accounting} for more information.
\Ticket{2728}

\item Round-robin scheduling is now used when there are multiple users
waiting to transfer files in the limits set by
\Macro{MAX\_CONCURRENT\_UPLOADS} and/or
\Macro{MAX\_CONCURRENT\_DOWNLOADS}.  Previously, the file transfer
queue was scheduled in first-in-first-out order, so one user with
many files to transfer could delay other users for as long as it took
to transfer those files.  Now, when choosing a new job to allow to
transfer, the first job belonging to the user who has least
recently been given an opportunity to transfer will be selected.
The old behavior, or variations on the new behavior, can be achieved
by configuring \Macro{TRANSFER\_QUEUE\_USER\_EXPR}.
\Ticket{3333}

\item \Condor{dagman} will now try twice to write a POST script terminate
event, rather than trying once and exiting. \Condor{dagman} exits, writing
a rescue DAG, if it is unable to write the event.
\Ticket{965}

\item The \Condor{gridmanager} now cleans up temporary files and directories
that are sometimes left by the batch\_gahp when executing a grid-type
\SubmitCmd{batch} job.
\Ticket{3276}

\item Added counts of nodes in various states to the \Condor{dagman}
node status file.  Refer to section~\ref{sec:DAG-node-status} for
more information.
\Ticket{2728}

\end{itemize}

\noindent Configuration Variable and ClassAd Attribute Additions and Changes:

\begin{itemize}

\item None.

\end{itemize}

\noindent Bugs Fixed:

\begin{itemize}

\item A change was made to more accurately implement the
minimum time defined by the configuration variable
\Macro{NEGOTIATOR\_CYCLE\_DELAY}. 
\Ticket{3332}

\item The \Prog{batch\_gahp} is no longer dependent on the Perl module 
\Code{XML::Simple} when submitting jobs to SGE.
\Ticket{3350}

\item The batch\_gahp now properly handles job X.509 proxies that aren't
in the old proxy format.
\Ticket{3362}

\item On 32-bit platforms a \Macro{STARTER\_RLIMIT\_AS} value larger than 4096 could
cause jobs to abort on startup.  Values larger than 2047 have no real meaning on 32-bit
platforms, so values larger than 2047 are treated as no limit on 32-bit platforms.
\Ticket{3309}

\end{itemize}

\noindent Known Bugs:

\begin{itemize}

\item None.

\end{itemize}

\noindent Additions and Changes to the Manual:

\begin{itemize}

\item None.

\end{itemize}


%%%%%%%%%%%%%%%%%%%%%%%%%%%%%%%%%%%%%%%%%%%%%%%%%%%%%%%%%%%%%%%%%%%%%%
\subsection*{\label{sec:New-7-9-2}Version 7.9.2}
%%%%%%%%%%%%%%%%%%%%%%%%%%%%%%%%%%%%%%%%%%%%%%%%%%%%%%%%%%%%%%%%%%%%%%

\noindent Release Notes:

\begin{itemize}

%\item HTCondor version 7.9.2 not yet released.
\item HTCondor version 7.9.2 released on December 11, 2012.
This release contains all of the bug fixes in the version 7.8.6 
stable release,
and most of the bug fixes in the
soon to be released version 7.8.7 stable release.

\end{itemize}


\noindent New Features:

\begin{itemize}

\item The permissions for the temporary execute directory of a job
have been tightened for vanilla universe jobs, 
such that only the owner of the job is allowed to see or
modify the contents.
\Ticket{3315}

\item Added experimental support for EC2 spot instances.
\Ticket{3209}

\item (This feature was added in version 7.9.1.)  
There are two new protocols for the submission of grid type EC2 jobs,
\Expr{euca3://} and \Expr{euca3s://}.
These protocols exist to work correctly when the resources do not support 
the \Param{InstanceInitiatedShutdownBehavior} parameter.
\Ticket{2974}

\item (This feature was added in version 7.9.1.)  
Added both a \Opt{-suppress\_notification},
a \Opt{-dont\_suppress\_notification} command line option,
and corresponding
\Macro{DAGMAN\_SUPPRESS\_NOTIFICATION} configuration variable
to \Condor{dagman} and \Condor{submit\_dag}.
This enables a user of DAGMan to stop email notification of job
events for jobs submitted by \Condor{dagman}. The value of
\MacroNI{DAGMAN\_SUPPRESS\_NOTIFICATION} defaults to \Expr{True},
so that jobs submitted
by \Condor{dagman} will not send email notification. 
\Ticket{3352}

\item The default for job notification email has changed
from \Expr{Complete} to \Expr{Never}. 
There is also a new configuration variable, \Macro{JOB\_DEFAULT\_NOTIFICATION},
which permits administrators to change the default for all jobs.
\Ticket{2155}

\item For platforms supporting cgroups,
resource limits can now be applied per job,
where a job may consist of multiple processes.
See section~\ref{sec:Resource-Limits-Cgroup} for details.
\Ticket{2734}

\end{itemize}

\noindent Configuration Variable and ClassAd Attribute Additions and Changes:

\begin{itemize}

\item The new configuration variable \Macro{MEMORY\_LIMIT}
supports implementing memory resource limits on a per-job basis under cgroups.
\Ticket{2734}

\end{itemize}

\noindent Bugs Fixed:

\begin{itemize}

\item \Condor{schedd} and \Condor{shadow} were not respecting the
\Macro{DAGManNodesMask} attribute. This caused extra events to be written to
the DAGMan node log.
\Ticket{3311}

\item Removed a spurious newline from the output of \Condor{submit}.
\Ticket{3316}

\item Fixed a bug that caused the \Condor{shadow} to set job attribute
\Attr{X509UserProxySubject} to the wrong value when the job's X.509
proxy file was updated. It incorrectly set the value to be 
the proxy's subject name, rather than to the correct value, which is
its identity.
\Ticket{3265}

\item The \Prog{batch\_gahp} no longer modifies the environment variable
\Env{LD\_LIBRARY\_PATH}.
In some instances, modifying \Env{LD\_LIBRARY\_PATH} caused the
batch system's command line tools to fail when run by the \Prog{batch\_gahp}.
\Ticket{3317}

\item Grid-type \SubmitCmd{batch} jobs now work properly on machines
where the gLite software has been installed.
\Ticket{3269}

\item The \Condor{shadow} would never print the allocated amount of
partitionable resources in the job log.
\Ticket{3318}

\item \Condor{who} would sometimes incorrectly display blank or partial
values in the PROGRAM column.
\Ticket{3314}

\end{itemize}

\noindent Known Bugs:

\begin{itemize}

\item None.

\end{itemize}

\noindent Additions and Changes to the Manual:

\begin{itemize}

\item None.

\end{itemize}


%%%%%%%%%%%%%%%%%%%%%%%%%%%%%%%%%%%%%%%%%%%%%%%%%%%%%%%%%%%%%%%%%%%%%%
\subsection*{\label{sec:New-7-9-1}Version 7.9.1}
%%%%%%%%%%%%%%%%%%%%%%%%%%%%%%%%%%%%%%%%%%%%%%%%%%%%%%%%%%%%%%%%%%%%%%

\noindent Release Notes:

\begin{itemize}

\item Condor version 7.9.1 released on October 22, 2012.

\item Condor no longer looks for its main configuration file in the
location \File{\MacroUNI{GLOBUS\_LOCATION}/etc/condor\_config}.
\Ticket{2830}

\item \Security This version contains an important security bug fix.  See below
for details of this and other bugs fixed.

\end{itemize}


\noindent New Features:

\begin{itemize}

\item There are two new protocols for the submission of grid type EC2 jobs,
\Expr{euca3://} and \Expr{euca3s://}.
These protocols exist to work correctly when the resources do not support 
the \Param{InstanceInitiatedShutdownBehavior} parameter.
\Ticket{2974}

\item \Condor{job\_router} can now submit the routed copy of jobs to a
different \Condor{schedd} than the one that serves as the source of
jobs to be routed.  The spool directories of the two
\Condor{schedds} must still be directly accessible to
\Condor{job\_router}.  This feature is enabled by using the new
optional configuration settings:

\begin{itemize}
\item \Macro{JOB\_ROUTER\_SCHEDD1\_SPOOL}
See definition at section~\ref{param:JobRouterSchedd1Spool}.
\item \Macro{JOB\_ROUTER\_SCHEDD2\_SPOOL}
See definition at section~\ref{param:JobRouterSchedd2Spool}.
\item \Macro{JOB\_ROUTER\_SCHEDD1\_NAME}
See definition at section~\ref{param:JobRouterSchedd1Name}.
\item \Macro{JOB\_ROUTER\_SCHEDD2\_NAME}
See definition at section~\ref{param:JobRouterSchedd2Name}.
\item \Macro{JOB\_ROUTER\_SCHEDD1\_POOL}
See definition at section~\ref{param:JobRouterSchedd1Pool}.
\item \Macro{JOB\_ROUTER\_SCHEDD2\_POOL}
See definition at section~\ref{param:JobRouterSchedd2Pool}.
\end{itemize}
\Ticket{3030}

\item The \Condor{job\_router} can now optionally transform jobs in place,
rather than creating a second transformed version (copy) of the job.
\Ticket{3185}

\item The \Condor{defrag} daemon now has a policy option implemented
by configuration to cancel the draining
of a machine that is in the Draining mode.  This can be used to effect
partial draining of machines.
\Ticket{2993}

\item Communication between the \Condor{c-gahp} and the \Condor{schedd} has
been improved. A large number of Condor-C jobs should no longer cause
other clients of the remote \Condor{schedd} to time out trying to get the
\Condor{schedd} daemon's attention.
\Ticket{2575}

\item \Condor{history} and \Condor{q} can now be told to read job records
from a user log, instead of parsing the history file or querying the
\Condor{schedd}.  This can be used to monitor the status of jobs with
reduced load on the \Condor{schedd}.
\Ticket{3188}

\item Eucalyptus 3.x support has been added to the EC2 GAHP.
\Ticket{2974}

\item File transfer remaps now support remapping directories.
\Ticket{3039}

\item The \Condor{schedd} can now dynamically spawn a local \Condor{startd}
to manage local universe jobs.
\Ticket{3129}

\item \Condor{q} \Opt{-jobads} will now respect the \Opt{-constraint} option.
\Ticket{3191}

\item Added BOSCO, a set of tools that makes it easy to use a Personal
Condor to run jobs on remote batch systems without administrator
assistance or manual installation of software on the remote systems.
See \URL{https://twiki.grid.iu.edu/bin/view/CampusGrids/BoSCO} for more
information about BOSCO.
\Ticket{2421}

\end{itemize}


\noindent Configuration Variable and ClassAd Attribute Additions and Changes:

\begin{itemize}

\item Dynamic slots now fill the values for attributes of with names
that begin with
\Attr{TotalSlot}, 
for configured local resources in a way consistent with standard resources
such as \Attr{TotalSlotCpus}.
Previously those values were all given the value zero on dynamic slots.
\Ticket{3229}

\item The \Condor{schedd} now advertises the value of configuration variable
\MacroNI{COLLECTOR\_HOST} as attribute \Attr{CollectorHost} in 
its daemon ClassAd.  This allows one to determine if a given
\Condor{schedd} reporting to a \Condor{collector} is flocking to that 
\Condor{collector} or not.
\Ticket{3202}

\item Added the attribute \Attr{DAGManNodesMask} to control the verboseness of
the log referred to by \Attr{DAGManNodesLog}.
\Ticket{3351}

\item The new configuration variable
\Macro{QUEUE\_SUPER\_USER\_MAY\_IMPERSONATE} specifies a regular
expression that matches the user names that
the queue super user may impersonate when managing jobs.  When not
set, the default behavior is to allow impersonation of any user who
has had a job in the queue during the life of the \Condor{schedd}.  For
proper functioning of the \Condor{shadow}, the \Condor{gridmanager}, and
the \Condor{job\_router}, this expression, if set, must match the owner
names of all jobs that these daemons will manage.
\Ticket{3030}

\item The new configuration variable \Macro{DEFRAG\_CANCEL\_REQUIREMENTS}
is an expression that specifies which draining machines should have 
draining be canceled.  
This defaults to \MacroUNI{DEFRAG\_WHOLE\_MACHINE\_EXPR}.  
This could be used to drain partial rather than whole machines.
\Ticket{2993}

\item The new submit command \SubmitCmd{use\_x509userproxy} can be set
to \Expr{True} to indicate that an X.509 user proxy is required for the job. 
If \SubmitCmd{x509userproxy} is not set, 
then the proxy file will be looked for in the standard locations.
\Ticket{3025}

\item If \Condor{submit} is used to submit an interactive job,
and the job is interrupted before the interactive job starts,
an attempt is made to immediately remove the interactive job from the queue.
Similarly, \Condor{ssh\_to\_job} has a new option \Opt{-remove-on-interrupt}.
\Ticket{3242}

\item Changes to were made to the ClassAd machine attributes 
\Attr{OpSys}, \Attr{OpSysVer}, \Attr{Distro}, as well as others,
in order to do a better job of identifying the operating system.
\Ticket{2366}

\item \Macro{GRIDMANAGER\_MAX\_SUBMITTED\_JOBS\_PER\_RESOURCE} can now be a
list, specifying different values for different hosts.
\Ticket{3220}

\item The new configuration parameter \Macro{GRIDMANAGER\_JOB\_PROBE\_RATE}
limits the number of job status requests sent to each remote resource.
\Ticket{3023}

\item The default value of \Macro{GRIDMANAGER\_JOB\_PROBE\_INTERVAL} has
changed from 300 to 60.
\Ticket{3023}.

\item The configuration parameters \Macro{CONDOR\_JOB\_POLL\_INTERVAL} and
\Macro{INFN\_JOB\_POLL\_INTERVAL} should no longer be used. Use
\Macro{GRIDMANAGER\_JOB\_PROBE\_INTERVAL\_CONDOR} and
\Macro{GRIDMANAGER\_JOB\_PROBE\_INTERVAL\_BATCH} instead.
\Ticket{3023}

\end{itemize}

\noindent Bugs Fixed:

\begin{itemize}

\item \Security Fixed a bug which allowed jobs submitted to the standard
universe to escalate privilege on the submit machine and execute code as 
\Login{root}.
(CVE-2012-5390)
\Ticket{3268}

\item A fix only invokes Globus callouts when actually needed, 
thereby avoiding a program segfault if
the call out mechanism is misconfigured or broken.
\Ticket{2104}

\item Fixed a bug in all daemons wherein the \Attr{DaemonStartTime} attribute 
in the ClassAd for all daemons would be reset to the current time when they are
reconfigured.
\Ticket{3235}

\item Fixed a bug wherein the \Arg{-dont\_use\_default\_node\_log} command line flag
to \Condor{submit\_dag} had no effect.
\Ticket{3352}

\item \Security Although not user-visible, 
there were multiple updates that removed places
in the code where potential buffer overruns could occur, 
thus preventing potential attacks.  
None of these overruns were known to be exploitable.

\item \Security Although not user-visible, 
there were updates to the code to improve
the error checking of system calls,
thereby removing some potential security threats.  
None were known to be exploitable.

\item \Security Although not user-visible, 
removed some code that was no longer used.
The presence of this code could have led to a Denial-of-Service attack,
which would allow an attacker to stop another user's jobs from running.

\item \Security Filesystem (FS) authentication was improved to check 
the UNIX permissions of the directory used for authentication.  
Without this, an attacker may have
been able to impersonate another submitter on the same submit machine.

\item The \Condor{negotiator} now checks the accountant log file for sanity
once only on start up,  
thereby increasing efficiency of iteration through 
the accountant ClassAd log structure.
\Ticket{3011}

\item The ClassAd functions \Procedure{splitUserName} and 
\Procedure{splitSlotName}
no longer leak a small amount of memory each time they are evaluated.  
This bug was introduced when these functions were added in Condor version 7.7.6.
\Ticket{3082}

\item There are several bug fixes for grid-type batch jobs:
  \begin{itemize}
  \item Monitoring the status of jobs submitted to PBS and SGE has been
    improved. \Ticket{3067} \Ticket{3157} \Ticket{3181}
  \item Job command-line arguments containing 
    left parenthesis, \verb@(@, right parenthesis, \verb@)@, 
    and ampersand, \verb@&@, characters are now handled properly. 
    \Ticket{3057}
  \item Removing PBS jobs that have just completed no longer causes the jobs
    to become held. \Ticket{3016}
  \item Added a work-around for a bug when submitting jobs to
    a Condor pool running Condor versions 7.7.6 through 7.8.2.
    A bug in \Condor{history} \Opt{-f} caused an error in determining
    a job's status.
    \Ticket{3133}
  \item Improved the handling of job files when the batch system has a shared
    file system. \Ticket{3195}
  \end{itemize}

\item Changes introduced in Condor version 7.9.0 caused jobs submitted by
\Condor{dagman} in the local universe to not write to the default node log file,
when \Macro{DAGMAN\_ALWAYS\_USE\_NODE\_LOG} was \Expr{True} (the default),
and a user log was also defined. This is fixed. 
\Ticket{3111}

\item Fixed a bug introduced in Condor version 7.9.0 that caused grid type
cream jobs to be held with a hold reason of 
\footnotesize
\begin{verbatim}
  CREAM_Delegate Error: Cannot set credentials in the gsoap-plugin context.
\end{verbatim}
\normalsize
\Ticket{3234}

\item Fixed a problem that could have caused the \Condor{collector} to crash
when receiving an invalid packet.
\Ticket{3161}

\end{itemize}

\noindent Known Bugs:

\begin{itemize}

\item None.

\end{itemize}

\noindent Additions and Changes to the Manual:

\begin{itemize}

\item None.

\end{itemize}


%%%%%%%%%%%%%%%%%%%%%%%%%%%%%%%%%%%%%%%%%%%%%%%%%%%%%%%%%%%%%%%%%%%%%%
\subsection*{\label{sec:New-7-9-0}Version 7.9.0}
%%%%%%%%%%%%%%%%%%%%%%%%%%%%%%%%%%%%%%%%%%%%%%%%%%%%%%%%%%%%%%%%%%%%%%

\noindent Release Notes:

\begin{itemize}

\item Condor version 7.9.0 released on August 16, 2012.

\end{itemize}


\noindent New Features:

\begin{itemize}

\item Machine slots can now be configured to identify and
divide customized local resources.
Jobs may then request these resources.
See section~\ref{sec:Configuring-SMP} for details.
\Ticket{2905}

\item Condor now supports and implements the caching of ClassAds 
to reduce memory footprints. 
This feature is experimental and is currently disabled by default.
It can be enabled by setting
the new configuration variable \Macro{ENABLE\_CLASSAD\_CACHING}
to \Expr{True}.
\Ticket{2541}
\Ticket{3127}

\item \Condor{status} now returns the \Condor{schedd} ClassAd directly 
from the \Condor{schedd} daemon,
if both options \Opt{-direct} and \Opt{-schedd} are given on the command line.
\Ticket{2492}

\item The new \Opt{-status} and \Opt{-echo} command line options to 
\Condor{wait} command cause it to show job start and terminate information,
and to print events to \Code{stdout}.
\Ticket{2926}

\item Added a \Expr{DEBUG} logging level output flag \Dflag{CATEGORY},
which causes Condor to include the logging level
flags in effect for each line of logged output.
\Ticket{2712}

\item \Condor{status} and \Condor{q} each have a new \Opt{-autoformat} option
to make some output format specifications easier than the existing
\Opt{-format} option.
See the \Condor{status} manual page located on page~\pageref{man-condor-status}
and the \Condor{q} manual page located on page~\pageref{man-condor-q} 
for details.
\Ticket{2941}

\item Enhanced the ClassAd log system to report the log line number 
on parse failures, 
and improved the ability to detect parse failures closer to 
the point of corruption.
\Ticket{2934}

\item Added an \Opt{-evaluate} option to \Condor{config\_val}, which causes the configured value queried from
a given daemon to be evaluated with respect to that daemon's ClassAd.
\Ticket{856}

\item Added code to \Condor{dagman},
such that a \Expr{VARS} assignment in a top-level DAG is applied to splices.
\Ticket{1780}

\item Condor now uses libraries from Globus 5.2.1.
\Ticket{2838}

\item When authenticating Condor daemons with GSI and
configuration variable \MacroNI{GSI\_DAEMON\_NAME} is undefined, 
Condor checks that the server name in the certificate matches the 
host name that the client is connecting to. 
When \MacroNI{GSI\_DAEMON\_NAME} is defined,
the old behavior is preserved: only certificates matching
\MacroNI{GSI\_DAEMON\_NAME} pass the authentication step, 
and no host name check is performed.  
The behavior of the host name check
may be further controlled with the new configuration variables
\MacroNI{GSI\_SKIP\_HOST\_CHECK} and
\MacroNI{GSI\_SKIP\_HOST\_CHECK\_CERT\_REGEX}.
\Ticket{1605}

\item Added new capability to \Condor{submit} to allow recursive macros in
submit description files. 
This facility allows one to update variables recursively. 
Before this new capability was added,
recursive definition would send \Condor{submit} into an
infinite loop of expanding the macro,
such that the expansion would fill up memory.
See section~\ref{macro-in-submit-description-file} for details.
\Ticket{406}

\item A DAGMan limitation and restriction has been removed.  
It is now permitted to define a \SubmitCmd{log} command using a macro,
within a node job's submit description file.
\Ticket{2428}

\item To enforce the dependencies of a DAG,
DAGMan now uses and watches only the default node
user log of the \Condor{dagman} job for events.  
DAGMan requests the \Condor{schedd} and \Condor{shadow} daemons to write each
event to this default log, 
in addition to writing to a log specified by the node job.
\Condor{dagman} writes POST script terminate events only to its default log;
these terminate events are not written to the user log.
This behavior can be turned off by setting the configuration variable
\Macro{DAGMAN\_ALWAYS\_USE\_NODE\_LOG} to \Expr{False}.
For correct behavior,
\MacroNI{DAGMAN\_ALWAYS\_USE\_NODE\_LOG} should be set to \Expr{False}
if \Condor{dagman} version 7.9.0 or later is submitting jobs 
to an older version of
a \Condor{schedd} daemon or of a \Condor{submit} executable.
\Ticket{2807}

\item \Condor{submit} has a new \Opt{-interactive} option for
platforms other than Windows,
which schedules and runs a job that provides a shell prompt
on the execute machine.
\Ticket{3088}

\end{itemize}

\noindent Configuration Variable and ClassAd Attribute Additions and Changes:

\begin{itemize}

\item The new configuration variables \Macro{MACHINE\_RESOURCE\_NAMES}
(see section~\ref{param:MachineResourceNames})
and \Macro{MACHINE\_RESOURCE\_<name>}
(see section~\ref{param:MachineResourceResourcename})
identify and specify the use of customized local machine resources.
\Ticket{2905}

\item The new configuration variable \MacroNI{ENABLE\_CLASSAD\_CACHING}
controls whether the new caching feature of ClassAds is used.
The default value is \Expr{False}.
\Ticket{3127}

\item The new configuration variable \Macro{CLASSAD\_LOG\_STRICT\_PARSING}
controls whether ClassAd log files such as the job queue
log are read with strict parse checking on ClassAd expressions.
\Ticket{3069}

\item The default value for configuration variable \Macro{USE\_PROCD}
is now \Expr{True} for the \Condor{master} daemon.  
This means that by
default the \Condor{master} will start a \Condor{procd} daemon to be used 
by it and all other daemons on that machine.
\Ticket{2911}

\item There is a new configuration variable used by the \Condor{starter}.
If \Macro{STARTER\_RLIMIT\_AS} is set to an integer value, 
the \Condor{starter}
will use the \Procedure{setrlimit} system call with the 
\Code{RLIMIT\_AS} parameter to
limit the virtual memory size of each process in the user job.  
The value of this configuration variable is a ClassAd expression, 
evaluated in the context of both the machine and job ClassAds, 
where the machine ClassAd is the \Expr{my} ClassAd, 
and the job ClassAd is the \Expr{target} ClassAd.
\Ticket{1663}

\item New configuration variables were added to to the \Condor{schedd} to
define statistics that count subsets of jobs. 
These variables have the form \Macro{SCHEDD\_COLLECT\_STATS\_BY\_<Name>},
and should be defined by a ClassAd expression that evaluates to a string.
See section~\ref{param:ScheddCollectStatsByName}
for the complete definition.
The optional configuration variable of the form
\Macro{SCHEDD\_EXPIRE\_STATS\_BY\_<Name>} can be used to set an expiration time,
in seconds, for each set of statistics.
\Ticket{2862}

\item The new \SubmitCmd{batch\_queue} submit description file command
and new job ClassAd attribute \Attr{BatchQueue} specify which job
queue to use for grid universe jobs of type
\SubmitCmd{pbs}, \SubmitCmd{lsf}, and \SubmitCmd{sge}.
\Ticket{2996}

\item The new configuration variable \Macro{GSI\_SKIP\_HOST\_CHECK} is
a boolean that controls whether a check is performed during
GSI authentication of a Condor daemon.  
When the default value \Expr{False},
the check is not skipped, so the daemon host name must match the
host name in the daemon's certificate, unless otherwise exempted
by values of \MacroNI{GSI\_DAEMON\_NAME} or
\MacroNI{GSI\_SKIP\_HOST\_CHECK\_CERT\_REGEX}.
When \Expr{True}, this check is skipped, and hosts will not be rejected
due to a mismatch of certificate and host name.
\Ticket{1605}

\item The new configuration variable
\MacroNI{GSI\_SKIP\_HOST\_CHECK\_CERT\_REGEX} may be set to a
regular expression.  GSI certificates of Condor daemons with a
subject name that are matched in full by this regular expression
are not required to have a matching daemon host name and certificate
host name.  The default is an empty regular expression, which will
not match any certificates, even if they have an empty subject name.
\Ticket{1605}

\end{itemize}

\noindent Bugs Fixed:

\begin{itemize}

\item Fixed a bug in which usage of cgroups incorrectly included the page cache 
in the maximum memory usage.
This bug fix is also included in Condor version 7.8.2.
\Ticket{3003}

\item The EC2 GAHP will now respect the value of the environment variable
\Env{X509\_CERT\_DIR} and the configuration variable
\Macro{GSI\_DAEMON\_TRUSTED\_CA\_DIR} for \emph{all} secure connections.
\Ticket{2823}

\item Condor will avoid selecting down (disabled) network interfaces.  Previously Condor could select a down interface over an up (active) interface.
\Ticket{2893}

\item Made logic in the \Condor{negotiator} that computes submitter limits 
properly aware of the configuration variable
\Macro{NEGOTIATOR\_CONSIDER\_PREEMPTION}.
\Ticket{2952}


\item Condor no longer back-dates file modification times by 3 minutes
when transferring job input files into the job spool directory or the job
execute directory.
\Ticket{2423}

\item Fixed a bug in which the use of a pipe in the configuration file 
on Windows platforms would cause a visible console window 
to show up whenever the configuration was read.
\Ticket{1534}

\end{itemize}

\noindent Known Bugs:

\begin{itemize}

\item None.

\end{itemize}

\noindent Additions and Changes to the Manual:

\begin{itemize}

\item Machine ClassAd attribute string values relating to \Attr{OpSys} have
been documented for Scientific Linux platforms.
\Ticket{2366}

\end{itemize}


%%%      PLEASE RUN A SPELL CHECKER BEFORE COMMITTING YOUR CHANGES!
%%%      PLEASE RUN A SPELL CHECKER BEFORE COMMITTING YOUR CHANGES!
%%%      PLEASE RUN A SPELL CHECKER BEFORE COMMITTING YOUR CHANGES!
%%%      PLEASE RUN A SPELL CHECKER BEFORE COMMITTING YOUR CHANGES!
%%%      PLEASE RUN A SPELL CHECKER BEFORE COMMITTING YOUR CHANGES!

%%%%%%%%%%%%%%%%%%%%%%%%%%%%%%%%%%%%%%%%%%%%%%%%%%%%%%%%%%%%%%%%%%%%%%
\section{\label{sec:History-7-8}Stable Release Series 7.8}
%%%%%%%%%%%%%%%%%%%%%%%%%%%%%%%%%%%%%%%%%%%%%%%%%%%%%%%%%%%%%%%%%%%%%%

This is a stable release series of Condor.
As usual, only bug fixes (and potentially, ports to new platforms)
will be provided in future 7.8.x releases.
New features will be added in the 7.9.x development series.

The details of each version are described below.

%%%%%%%%%%%%%%%%%%%%%%%%%%%%%%%%%%%%%%%%%%%%%%%%%%%%%%%%%%%%%%%%%%%%%%
\subsection*{\label{sec:New-7-8-0}Version 7.8.0}
%%%%%%%%%%%%%%%%%%%%%%%%%%%%%%%%%%%%%%%%%%%%%%%%%%%%%%%%%%%%%%%%%%%%%%

\noindent Release Notes:

\begin{itemize}

\item Condor version 7.8.0 not yet released.
%\item Condor version 7.8.0 released on Month Date, 2012.

\end{itemize}


\noindent New Features:

\begin{itemize}

\item None.

\end{itemize}

\noindent Configuration Variable and ClassAd Attribute Additions and Changes:

\begin{itemize}

\item None.

\end{itemize}

\noindent Bugs Fixed:

\begin{itemize}

\item None.

\end{itemize}

\noindent Known Bugs:

\begin{itemize}

\item None.

\end{itemize}

\noindent Additions and Changes to the Manual:

\begin{itemize}

\item None.

\end{itemize}



%%%      PLEASE RUN A SPELL CHECKER BEFORE COMMITTING YOUR CHANGES!
%%%      PLEASE RUN A SPELL CHECKER BEFORE COMMITTING YOUR CHANGES!
%%%      PLEASE RUN A SPELL CHECKER BEFORE COMMITTING YOUR CHANGES!
%%%      PLEASE RUN A SPELL CHECKER BEFORE COMMITTING YOUR CHANGES!
%%%      PLEASE RUN A SPELL CHECKER BEFORE COMMITTING YOUR CHANGES!

%%%%%%%%%%%%%%%%%%%%%%%%%%%%%%%%%%%%%%%%%%%%%%%%%%%%%%%%%%%%%%%%%%%%%%
\section{\label{sec:History-7-7}Development Release Series 7.7}
%%%%%%%%%%%%%%%%%%%%%%%%%%%%%%%%%%%%%%%%%%%%%%%%%%%%%%%%%%%%%%%%%%%%%%

This is the development release series of Condor.
The details of each version are described below.

%%%%%%%%%%%%%%%%%%%%%%%%%%%%%%%%%%%%%%%%%%%%%%%%%%%%%%%%%%%%%%%%%%%%%%
\subsection*{\label{sec:New-7-7-0}Version 7.7.0}
%%%%%%%%%%%%%%%%%%%%%%%%%%%%%%%%%%%%%%%%%%%%%%%%%%%%%%%%%%%%%%%%%%%%%%

\noindent Release Notes:

\begin{itemize}

\item Condor version 7.7.0 released on July 29, 2011.
This developer release contains all bug fixes from Condor version 7.6.2.

\end{itemize}


\noindent New Features:

\begin{itemize}

\item A full port of Condor is available for RedHat Enterprise Linux 6
on the x86\_64 processor.
A full port includes support for the standard universe.

\item The matchmaking attributes \Attr{SubmittersUserResourcesInUse}
and \Attr{RemoteuserResourcesInUser} are now biased by slot weights.

\item \Condor{submit} now accepts the new command line option \Opt{-addr},
naming the IP address of the \Condor{schedd} to submit to.

\item CONDOR\_VIEW\_HOST can now support more than one condor view host.

\item The vmgahp now is dynamically linked to libvirt.  We believe
this makes it more portable.

\item Removed the condor\_reschedule\_schedd and condor\_master\_off
programs.  These have been replaced many years ago by the more general
condor\_reschedule and condor\_off commands.

\item New parameter SHADOW\_RUN\_UNKNOWN\_USER\_JOBS.  This defaults
to false, when true, allows the shadow to run jobs remotely submitted
from users not in the local passwd file.
 
\item On Windows, improve the ability of the starter and shadow
to clean up the execute directory if jobs have changed the ACLs
or permissions on files they have created.

\item The security parameters DENY\_CLIENT and HOSTDENY\_CLIENT
now also look for the prefixes TOOL and SUBMIT.

\item Condor\_submit now sets a default RequestMemory size.

\item CREAM submission performance has been substantially
improved by batching submit requests.

\item condor\_q -better now has cleaner output, and informs
the user when negotiation has not happened yet.

\item Many improvements to the Condor init scripts.

\item Deltacloud support updated to deltacloud version 0.8.

\item New parameter NEGOTIATOR\_UPDATE\_AFTER\_CYCLE, defaults 
to false.  If set to true, it will force the Negotiator to publish an update ad to the Collector at the end of every negotiation cycle. This is very useful
    if monitoring LastNegotiationCycle statistics.

% gittrac #1960
\item VM universe submit files no longer supports
automatic creation of cdrom images from text input file.
Users must now explicitly create ISO images and transfer them
with the job.

\item \Condor{q} now supports the new option \Opt{-stream-results}.
  When this option is specified, \Condor{q} displays results as they
  are fetched from the job queue, rather than buffering up the query
  results before displaying anything.

% gittrac #1871 
% gittrac #2295
\item The new submit description file command \SubmitCmd{stack\_size} 
  applies to Linux jobs that are not running in the standard universe. 
  It sets the allocation of stack space to be other than the default
  value, which is unlimited.
  It also advertises the job ClassAd attribute \AdAttr{StackSize}.

% gittrac #1550
\item The new ClassAd function \Code{stringListsIntersect} evaluates to 
  \Expr{True} if two strings of delimited elements have any matching elements,
  and it evaluates to \Expr{False} otherwise.

% gittrac #1821
\item The grid universe now supports the \SubmitCmd{ec2} resource type,
  which uses the EC2 Query (REST) API to start virtual machines on cloud
  resources.

% gittrac #2090 
\item The behavior of DAGMan has changed, 
such that if multiple definitions of a VARS macroname 
for a specific node within a DAG input exist,
a warning is written to the log, of the format
\begin{verbatim}
Warning: VAR <macroname> is already defined in job <JobName>
Discovered at file "<DAG input file name>", line <line number>
\end{verbatim}
See section ~\ref{dagman:VARS} for details.

% gittrac #2297
\item The version number for ClassAds now matches the Condor version number. 

% gittrac #2259
\item When \Prog{glexec} fails to execute a job,
diagnostic error messages produced by \Prog{glexec} used to be discarded.
These error messages are now displayed in the log of the \Condor{starter} 
and in the job's hold reason. 

% gittrac #2185
\item New submit description file commands
\SubmitCmd{periodic\_hold\_reason}, \SubmitCmd{periodic\_hold\_subcode},
\SubmitCmd{on\_exit\_hold\_reason}, and \SubmitCmd{on\_exit\_hold\_subcode}
permit the job to set a hold reason string and subcode number.
Similarly, the system job policy can specify the reason and subcode 
using \Macro{SYSTEM\_PERIODIC\_HOLD\_REASON} and 
\Macro{SYSTEM\_PERIODIC\_HOLD\_SUBCODE}.
In addition, the \Condor{hold} command now accepts a \Opt{-subcode} option,
which is used to set the job attribute \Attr{HoldReasonSubCode}. 

\end{itemize}


\noindent Configuration Variable and ClassAd Attribute Additions and Changes:

\begin{itemize}

\item For a job with an X.509 proxy credential, the new job ClassAd
attribute \AdAttr{X509UserProxyEmail} is the email address extracted
from the proxy.

% gittrac 2067
\item On Linux execute machines with kernel version more recent than 2.6.27,
the proportional set size (PSS) in Kbytes summed across all
processes in the job is now reported in the attribute
\AdAttr{ProportionalSetSizeKb}.  If the execute machine does not
support monitoring of PSS or PSS has not yet been measured, this
attribute will be undefined.  PSS differs from \AdAttr{ImageSize} in
how memory shared between processes is accounted.  The PSS for one
process is the sum of that process' memory pages divided by the
number of processes sharing each of the pages.  \AdAttr{ImageSize} is
the same, except there is no division by the number of processes
sharing the pages.

% gittrac #1755
\item The new configuration variable \Macro{DAGMAN\_USE\_STRICT} 
turns warnings into errors, as defined in section~\ref{param:DAGManUseStrict}.

% gittrac #2006
\item The \Condor{schedd} now publishes performance-related statistics.
  Page~\pageref{sec:Scheduler-ClassAd-Attributes} in Appendix A contains
  definitions for these new attributes:
  \begin{itemize}
    \item \Attr{DetectedMemory}
    \item \Attr{DetectedCpus}
    \item \Attr{UpdateInterval}
    \item \Attr{WindowedStatWidth}
    \item \Attr{ExitCode<N>}
    \item \Attr{ExitCodeCumulative<N>}
    \item \Attr{JobsSubmitted}
    \item \Attr{JobsSubmittedCumulative}
    \item \Attr{JobsStarted}
    \item \Attr{JobsStartedCumulative}
    \item \Attr{JobsCompleted}
    \item \Attr{JobsCompletedCumulative}
    \item \Attr{JobsExited}
    \item \Attr{JobsExitedCumulative}
    \item \Attr{ShadowExceptions}
    \item \Attr{ShadowExceptionsCumulative}
    \item \Attr{JobSubmissionRate}
    \item \Attr{JobStartRate}
    \item \Attr{JobCompletionRate}
    \item \Attr{MeanTimeToStart}
    \item \Attr{MeanTimeToStartCumulative}
    \item \Attr{MeanRunningTime}
    \item \Attr{MeanRunningTimeCumulative}
    \item \Attr{SumTimeToStartCumulative}
    \item \Attr{SumRunningTimeCumulative}
  \end{itemize}

% gittrac #1930
\item For Windows platforms, the \Condor{startd} now publishes the 
ClassAd attribute \Attr{DotNetVersions},
containing a comma separated list of installed .NET versions.

\end{itemize}

\noindent Bugs Fixed:

\begin{itemize}

\item Fixed a bug in which the startd can get stuck in a
loop trying to execute an invalid (i.e. non-existent) Daemon ClassAd Hook job.

\item Fixed bug that would cause the Startd to incorrectly
report Benchmarking activity instead of Idle activity when there
is a problem launching the benchmarking programs.

\item On Windows only, fixed a rare bug that could cause
a sporadic access violation when a Condor daemon spawned another process.

\item Fixed a bug introduced in 7.5.5 which caused the
schedd to die managing parallel jobs.

\item Fixed bug throughout classads where expressions expecting
a floating point value returned an error if they got a boolean
value.  This is common in RANK expressions.

\item The Startd now looks up the Keyboard daemon address
on every update.  This fixed problems if the Keyboard
daemon is restarted during the startd lifespan.

\item Fixed bug in Condor hold that happened if the hold
reason contained a double quote character.

\item Fixed a bug introduced in 7.5.6 that
caused any startd cron job with non-empty ARGS to fail.
Also, the ENV specified for startd cron jobs was ignored.

\item Fixed bug in amazon grid manager which would cause it to crash.

\item Fixed bug in rpm init script. A status request would always report Condor inactive, and a shutdown request would not report failure if there was a
timeout shutting down Condor.

\item File transfer plugins now have a correctly set environment.

\item Fixed problem with detecting IBM Java Virtual Machines whose
version strings have embedded newlines.

\item condor\_q -analyze now works with new classad built-in functions.

\item Standalone checkpointing now works with compressed checkpoints again.
This had been broken in 7.5.4

\item If the shadow cannot write to the user log, the job
is now put on hold.

%gittrac 1962
\item On Windows, net stop condor would sometimes cause the
master to crash.  This is now fixed.

\item Fixed bug in condor\_q -run so that it displays
the hostname correctly for local and scheduler universe jobs.

% gittrac #1928
\item \AdAttr{JobUniverse} was effectively a required attribute for
  jobs created via the Fetch Work hook,
  due to the need to set the \MacroNI{IS\_VALID\_CHECKPOINT\_PLATFORM}
  expression, such that it would not evaluate to \Expr{Undefined}.
  Now the default \MacroNI{IS\_VALID\_CHECKPOINT\_PLATFORM} expression
  evaluates to \Expr{True} when \AdAttr{JobUniverse} is not defined.

% gittrac #1943
\item When there are multiple cpus but only one slot, the slot name no
longer begins with \Expr{slot1@}.

% gittrac #1805 
\item The tool \Condor{advertise} seemed to be trying too hard to resolve
host names. This was fixed to only do the minimally necessary 
number of look ups.

\end{itemize}

\noindent Known Bugs:

\begin{itemize}

\item None.

\end{itemize}

\noindent Additions and Changes to the Manual:

\begin{itemize}

\item None.

\end{itemize}


%%%      PLEASE RUN A SPELL CHECKER BEFORE COMMITTING YOUR CHANGES!
%%%      PLEASE RUN A SPELL CHECKER BEFORE COMMITTING YOUR CHANGES!
%%%      PLEASE RUN A SPELL CHECKER BEFORE COMMITTING YOUR CHANGES!
%%%      PLEASE RUN A SPELL CHECKER BEFORE COMMITTING YOUR CHANGES!
%%%      PLEASE RUN A SPELL CHECKER BEFORE COMMITTING YOUR CHANGES!

%%%%%%%%%%%%%%%%%%%%%%%%%%%%%%%%%%%%%%%%%%%%%%%%%%%%%%%%%%%%%%%%%%%%%%
\section{\label{sec:History-7-6}Stable Release Series 7.6}
%%%%%%%%%%%%%%%%%%%%%%%%%%%%%%%%%%%%%%%%%%%%%%%%%%%%%%%%%%%%%%%%%%%%%%

This is a stable release series of Condor.
As usual, only bug fixes (and potentially, ports to new platforms)
will be provided in future 7.6.x releases.
New features will be added in the 7.7.x development series.

The details of each version are described below.

%%%%%%%%%%%%%%%%%%%%%%%%%%%%%%%%%%%%%%%%%%%%%%%%%%%%%%%%%%%%%%%%%%%%%%
\subsection*{\label{sec:New-7-6-0}Version 7.6.0}
%%%%%%%%%%%%%%%%%%%%%%%%%%%%%%%%%%%%%%%%%%%%%%%%%%%%%%%%%%%%%%%%%%%%%%

\noindent Release Notes:

\begin{itemize}

\item Condor version 7.6.0 not yet released.
%\item Condor version 7.6.0 released on Month Date, 2011.

% gittrac #2016
\item Prior to Condor version 7.5.0, commenting out \MacroNI{PREEN} in the
  default configuration file disabled \Condor{preen}.  
  Starting in Condor version 7.5.0,
  there was an internal default value for \MacroNI{PREEN}, so if
  the configuration variable was not set in any configuration file,
  \Condor{preen} would still run.
  To disable this functionality, \MacroNI{PREEN} can be explicitly set to
  nothing.

\end{itemize}


\noindent New Features:

\begin{itemize}

\item Condor can now create and manage virtual machine instances in a
cloud service via Deltacloud. This is done via the new
\SubmitCmd{deltacloud} grid type in the grid universe.
See section ~\ref{sec:Deltacloud} for details.

% gittrac #1931
\item Improved scalability of submission of cream grid type jobs.

\end{itemize}

\noindent Configuration Variable and ClassAd Attribute Additions and Changes:

\begin{itemize}

\item The new configuration variable \Macro{DELTACLOUD\_GAHP} specifies
where the \Prog{deltacloud\_gahp} binary is located. This binary is used to
manage deltacloud grid type jobs in the grid universe.
In a normal Condor installation, the value should be
\File{\$(SBIN)/deltacloud\_gahp}.

\item Several new job ClassAd attributes have been added to support
the deltacloud grid type in the grid universe.
These attributes are taken from the submit description file:
\AdAttr{DeltacloudUsername},
\AdAttr{DeltacloudPasswordFile},
\AdAttr{DeltacloudImageId},
\AdAttr{DeltacloudRealmId},
\AdAttr{DeltacloudHardwareProfile},
\AdAttr{DeltacloudHardwareProfileCpu},
\AdAttr{DeltacloudHardwareProfileMemory},
\AdAttr{DeltacloudHardwareProfileStorage},
\AdAttr{DeltacloudKeyname}, and
\AdAttr{DeltacloudUserData}.
%\AdAttr{DeltacloudRetryTimeout},
These attributes are set by Condor when the instance runs:
\AdAttr{DeltacloudAvailableActions},
\AdAttr{DeltacloudPrivateNetworkAddresses},
\AdAttr{DeltacloudPublicNetworkAddresses}.
See section ~\ref{sec:Deltacloud} for details of running jobs under
Deltacloud, and see section ~\ref{sec:Job-ClassAd-Attributes}
for definitions of these job ClassAd attributes.

% gittrac #2024
\item The configuration variable \Macro{JAVA\_MAXHEAP\_ARGUMENT} 
  has been removed. 
  This means that Java universe jobs will now run with the JVM's 
  default maximum heap setting,
  unless specified otherwise by the administrator using the configuration
  of \Macro{JAVA\_EXTRA\_ARGUMENTS},
  or by the job via 
  \SubmitCmd{java\_vm\_args} in the submit description file
  as described in section~\ref{sec:Java}.

% gittrac #2066
\item The configuration variable \Macro{TRUST\_UID\_DOMAIN}
  was set to \Expr{True} in the default \File{condor\_config.local}
  in the rpm and debian packages.  This is no longer the case.
  This setting will therefore use the default value \Expr{False}.

\item The configuration variable \Macro{NEGOTIATOR\_INTERVAL} was set
  to 20 in the default \File{condor\_config.local} in the rpm and
  debian packages.  This is no longer the case.  This setting
  therefore will use the default value 60.

\end{itemize}

\noindent Bugs Fixed:

\begin{itemize}

% gittrac #1957
\item Fixed a bug in \Condor{dagman} that caused it to fail when in recovery
mode in the face of a specific combination of node job failures with
retries.

% gittrac #1991
\item Fixed a bug that resulted in the spooled user log not being
  fetched by \Condor{transfer\_data}.  Prior to Condor version 7.5.4, this
  problem affected spooled jobs submitted with an explicit list of
  output files to transfer.  In Condor version 7.5.4, this problem also
  affected spooled jobs that auto-detected output files.

% gittrac #1985
\item When a job is submitted with the \Opt{-spool} option to \Condor{submit},
the \Condor{schedd} now correctly writes the submit event to the user log 
in its spool directory. 
Previously, it would try to write the event using the user
log path given to \Condor{submit} by the user, 
which \Condor{submit} may not have access to.

% gittrac #2001
\item Fixed a file descriptor leak in the \Condor{vm-gahp}. The leak would
cause the daemon to fail if a VMware job ran for more than 16 hours on a
Linux machine.

%gittrac #2017
\item Fixed a bug in \Condor{dagman} that caused it to treat any dollar
sign in the log file name of a node job's submit description file
as an illegal \Condor{dagman} macro.
Now only the sequence of characters \Expr{\$(} delimits a macro.

\end{itemize}

\noindent Known Bugs:

\begin{itemize}

\item There are two known issues related to the automatic creation
of checkpoints with the Condor checkpointing library in 
Condor version 7.6.0.
The first is that compression of
standalone checkpoints is disabled for 32-bit binaries.
It was always disabled previously, for 64-bit binaries.
A standalone checkpoint is one that is run outside
of Condor's standard universe.  The second problem has to do with compressed
32-bit checkpoint files within the standard universe.
If a user requests a compressed 32-bit checkpoint in the standard universe,
the resulting checkpoint will not be compressed.
As with standalone checkpoints, this has never been supported
in 64-bit binaries.  We hope to fix both problems in 
Condor version 7.6.1.

\end{itemize}

\noindent Additions and Changes to the Manual:

\begin{itemize}

\item None.

\end{itemize}


% as of April 2012, Karen no longer wants to include these older
% version histories with the 7.4 and 7.5 manuals.
%%%%      PLEASE RUN A SPELL CHECKER BEFORE COMMITTING YOUR CHANGES!
%%%      PLEASE RUN A SPELL CHECKER BEFORE COMMITTING YOUR CHANGES!
%%%      PLEASE RUN A SPELL CHECKER BEFORE COMMITTING YOUR CHANGES!
%%%      PLEASE RUN A SPELL CHECKER BEFORE COMMITTING YOUR CHANGES!
%%%      PLEASE RUN A SPELL CHECKER BEFORE COMMITTING YOUR CHANGES!

%%%%%%%%%%%%%%%%%%%%%%%%%%%%%%%%%%%%%%%%%%%%%%%%%%%%%%%%%%%%%%%%%%%%%%
\section{\label{sec:History-7-5}Development Release Series 7.5}
%%%%%%%%%%%%%%%%%%%%%%%%%%%%%%%%%%%%%%%%%%%%%%%%%%%%%%%%%%%%%%%%%%%%%%

This is the development release series of Condor.
The details of each version are described below.

%%%%%%%%%%%%%%%%%%%%%%%%%%%%%%%%%%%%%%%%%%%%%%%%%%%%%%%%%%%%%%%%%%%%%%
\subsection*{\label{sec:New-7-5-6}Version 7.5.6}
%%%%%%%%%%%%%%%%%%%%%%%%%%%%%%%%%%%%%%%%%%%%%%%%%%%%%%%%%%%%%%%%%%%%%%

\noindent Release Notes:

\begin{itemize}

\item Condor version 7.5.6 released on March 21, 2011.

\item What used to be known as the \Condor{startd} and \Condor{schedd} cron
  mechanisms are now collectively called \Term{Daemon ClassAd Hooks}.
  The significant changes in this Condor version 7.5.6 release are 
  given in the New Features section.

% gittrac #1935
\item In the release directory, the subdirectory \File{lib/glite/} has
  been moved to \File{libexec/glite/}.

% gittrac #1897
\item This development series of Condor is no longer officially released 
  for the platforms PowerPC AIX, PowerPC-64 SLES 9, PowerPC MacOS 10.4, 
  Solaris 5.9 on all architectures, 
  Solaris 5.10 on all architectures, 
  Itanium IA64 RHEL 3, PS3 (PowerPC) YDL 5.0, and x86 Debian 4.

% gittrac #1924
\item Support for GCB has been removed.

% gittrac #1661
\item The default Unix Sys-V init script has been completely reworked.
  In addition to new features, this changes the following:
  \begin{itemize}
    \item The default location of the Condor configuration file is now
      \File{/etc/condor/condor\_config}.  This location can be changed by
      editing the \File{sysconfig} file or the init script itself.
    \item The default location of the Condor installation is now
      \File{/usr/}, with binaries in \File{/usr/bin} and \File{/usr/sbin}.
      These locations can also be changed by editing the \File{sysconfig} file
      or the init script itself.
  \end{itemize}

\end{itemize}


\noindent New Features:

\begin{itemize}

% gittrack 1800
\item Condor no longer relies on DNS to determine its IP address.
  Instead, it examines the list of system network devices.

% gittrac #1754
\item \Condor{dagman} now gives a warning if a node category has no
nodes assigned to it or no throttle set.

% gittrac #1855
\item \Condor{dagman} now has a \Env{\$MAX\_RETRIES} macro for PRE and
POST script arguments.
Also, \Condor{dagman} now prints a warning if an unrecognized macro is
used for a PRE or POST script argument.
See ~\pageref{dagman:SCRIPT} for details.

% gittrac #1886
\item The \Condor{schedd} is now more efficient in handling the exit of
  \Condor{shadow} processes, when there are large numbers of 
  \Condor{shadow} processes.

%gittrac #1628
\item Condor's Chirp protocol has been updated with new commands.
 The Chirp C++ client
 and \Condor{chirp} command are updated to use the new commands.
  See section ~\ref{man-condor-chirp} for details on the new commands.

\item The Daemon ClassAd Hooks mechanism is described in
section~\ref{sec:daemon-classad-hooks},
with configuration variables defined in section~\ref{sec:Config-hooks}.
The mechanism has the following new features:
  \begin{itemize}
    %gittrac #1086
    \item The \Condor{startd}'s benchmarks are no longer hard coded into
    the \Condor{startd}.  Instead, the benchmarks are now implemented
    via the Daemon ClassAd Hooks mechanism.  Two new programs are
    shipped with Condor version 7.5.6:
    \Condor{mips} and \Condor{kflops}.
    These programs are in  the \File{libexec} directory). 
    They implement the original mips and kflops benchmarks for this 
    new implementation.
    Additional benchmarks can now easily be implemented;
    the list of benchmarks is controlled
    via the new \Macro{BENCHMARKS\_JOBLIST} configuration variable.

  \item Several fixes to the the mips and kflops benchmarks should
    increase the reproducibility of their results.

  %gittrac #1837
  \item Two new job types have been implemented in the Daemon ClassAd
    Hooks mechanism.  They are called \Expr{OneShot} and \Expr{OnDemand}.
    Currently, \Expr{OnDemand} is used only by the new \Expr{BENCHMARKS} 
    mechanism.
  \end{itemize}

\item \Condor{dagman} now prints  out all boolean configuration
variable values as \Expr{True} or \Expr{False},
instead of 1 or 0 within the \File{dagman.out} file.

% gittrac #434
\item Because of the new \Macro{DAGMAN\_VERBOSITY} configuration setting
(see section~\ref{sec:DAGMan-Config-File-Entries}),
the \Opt{-debug} flag is no longer propagated from a top-level DAG to a
sub-DAG; furthermore, \Opt{-debug} is no longer set in a
\File{.condor.sub} file unless it is set on the \Condor{submit\_dag}
command line.

% gittrac #1790
\item When job ClassAd attributes are modified via \Condor{qedit}, 
the changes are now propagated to the \Condor{shadow} and \Condor{gridmanager}.
This allows a user's changes to the job ClassAd to affect the job policy 
expressions while the job is managed by these daemons.

\item Several improvements for CREAM grid jobs:
  \begin{itemize}
  % gittrac #1936
  \item CREAM commands are retried if the server closes the connection
    prematurely.
  % gittrac #940
  \item All jobs going to a CREAM server share the same lease handle.
  % gittrac #1931
  \item Multiple CREAM status requests for single jobs are now batched
    into a single command to the server.
  % gittrac #958
  \item When there are too many commands to be issued to a CREAM server
    simultaneously, new job submissions have lower priority than commands
    operating on existing jobs.
  \end{itemize}

% gittrac #1664  needs documentation?
\item The new script \Condor{gather\_info}, located in \File{bin/},
  creates reports with 
  information from a Condor pool about a specific job ID.
  It also gathers some understanding of the pool under which it runs.

%gittrac #1393
\item Added support for hierarchical accounting groups and group quotas.

%gittrac #1670
\item \Condor{q} -better-analyze now identifies jobs that have not yet been 
  considered by matchmaking, instead of characterizing them as not 
  matching \emph{for unknown reasons}.

% gittrac #1661
\item The default Unix Sys-V init script has been completely reworked.
  The new version should now work on all Unix and Linux systems.
  Major features and changes in the new script:
  \begin{itemize}
    \item Supports the use of a Linux-style \File{sysconfig} file
    \item Supports the use of a Linux-style PID file
    \item Supports the following commands:
      \begin{itemize}
         \item start
         \item stop
         \item restart
         \item try-restart
         \item reload
         \item force-reload
         \item status
      \end{itemize}
    \item The default location of the Condor configuration file is now
     \File{/etc/condor/condor\_config}.  This location can be changed by
      editing the \File{sysconfig} file or the init script itself.
    \item The default location of the Condor installation is now
      \File{/usr/}, with binaries in \File{/usr/bin} and \File{/usr/sbin}.
      These locations can be changed by editing the \File{sysconfig} file
      or the init script itself.
  \end{itemize}

\end{itemize}

\noindent Configuration Variable and ClassAd Attribute Additions and Changes:

\begin{itemize}

% gittrac #1935
\item The default value of configuration variable \Macro{GLITE\_LOCATION}
  has changed to \verb|$(LIBEXEC)/glite|. This reflects the change made in the
  layout of the Condor release files.

% gittrac 1800
\item Values for configuration variables \Macro{NETWORK\_INTERFACE} and
  \Macro{PRIVATE\_NETWORK\_INTERFACE} may now specify a network
  device name or an IP address.  The asterisk character (\verb|*|)
  may be used as a wild card in either a name or IP address.
  This makes it easier to apply the same
  configuration to a large number of machines, because the IP address
  does not have to be customized for each host.

% gittrac #1812
\item The new configuration variable
  \Macro{DELEGATE\_JOB\_GSI\_CREDENTIALS\_LIFETIME} specifies the
  maximum number of seconds for which delegated job proxies should be
  valid.  The default is one day.  A value of 0 indicates that the
  delegated proxy should be valid for as long as allowed by the
  credential used to create the proxy; this was the behavior in
  previous releases of Condor.  This configuration variable currently
  only applies to proxies delegated for non-grid jobs and Condor-C
  jobs.  It does not currently apply to globus grid jobs.  The job may
  override this configuration variable by using the
  \SubmitCmd{delegate\_job\_GSI\_credentials\_lifetime} submit description file
  command.

\item The new configuration variable
  \Macro{DELEGATE\_JOB\_GSI\_CREDENTIALS\_REFRESH} specifies a
    floating point number between 0 and 1 that indicates when
    delegated credentials with limited lifetime should be renewed, as
    a fraction of the delegated lifetime.  The default is 0.25.

\item The new configuration variable
  \Macro{SHADOW\_CHECKPROXY\_INTERVAL} specifies the number of
  seconds between tests to see if the job proxy has been updated or
  should be refreshed.  The default is 600 (10 minutes).  Previously,
  the \Condor{shadow} checked for proxy updates once per minute.

% gittrac #1698
\item Daemon ClassAd Hooks no longer support what was identified as 
  the \emph{old} syntax.  
  Due to this, variables
  \Macro{STARTD\_CRON\_JOBS} and \Macro{HAWKEYE\_JOBS} no longer exist.  
  In previous versions of Condor, the \Condor{startd} would issue a
  warning if this syntax was found, but, starting with 7.5.6, any use
  of these macros will be ignored.

% gittrac #434
\item New configuration variables \Macro{DAGMAN\_VERBOSITY},
\Macro{DAGMAN\_MAX\_PRE\_SCRIPTS}, \Macro{DAGMAN\_MAX\_POST\_SCRIPTS},
and \Macro{DAGMAN\_ALLOW\_LOG\_ERROR}
are defined in section~\ref{sec:DAGMan-Config-File-Entries}.

% gittrac #1719
\item The new configuration variable
\Macro{STARTD\_PUBLISH\_WINREG} can contain a list of Windows 
registry key names, 
whose values will be published in the \Condor{startd} daemon's ClassAd.

\item The new configuration variable
\Macro{CONDOR\_VIEW\_CLASSAD\_TYPES} is a string list that specifies
the types of the ClassAds that will be forwarded to
the location defined by \MacroNI{CONDOR\_VIEW\_HOST}. 
See the definition at section~\ref{sec:Collector-Config-File-Entries}.

% gittrac  #677
\item Added a \Opt{-local-name} command line option to
  \Condor{config\_val} to inspect the values of attributes that use
  local names.


\end{itemize}

\noindent Bugs Fixed:

\begin{itemize}

% gittrac #1892
\item Fixed a bug for parallel universe jobs,
  introduced in Condor version 7.5.5,
  where the \Condor{schedd} would crash under certain conditions when
  a parallel job was removed or exited.

% gittrac #1909
\item Fixed a memory leak in the \Condor{quill} daemon.

% gittrac #1935
\item Fixed a problem in Condor version 7.5.5 release,
  in which binaries used for the grid universe's pbs and lsf grid types 
  were not marked as executable.

% gittrac #1900
\item Fixed a bug introduced in Condor version 7.5.5
that caused running \SubmitCmd{vanilla}, \SubmitCmd{java},
and \SubmitCmd{vm} universe jobs to leave the queue when held.

% gittrac 1826
\item A bug has been fixed that caused SOAP transactions in the
  \Condor{schedd} daemon to result in a log message of the form 
\begin{verbatim}
Timer <X> not found
\end{verbatim}
  This bug is not known to have produced any other undesired behaviors.

% gittrac #1869
\item The job ClassAd attribute \AdAttr{JobLeaseDuration} is now set 
appropriately when a Condor-C job is forwarded to a remote pool.
Previously, a default value was not supplied,
causing jobs to be unnecessarily killed if the
submit and execute machines temporarily lost contact with each other.

% gittrac #1586
\item Fixed a bug that caused \Condor{dagman} to sometimes falsely
report that a cycle existed in a DAG.

% gittrac #741
\item Using the \Condor{hold} command on a Windows platform job managed
by \Condor{dagman} no longer removes the node job of the DAG.
This behavior on Windows now matches the behavior on other platforms.

% gittrac #1490
\item Using the \Condor{hold} command followed by the \Condor{remove} 
command on a job managed by \Condor{dagman}
now removes node jobs of the DAG, rather than leaving them as orphans.

\item A bug has been fixed in the \Condor{config\_val} program,
  which caused it to try to contact the \Condor{collector} before printing
  usage information, if the command line was syntactically incorrect.

% gittrac #1922
\item A bug has been fixed that caused Condor daemons to crash in
  response to DNS look up failures when receiving connections.  The
  crash occurred during authorization of the new connection.  
  This problem was introduced in Condor version 7.5.4.

% gittrac #1806
\item Fixed a bug that caused \Condor{submit} to silently ignore parts
of attribute values if an equals sign was omitted.

% gittrack 1915
\item Starting in Condor version 7.5.5, the \Condor{schedd} daemon would
  sometimes generate an error message and exit abnormally when
  shutting down.  The error message contained the following text:
\begin{verbatim}
ERROR ``Assertion ERROR on (m_ref_count == 0)''
\end{verbatim}

% gittrack 1913
\item Changes to the \Condor{negotiator} daemon's address were 
  not published to the \Condor{collector} until the \Condor{negotiator} daemon
  was reconfigured or restarted.  
  This was a problem in some situations when using \Condor{shared\_port}.

% gittrack #1945
\item A bug introduced in 7.5.5 resulted in failure to advance the
  flocking level due to lack of activity from one of the negotiators
  in the flocking list.

%gittrac #1875
\item Fixed a Windows-specific problem where the main daemon loop can 
  get into a state where it is busy waiting.

%gittrac #1915
\item Fixed \Condor{schedd} exception on shutdown caused by bad reference count.

%gittrac #677
\item Releases of Condor with versions from 7.5.0 to 7.5.5 incorrectly
  implemented the macro language used for configuration with variables
  having \Expr{LOCAL.} at the prefix. This was a regression
  from the Condor 7.4 series. It is now fixed and the functionality
  has been restored.

\end{itemize}

\noindent Known Bugs:

\begin{itemize}

% gittrac #1910
\item If a cycle exists in the set of jobs to be removed defined by 
the job ClassAd attribute \Attr{OtherJobRemoveRequirements},
removing any of the jobs in the set will cause the
\Condor{schedd} to go into an infinite loop.
\Attr{OtherJobRemoveRequirements} is defined on
page ~\pageref{attribute-OtherJobRemoveRequirements}.

% gittrac #1751
\item In a \Condor{dagman} workflow, if a splice contains nothing but
another splice, parsing the DAG will fail.  This can be worked around
by putting any non-splice job, including a DAG-level NOOP job,
into the offending splice.  
This bug has apparently existed since the splice
feature was introduced in \Condor{dagman}.

% gittrac #1947
\item If an individual Daemon ClassAd Hook manager is not \emph{named},
  the jobs under it will attempt to use incorrectly named configuration
  variables.
  For example, the following correct configuration will \emph{not} work,
  because the Daemon ClassAd Hook manager will fail to look up the job's
  executable variable, given the error in configuration variable naming:

\begin{verbatim}
STARTD_CRON_JOBLIST = TEST
...
STARTD_CRON_TEST_MODE       = periodic
STARTD_CRON_TEST_EXECUTABLE = $(LIBEXEC)/test
...
\end{verbatim}

Condor version 7.5.6 and all previous 7.x Condor versions will incorrectly 
name the variables from this example \MacroNI{STARTD\_TEST\_MODE} and
\MacroNI{STARTD\_TEST\_EXECUTABLE} instead.
If instead, the Daemon ClassAd Hook Manager had been named, 
using the no-longer-supported \MacroNI{STARTD\_CRON\_NAME},
the code works as expected.  For example:

\begin{verbatim}
STARTD_CRON_NAME = HAWKEYE
HAWKEYE_JOBLIST  = TEST
...
HAWKEYE_TEST_MODE       = periodic
HAWKEYE_TEST_EXECUTABLE = $(LIBEXEC)/test
...
\end{verbatim}

This \emph{old} behavior is, as of Condor version 7.5.6,
documented as unsupported and is going away,
primarily because it is confusing.
But, for this release, it still works.
It is believed that this same behavior exists in all 7.x releases of Condor,
but because the naming feature is used, the incorrect behavior went undetected.

This affects the \MacroNI{STARTD\_CRON} and \MacroNI{SCHEDD\_CRON}
Daemon ClassAd Hook managers, and will be fixed in Condor version 7.6.0.

\end{itemize}

\noindent Additions and Changes to the Manual:

\begin{itemize}

\item None.

\end{itemize}


%%%%%%%%%%%%%%%%%%%%%%%%%%%%%%%%%%%%%%%%%%%%%%%%%%%%%%%%%%%%%%%%%%%%%%
\subsection*{\label{sec:New-7-5-5}Version 7.5.5}
%%%%%%%%%%%%%%%%%%%%%%%%%%%%%%%%%%%%%%%%%%%%%%%%%%%%%%%%%%%%%%%%%%%%%%

\noindent Release Notes:

\begin{itemize}

\item Condor version 7.5.5 released on January 26, 2011.

\item This version of Condor uses a different layout in the spool
  directory for storing files belonging to jobs that are in the queue.
  Conversion of the spool directory is automatic when upgrading, but
  be aware that \emph{downgrading to a previous version of Condor
  requires extra effort}.  The procedure for downgrading is either
  to drain all jobs with spooled files from the queue, or to manually
  convert the spool back to the older format.  To manually convert
  back to the older format, stop Condor and back up the spool directory
  in case of problems.  Then move all subdirectories matching the form
  \verb|$(SPOOL)/<#>/<#>/cluster<#>.proc<#>.subproc<#>| into
  \verb|$(SPOOL)|.  Also do this for any files of the form
  \verb|$(SPOOL)/<#>/cluster<#>.ickpt.subproc<#>|.  Edit
  \verb|$(SPOOL)/job_queue.log| with a text editor, and change all
  references to the old paths to the new paths.  Then, remove
  \verb|$(SPOOL)/spool_version|.  Finally, start up Condor.

\item For those who compile Condor from the source code rather than
  using packages of pre-built executables, be aware that in this
  release Condor is built using \Prog{cmake} instead of \Prog{imake}.
  See the \File{README.building} file for the new instructions on how
  to build Condor.

\item This release note serves to remind users that as of Condor version 7.5.1,
  the RPMs come with native packaging.  
  Therefore, items are in different locations, as given by FHS locations,
  such as \File{/usr/bin}, \File{/usr/sbin}, \File{/etc}, and \File{/var/log}.  
  Please see section~\ref{sec:install-rpms} for installation documentation.

\item Quill is now available only within the source code distribution 
  of Condor.
  It is no longer included in the builds of Condor provided by UW,
  but it is available as a feature that can be enabled by those who compile
  Condor from the source code.
  Find the code within the \File{condor\_contrib} directory, in the
  directories \File{condor\_tt} and \File{condor\_dbmsd}. 

%\item Windows packages are not included in this release.  We expect to
%  release Windows packages in Condor version 7.5.6.

\item The AIX 5.2 packages in this release have been found to be
  incompatible with AIX 5.3.

\item We are planning to drop support for AIX.  Please contact us if
  this is a problem for you.

\item The directory structure within the Unix tar file package of Condor
  has changed.  Previously, the tar file contained a top level
  directory named \File{condor-<\emph{version}>}.  The top level
  directory is now the same as the tar file name, but without the
  \File{.tar.gz} extension.

\item On Unix platforms, the following executables used to be located in both
  the \File{sbin} and \File{bin} directories,
  but are now only located in the \File{bin}
  directory: \Prog{condor}, \Condor{checkpoint}, \Condor{reschedule}, and
  \Condor{vacate}.

\item The size of the Condor installation has increased by as much as
  60\% compared to Condor version 7.5.4.  We hope to eliminate most of this
  increase in Condor version 7.5.6.

\item Previously, packages containing debug symbols were available for
  many Unix platforms.  In this release, the debug packages contain
  full, `unstripped' executables instead of just the debug symbols.

\item The contents of the Windows zip and MSI packages of Condor have
  changed.  The \File{lib} and \File{libexec} folders no longer exist,
  and all contents previously within them are now in \File{bin}.
  \Condor{setup} and \Condor{set\_acls} have been moved from the top
  level directory into \File{bin}.

\item The Windows MSI installer for Condor version 7.5.5 requires that 
  all previous
  MSI installations of Condor be uninstalled.  Before uninstalling
  previous versions, make backup copies of configuration files.  
  Any settings that
  need to be preserved must be reapplied to the configuration of the
  new installation.

\item The following list itemizes changes included in this Condor version
  7.5.5 release that belong to Condor version 7.4.5.  That stable series
  version will not yet have been released as this development version 
  is released.
  \begin{itemize}

  % gittrac #1713
  % gittrac #1715
  \item \Condor{dagman} now prints a message in the \File{dagman.out} file
  whenever it truncates a node job user log file.
  \Condor{dagman} now prints additional diagnostic information in the
  case of certain log file errors.

  % gittrac #1750
  \item Fixed a bug in which
  a network disconnect between the submit machine and execute
  machine during the transfer of output files caused the
  \Condor{starter} daemon to immediately give up, rather than waiting
  for the \Condor{shadow} to reconnect.  This problem was introduced
  in Condor version 7.4.4.

  % gittrac #1743
  \item Fixed a bug in which
  if \Condor{ssh\_to\_job} attempted to connect to a job while the
  job's input files were being transferred, this caused the file
  transfer to fail, which resulted in the job returning to the idle
  state in the queue.

  % gittrac #1785
  \item In privsep mode, the transfer of output failed if a job's execute
  directory contained symbolic links to non-existent paths.

  \end{itemize}
\end{itemize}


\noindent New Features:

\begin{itemize}

% gittrac 1069
\item Negotiation is now handled asynchronously in the \Condor{schedd} daemon.
This means that the \Condor{schedd} remains responsive during 
negotiation and is less prone to falling behind on communication 
with \Condor{shadow} processes.

% gittrac 1707
\item Improved monitoring and avoidance of a \Term{lock convoy} problem
observed when there were more than 30,000 \Condor{shadow} processes.
At this scale,
locking the \Condor{shadow} daemon's log on each write to the log file
has been observed
on Linux platforms to sometimes result in a situation where the system does
very little productive work, and is instead consumed by rapid context
switching between the \Condor{shadow} daemons that are waiting for the lock.

% gittrac 1706
\item On Linux platforms, if the \Condor{schedd} daemon's spool directory is
  on an ext3 file system, Condor can now scale to a larger number
  of spooled jobs.  Previously, Condor created two subdirectories
  within the spool directory for each spooled job and for each running
  job.  The ext3 file system only supports 31,997 subdirectories.  This
  effectively limited the number of spooled jobs to less than 16,000.
  Now, Condor creates a hierarchy of subdirectories within
  the spool directory, to increase the limit on the number of spooled jobs
  in ext3 to 320,000,000, which is likely to be larger than other limits
  on the size of the job queue, such as memory.

%gittrac 793
\item The \Condor{shadow} daemon uses less memory than it has since 
Condor version 7.5.0.
Memory usage should now be similar to the 7.4 series.

%gittrac 1752
\item The \Condor{dagman} and \Condor{submit\_dag} command-line flag
\Opt{-DumpRescue} causes the dump of an incomplete Rescue DAG,
when the parsing of the DAG input file fails.
This may help in figuring out what went wrong.
See section~\ref{sec:DAGMan-rescue} for complete details on Rescue DAGs.

%gittrac #1077
\item \Condor{dagman} now has the capability to create the
\File{jobstate.log} file needed for the Pegasus workflow manager.
See section~\ref{sec:DAGJobstateLog} for details.

%gittrac 1745
\item \Condor{wait} can now work on jobs with logs that are only
  readable by the user running \Condor{wait}.  Previously, write
  access to the job's user log was required.

%gittrac #1641
\item Added a new value for the job ClassAd attribute \Attr{JobStatus}.
The \Expr{TRANSFERRING\_OUTPUT} status is used
when transferring a job's output files after the job has finished running.
Jobs with this status will have their \Attr{JobStatus} attribute set to 6.
The standard \Condor{q} display will show the job's status as \Expr{>}.

\end{itemize}

\noindent Configuration Variable and ClassAd Attribute Additions and Changes:

\begin{itemize}

% gittrac 1707
\item The new configuration variable \Macro{LOCK\_DEBUG\_LOG\_TO\_APPEND}
controls whether a daemon's debug lock is used when appending to the log.
When the default value of \Expr{False},
the debug lock is only used when rotating the log file.
When \Expr{True}, the debug lock is used when writing to
the log as well as when rotating the log file.
See section~\ref{param:LockDebugLogToAppend} for the complete definition.

%gittrac 1552
\item The new configuration variable
  \Macro{LOCAL\_CONFIG\_DIR\_EXCLUDE\_REGEXP} may be set to a regular
  expression that specifies file names to ignore when looking for
  configuration files within the directories specified via
  \MacroNI{LOCAL\_CONFIG\_DIR}.  
  See section~\ref{param:LocalConfigDirExcludeRegexp} for the 
  complete definition.

\end{itemize}

\noindent Bugs Fixed:

\begin{itemize}

\item In previous versions of Condor, the \Condor{starter} could not 
write the \File{.machine.ad} and \File{.job.ad} files to the \File{execute}
directory when PrivSep was enabled.  This has now been fixed, and these files
are correctly emitted in all cases.

% gittrac 1773
\item Since Condor version 7.5.2, the speed of \Condor{q} was not as high
  as earlier 7.5 and 7.4 releases,
  especially when retrieving large numbers of jobs.
  Viewing 100K jobs took about four times longer.
  This release fixes the performance,
  making it about the same as before Condor version 7.5.2.

% gittrac #1737
\item A bug introduced in Condor version 7.5.4 prevented parallel 
universe jobs with multiple \SubmitCmd{queue} statements in 
the submit description file from working with \Condor{dagman}.
This is now fixed.

% gittrac #1681
\item Improved the way Condor daemons send heartbeat messages to their parent
process.  This resolves a problem observed on busy submit machines using the
\Condor{shared\_port} daemon.  The \Condor{master} daemon sometimes incorrectly
determined that the \Condor{schedd} was hung, and would kill and restart it.

% gittrac #1688
\item When the configuration variable \Macro{NOT\_RESPONDING\_WANT\_CORE}
is \Expr{True},
the \Condor{master} daemon now follows up with a \Expr{SIGKILL},
if the child process does not exit within ten minutes of receiving
\Expr{SIGABRT}.
This addresses observed cases in
which the child process hangs while writing a core file.

% gittrac #1720
\item Host name-based authorization failed in Condor version 7.5.4
under Mac OS X 10.4,
because look ups of the host name for incoming connections failed.

% gittrac #1724
\item A bug introduced in Condor version 7.5.0 caused
the attributes \AdAttr{MyType} and \AdAttr{TargetType}
in offline ClassAds to get set to \Expr{"(unknown type)"}
when the offline ClassAd was matched to a job.

% gittrac #1715
\item \Condor{dagman} now excepts in the case of certain log file errors,
when continuing would be likely to put \Condor{dagman} into an incorrect 
internal state.

% gittrac #1762
\item Fixed a bug that caused DAG node jobs to have their coredumpsize
limit set according to the \MacroNI{CREATE\_CORE\_FILES} configuration
variable, rather than the user's coredumpsize limit.

% gittrac #1777
\item Fixed a case introduced in Condor version 7.5.4 on Windows platforms,
in which the following spurious log message was produced:
\begin{verbatim}
SharedPortEndpoint: Destructor: Problem in thread shutdown notification: 0
\end{verbatim}

% gittrac #1088
\item Since Condor version 7.4.1,
Condor-C jobs submitted without file transfer enabled could
fail with the following error in the \Condor{starter} log:
\begin{verbatim}
FileTransfer: DownloadFiles called on server side
\end{verbatim}

% gittrac #1796
\item Fixed a memory leak caused by use of the ClassAd \Expr{eval()}
  function.  This problem was introduced in Condor version 7.5.2.

% gittrac #1804
\item Fixed a bug that could cause the \Condor{negotiator} daemon to
  crash when groups are configured with
  \MacroNI{GROUP\_QUOTA\_DYNAMIC\_<group\_name>}, or when
  \MacroNI{GROUP\_QUOTA\_<group\_name>} is not defined to be something
  greater than 0.

% gittrac #1809
\item Fixed a bug that caused random characters to appear for the
  value of \Expr{AuthMethods} when logging with \Expr{D\_FULLDEBUG}
  and \Expr{D\_SECURITY} enabled.
  This problem was introduced in Condor version 7.5.3.

% gittrac #1849
\item Fixed a memory leak in the  \Condor{schedd} 
introduced in Condor version 7.5.4.

% gittrac #1866
\item Fixed a problem introduced in Condor version 7.5.4 that could cause the
  \Condor{schedd} daemon to enter an infinite loop while in the
  process of shutting down.  For the problem to happen, it was
  necessary for flocking to have been enabled.

% gittrac #1828
\item Configuration variable \MacroNI{SCHEDD\_QUERY\_WORKERS} was effectively 
  ignored when \Condor{q} authenticated itself to the \Condor{schedd}.
  The query was always
  processed in the main \Condor{schedd} process rather than in a sub-process.
  This problem has existed since before Condor version 7.0.0.

% gittrac #1844
\item Fixed a problem affecting jobs that store their output in the
  \Condor{schedd}'s spool directory.  The problem affected jobs that
  include an empty directory in their list of output files to
  transfer.  This problem was introduced in Condor version 7.5.4,
  when support for the transfer of directories was added.

% gittrac #1749
\item Fixed a problem affecting the \Condor{master} daemon since
  Condor version 7.5.3.  
  The \Condor{master} daemon would crash if it was instructed
  to shut down a daemon that was not currently running,
  but which was waiting to be restarted.

% gittrac #1650
\item Fixed a bug in \Condor{submit} that prevented the submission of
multiple \SubmitCmd{vm} universe jobs in a single submit file.

% gittrac #1761
\item Fixed a bug in the \Condor{schedd} that could cause it to temporarily
under count the number of running local and scheduler universe jobs. 
In Condor version 7.5.4, 
this under counting could cause the daemon to crash.

% gittrac #1695
\item Fixed a bug that could cause the \Condor{gridmanager} to crash if
a GAHP server did not behave as expected on start up.

% gittrac #1747
\item Improved the hold reason reported in several failure cases for 
CREAM grid jobs.

% gittrac #1771
\item The \Attr{KFlops} attribute reported by 
\begin{verbatim}
  condor_status -run -total 
\end{verbatim}
could erroneously be reported as negative.  This has been fixed.

% gittrac #1867
\item Since Condor version 7.5.4, the refreshing of the proxy for the job in the
  remote queue did not work in Condor-C.  Therefore, if the original job proxy
  expired, the job was halted and put on hold, even if the proxy had
  been renewed on the submit machine.

\end{itemize}

\noindent Known Bugs:

\begin{itemize}

% gittrac #1900
\item In Condor version 7.5.5, when a running job is put on hold, the job
  is removed from the job queue.

\end{itemize}

\noindent Additions and Changes to the Manual:

\begin{itemize}

\item None.

\end{itemize}



%%%%%%%%%%%%%%%%%%%%%%%%%%%%%%%%%%%%%%%%%%%%%%%%%%%%%%%%%%%%%%%%%%%%%%
\subsection*{\label{sec:New-7-5-4}Version 7.5.4}
%%%%%%%%%%%%%%%%%%%%%%%%%%%%%%%%%%%%%%%%%%%%%%%%%%%%%%%%%%%%%%%%%%%%%%

\noindent Release Notes:

\begin{itemize}

\item Condor version 7.5.4 released on October 20, 2010.

\item All of the bug fixes and features which are in
Condor version 7.4.4 are in this 7.5.4 release.

% gittrac #1539
\item The release now contains all header files necessary to compile
code that uses the job log reading and writing utilities contained
in libcondorapi. Some headers were missing starting in Condor 7.5.1.

\end{itemize}


\noindent New Features:

\begin{itemize}

% gittrac #1447
\item Concurrency limits now work with parallel universe jobs
scheduled by the dedicated scheduler.

% gittrac #1522
\item Transfer of directories is now supported by
  \SubmitCmd{transfer\_input\_files} and
  \SubmitCmd{transfer\_output\_files} for non-grid universes and
  Condor-C.  The auto-selection of output files, however, remains the
  same: new directories in the job's output sandbox are \emph{not}
  automatically selected as outputs to be transferred.

% gittrac #1520
\item Paths other than simple file names with no directory information
  in \SubmitCmd{transfer\_output\_files} previously did not have well
  defined behavior.  Now, paths are supported for non-grid universes
  and Condor-C.  When a path to an output file or directory is
  specified, this specifies the path to the file on the execute side.
  On the submit side, the file is placed in the job's initial working
  directory and it is named using the base name of the original path.
  For example, \File{path/to/output\_file} becomes \File{output\_file}
  in the job's initial working directory.  The name and path of the
  file that is written on the submit side may be modified by using
  \SubmitCmd{transfer\_output\_remaps}.

% gittrac #991
\item The \Condor{shared\_port} daemon is now supported on Windows platforms.

% gittrac #1257
\item Jobs can now by submitted to multiple EC2 servers via the amazon
grid type. The server's URL must be specified via the \SubmitCmd{grid\_resource}
submit description file command for each job.
See section~\ref{sec:Amazon} for details.

% gittrac #1545
\item The grid universe's amazon grid type can now be used to submit
virtual machine jobs to Eucalyptus systems via the EC2 interface.

%gittrac #1179
\item \Condor{q} now uses the queue-management API's projection feature when 
  used with \Opt{-run}, \Opt{-hold}, \Opt{-goodput}, \Opt{-cputime}, 
  \Opt{-currentrun}, and \Opt{-io} options when called with no display options
  or with \Opt{-format}. 

% gittrac #1460
\item Decreased the CPU utilization of \Condor{dagman} when it is
	submitting ready jobs into Condor.

%gittrac #1479
\item \Condor{dagman} now logs the number of queued jobs in the DAG
that are on hold,
as part of the DAG status message in the \File{dagman.out} file.

%gittrac #825
\item \Condor{dagman} now logs a note in the \File{dagman.out} file
when the \Condor{submit\_dag} and \Condor{dagman} versions differ,
even if the difference is permissible.

%gittrac #1483
\item Added the capability for \Condor{dagman} to create and periodically
rewrite a file that lists the status of all nodes within a DAG.
Alternatively, the file may be continually updated as the DAG runs.
See section~\ref{sec:DAG-node-status} for details.

%gittrac #1560
\item The \Condor{schedd} daemon now uses a better algorithm for
determining which flocking level is being negotiated.  No special
configuration is required for the new algorithm to work.  In the
past, the algorithm depended on DNS and the
configuration variables \MacroNI{NEGOTIATOR\_HOST} and
\MacroNI{FLOCK\_NEGOTIATOR\_HOSTS}.  In some networking environments,
such as that of a multi-homed central manager, it was difficult to
configure things correctly.  When wrongly configured, negotiation
would be aborted with the message, \Expr{Unknown negotiator}.  The new
algorithm is only used when the \Condor{negotiator} is version 7.5.4 or
newer.  Of course, the \Condor{schedd} daemon must still be configured to
authorize the \Condor{negotiator} daemon at the \DCPerm{NEGOTIATOR}
authorization level.

% gittrack #1496
\item \Condor{advertise} has a new option, \Opt{-multiple}, which
allows multiple ClassAds to be published.  This is more efficient than
publishing each ClassAd in a separate invocation of \Condor{advertise}.

% gittrack #1647
\item The \Condor{job\_router} is no longer restricted to routing only vanilla
universe jobs.  It also now automatically avoids recursively routing jobs.

% gittrac #1441
\item The \Condor{schedd} now writes the submit event to the user job log.
Previously, \Condor{submit} wrote the event.

% gittrac #1665
\item The \Condor{schedd} daemon now scales better when there are many
job auto clusters.

% gittrac #1173
\item The \Condor{q} command with option \Opt{-run}, \Opt{-hold}, 
\Opt{-goodput}, \Opt{-cputime}, \Opt{-currentrun} or \Opt{-io}
is now much more efficient in its communication with the \Condor{schedd}.

\end{itemize}

\noindent Configuration Variable and ClassAd Attribute Additions and Changes:

\begin{itemize}

% gittrac #1545
\item The new configuration variable \Macro{SOAP\_SSL\_SKIP\_HOST\_CHECK}
can be used to disable the standard check that a SOAP server's host name
matches the host name in its X.509 certificate. This is useful when submitting
grid type amazon jobs to Eucalyptus servers, which often have certificates
with a host name of \Expr{localhost}.

% gittrac #61
\item Added default values for \MacroNI{<SUBSYS>\_LOG} configuration variables.
  If a \MacroNI{<SUBSYS>\_LOG} configuration variable is not set in 
  files \File{condor\_config} or \File{condor\_config.local},
  it will default to \File{\$(LOG)/<SUBSYS>LOG}.

%gittrac #1385
\item The new job ClassAd attribute \AdAttr{CommittedSuspensionTime}
is a running total of the number of seconds the job has spent in
suspension during time in which the job was not evicted without a
checkpoint.  This complements the existing attribute
\AdAttr{CumulativeSuspensionTime}, which includes all time spent in
suspension, regardless of job eviction.

%gittrack #1385
\item The new job ClassAd attributes \AdAttr{CommittedSlotTime} and
\AdAttr{CumulativeSlotTime} are just like \AdAttr{CommittedTime} and
\AdAttr{RemoteWallClockTime} respectively, except the new attributes
are weighted by the \AdAttr{SlotWeight} of the machine(s) that ran
the job.

%gittrack #1385
\item The new configuration variable
\Macro{SYSTEM\_JOB\_MACHINE\_ATTRS} specifies a list of machine
attributes that should be recorded in the job ClassAd.  The default
attributes are \Attr{Cpus} and \Attr{SlotWeight}.  When there are
multiple run attempts, history of machine attributes from previous
run attempts may be kept.  The number of run attempts to store is
specified by the new configuration variable
\Macro{SYSTEM\_JOB\_MACHINE\_ATTRS\_HISTORY\_LENGTH}, which defaults
to 1.  A machine attribute named \Attr{X} will be inserted into the
job ClassAd as an attribute named \Attr{MachineAttrX0}.  The previous
value of this attribute will be named \Attr{MachineAttrX1}, the
previous to that will be named \Attr{MachineAttrX2}, and so on, up to
the specified history length.  Additional attributes to record may be
specified on a per-job basis by using the new \SubmitCmd{job\_machine\_attrs}
submit file command.  The history length may also be extended on a
per-job basis by using the new submit file command
\SubmitCmd{job\_machine\_attrs\_history\_length}.

% gittrac 1458
\item The new configuration variable
  \Macro{NEGOTIATION\_CYCLE\_STATS\_LENGTH} specifies how many
  recent negotiation cycles should be included in the history that is
  published in the \Condor{negotiator}'s ClassAd.  The default is 3.  See
  page~\pageref{param:NegotiationCycleStatsLength} for the
  definition of this configuration variable, and see
  page~\pageref{attr:LastNegotiationCycleActiveSubmitterCount<X>} for a
  list of attributes that are published.

%gittrac #1560
\item The configuration variable \Macro{FLOCK\_NEGOTIATOR\_HOSTS} is now
optional.  Previously, the \Condor{schedd} daemon refused to flock without
this setting.  When this is not set, the addresses of the flocked
\Condor{negotiator} daemons are found by querying the flocked 
\Condor{collector} daemons.
Of course, the \Condor{schedd} daemon must still be configured to
authorize the \Condor{negotiator} daemon at the \DCPerm{NEGOTIATOR}
authorization level.  Therefore, when using host-based security,
\MacroNI{FLOCK\_NEGOTIATOR\_HOSTS} may still be useful as a macro for inserting
the negotiator hosts into the relevant authorization lists.

%gittrack #1312
\item The configuration variable \MacroNI{FLOCK\_HOSTS} is no longer used.
For backward compatibility, this setting used to be treated as a default
for \MacroNI{FLOCK\_COLLECTOR\_HOSTS} and \MacroNI{FLOCK\_NEGOTIATOR\_HOSTS}.

% gittrac #1257
\item The configuration variable \MacroNI{AMAZON\_EC2\_URL} is now only used
for previously-submitted jobs when upgrading Condor to version 7.5.4 or
beyond. New grid type amazon jobs must specify which EC2 service to use
by setting the \SubmitCmd{grid\_resource} submit description file command.

%gittrack #121
\item The new job ClassAd attribute \AdAttr{NumPids} is the total number of 
 child processes a running job has.

%gittrac #1480
\item The new configuration variable \MacroNI{DAGMAN\_MAX\_JOB\_HOLDS}
specifies the maximum number of times a DAG node job is allowed to go
on hold.  See section~\ref{param:DAGManMaxJobHolds} for details.

% gittrac #1652
\item The configuration variable \Macro{STARTD\_SENDS\_ALIVES} now only
needs to be set for the \Condor{schedd} daemon. Also, the default value has
changed to \Expr{True}.

% gittrac #1593
\item The job ClassAd attributes \SubmitCmd{amazon\_user\_data} and
\SubmitCmd{amazon\_user\_data\_file} can now both be used for the same
job. When both are provided, the two blocks of data are concatenated,
with the value of the one specified by \SubmitCmd{amazon\_user\_data}
occurring first.

% gittrac #1653
\item The new configuration variable \Macro{GRAM\_VERSION\_DETECTION}
can be used to disable Condor's attempts to distinguish between \Expr{gt2}
(GRAM2) and \Expr{gt5} (GRAM5) servers.
The default value is \Expr{True}.
If set to \Expr{False}, Condor trusts the \Expr{gt2} or \Expr{gt5} value
provided in the job's \SubmitCmd{grid\_resource} attribute.

% gittrac #1390
\item The new job ClassAd attribute \AdAttr{ResidentSetSize} is an integer
measuring the amount of physical memory in use by the job on the execute
machine in kilobytes.

% gittrac #1502
\item The new job ClassAd attribute \AdAttr{X509UserProxyExpiration} is an
integer representing when the job's X.509 proxy credential will expire,
measured in the
number of seconds since the epoch (00:00:00 UTC, Jan 1, 1970).

% gittrac #1315
\item The new configuration variable \Macro{SCHEDD\_CLUSTER\_MAXIMUM\_VALUE}
is an upper bound on assigned job cluster ids. If set to
value $M$, the maximum job cluster id assigned to any job will be $M-1$.
When the maximum id is reached, assignment of cluster ids will wrap around 
back to \MacroNI{SCHEDD\_CLUSTER\_INITIAL\_VALUE}. The default value is zero,
which does not set a maximum cluster id. 

% gittrac #1348
% gittrac #1487
\item The default value of configuration variable 
\MacroNI{MAX\_ACCEPTS\_PER\_CYCLE} has been changed from 1 to 4.

%gittrac #1310
\item The configuration variable \Macro{NEW\_LOCKING}, introduced in
  Condor version 7.5.2, has been changed to
  \Macro{CREATE\_LOCKS\_ON\_LOCAL\_DISK} and now defaults to \Expr{True}.

\end{itemize}

\noindent Bugs Fixed:

\begin{itemize}

% gittrac #1766
\item Fixed a bug that occurred with x64 flavors of the Windows operating system. 
  Condor was setting the default value of \AdAttr{Arch} to \Expr{INTEL} when it 
  should have been \Expr{X86\_64}.  This was a consequence of the fact that the 
  Condor runs in the WOW64 sandbox on 64-bit Windows.  This was fixed so that
  \AdAttr{Arch} would contain the value for the native architecture rather than 
  the WOW64 sandbox architecture.

% gittrac #1500
\item Fixed a bug in the user privilege switching code in Windows that 
  caused the \Condor{shadow} daemon to except when the \Condor{schedd} 
  daemon attempted to re-use it. 

% gittrack #1667
\item Fixed the output in the \Condor{master} daemon log file to be
  clearer when an authorized user tries to use \Condor{config\_val}
  \Opt{-set} and \Macro{ENABLE\_PERSISTENT\_CONFIG} is \Expr{False}.
  The previous
  output implied that the operation succeeded when, in fact, it did not.

% gittrac #1523
\item Since Condor version 7.5.2,
  the following \Condor{job\_router} features were
  effectively non-functional: \Attr{UseSharedX509UserProxy},
  \Attr{JobShouldBeSandboxed}, and \Attr{JobFailureTest}.

% gittrack #1561
\item The submit description file command \SubmitCmd{copy\_to\_spool}
  did not work properly in Condor version 7.5.3.
  When sending the executable to the execute machine, it was
  transferred from the original path rather than from the spooled copy
  of the file.

% gittrack #1521
\item When output files were auto-selected and spooled, Condor-C and
  \Condor{transfer\_data} would copy back both the output files and
  all other contents of the job's spool directory, which typically
  included the spooled input and the user log.  
  Now, only the output files are retrieved.
  To adjust which files are retrieved, the job
  attribute \Attr{SpooledOutputFiles} can be manipulated, but this
  typically should be managed by Condor.

% gittrac 1139
\item The \Condor{master} daemon now invalidates its ClassAd,
  as represented in the \Condor{collector} daemon, before it shuts down.

% gittrac #1563
\item Fixed a bug that caused \SubmitCmd{vm} universe jobs to not run
if the VMware \File{.vmx} file contained a space.

% gittrac #1549
\item Fixed a bug introduced in Condor version 7.5.1 that caused integers 
in ClassAd expressions that had leading zeros to be read as octal (base eight).

% gittrac #1516
\item Fixed a bug introduced in Condor version 7.5.1 that did not recognize 
a semicolon as a separator of function arguments in ClassAds.

% gittrac #1544
\item Fixed a bug that caused integers larger than $\pm2^{31}$ in a ClassAd
expression to be parsed incorrectly. Now, when these integers are
encountered, the largest 32-bit integer (with matching sign) is used.

% gittrac #1537
\item Fixed a bug that caused the \Condor{gridmanager} to exit when
receiving badly-formatted error messages from the \Prog{nordugrid\_gahp}.

% gittrac #1342
% gittrac #1644
\item Fixed a problem affecting the use of version 7.5.3 \Condor{startd} and
  \Condor{master} daemons in a pool with a \Condor{collector} from before
  version 7.5.2.  On shutdown, the \Condor{startd} and the \Condor{master}
  caused all \Condor{startd} and \Condor{master} ClassAds, respectively,
  to be removed from the \Condor{collector}.

% gittrac #1590
\item Fixed a bug that caused delegation of an X.509 RFC proxy between
two Condor processes to fail.

% gittrac #1563
\item Fixed a bug in \Condor{submit} that would cause failures if a file
name containing a space was used with the submit description file commands
\SubmitCmd{append\_files}, \SubmitCmd{jar\_files} or
\SubmitCmd{vmware\_dir}.

% gittrac #890
\item Fixed a bug that could cause the \Condor{gridmanager} to lock up if
a GAHP server it was using wrote a large amount of data to its \File{stderr}.

% gittrac #1653
% gittrac #1475
\item Fixed a bug that could cause the \Condor{gridmanager} to wrongly
conclude that a \Expr{gt2} (that is, GRAM2) server was a \Expr{gt5}
(that is, GRAM5) server.
Such a conclusion can be disastrous, as Condor's mechanisms to
prevent overloading a \Expr{gt2} server are then disabled. The new
configuration variable \Macro{GRAM\_VERSION\_DETECTION} can be used 
to disable Condor's attempts to distinguish between the two.

% gittrac #1689
\item Fixed a bug introduced in Condor version 7.5.3. 
When file transfer failed for a \SubmitCmd{grid} universe job of grid type 
cream,
Condor would write a hold event to the job log,
but not actually put the job on hold.

% gittrac #1694
\item Fixed a bug in the \Condor{gridmanager} that could cause it to crash
while handling cream grid type jobs destined for different resources.

% gittrac #1481
\item Fixed a bug that prevented the \Condor{shadow} from managing
additional jobs after its first job completed when 
\Macro{SEC\_ENABLE\_MATCH\_PASSWORD\_AUTHENTICATION} was set to \Expr{True}.

% gittrac #1533
\item The timestamps in the log defined by \Macro{PROCD\_LOG}
now print the real time.

% gittrac #1580
\item Fixed how some daemons advertise themselves to the \Condor{collector}.
Now, all daemons set the attribute \AdAttr{MyType} to indicate what
type of daemon they are.

% gittrac #1630
\item \Condor{chirp} no longer crashes on a put operation,
if the remote file name is omitted.

% gittrac #1489
% gittrac #1494
\item Fixed the packaging of Hadoop File System support in Condor. This includes
updating to HDFS 0.20.2 and making the HDFS web interface work properly.

% gittrac #1717
\item Condor no longer tries to invoke \Prog{glexec} if the job's X.509 proxy
is expired.

\end{itemize}

\noindent Known Bugs:

\begin{itemize}

% gittrac #1720
\item Using host names for host-based authentication,
such as in the definitions of configuration variables 
\MacroNI{ALLOW\_*} and \MacroNI{DENY\_*},
does not work on Mac OS X 10.4.
Later versions of the OS are not affected.
As a work around, IP addresses can be used instead of host names.

\end{itemize}

\noindent Additions and Changes to the Manual:

\begin{itemize}

\item None.

\end{itemize}


%%%%%%%%%%%%%%%%%%%%%%%%%%%%%%%%%%%%%%%%%%%%%%%%%%%%%%%%%%%%%%%%%%%%%%
\subsection*{\label{sec:New-7-5-3}Version 7.5.3}
%%%%%%%%%%%%%%%%%%%%%%%%%%%%%%%%%%%%%%%%%%%%%%%%%%%%%%%%%%%%%%%%%%%%%%

\noindent Release Notes:

\begin{itemize}

\item Condor version 7.5.3 released on June 29, 2010.

\end{itemize}


\noindent New Features:

\begin{itemize}

% gittrac 1274
\item \Condor{q} \Opt{-analyze} now notices the \Opt{-l} option, and if both
are given, then the analysis prints out the list of machines
in each analysis category.

% gittrac 1302
\item The behavior of macro expansion in the configuration file has
  changed.  Previously, most macros were effectively treated as
  undefined unless explicitly assigned a value in the configuration
  file.  Only a small number of special macros had pre-defined values
  that could be referred to via macro expansion.  Examples include
  \MacroNI{FULL\_HOSTNAME} and \MacroNI{DETECTED\_MEMORY}.  Now, most
  configuration settings that have default values can be referred to
  via macro expansion.  There are a small number of exceptions where
  the default value is too complex to represent in the current
  implementation of the configuration table.  Examples include the
  security authorization settings. All such configuration settings
  will also be reported as undefined by \Condor{config\_val} unless
  they are explicitly set in the configuration file.

% gittrac 1131
\item Unauthenticated connections are now identified as
  \verb|unauthenticated@unmapped|.  Previously, unauthenticated
  connections were not assigned a name, so some authorization policies
  that needed to distinguish between authenticated and unauthenticated
  connections were not expressible.  Connections that are
  authenticated but not mapped to a name by the mapfile used to be
  given the name \verb|auth-method@unmappeduser|, where
  \emph{auth-method} is the authentication method that was used.  Such
  connections are now given the name \verb|auth-method@unmapped|.
  Connections that match \verb|*@unmapped| are now forbidden from
  doing operations that require a user id, regardless of configuration
  settings.  Such operations include job submission, job removal, and
  any other job management commands that modify jobs.

% gittrac 1131
\item There has been a change of behavior when authentication fails.
  Previously, authentication failure always resulted in the command
  being rejected, regardless of whether the ALLOW/DENY settings
  permitted unauthenticated access or not.  This is still true if either
  the client or server specifies that authentication is required.
  However, if both sides specify that authentication is not required
  (i.e. preferred or optional), then authentication failure only results
  in the command being rejected if the ALLOW/DENY settings reject
  unauthenticated access.  This change makes it possible to have some
  commands accept unauthenticated users from some network addresses
  while only allowing authenticated users from others.

\item Improved log messages when failing to authenticate requests.  At
  least the IP address of the requester is identified in all cases.

% gittrac 1357
\item The new submit file command \SubmitCmd{job\_ad\_information\_attrs}
may be used to specify attributes from the job ad that should be saved
in the user log whenever a new event is being written.  See
page~\pageref{man-condor-submit-job-ad-information-attrs} for details.

%gittrac 1391
\item Administrative commands now support the \Opt{-constraint} option, which
  accepts a ClassAd expression.  This applies to \Condor{checkpoint},
  \Condor{off}, \Condor{on}, \Condor{reconfig}, \Condor{reschedule},
  \Condor{restart}, \Condor{set\_shutdown}, and \Condor{vacate}.

%gittrac #1351
\item File transfer plugins can be used for vm universe jobs. Notably,
  \Expr{file://} URLs can be used to allow VM image files to be pre-staged
  on the execute machine. The submit description file command 
  \SubmitCmd{vmware\_dir} is now optional.
  If it is not given, then all relevant VMware image files
  must be listed in \SubmitCmd{transfer\_input\_files}, possibly as URLs.

%gittrac #1403
\item File transfers for CREAM grid universe jobs are now initiated by
  the \Condor{gridmanager}. This removes the need for a GridFTP server
  on the client machine.

%gittrac #1403
\item Improved the parallelism of file transfers for nordugrid jobs.

%gittrac #1298
\item Removed the distinction between regular and full reconfiguration
  of Condor daemons. Now, all reconfigurations are full and require the
  WRITE authorization level. \Condor{reconfig} accepts but ignores the
  \Opt{-full} command-line option.

\item The \Prog{batch\_gahp}, used for pbs and lsf grid universe jobs, has been
updated from version 1.12.2 to 1.16.0.

\item \Condor{dagman} now prints a message to the \File{dagman.out} file
when it truncates a node job user log file.

%gittrac 1410
\item \Condor{dagman} now allows node categories to include
nodes from different splices.  See section~\ref{sec:DAG-node-category}
for details.

%gittrac 1410
\item \Condor{dagman} now allows category throttles in splices to
be overridden by higher levels in the DAG splicing structure.
See section~\ref{sec:DAG-node-category} for details.

%gittrac 1158
\item Daemon logs can now be rotated several times instead of only once 
  into a single \File{.old} file. In order to do so, the newly introduced 
  configuration variable \Macro{MAX\_NUM\_<SUBSYS>\_LOG} needs to be set 
  to a value greater than 1. The file endings will be ISO timestamps, and
  the oldest rotated file will still have the ending \File{.old}.
 

\end{itemize}

\noindent Configuration Variable and ClassAd Attribute Additions and Changes:

\begin{itemize}

\item The new configuration variable \Macro{JOB\_ROUTER\_LOCK}  specifies a
  lock file used to
  ensure that multiple instances of the \Condor{job\_router} never run
  with the same values of \MacroNI{JOB\_ROUTER\_NAME}.
  Multiple instances running
  with the same name could lead to mismanagement of routed jobs.

\item The new configuration variable \Macro{ROOSTER\_MAX\_UNHIBERNATE}
  is an integer
  specifying the maximum number of machines to wake up per cycle.
  The default value of 0 means no limit.

\item The new configuration variable \Macro{ROOSTER\_UNHIBERNATE\_RANK}
  is a ClassAd
  expression specifying which machines should be woken up first in a
  given cycle.  Higher ranked machines are woken first.
  If the number of machines to be woken up is limited by
  \MacroNI{ROOSTER\_MAX\_UNHIBERNATE}, the rank may be used for
  determining which machines are woken before reaching the limit.

% gittrac 1228
\item The new configuration variable \Macro{CLASSAD\_USER\_LIBS}
  is a list of libraries
  containing additional ClassAd functions to be used during ClassAd
  evaluation.

% gittrac 1375
\item The new configuration variable \MacroNI{SHADOW\_WORKLIFE}
  specifies the number of seconds after which the \Condor{shadow} will exit,
  when the current job finishes, instead of fetching a new job to
  manage.  Having the \Condor{shadow} continue managing jobs helps
  reduce overhead and can allow the \Condor{schedd} to achieve higher
  job completion rates.  The default is 3600, one hour.  The value 0
  causes \Condor{shadow} to exit after running a single job.

%gittrac 1158  
\item The new configuration variable \Macro{MAX\_NUM\_<SUBSYS>\_LOG} 
  will determine how often the daemon log of \MacroNI{SUBSYS} will rotate.
  The default value is 1 which leads to the old behavior of a single 
  rotation into a \File{.old} file.

\end{itemize}

\noindent Bugs Fixed:

\begin{itemize}

% gittrac 1332
\item Configuration variables with a default value of 0
  that were not defined in the configuration file
  were treated as though they were undefined by \Condor{config\_val}.
  Now Condor treats this case like any other:
  the default value is displayed.

% gittrac #1203
\item Starting in Condor version 7.5.1,
  using literals with a logical operator
  in a ClassAd expression (for example, \Expr{1 || 0}) caused the expression
  to evaluate to the value \Expr{ERROR}. The previous behavior has been
  restored: zero values are treated as \Expr{False},
  and non-zero values are treated as \Expr{True}.


% gittrac 1378
\item Starting in Condor version 7.5.0,
  the \Condor{schedd} no longer supported queue
  management commands when security negotiation was disabled,
  for example, if \Expr{SEC\_DEFAULT\_NEGOTIATION = NEVER}.

% gittrac 1395
\item Fixed a bug introduced in Condor version 7.5.1.
ClassAd string literals containing
characters with negative ASCII values were not accepted.

% gittrac #1334
\item Fixed a bug introduced in Condor version 7.5.0,
which caused Condor to not renew
job leases for CREAM grid jobs in most situations.

% gittrac #1331
\item Question marks occurring in a ClassAd string are no longer preceded
by a backslash when the ClassAd is printed.

\end{itemize}

\noindent Known Bugs:

\begin{itemize}

\item None.

\end{itemize}

\noindent Additions and Changes to the Manual:

\begin{itemize}

\item None.

\end{itemize}


%%%%%%%%%%%%%%%%%%%%%%%%%%%%%%%%%%%%%%%%%%%%%%%%%%%%%%%%%%%%%%%%%%%%%%
\subsection*{\label{sec:New-7-5-2}Version 7.5.2}
%%%%%%%%%%%%%%%%%%%%%%%%%%%%%%%%%%%%%%%%%%%%%%%%%%%%%%%%%%%%%%%%%%%%%%

\noindent Release Notes:

\begin{itemize}

\item Condor version 7.5.2 released on April 26, 2010.

% gittrac 1003
\item Condor no longer supports SuSE 8 Linux on the Itanium 64 architecture.

% gittrac #600
\item The following submit description file commands are no longer recognized.
Their functionality is replaced by the command \SubmitCmd{grid\_resource}.
\begin{description}
  \item{\SubmitCmd{grid\_type}}
  \item{\SubmitCmd{globusscheduler}}
  \item{\SubmitCmd{jobmanager\_type}}
  \item{\SubmitCmd{remote\_schedd}}
  \item{\SubmitCmd{remote\_pool}}
  \item{\SubmitCmd{unicore\_u\_site}}
  \item{\SubmitCmd{unicore\_v\_site}}
\end{description}

\end{itemize}


\noindent New Features:

\begin{itemize}

% gittrac 1231
% gittrac 1287
\item The \Condor{schedd} daemon uses less disk bandwidth when logging
updates to job ClassAds from running jobs and also when removing jobs
from the queue and handling job eviction and \Condor{shadow} exceptions.
This should improve performance in situations where
disk bandwidth is a limiting factor.
Some cases of updates to the job user log
have also been optimized to be less disk intensive.

% gittrac 1288
\item The \Condor{schedd} daemon uses less CPU when scheduling
some types of job queues.  Most likely to benefit from this improvement is
a large queue of short-running, non-local, and non-scheduler universe jobs,
with at least one idle local or scheduler universe job.

% gittrac 1266
\item The \Condor{schedd} automatically grants the \Condor{startd}
  authority to renew leases on claims and to evict claims.
  Previously, this required that the \Condor{startd} be trusted for
  general \DCPerm{DAEMON}-level command access.  Now this only
  requires \DCPerm{READ}-level command access.  The specific commands
  that the \Condor{startd} sends to the \Condor{schedd} can
  effectively only operate on the claims associated with that \Condor{startd},
  so this change does not open up these operations to access by anyone
  with \DCPerm{READ} access.  It reduces the level of trust that
  the \Condor{schedd} must have in the \Condor{startd}.

% gittrac 834
\item The \Condor{procd}'s log now rotates if logging is activated. 
  The default maximum size is 10Mbytes. To change the default,
  use the configuration variable \Macro{MAX\_PROCD\_LOG}.  

%gittrac 1310
\item For Unix systems only, 
  user job log and global job event log lock files can now optionally 
  be created in a directory on a 
  local drive by setting \MacroNI{NEW\_LOCKING} to \Expr{True}. 
  See section~\ref{param:NewLocking} for 
  the details of this configuration variable.
  
%gittrac 507
\item \Condor{dagman} and \Condor{submit\_dag} now default to lazy
creation  of the \File{.condor.sub} files for nested DAGs.
\Condor{submit\_dag} no longer creates them, and \Condor{dagman}
itself creates the files as the DAG is run.
The previous "eager" behavior can
be obtained with a combination of command-line and configuration settings.
There are several advantages to the "lazy" submit file creation:
\begin{itemize}
\item The DAG file for a nested DAG does not have to exist until that node
is ready to run, so the DAG file can be dynamically created by earlier
parts of the top-level DAG (including by the PRE script of the nested
DAG node).
\item It is now possible to have nested DAGs within splices, which is not
possible with "eager" submit file creation.
\end{itemize}

\end{itemize}

\noindent Configuration Variable and ClassAd Attribute Additions and Changes:

\begin{itemize}

%gittrac 507
\item The new configuration variable
\MacroNI{DAGMAN\_GENERATE\_SUBDAG\_SUBMITS} controls whether
\Condor{dagman} itself generates the \File{.condor.sub} files for
nested DAGs, rather than relying on \Condor{submit\_dag} "eagerly"
creating them.  See section~\ref{param:DAGManGenerateSubDagSubmits} for
more information.

%gittrac 1310
\item The new configuration variable \Macro{NEW\_LOCKING} can specify that
  job user logs and the global job event log to be written to a local drive,
  avoiding locking problems with NFS.
  See section~\ref{param:NewLocking} for 
  the details of this configuration variable.
\end{itemize}

\noindent Bugs Fixed:

\begin{itemize}

% gittrac 1300
\item The \Condor{job\_router} failed to work on SLES 9 PowerPC,
AIX 5.2 PowerPC,
and YDL 5 PowerPC due to a problem in how it detected EOF in the job queue log.

% gittrac 742
\item When jobs are removed, the \Condor{schedd} sometimes did not
  quickly reschedule a different job to run on the slot to which the
  removed job had been matched.  Instead, it would take up to
  \MacroNI{SCHEDD\_INTERVAL} seconds to do so.

% gittrac #1279
% Not documenting gittrac #1280, as it was completely covered up by
% #1279.
\item Fixed a bug introduced in Condor version 7.5.1 that caused the
\Prog{gahp\_server} to crash when
first communicating with most gt2 or gt5 GRAM servers.

\end{itemize}

\noindent Known Bugs:

\begin{itemize}

\item None.

\end{itemize}

\noindent Additions and Changes to the Manual:

\begin{itemize}

\item None.

\end{itemize}


%%%%%%%%%%%%%%%%%%%%%%%%%%%%%%%%%%%%%%%%%%%%%%%%%%%%%%%%%%%%%%%%%%%%%%
\subsection*{\label{sec:New-7-5-1}Version 7.5.1}
%%%%%%%%%%%%%%%%%%%%%%%%%%%%%%%%%%%%%%%%%%%%%%%%%%%%%%%%%%%%%%%%%%%%%%

\noindent Release Notes:

\begin{itemize}

\item Condor version 7.5.1 released on March 2, 2010.

\item Some, but not all of the bug fixes and features which are in
Condor version 7.4.2, are in this 7.5.1 release.

\item The Condor release is now available as a proper RPM or Debian
package.

\item Condor now internally uses the version of New ClassAds provided
as a stand-alone library (\URL{http://www.cs.wisc.edu/condor/classad/}).
Previously, Condor 
used an older version of ClassAds that was heavily tied to the Condor 
development libraries. This change should be transparent in the 
current development series. In the next development series (7.7.x),
Condor  will begin to use features of New ClassAds that were unavailable in 
Old ClassAds. 
Section~\ref{sec:classad-newandold} details the differences.

\item HPUX 11.00 is no longer a supported platform.

\end{itemize}


\noindent New Features:

\begin{itemize}

% gittrac #1102
\item A port number defined within \Macro{CONDOR\_VIEW\_HOST} may now use 
  a shared port.

% gittrac #1104
\item The \Condor{master} no longer pauses for 3 seconds after starting
  the \Condor{collector}.  However, if the configuration variable
  \MacroNI{COLLECTOR\_ADDRESS\_FILE} defines a file, 
  the \Condor{master} will wait for that file to be created
  before starting other daemons.

% gittrac #1144
\item In the grid universe, Condor can now automatically distinguish
between GRAM2 and GRAM5 servers, that is grid types \SubmitCmd{gt2} and
\SubmitCmd{gt5}.
Users can submit jobs using a grid type of \SubmitCmd{gt2} or \SubmitCmd{gt5}
for either type of server.

% gittrac #938
\item Grid universe jobs using the CREAM grid system now batch up
common requests into larger single requests.  This
reduces network traffic, increases the number of parallel tasks
the Condor can handle at once, and reduces the load on the remote
gatekeeper.

% gittrac #1100
\index{submit commands!cream\_attributes}
\item The new submit description file command \SubmitCmd{cream\_attributes}
sets additional attribute/value pairs for the CREAM job description
that Condor creates when submitting a grid universe job 
destined for the CREAM grid system.

% gittrac #1138
\item The \Condor{q} command with option \Opt{-analyze} is now performs
the same analysis as previously occurred with the \Opt{-better-analyze} option.
Therefore, the output of \Condor{q} with the \Opt{-analyze} option
has different output than before.
The \Opt{-better-analyze} option is still recognized and behaves the same
as before, though it may be removed from a future version.

% gittrac #1169
\item Security sessions that are not used for longer than an hour are
now removed from the security session cache to limit memory usage.

% gittrac #1169
\item The number of security sessions in the cache is now advertised in
the daemon ClassAd as \Attr{MonitorSelfSecuritySessions}.

% gittrac #1078
\item \Condor{dagman} now has the capability to run DAGs containing nodes
that are declared to be NOOPs -- for these nodes, a job is never actually
submitted.  See section~\ref{dagman:NOOP} for information.

% gittrac #1128
\index{submit commands!vm\_macaddr}
\item The submit file attribute \SubmitCmd{vm\_macaddr} can now be used to set
the MAC address for vm universe jobs that use VMware. The range of valid
MAC addresses is constrained by limits imposed by VMware.

% gittrac #1173
\item The \Condor{q} command with option \Opt{-globus}
is now much more efficient in its communication with the \Condor{schedd}.

\end{itemize}

\noindent Configuration Variable and ClassAd Attribute Additions and Changes:

\begin{itemize}

% gittrac #1242
\item The new configuration variable \MacroNI{STRICT\_CLASSAD\_EVALUATION}
controls whether new or old ClassAd expression evaluation semantics are
used. In new ClassAd semantics, an unscoped attribute reference is only
looked up in the local ad. The default is False (use old ClassAd semantics).

% gittrac #221
\item The configuration variable
\MacroNI{DELEGATE\_FULL\_JOB\_GSI\_CREDENTIALS} now applies to all proxy
delegations done between Condor daemons and tools.
The value is a boolean and defaults to \Expr{False},
which means that when doing delegation Condor will now create a limited proxy
instead of a full proxy.

\item The new configuration variable
  \MacroIndex{SEC\_DEFAULT\_SESSION\_LEASE}
  \Macro{SEC\_<access-level>\_SESSION\_LEASE} specifies the maximum
  number of seconds an unused security session will be kept in a daemon's
  session cache before being removed to save memory.  The default is 3600.
  If the server and client have different configurations, the smaller
  one will be used.

\end{itemize}

\noindent Bugs Fixed:

\begin{itemize}

% gittrac #1141
\item The default value for \Macro{SEC\_DEFAULT\_SESSION\_DURATION}
  was effectively 3600 in Condor version 7.5.0.
  This produced longer than desired
  cached sessions for short-lived tools such as \Condor{status}.
  It also produced shorter than possibly desired cached sessions for
  long-lived daemons such as \Condor{startd}.  
  The default has been restored to what it was before Condor version 7.5.0,
  with the exception of \Condor{submit},
  which has been changed from 1 hour to 60 seconds.
  For command line tools, the default is 60 seconds,
  and for daemons it is 1 day.

% gittrac #1142
\item \MacroNI{SEC\_<access-level>\_SESSION\_DURATION} previously did
  not support integer expressions, but it did not detect invalid
  input, so the use of an expression could produce unexpected results.
  Now, like other integer configuration variables,
  a constant expression can be used and input is fully validated.

% gittrac #1196
\item The configuration variable \MacroNI{LOCAL\_CONFIG\_DIR} is no longer
ignored if defined in a local configuration file.

% gittrac #767
\item Removed the incorrect issuing of the following Condor version 7.5.0 
  warning to the
  \Condor{starter}'s log, even when the outdated, and no longer used
  configuration
  variable \MacroNI{EXECUTE\_LOGIN\_IS\_DEDICATED} was not defined.

\begin{verbatim}
WARNING: EXECUTE_LOGIN_IS_DEDICATED is deprecated.
Please use DEDICATED_EXECUTE_ACCOUNT_REGEXP instead.
\end{verbatim}


\end{itemize}

\noindent Known Bugs:

\begin{itemize}

\item None.

\end{itemize}

\noindent Additions and Changes to the Manual:

\begin{itemize}

\item None.

\end{itemize}


%%%%%%%%%%%%%%%%%%%%%%%%%%%%%%%%%%%%%%%%%%%%%%%%%%%%%%%%%%%%%%%%%%%%%%
\subsection*{\label{sec:New-7-5-0}Version 7.5.0}
%%%%%%%%%%%%%%%%%%%%%%%%%%%%%%%%%%%%%%%%%%%%%%%%%%%%%%%%%%%%%%%%%%%%%%

\noindent Release Notes:

\begin{itemize}

\item All bug fixes and features which are in 7.4.1 are in this 7.5.0 release.

\end{itemize}


\noindent New Features:

\begin{itemize}

% gittrac #892
\item Added the new daemon \Condor{shared\_port} for Unix platforms 
  (except for HPUX).
  It allows Condor daemons to share a
  single network port.  This makes opening access to Condor through a
  firewall easier and safer.  It also increases the scalability of a
  submit node by decreasing port usage. See
  section~\ref{sec:Config-shared-port} for more information.

% gittrac #960
\item Improved CCB's handling of rude NAT/firewalls that silently drop
TCP connections.

% gittrac #968
\item Simplified the publication of daemon addresses.
  \Attr{PublicNetworkIpAddr} and \Attr{PrivateNetworkIpAddr} have been removed.
  \Attr{MyAddress} contains both public and private addresses.  For now,
  \Attr{<Subsys>IpAddr} contains the same information.  In a future release,
  the latter may be removed.

% gittrac #975
\item Changes to \MacroNI{TCP\_FORWARDING\_HOST},
  \MacroNI{PRIVATE\_NETWORK\_ADDRESS}, and
  \MacroNI{PRIVATE\_NETWORK\_NAME} can now be made without requiring a
  full restart.  It may take up to one \Condor{collector} update interval 
  for the changes to become visible.

% gittrac #1002
\item Network compatibility with Condor prior to 6.3.3 is no longer
  supported unless \MacroNI{SEC\_CLIENT\_NEGOTIATION} is set to
  \Expr{NEVER}.  This change removes the risk of communication errors
  causing performance problems resulting from automatic fall-back to the
  old protocol.

% gittrac #930
\item For efficiency, authentication between the \Condor{shadow} and
  \Condor{schedd} daemons is now able to be cached and reused in more
  cases.  Previously, authentication for updating job information was
  only cached if read access was configured to require authentication.

\item \Condor{config\_val} will now report the default value for
  configuration variables that are not set in the configuration files.

% gittrac #939
\item The \Condor{gridmanager} now uses a single status call to obtain
the status of all CREAM grid universe jobs from the remote server.

% gittrac #955
\item The \Condor{gridmanager} will now retry CREAM commands that time out.

% gittrac #941
\item Forwarding a renewed proxy for CREAM grid universe jobs to the
remote server is now much more efficient.

\end{itemize}

\noindent Configuration Variable and ClassAd Attribute Additions and Changes:

\begin{itemize}

% gittrac #997
\item Removed the configuration variable 
  \MacroNI{COLLECTOR\_SOCKET\_CACHE\_SIZE}.
  Configuration of this parameter used to be mandatory to enable TCP updates
  to the \Condor{collector}.  Now no special configuration of the
  \Condor{collector} is required to allow TCP updates, but it is
  important to ensure that there are sufficient file descriptors for
  efficient operation.  See section~\ref{sec:tcp-collector-update} for
  more information.

% gittrac #892
\item The new configuration variable \MacroNI{USE\_SHARED\_PORT} 
  is a boolean value that specifies
  whether a Condor process should rely on the \Condor{shared\_port} daemon for
  receiving incoming connections.  Write access to
  \Macro{DAEMON\_SOCKET\_DIR} is required for this to take effect.
  The default is \Expr{False}.  If set to \Expr{True}, \MacroNI{SHARED\_PORT}
  should be added to \MacroNI{DAEMON\_LIST}.  See
  section~\ref{sec:Config-shared-port} for more information.

% gittrac #960
\item Added the new configuration variable \MacroNI{CCB\_HEARTBEAT\_INTERVAL}.
  It is the maximum
  number of seconds of silence on a daemon's connection to the CCB server
  after which it will ping the server to verify that the connection still
  works.  
  The default value is 1200 (20 minutes).
  This feature serves to both speed
  up detection of dead connections and to generate a guaranteed minimum
  frequency of activity to attempt to prevent the connection from being
  dropped.

\end{itemize}

\noindent Bugs Fixed:

\begin{itemize}

\item Fixed problem with a ClassAd debug function,
so it now properly emits debug information for ClassAd \Code{IfThenElse}
clauses.

\end{itemize}

\noindent Known Bugs:

\begin{itemize}

\item None.

\end{itemize}

\noindent Additions and Changes to the Manual:

\begin{itemize}

\item None.

\end{itemize}

%%%%      PLEASE RUN A SPELL CHECKER BEFORE COMMITTING YOUR CHANGES!
%%%      PLEASE RUN A SPELL CHECKER BEFORE COMMITTING YOUR CHANGES!
%%%      PLEASE RUN A SPELL CHECKER BEFORE COMMITTING YOUR CHANGES!
%%%      PLEASE RUN A SPELL CHECKER BEFORE COMMITTING YOUR CHANGES!
%%%      PLEASE RUN A SPELL CHECKER BEFORE COMMITTING YOUR CHANGES!

%%%%%%%%%%%%%%%%%%%%%%%%%%%%%%%%%%%%%%%%%%%%%%%%%%%%%%%%%%%%%%%%%%%%%%
\section{\label{sec:History-7-4}Stable Release Series 7.4}
%%%%%%%%%%%%%%%%%%%%%%%%%%%%%%%%%%%%%%%%%%%%%%%%%%%%%%%%%%%%%%%%%%%%%%

This is a stable release series of Condor.
As usual, only bug fixes (and potentially, ports to new platforms)
will be provided in future 7.4.x releases.
New features will be added in the 7.5.x development series.

The details of each version are described below.

%%%%%%%%%%%%%%%%%%%%%%%%%%%%%%%%%%%%%%%%%%%%%%%%%%%%%%%%%%%%%%%%%%%%%%
\subsection*{\label{sec:New-7-4-0}Version 7.4.0}
%%%%%%%%%%%%%%%%%%%%%%%%%%%%%%%%%%%%%%%%%%%%%%%%%%%%%%%%%%%%%%%%%%%%%%

\noindent Release Notes:

\begin{itemize}

\item The default configuration file now uses
  \MacroNI{ALLOW}/\MacroNI{DENY} in place of
  \MacroNI{HOSTALLOW}/\MacroNI{HOSTDENY}.  We recommend making this
  same change throughout your other configuration files.  That way,
  if your policy depends on the default policy, it should continue to
  work as it did before.  The behavior of these configuration settings
  remains unchanged.  The \MacroNI{ALLOW}/\MacroNI{DENY} lists are
  added to the \MacroNI{HOSTALLOW}/\MacroNI{HOSTDENY} lists to form the
  authorization policy.  Both styles support the same syntax.  We would
  like to phase out \MacroNI{HOSTALLOW}/\MacroNI{HOSTDENY} to simplify
  the configuration.

\item As of 7.3.2, \Condor{q} \Opt{-xml} output no longer begins with
non-XML consisting of two blank lines followed by a line such as the
following:

\begin{verbatim}
-- Submitter: schedd-name : <IP> : hostname
\end{verbatim}

\end{itemize}


\noindent New Features:

\begin{itemize}

\item Added \MacroNI{SCHEDD\_JOB\_QUEUE\_LOG\_FLUSH\_DELAY} to set an
upper bound in seconds on how long it takes for changes to the job
ad to be visible to JobRouter and Quill.  The default is 5.
Previously, there was no upper limit.  Typically, other activity in
the job queue, such as jobs being submitted or completed would cause
buffered data to be flushed to disk, so the effective upper bound was
a function of how busy the job queue was.

\item \Condor{q} \Arg{-analyze} and \Arg{-better-analyze} now provide
  analysis for scheduler and local universe jobs.  Specifically, the
  \MacroNI{START\_SCHEDULER\_UNIVERSE} and
  \MacroNI{START\_LOCAL\_UNIVERSE} expressions are checked.

\item Added \Attr{TotalLocalRunningJobs}, \Attr{TotalLocalIdleJobs},
\Attr{TotalSchedulerRunningJobs}, and \Attr{TotalSchedulerIdleJobs}
to the published ad for the \Condor{schedd}.  This means that
\Condor{q} \Opt{-analyze} can still give helpful information about
why local or scheduler universe jobs are idle when
\MacroNI{START\_LOCAL\_UNIVERSE} or
\MacroNI{START\_SCHEDULER\_UNIVERSE} refer to these attributes.

% #130
\item Add support for KVM in the vm universe.

% #688
\item The \Condor{vm-gahp} now links with libvirt, rather than calling
virsh command-line tools.

% #760
\item Greatly improved \Condor{gridmanager}'s scalability when handling
many grid-type gt2 grid universe jobs.

\item Latency in submitting and cleaning up Condor-C (grid-type 'condor')
has been improved.

\end{itemize}

\noindent Configuration Variable Additions and Changes:

\begin{itemize}

\item The default configuration file now uses
  \MacroNI{ALLOW}/\MacroNI{DENY} in place of
  \MacroNI{HOSTALLOW}/\MacroNI{HOSTDENY}.  See the release notes above
  for more information.

\item The default value for \MacroNI{MAX\_JOBS\_RUNNING} has changed.
  Previously, it was 200.  Now it is an expression that depends on the
  total amount of memory and the operating system.  The default
  expression requires 1MB of RAM on the submit machine per running
  job.  In some environments and configurations, this is overly
  generous and can be cut by as much as 50\%.  Under Windows, the
  number of running jobs is still capped at 200.  If you wish to set
  it higher, it is recommended to use a 64-bit version of Windows.
  Under unix, the maximum default is now 10,000.  To scale higher, it
  is recommended to examine the system ephemeral port range and extend
  it so that there are at least 2.1 ports per running job.

\item The default value of \MacroNI{RESERVED\_SWAP} is now 0, which
  disables the \Condor{schedd}'s check for sufficient swap space
  before starting more jobs.  The new expression for
  \MacroNI{MAX\_JOBS\_RUNNING} has a more appropriate memory check, so
  we are phasing out \MacroNI{RESERVED\_SWAP}.  In case
  \MacroNI{RESERVED\_SWAP} is enabled, we have changed the default value
  of \MacroNI{SHADOW\_SIZE\_ESTIMATE} to 800 (kilobytes).  Previously,
  it was 200 if not set but it was set to 1800 in the example configuration
  file.

\item The default values of \MacroNI{START\_LOCAL\_UNIVERSE} and
  \MacroNI{START\_SCHEDULER\_UNIVERSE} have changed.  Previously,
  these were simply set to \Expr{True}.  Now, they are configured to allow
  up to 200 local universe and 200 scheduler universe jobs to run.

\item The default value of
  \MacroNI{GRIDMANAGER\_MAX\_SUBMITTED\_JOBS\_PER\_RESOURCE} has
  changed from 100 to 1000.

\item The default \MacroNI{NEGOTIATOR\_INTERVAL} has changed from 300 to 60.

\item The default value of \MacroNI{ENABLE\_GRID\_MONITOR} has been
  changed from \Expr{False} to \Expr{True}.  Previously, this variable
  was assigned to \Expr{True} in the example configuration file, so
  many installations would have been using the grid monitor even
  though the default when the variable is not defined in the
  configuration file was not to use the grid monitor.

% #631
\item \MacroNI{VM\_VERSION} has been removed, as has the machine ad
attribute of the same name. \MacroNI{START\_ATTRS} can be used to
add virtual machine version information to the machine ad.

\end{itemize}

\noindent Bugs Fixed:

\begin{itemize}

\item Under windows, if a job ClassAd update from the starter caused the
job periodic hold or remove policy to become true, there was a 20-second
deadlock between the shadow, startd, and starter.

%gittrac #622
\item Fixed a bug that could cause \Condor{dagman} to generate an
illegal rescue DAG if it read events incorrectly in recovery mode
(\Condor{dagman} now checks for events that violate DAG semantics
when reading events in recovery mode, and exits without creating a
rescue DAG if it reads such an event).

% gittrac #744
\item Fixed a bug that could cause \Condor{dagman} to abort if it saw
a combination of a terminated and an aborted event on a node with
retries.

\item Changed some log file warnings in \Condor{dagman} to not be
printed at the default verbosity setting.

\item Fixed a bug originating in 7.1.4.  When a user submitted a job
with an executable that did not have execute permission enabled, and
Condor was running as root, and transfer-files mode was being used,
Condor failed to automatically turn on execute permission after
transferring the file.

\item Fixed a bug from 7.3.2.  The configuration variable
\MacroNI{COUNT\_HYPERTHREAD\_CPUS} was ignored and was effectively
treated as False in all cases.

\item JobRouter was not able to see matchmaking diagnostic attributes
such as \Attr{LastRejMatchTime}.  Therefore, when evaluating policy
expressions that referred to these attributes, they were effectively
treated as though undefined.  Quill was also not able to see these
attributes.

% #822
\item Fixed a bug introduced in 7.3.2 that could cause the
\Condor{gridmanager} to crash repeatedly on startup if the job queue
contained grid-type gt2 jobs that had been previously submitted.

% #724, #774
\item Fixed two bugs related to VOMS introduced in 7.3.2. The first
prevented jobs with X509 proxies from being submitted on platforms
on which Condor doesn't support VOMS. The second prevented submission
of jobs with VOMS proxies if the authenticity of the VOMS extensions
couldn't be verified.

\item Fixed a bad default in \Attr{batch\_gahp.config} that prevented
Condor from observing job state changes for grid-type pbs and lsf grid
universe jobs.

% #748
\item Fixed a bug that caused Condor-C jobs to fail if
\Attr{transfer\_executable} was set to \Opt{False}.

% #784
\item Fix a bug that caused Condor-C jobs to fail if the executable
or stdin/out/err filenames contained a comma.

% #460
\item File transfer for grid-type gt4 jobs requires an empty directory
in \Attr{/tmp}, which the \Condor{gridmanager} creates. If this directory
is deleted, the \Condor{gridmanager} will now recreate it.

\end{itemize}

\noindent Known Bugs:

\begin{itemize}

\item None.

\end{itemize}

\noindent Additions and Changes to the Manual:

\begin{itemize}

\item None.

\end{itemize}


% as of April 2011, Karen no longer wants to include these older
% version histories with the 7.6 and beyond manuals.
%%%%      PLEASE RUN A SPELL CHECKER BEFORE COMMITTING YOUR CHANGES!
%%%      PLEASE RUN A SPELL CHECKER BEFORE COMMITTING YOUR CHANGES!
%%%      PLEASE RUN A SPELL CHECKER BEFORE COMMITTING YOUR CHANGES!
%%%      PLEASE RUN A SPELL CHECKER BEFORE COMMITTING YOUR CHANGES!
%%%      PLEASE RUN A SPELL CHECKER BEFORE COMMITTING YOUR CHANGES!

%%%%%%%%%%%%%%%%%%%%%%%%%%%%%%%%%%%%%%%%%%%%%%%%%%%%%%%%%%%%%%%%%%%%%%
\section{\label{sec:History-7-3}Development Release Series 7.3}
%%%%%%%%%%%%%%%%%%%%%%%%%%%%%%%%%%%%%%%%%%%%%%%%%%%%%%%%%%%%%%%%%%%%%%

This is the development release series of Condor.
The details of each version are described below.

%%%%%%%%%%%%%%%%%%%%%%%%%%%%%%%%%%%%%%%%%%%%%%%%%%%%%%%%%%%%%%%%%%%%%%
\subsection*{\label{sec:New-7-3-1}Version 7.3.1}
%%%%%%%%%%%%%%%%%%%%%%%%%%%%%%%%%%%%%%%%%%%%%%%%%%%%%%%%%%%%%%%%%%%%%%

\noindent Release Notes:

\begin{itemize}

\item None.

\end{itemize}


\noindent New Features:

\begin{itemize}

\item None.

\end{itemize}

\noindent Configuration Variable Additions and Changes:

\begin{itemize}

\item Made the \Condor{schedd} more efficient in how it stores
information about `dollar dollar' expansions in the job ClassAd.
Also made it more efficient in how it contacts the \Condor{negotiator}
to submit `reschedule' requests.

\item Added \Macro{UPDATE\_OFFSET} to cause the \Condor{startd} to
  delay the initial (and all further) updates that it sends to the
  \Condor{collector}.  See \ref{param:UpdateOffset} for more details.

\end{itemize}

\noindent Bugs Fixed:

\begin{itemize}

\item The \Condor{schedd} was failing to send `reschedule' commands to
flocked negotiators, so unless some other schedd in the negotiator's
pool sent it a reschedule command, negotiation cycles would only
happen every \Macro{NEGOTIATOR\_INTERVAL}.

\end{itemize}

\noindent Known Bugs:

\begin{itemize}

\item None.

\end{itemize}

\noindent Additions and Changes to the Manual:

\begin{itemize}

\item None.

\end{itemize}

%%%%%%%%%%%%%%%%%%%%%%%%%%%%%%%%%%%%%%%%%%%%%%%%%%%%%%%%%%%%%%%%%%%%%%
\subsection*{\label{sec:New-7-3-0}Version 7.3.0}
%%%%%%%%%%%%%%%%%%%%%%%%%%%%%%%%%%%%%%%%%%%%%%%%%%%%%%%%%%%%%%%%%%%%%%

\noindent Release Notes:

\begin{itemize}

\item Updated the DRMAA version.
This new version is compliant with GFD.133,
the DRMAA 1.0 grid recommendation standard.
Three new functions were added to meet the specification's requirements,
and several bugs were fixed.

\end{itemize}


\noindent New Features:

\begin{itemize}

\item Added support for using any recognized script as an executable
in a submit file on Windows. For more information please see
section~\ref{sec:windows-scripts-as-executables} on
page~\pageref{sec:windows-scripts-as-executables}.

\item Added CCB, the Condor Connection Broker.  It is similar in
functionality to GCB, the Generic Connection Broker, but it has
several advantages, including ease of use and working on Windows as
well as Unix platforms.
GCB continues to work, but we may remove
it some time in the 7.3 development series.  The main missing feature
in CCB at the moment that prevents it from replacing GCB,
is support for connectivity from one private network to another.
CCB only works
when connecting from a public network to a private one.  For example,
jobs may be sent from a \Condor{schedd} on the public internet to 
\Condor{startd} daemons on a
private network, if the \Condor{startd} daemons are configured
to use a CCB server that is accessible to the \Condor{schedd} daemon.
However, if the \Condor{schedd} daemon is on one private
network and the \Condor{startd} daemons are on a different private network,
CCB does not help.  For more information on CCB, see page~\pageref{sec:CCB}.

\item Added support for CPU affinity on Linux and Windows.

\item Added support for \Condor{q}'s \Opt{-better-analyze} on Windows.

\item Added \MacroNI{WANT\_HOLD}.  When \MacroNI{PREEMPT} becomes
true, if \MacroNI{WANT\_HOLD} is true, the job is put on hold for the
reason (optionally) specified by \MacroNI{WANT\_HOLD\_REASON} and
\MacroNI{WANT\_HOLD\_SUBCODE}.  These policy expressions are evaluated
by the execute machine.  As usual, the job owner may specify
\AdAttr{periodic\_release} and/or \AdAttr{periodic\_remove}
expressions to react to specific hold states automatically.

\item Added the ClassAd function \Procedure{debug}.
See section~ \ref{sec:classadFunctions} for the details of this function.

\item Log messages have been made more clear.
% Includes: Give a clear warning instead of a terse error, when lacking a COLLECTOR.

\item The \Condor{schedd} can now use MD5 checksums to avoid storing
multiple copies of the same executable in its \Macro{SPOOL} directory.
Note that this feature only affects executables sent to the
\Condor{schedd} via the \SubmitCmd{copy\_to\_spool} submit macro.

% gittrac #197
\item Reduced the number of sleeps \Condor{dagman} does to maintain log
file consistency when a DAG uses multiple user logs for node jobs.
(DAGMan now does one sleep per submit cycle instead of one sleep for
each submit.)

% gittrac #166, #208
\item Added the \Opt{-import\_env} command-line flag to
\Condor{submit\_dag}.  This explicitly puts the submittor's environment
into the \File{.condor.sub} file.

\end{itemize}

\noindent Configuration Variable Additions and Changes:

\begin{itemize}

\item \MacroNI{OPEN\_VERB\_FOR\_<EXT>\_FILES} has been added to allow
the default interpreter for scripts with an extension \textit{EXT} to
be changed.  For more information please see
section~\ref{sec:windows-scripts-as-executables} on
page~\pageref{sec:windows-scripts-as-executables}.

\item \MacroNI{CCB\_ADDRESS} configures a daemon to use one or more
CCB servers to allow communication with Condor components outside of
the private network.  See page~\pageref{sec:CCB}.

\item \MacroNI{MAX\_FILE\_DESCRIPTORS} (UNIX only) specifies the
required file descriptor limit for a Condor daemon.  File descriptors
are a system resource used for open files and for network connections.
Condor daemons that make many simultaneous network connections may
require an increased number of file descriptors.  For example, see
page~\pageref{sec:CCB} for information on file descriptor requirements
of CCB.

\item The new configuration variables \Macro{ENFORCE\_CPU\_AFFINITY} and 
\Macro{SLOTx\_CPU\_AFFINITY} have been added on Linux to allow for
condor to lock slots to given CPUs.

\item The new configuration variable \Macro{DEBUG\_TIME\_FORMAT}
  allows a custom specification for the format of the time
  printed at the start of each line in a daemon's log file.
  See \ref{param:DebugTimeFormat} for the complete definition of
  this variable.

\item The new configuration variable \Macro{SHARE\_SPOOLED\_EXECUTABLES}
  is a boolean value that determines whether the \Condor{schedd} will
  use MD5 checksums to avoid storing multiple copies of the same
  executable in the \Macro{SPOOL} directory. The default setting is
  \Expr{True}.

\end{itemize}

\noindent Bugs Fixed:

\begin{itemize}

\item This release is incompatible when communicating with previous
versions of Condor if the new CCB feature is enabled or if
\Macro{PRIVATE\_NETWORK\_NAME} is configured.

\end{itemize}

\noindent Known Bugs:

\begin{itemize}

\item None.

\end{itemize}

\noindent Additions and Changes to the Manual:

\begin{itemize}

\item None.

\end{itemize}

%%%%      PLEASE RUN A SPELL CHECKER BEFORE COMMITTING YOUR CHANGES!
%%%      PLEASE RUN A SPELL CHECKER BEFORE COMMITTING YOUR CHANGES!
%%%      PLEASE RUN A SPELL CHECKER BEFORE COMMITTING YOUR CHANGES!
%%%      PLEASE RUN A SPELL CHECKER BEFORE COMMITTING YOUR CHANGES!
%%%      PLEASE RUN A SPELL CHECKER BEFORE COMMITTING YOUR CHANGES!

%%%%%%%%%%%%%%%%%%%%%%%%%%%%%%%%%%%%%%%%%%%%%%%%%%%%%%%%%%%%%%%%%%%%%%
\section{\label{sec:History-7-2}Stable Release Series 7.2}
%%%%%%%%%%%%%%%%%%%%%%%%%%%%%%%%%%%%%%%%%%%%%%%%%%%%%%%%%%%%%%%%%%%%%%

This is a stable release series of Condor.
As usual, only bug fixes (and potentially, ports to new platforms)
will be provided in future 7.2.x releases.
New features will be added in the 7.3.x development series.

The details of each version are described below.


%%%%%%%%%%%%%%%%%%%%%%%%%%%%%%%%%%%%%%%%%%%%%%%%%%%%%%%%%%%%%%%%%%%%%%
\subsection*{\label{sec:New-7-2-0}Version 7.2.0}
%%%%%%%%%%%%%%%%%%%%%%%%%%%%%%%%%%%%%%%%%%%%%%%%%%%%%%%%%%%%%%%%%%%%%%

\noindent Release Notes:

\begin{itemize}

\item A bug in some older Xen kernels can result in Condor errors
due to a broken assumption in the \Condor{procd} daemon.
See the FAQ entry at section~ \ref{sec:xen-jiffies-bug} for details.

\item A problem has been discovered when using snapshot disks with 
\SubmitCmd{vm} universe VMware jobs,
if the path that the \Condor{vm-gahp} uses to refer to the
virtual machine's VMX file contains a symbolic link.
See the FAQ entry at section~ \ref{sec:vmware-symlink-bug} for details.

\item The name of the Amazon EC2 GAHP binary has changed from
\Prog{amazon-gahp} to \Prog{amazon\_gahp}. This makes it consistent
with the naming of other Condor binaries.

\end{itemize}


\noindent New Features:

\begin{itemize}

\item The default \SubmitCmd{universe} for jobs is now 
\SubmitCmd{vanilla}, instead of \SubmitCmd{standard}.
The default can be changed using the configuration variable
\Macro{DEFAULT\_UNIVERSE}.

\item VMware \SubmitCmd{vm} universe jobs now have any BIOS settings saved in
an \File{nvram} file in the \SubmitCmd{vmware\_dir} given in the
job's submit file transferred to the execute machine, so that they
apply to the job's execution.

\item Daemons that become unresponsive are now killed using the
SIGABRT signal, which causes a core file to be dropped.
Setting the configuration variable \Macro{NOT\_RESPONDING\_WANT\_CORE}
to \Expr{False} will revert to the previous behavior that used
the SIGKILL signal.

\item The \Condor{job\_router} and the
\Condor{q} command with the \Opt{-better-analyze} option now
support more ClassAd functions than they previously did.  They now
support all ClassAd functions, except for those with names beginning
with the string \Code{stringList}.

\end{itemize}

\noindent Configuration Variable Additions and Changes:

\begin{itemize}

\item The HAD configuration variable \MacroNI{NEGOTIATOR\_STATE\_FILE}
has changed its name to \MacroNI{STATE\_FILE}.

\end{itemize}

\noindent Bugs Fixed:

\begin{itemize}

\item \Security A flaw was found and fixed that could allow an unauthenticated
user to cause Condor daemons to shut down,
and could allow running jobs to be removed from the queue.

% PR 952
\item Fixed a bug that caused \Condor{dagman} to stay in the Condor queue,
if \Condor{dagman} was accidentally submitted with an empty DAG input file.

% PR 959
\item \Condor{submit\_dag} now generates a \File{.condor.sub} file with
the submit description file command \SubmitCmd{copy\_to\_spool}
set to \Expr{True}, to ease version upgrades while
large DAGs are running.

\item Fixed a problem in the \Condor{startd} when using
\MacroNI{STARTD\_SLOT\_EXPRS} for attributes that are sometimes
present and sometimes absent from the machine ClassAd.  This is most
typical of attributes that enter the machine ClassAd from the job, via
\MacroNI{STARTD\_JOB\_EXPRS}.  When the attribute went away from slot X
(for example, because the job on slot X finished), the corresponding
\MacroNI{SlotX\_<AttributeName>} attribute was not reliably removed from
all of the other slots.

\item Removed some redundant information from the \Condor{startd} 
advertisements to the \Condor{collector}, 
from within the private ClassAd that is not user-visible.
This fix reduces UDP traffic and memory usage generated by
the \Condor{startd} by about 20\Percent\
in the \Condor{collector} and \Condor{negotiator} daemons.

\item Fixed the \Condor{master} daemon to correctly preserve all command-line
arguments when restarting itself.  In some cases, not preserving \Code{argv[0]}
confused external utilities that monitor the \Condor{master} process by looking
at the output of \Prog{ps} or similar programs.  Also, not preserving
\Opt{-pid} and \Opt{-runfor} could cause unexpected behavior.

\item Fixed a bug that exhibited itself when
the configuration variable \MacroNI{NEGOTIATOR\_CONSIDER\_PREEMPTION}
was set to \Expr{False}, in which jobs
would not be matched to slots in the backfill state.  Corrected, slots doing
backfill are included in the matchmaking process.

\item The \Condor{job\_router} did not work while managing jobs from
multiple users when read access to the \Condor{schedd} required
authentication.  The \Condor{job\_router} was also not able to use
authentication methods other than FS.  Now it can use any
authentication method, as long as the resulting identity is listed in
the configuration variable
\MacroNI{QUEUE\_SUPER\_USERS} or the \Condor{job\_router} and
\Condor{schedd} are running as a Personal Condor in non-root mode.

% Commented out by Karen, as it provides no relevant information
% in the given form.
% \item Fixed a number of memory leaks.

\item Fixed a bug in the \Condor{schedd} daemon that could cause it to write
  an incorrect Unique ID to the event log's header.

\item Fixed a bug in the user log reader API that could cause it to
  incorrectly return a ULOG\_NO\_EVENT in rare cases.

\item Fixed a bug in the user log reader API that could cause it to
  crash if the application attempted to re-initialize the ReadUserLog
  object.  The code now detects this condition, and returns an error
  when the application attempts to re-initialization an already
  initialized ReadUserLog object.

\item Fixed a bug that limited the size of \File{stdin}, \File{stdout},
and \File{stderr} files in the vanilla universe to 2GBytes.

\item Fixed a bug that could cause the \Condor{starter} to EXCEPT upon 
completion or eviction of a \SubmitCmd{vm} universe job.
The error message that appeared in the \File{StarterLog} file was
\begin{verbatim}
  Write_Pipe: invalid pipe end
\end{verbatim}

\item When a held job is removed, the values of the attributes
\Attr{HoldReason}, \Attr{HoldReasonCode} and \Attr{HoldReasonSubCode}
are moved to \Attr{LastHoldReason}, \Attr{LastHoldReasonCode} and
\Attr{LastHoldReasonSubCode}. Before, a hold reason could be lost if a
removed job was subsequently held.

\item The executable attribute for amazon grid universe jobs no longer
needs to be a valid file path.

\item Improved error reporting when a Xen or VMware command fails in the
\SubmitCmd{vm} universe.

\item For \SubmitCmd{vm} universe jobs,
virtual floppy disks are no longer disabled.

\item Fixed a bug introduced in Condor 7.1.4 that caused Condor to
ignore the virtual machine status reported by Xen in the \SubmitCmd{vm} universe.

\item Fixed a 20-second delay in the start up of the \Condor{c-gahp} and
the \Condor{vm-gahp}.

\item Fixed a bug which caused the net mask to be published
  into the machine ClassAd incorrectly.

\item Fixed a bug introduced in Condor 7.1.4 which could cause any
  Condor daemon to crash if the level of debugging output \MacroNI{D\_ALL}
  is enabled when a \Condor{reconfig} command is issued.

\item Fixed a bug introduced in Condor 7.1.4 which caused standard universe
jobs to fail to start up, if security authentication, but not encryption was
enabled between the submit side and the execute side.

% Commented out by Karen, as it gives no relevant information to any
% reader of this version history.  
%\item Many bugs fixed in the \Condor{job\_router} hooks.

\item Fixed a bug with streaming \File{stdin}, \File{stdout}, and
\File{stderr} when using \Prog{glexec}.

% Commented out by Karen, as it gives no relevant information to any
% reader of this version history, and has nothing to do with bugs fixed.
% \item Many improvements in error propagation and debugging output.

\end{itemize}

\noindent Known Bugs:

\begin{itemize}

\item None.

\end{itemize}

\noindent Additions and Changes to the Manual:

\begin{itemize}

\item Initial documentation for dynamic provisioning is available
in section~ \ref{sec:SMP-dynamicprovisioning}.

\end{itemize}


%%%%      PLEASE RUN A SPELL CHECKER BEFORE COMMITTING YOUR CHANGES!
%%%      PLEASE RUN A SPELL CHECKER BEFORE COMMITTING YOUR CHANGES!
%%%      PLEASE RUN A SPELL CHECKER BEFORE COMMITTING YOUR CHANGES!
%%%      PLEASE RUN A SPELL CHECKER BEFORE COMMITTING YOUR CHANGES!
%%%      PLEASE RUN A SPELL CHECKER BEFORE COMMITTING YOUR CHANGES!

%%%%%%%%%%%%%%%%%%%%%%%%%%%%%%%%%%%%%%%%%%%%%%%%%%%%%%%%%%%%%%%%%%%%%%
\section{\label{sec:History-7-1}Development Release Series 7.1}
%%%%%%%%%%%%%%%%%%%%%%%%%%%%%%%%%%%%%%%%%%%%%%%%%%%%%%%%%%%%%%%%%%%%%%

This is the development release series of Condor.
The details of each version are described below.


%%%%%%%%%%%%%%%%%%%%%%%%%%%%%%%%%%%%%%%%%%%%%%%%%%%%%%%%%%%%%%%%%%%%%%
\subsection*{\label{sec:New-7-1-0}Version 7.1.0}
%%%%%%%%%%%%%%%%%%%%%%%%%%%%%%%%%%%%%%%%%%%%%%%%%%%%%%%%%%%%%%%%%%%%%%

\noindent Release Notes:

\begin{itemize}

\item None.

\end{itemize}


\noindent New Features:

\begin{itemize}

\item Updated \AdAttr{Arch} to better reflect the architectures supported 
      under Windows. The three most commonly seen values will now be: 
      \AdStr{INTEL} (aka x68), \AdStr{IA64} (Intel Itanium), and 
      \AdStr{X86\_64} (for both AMD and Intel 64-bit processors).
      For further information, please see the unnumbered subsection 
      labeled Machine ClassAd Attributes on 
      page~\pageref{sec:Machine-ClassAd-Attributes}.

\item The Windows MSI installer now supports extended VM Universe 
      options. These new options include: the ability to set the 
      networking type; how much memory VM Universe can use on a host; 
      and the ability to set the version of VMware installed on the host.

\item The \Condor{status} and \Condor{q} command line tools now have a
version option which prints the version of those specific tools.  This
can be useful when multiple versions of Condor are installed on the
same machine.

\end{itemize}

\noindent Configuration Variable Additions and Changes:

\begin{itemize}

\item None.

\end{itemize}

\noindent Bugs Fixed:

\begin{itemize}

\item The Build ID is now included in the Windows' \Condor{version} print
      out, as well as the logs.

\item \Condor{quill}, as well as \Condor{dbmsd}, correctly register 
      themselves with the Window Firewall.

\item Added \AdStr{WINNT60} (for Vista) to the documented list of 
      possible values for \AdAttr{OpSys}.

\end{itemize}

\noindent Known Bugs:

\begin{itemize}

\item None.

\end{itemize}

\noindent Additions and Changes to the Manual:

\begin{itemize}

\item None.

\end{itemize}


%%%%      PLEASE RUN A SPELL CHECKER BEFORE COMMITTING YOUR CHANGES!
%%%      PLEASE RUN A SPELL CHECKER BEFORE COMMITTING YOUR CHANGES!
%%%      PLEASE RUN A SPELL CHECKER BEFORE COMMITTING YOUR CHANGES!
%%%      PLEASE RUN A SPELL CHECKER BEFORE COMMITTING YOUR CHANGES!
%%%      PLEASE RUN A SPELL CHECKER BEFORE COMMITTING YOUR CHANGES!

%%%%%%%%%%%%%%%%%%%%%%%%%%%%%%%%%%%%%%%%%%%%%%%%%%%%%%%%%%%%%%%%%%%%%%
\section{\label{sec:History-7-0}Stable Release Series 7.0}
%%%%%%%%%%%%%%%%%%%%%%%%%%%%%%%%%%%%%%%%%%%%%%%%%%%%%%%%%%%%%%%%%%%%%%

This is a stable release series of Condor.
It is based on the 6.9 development series.
All new features added or bugs fixed in the 6.9 series are available
in the 7.0 series.
As usual, only bug fixes (and potentially, ports to new platforms)
will be provided in future 7.0.x releases.
New features will be added in the 7.1.x development series.

On backwards compatibility:
we believe that Condor 7.0.x and 6.8.x are wire-compatible, 
and can be freely mixed between computers in a Condor pool. 
However, we do not regularly test this compatibility and cannot guarantee it, 
so we recommend using a single release of Condor when possible. 
Please note that although you can mix Condor 7.0.x and 6.8.x in a pool, 
you cannot mix them on a single computer. 
That is, a \Condor{master} daemon running 6.8.x cannot run Condor daemons 
from version 7.0.x, or vice-versa.

The details of each version are described below.


%%%%%%%%%%%%%%%%%%%%%%%%%%%%%%%%%%%%%%%%%%%%%%%%%%%%%%%%%%%%%%%%%%%%%%
\subsection*{\label{sec:New-7-0-6}Version 7.0.6}
%%%%%%%%%%%%%%%%%%%%%%%%%%%%%%%%%%%%%%%%%%%%%%%%%%%%%%%%%%%%%%%%%%%%%%

\noindent Release Notes:

\begin{itemize}

\item None.

\end{itemize}


\noindent New Features:

\begin{itemize}

\item None.

\end{itemize}

\noindent Configuration Variable Additions and Changes:

\begin{itemize}

\item None.

\end{itemize}

\noindent Bugs Fixed:

\begin{itemize}

\item In some rare cases, the \Condor{startd} failed to fully preempt jobs.
The job itself was killed, but the \Condor{starter} process watching over
it would not be killed.  The slot would then stay in the Preempting state
indefinitely.

\end{itemize}

\noindent Known Bugs:

\begin{itemize}

\item None.

\end{itemize}

\noindent Additions and Changes to the Manual:

\begin{itemize}

\item None.

\end{itemize}


%%%%%%%%%%%%%%%%%%%%%%%%%%%%%%%%%%%%%%%%%%%%%%%%%%%%%%%%%%%%%%%%%%%%%%
\subsection*{\label{sec:New-7-0-5}Version 7.0.5}
%%%%%%%%%%%%%%%%%%%%%%%%%%%%%%%%%%%%%%%%%%%%%%%%%%%%%%%%%%%%%%%%%%%%%%

\noindent Release Notes:

This release contains many bug fixes and some improvements to error handling
of Local Universe jobs. Note that some of the bug fixes are
security-related; therefore, we recommend sites either upgrade Condor, or
restrict permissions on who is allowed to submit Condor jobs to trusted
users. Bug fixes that are security related are clearly marked in the Bugs
Fixed section below along with a description of the potential security
impact. The Condor Project believes in the full disclosure of information,
and therefore complete vulnerability details can be found at
\URL{http://www.cs.wisc.edu/condor/security/}. However, in order to give an
adequate upgrade window for production installations, we will delay posting
the full vulnerability details fixed in this release for 30 days (until the
week of November 3rd 2008). 


\noindent New Features:

\begin{itemize}

\item Local universe jobs now go on hold for the same specific reasons that
vanilla jobs may go on hold.  Examples are missing input or executable files.
Previously, when local universe jobs failed in this manner,
the jobs returned to the idle state in the job queue,
repetitively attempting to run, 
and failing over and over until the job is removed.

\item Local universe jobs now have the ClassAd attribute \Attr{NumShadowStarts}.
Although local universe jobs do not have a \Condor{shadow} process, 
this attribute
is introduced to keep management of local universe as similar to
vanilla universe as possible.  For local universe jobs, this attribute
is identical to the attribute \Attr{JobRunCount}, 
which indicates how many times a
local \Condor{starter} process has been created to run the job.

\end{itemize}

\noindent Configuration Variable Additions and Changes:

\begin{itemize}

\item None.

\end{itemize}

\noindent Bugs Fixed:

\begin{itemize}

\item \Security A flaw was found and fixed in the way Condor processes user submitted
jobs. It was possible for a user who had permissions to submit jobs into
Condor to do so in a way that could cause that job to run as any other
non-root user. We have not had any reported incidents exploiting this
flaw. (CVE-2008-3826)

\item \Security A stack-based buffer overflow flaw was found and fixed in the
\Condor{schedd} daemon. A user who had permissions to submit a job could
do so in a manner that could cause the \Condor{schedd} to crash, or
potentially, execute arbitrary code on the submit machine with the
\Condor{schedd}'s identity. We are not aware of any known exploits for
this flaw. We have not had any reported incidents exploiting this flaw.
(CVE-2008-3828)

\item \Security A denial-of-service flaw was found and fixed in the \Condor{schedd}
daemon. A user who had permissions to submit a job could have done so in a
manner that would cause \Condor{schedd} to crash. 
We have not had any reported incidents exploiting this flaw. (CVE-2008-3829)

\item \Security A flaw was found and fixed in the way Condor processes allow and deny 
net masks for access control. 
If Condor's configuration file contained overlapping net masks in 
the allow or deny rules, it could have caused those rules to be ignored, 
potentially allowing unintended access to users in Condor's 
deny authorization lists. 
We have not had any reported incidents exploiting this flaw. (CVE-2008-3830) 

% Fixed by Red Hat
\item Fixed a segmentation fault bug with \Condor{submit} \Opt{-dump} when
\Expr{universe=grid} or \Expr{x509userproxy=<anything>}.

\item Fixed a stack overflow bug in the \Condor{negotiator} daemon.

\item Fixed \Condor{submit} \Opt{-dump} such that it would function with the
standard universe.

\item Fixed a memory leak in the \Condor{startd}, which occurred during the
handling of a \Condor{reconfig} command.

\item When the configuration variable \Macro{NEGOTIATOR\_CONSIDER\_PREEMPTION}
is defined to be \Expr{False},
this no longer results in machines in the Owner state being ignored
during matchmaking.  Previously, even if \MacroNI{START} was \Expr{True},
machines in the Owner state were disregarded.

\item Setting \Attr{JobLeaseDuration} to be less than 15 minutes caused the
\Condor{schedd} daemon to abort and restart the next time a
\Condor{reconfig} command was executed.
The error message in the \Condor{schedd} log appeared as:

\footnotesize
\begin{verbatim}
ALIVE_INTERVAL in the condor configuration is too high (300).
\end{verbatim}
\normalsize

\item Fixed a slow memory leak affecting the \Condor{startd},
\Condor{schedd}, and \Condor{collector} daemons.  This leak would probably
require many months of continuous operation before causing noticeable problems.

\item Fixed a bug that caused a \Condor{schedd} daemon crash.
The crash occurred during a fast shut down of the
\Condor{schedd} daemon as it dealt with local universe
jobs or with any job that required reconnection when
the \Condor{schedd} daemon started up.

\item Local and scheduler universe jobs were failing to increment the
\Attr{JobRunCount} attribute in the job ClassAd when an attempt to run
the job was made.  This problem was introduced in 6.9.5.

\item Some rare types of failures during file transfer caused the
Condor daemon conducting the transfer to hang indefinitely.  For
example, if the file transfer process created by the \Condor{schedd}
was killed by an administrator or crashed due to an internal error,
the \Condor{schedd} would become unresponsive.

\item GCB was updated, fixing minor bugs with GCB temporary files 
(typically the file(s) \File{/tmp/gcb-inherit-*}).
These bugs did not impact GCB functionality.  Earlier
versions would leave temporary files behind. Temporary files would have
permissions of 000.  With the fix, under normal operations the files should be
deleted, and the \Login{condor} user should have read and write access to the files.

\item Evaluation of the configuration variable \Macro{STARTD\_AD\_REEVAL\_EXPR}
did not work for many types of expressions.
The problem resulted in the following message in the 
\Condor{negotiator} daemon log:

\begin{verbatim}
Can't evaluate STARTD_AD_REEVAL_EXPR  ...
\end{verbatim}

\item Reconnecting to parallel universe jobs after a restart of the
\Condor{schedd} daemon, would sometimes fail.  The failure was caused
by the \Condor{shadow} trying to connect to the address of the
previous instance of the \Condor{schedd} rather than the address of
the current instance.

\item Made the \Condor{gridmanager} less aggressive in forwarding refreshed
proxies for gt2 grid universe jobs. Now, the refreshed proxy will not be
forwarded until the old proxy has less than six hours of life until
expiration.

\item Fixed a bug in the \Condor{gridmanager} that could result in job
status updates from the Grid Monitor to be ignored.

\item The Grid Monitor no longer changes the last-modified time of GRAM
state files whose job's status is FAILED. This should make it easier for
file cleaners to remove the the GRAM state files of old, abandoned jobs.

\item Fixed a problem that could cause flocked jobs to fail due to
authorization errors in the \Condor{starter}. Such failures were more
likely to occur for long-running jobs or if the \Condor{schedd} were
issued a full reconfig during the job's execution.

\item Fixed a \Condor{gridmanager} crash on Windows. This crash only
appeared if \Macro{GRIDMANAGER\_DEBUG} were set to a higher level than
the default.

\item In PrivSep mode, a job would previously fail if it created a
symlink in its sandbox pointing to a file owned by a UID other than
that used to run the job. This behavior has been fixed.

\end{itemize}

\noindent Known Bugs:

\begin{itemize}

\item None.

\end{itemize}

\noindent Additions and Changes to the Manual:

\begin{itemize}

\item Descriptions of previously undocumented Condor Perl module subroutines
  have been added to the manual.  See section~\ref{sec:condor-pm}.

\end{itemize}



%%%%%%%%%%%%%%%%%%%%%%%%%%%%%%%%%%%%%%%%%%%%%%%%%%%%%%%%%%%%%%%%%%%%%%
\subsection*{\label{sec:New-7-0-4}Version 7.0.4}
%%%%%%%%%%%%%%%%%%%%%%%%%%%%%%%%%%%%%%%%%%%%%%%%%%%%%%%%%%%%%%%%%%%%%%

\noindent Release Notes:

\begin{itemize}

\item This release fixes a problem causing possible incorrect handling
of wild cards in authorization lists.
Examples of the configuration variables that specify
authorization lists are
\begin{verbatim}
  ALLOW_WRITE
  DENY_WRITE
  HOSTALLOW_WRITE
  HOSTDENY_WRITE
\end{verbatim}
If a configuration variable uses the asterisk character
(\verb@*@) in configuration variables that specify the
authorization policy, it is advisable to upgrade.
This is especially true for the 
use of wild cards in any \MacroNI{DENY} list,
since this problem could result in
access being allowed, when it should have been denied.
This issue affects all previous versions of Condor.

\item The default daemon-to-daemon security session duration has been
changed from 100 days to 1 day. This should reduce memory usage in the
\Condor{collector} in pools with short-lived \Condor{startd}s (e.g. 
glidein pools or pools whose machines are rebooted every night).

\end{itemize}


\noindent New Features:

\begin{itemize}

\item Added functionality to periodically update timestamps on lock files. 
This prevents administrative programs from deleting in-use lock files and
causing undefined behavior.

\item When the configuration variable \Macro{SCHEDD\_NAME} ends in 
the \verb$@$ symbol,
Condor will no longer append the fully qualified
host name to the value.
This makes it possible to configure a high availability
job queue that works with the remote submission of jobs.

\end{itemize}

\noindent Configuration Variable Additions and Changes:

\begin{itemize}

\item Added configuration variable: \Macro{LOCK\_FILE\_UPDATE\_INTERVAL}.
Please see page~\pageref{param:LockFileUpdateInterval} for a complete
description.

\item Changed the default value of configuration variable
\Macro{SEC\_DEFAULT\_SESSION\_DURATION} from 8640000 seconds (100 days)
to 86400 seconds (1 day).

\end{itemize}

\noindent Bugs Fixed:

\begin{itemize}

\item Fixed a bug in the \Condor{c-gahp} that caused it to fail repeatedly
on Windows, if more than two Condor-C jobs were submitted at the same time.

\item Fixed a problem that caused the \Condor{collector}'s memory usage
to increase dramatically, if \Condor{findhost} was run repeatedly.

\item Fixed a bug where Windows jobs suspended by Condor would never
be continued, despite log files indicating successful continuation.
This problem has existed since the 6.9.2 release of Condor.

%PR 937
\item Fixed a problem that could cause \Condor{dagman} to core dump
if straced, especially if the \File{dagman.out} file is on a shared
file system.

\item Fixed a problem introduced in 7.0.1 that could cause the \Condor{schedd}
daemon to crash when starting parallel or MPI universe jobs.  In some cases,
the problem would result in the following log message:

\footnotesize
\begin{verbatim}
ERROR ``Assertion ERROR on (mrec->request_claim_sock == sock)'' \
  at line 1361 in file dedicated_scheduler.C
\end{verbatim}
\normalsize

\item The \Condor{procd} daemon now periodically updates the timestamps on
the named pipe file system objects that it uses for communication.
This prevents these objects from being cleaned up by programs like
\Prog{tmpwatch}, which would result in Condor daemon exceptions.

\item Fixed a problem introduced in Condor 7.0.2 that would cause daemons
to fail on start up on Windows 2000.

\item Fixed a problem where standard universe jobs would fail to start
when using PrivSep, if the \Macro{PROCD\_ADDRESS} configuration variable was not
defined.

\item If the X509 proxy of a vanilla universe job has been refreshed, the
updated file will no longer be returned when the job completes.

\item If ClassAd attributes \Attr{StreamOut} or \Attr{StreamErr} are
missing from the job ClassAd of a grid universe job,
the default value for these attributes is now \Expr{False}.

\end{itemize}

\noindent Known Bugs:

\begin{itemize}

\item A bug in 7.0.4 affects jobs using Condor file transfer on submit
machines that are configured to deny write access from execute
machines.  The result is that output from jobs may fail to be copied
back to the submit machine.  The problem may or may not affect jobs
that run for less than eight hours, but it definitely will affect jobs
that run for more than eight hours.  An example of a configuration
vulnerable to this problem is one where DAEMON level access is allowed
to all execute nodes but WRITE level access is not.  When the problem
happens, the \Condor{shadow} log will contain a line like the following:

\begin{verbatim}
DaemonCore: PERMISSION DENIED to unknown user from host ...
for command 61001 (FILETRANS_DOWNLOAD), access level WRITE
\end{verbatim}

The workaround for this problem is to allow WRITE access from the
execute nodes.  If the existing configuration requires WRITE access to
be authenticated, then simply add WRITE access by the authenticated
condor identities associated with all execute nodes.  If WRITE access
is not currently required to be authenticated, then allow
unauthenticated WRITE access from all worker nodes.  Note that this
does \emph{not} imply that execute nodes will be able to modify the
job queue without authenticating.  Remote commands that modify the job
queue (for example, \Condor{submit} or \Condor{qedit}) always require that the
user be authenticated, no matter what configuration options are used;
if no method of remote authentication can succeed in the pool for
WRITE operations, then commands that modify the job queue can only run
on the submit machine.

\end{itemize}

\noindent Additions and Changes to the Manual:

\begin{itemize}

\item None.

\end{itemize}

%%%%%%%%%%%%%%%%%%%%%%%%%%%%%%%%%%%%%%%%%%%%%%%%%%%%%%%%%%%%%%%%%%%%%%
\subsection*{\label{sec:New-7-0-3}Version 7.0.3}
%%%%%%%%%%%%%%%%%%%%%%%%%%%%%%%%%%%%%%%%%%%%%%%%%%%%%%%%%%%%%%%%%%%%%%

\noindent Release Notes:

\begin{itemize}

\item This is a bug fix release.  A bug in Condor version 7.0.2 sometimes caused
the \Condor{schedd} to become unresponsive for 20 seconds when starting
the \Condor{shadow} to run a job.
Therefore, anyone running 7.0.2 is strongly encouraged to upgrade.

\end{itemize}


\noindent New Features:

\begin{itemize}

\item None.

\end{itemize}

\noindent Configuration Variable Additions and Changes:

\begin{itemize}

\item The configuration variable \Macro{VALID\_SPOOL\_FILES} now automatically 
includes \File{SCHEDD.lock},
the lock file used for high availability \Condor{schedd} fail over.  Other
high availability lock files are not currently included.

\end{itemize}

\noindent Bugs Fixed:

\begin{itemize}

\item Fixed a problem sometimes causing minutes or more of lag between
the time of job suspension or unsuspension and the corresponding entries
in the job user log.

\item Fixed a problem in \Condor{q} \Opt{-better-analyze} handling
requirements expressions containing  the expression \Expr{=!= UNDEFINED}.

\item Configuration variable \Macro{GRIDMANAGER\_GAHP\_CALL\_TIMEOUT}
is now recognized for nordugrid grid universe jobs.

\item Fixed a bug that could cause the \Condor{schedd} daemon to abort
and restart some time after a graceful restart,
when jobs to which the \Condor{schedd} daemon reconnected were preempted.

\item Fixed a bug causing failure to reconnect to jobs which use
\Expr{\$\$([\textit{expression}])}
in their ClassAds.  The jobs would go on
hold with the hold reason:
\AdStr{Cannot expand \$\$([\textit{expression}]).}

\item Fixed a bug in Condor version 7.0.2 that sometimes caused 
the \Condor{schedd} daemon to become
unresponsive for 20 seconds when starting the \Condor{shadow} daemon
to run a job.

\end{itemize}

\noindent Known Bugs:

\begin{itemize}

\item None.

\end{itemize}

\noindent Additions and Changes to the Manual:

\begin{itemize}

\item See 
  section~\ref{sec:WebService-Implementation}
  for documentation on finding the port number the \Condor{schedd} daemon
  is listening on for use with the web service API.

\end{itemize}


%%%%%%%%%%%%%%%%%%%%%%%%%%%%%%%%%%%%%%%%%%%%%%%%%%%%%%%%%%%%%%%%%%%%%%
\subsection*{\label{sec:New-7-0-2}Version 7.0.2}
%%%%%%%%%%%%%%%%%%%%%%%%%%%%%%%%%%%%%%%%%%%%%%%%%%%%%%%%%%%%%%%%%%%%%%

\noindent Release Notes:

\begin{itemize}

\item On Unix, Condor no longer requires its \Macro{EXECUTE} directory to
be world-writable, as long as it is not on a root-squashed NFS mount and is
owned by the user given in the \Macro{CONDOR\_IDS} setting (or by Condor's
real UID, if not started as \Login{root}). Condor will automatically remove
world-writability from existing \MacroNI{EXECUTE} directories where possible.
Note: The \MacroNI{EXECUTE} directory has never been required to be
world-writable on Windows.

\item With this release, a binary package for IA64 SUSE Linux Enterprise 8
will no longer be made available.

\end{itemize}


\noindent New Features:

\begin{itemize}

\item A clipped port to FreeBSD 7.0 x86 and x86\_64 is available, but at this
time, it is not available for download as a binary package.

\item Previously, \Condor{q} \Opt{-better-analyze} was supported on most
but not all versions of Linux.  It is now supported on all Unix platforms,
but not yet on Windows platforms.

\end{itemize}

\noindent Configuration Variable Additions and Changes:

\begin{itemize}

\item The new configuration variable
\Macro{GRIDMANAGER\_MAX\_WS\_DESTROYS\_PER\_RESOURCE} limits the number
of simultaneous WS destroy commands issued to a given server for grid
universe jobs of type gt4. The default value is 5.

\end{itemize}

\noindent Bugs Fixed:

\begin{itemize}

\item Fixed a bug in the standard universe where if a Linux machine was
  configured to use the Network Service Cache Daemon (nscd), taking
  a checkpoint would be deferred indefinitely.

\item Fixed a bug that caused the Quill daemon to crash.

\item Fixed bug that prevented Quill, when running on a
  Windows host, from successfully updating the database.

\item Fixed a bug that prevented Quill's \Condor{dbmsd} daemon from proper
  shutting down upon request when running on Windows platforms.

% condor-admin 17847
\item Fixed a bug that caused Stork to be completely broken.

\item As a back port from Condor versions 7.1,
  the Windows Installer is now completely
  internationalized: it will no longer fail to install because of a
  missing "Users" group; instead, it will use the regionally appropriate
  group.

\item As a back port from Condor versions 7.1,
  interoperability with Samba (as a PDC) has been improved.
  Condor uses a fast form of login during credential validation.
  Unfortunately, this login procedure fails under Samba,
  even if the credentials are valid.  The new behavior is to attempt
  the fast login, and on failure, fall back to the slower form.

\item As a back port from Condor versions 7.1,
  Windows slot users no longer have the
  Batch Privilege added, nor does Condor first attempt a Batch login
  for slot users.  This was causing permission problems on hardened
  versions of Windows, such as Windows Sever 2003, in that not
  interactive users lacked the permission to run batch files 
  (via the \Prog{cmd.exe} tool). 
  This affected any user submitting jobs that used
  batch files as the executable.

\item Fixed a bug that could sometimes cause the \Condor{schedd}
  to either EXCEPT or crash shortly after a user issues a \Condor{rm}
  command with the \Opt{-forcex} option.

\item \Condor{history} in a Quill environment,
  when given the \Opt{-constraint} option,
  would ignore attributes from the vertical schema.  This has been fixed.

\item In Unix, when started as \Login{root},
  the \Condor{master} now changes the
  effective user id back to \Login{root} (instead of condor)
  when restarting itself.
  This occurs for example due to the command \Condor{restart}.
  This makes no difference unless the \Condor{master} is wrapped
  with a script, and the script expects to be run as \Login{root}
  not only on initial start up, but on restart as well.

\item The dedicated scheduler would sometimes take two negotiation cycles
  to acquire all the machines it needed to run a job.
  This has been now fixed.

% PR 938
\item \Condor{dagman} no longer prints "Argument added" and
  "Retry Abort Value" diagnostic messages at the default verbosity,
  to reduce the size of the \File{dagman.out} file and the start up time
  for very large DAGs.

\item \Condor{dagman} now prints a few fatal parse errors at lower
  verbosity settings than it did previously.

\item \Condor{preen} no longer deletes \Prog{MyProxy} password files in the
  Condor spool directory.

\item When using TCP updates (UDP updates are the default), the
  \Condor{collector} would sometimes freeze for 20 seconds when
  receiving an invalidation notice.  
  The notice is received when Condor is being turned off
  on a machine in the pool.

\item Fixed a case in which the \Condor{schedd}'s job queue log file
could get corrupted when encountering errors writing to the disk such
as `out of space'.  This type of corruption was detected by the
\Condor{schedd} the next time it restarted and read the file to
restore the job queue, so you would only have been affected by this
problem if your \Condor{schedd} refused to start up until you fixed or
removed the job queue log file.  This bug has existed in all versions
of Condor, but it became more likely to occur in 6.9.4.

\item The configuration setting \MacroNI{JAVA} may now contain spaces.
Previously, this did not work.

\item Fixed a problem that caused occasional failure to detect hung
Condor daemons.

\item Fixed a file descriptor leak in the negotiator.  The leak happened
  whenever the negotiator failed to initiate the NEGOTIATE command to
  a \Condor{schedd}, for example if security negotiation failed
  with the \Condor{schedd}.
  Under Unix, this would eventually cause the \Condor{negotiator} to run out of
  file descriptors, exit, and restart.  This bug affected all previous
  versions of Condor.

\item Fixed several bugs in the user log reader that caused it to
  generate an invalid persisted state if no events had been read in.
  When read back in, this persisted state would cause the reader to
  segfault during initialization.

\item Fixed a bug causing communication problems if different portions
of a Condor pool were configured with different values of
\MacroNI{SEC\_DEFAULT\_SESSION\_DURATION}.  This bug affects all
previous versions of Condor.  The client side of the connection was
always using its own security session duration, even if the server's
duration was shorter.  Among other potential problems, this was
observed to cause file transfer failures when the starter was
configured with a longer session duration than the shadow.

\item Fixed a bug in the user log writer that was causing the writing
  of events to the global event log fail in some conditions.

\item In the grid universe, submission of nordugrid jobs is now properly
throttled by configuration parameters
\Macro{GRIDMANAGER\_MAX\_SUBMITTED\_JOBS\_PER\_RESOURCE} and
\Macro{GRIDMANAGER\_MAX\_PENDING\_SUBMITS\_PER\_RESOURCE}.

\item The NorduGrid GAHP server can now properly extract job execution
information from newer NorduGrid servers. Previously, the GAHP could crash
when talking to newer servers.

\item Fixed a bug that caused \Condor{config\_val} \Opt{-set} or
  \Opt{-rset} to fail if security negotiation was turned off.
  This happens, for example, if
  \Expr{SEC\_DEFAULT\_NEGOTIATION = NEVER}.
  This bug was introduced in Condor 7.0.0.

\item Fixed a bug that could cause incorrect IP addresses to be advertised
when the \Condor{collector} was on a multi-homed host.

\item Fixed a problem where unexpected ownership and permissions on files
inside a job's working directory could cause the \Condor{starter} to EXCEPT.

\item Improved the speed at which the \Condor{startd} can handle claim
requests, particularly when the \Condor{startd} manages a large number
of slots.

\item Fixed an error in the way the \Condor{procd} calculates image size for
jobs that involve multiple processes. Previously the maximum image size for
any single process was being used. Now the image size sum across all
processes is used.

\item The \Condor{procd} no longer truncates its log file on start up.
  Enabling a log file for the \Condor{procd} is only recommended for
  debugging, since it is not rotated to conserve disk space.

\item Fixed a problem present in Condor 7.0.1 and 7.1.0 where the
\Condor{startd} will crash upon deactivating or releasing a COD claim.

\item Condor on Windows can now correctly handle job image size when
processes are created that allocate more than 2GB of address space.

\item The \Macro{JOB\_INHERITS\_STARTER\_ENVIRONMENT} setting now works when
the \Macro{GLEXEC\_STARTER} feature is in use.

\item Fixed a problem causing \Condor{schedd} to perform poorly when
handling large job queues in which there are any idle local or
scheduler universe jobs (for example, Condor cron jobs).

\item Sped up \Condor{schedd} graceful shutdown when disconnecting
from running jobs that have job leases.  Previously, it would only
disconnect from one such job at a time, so if there were a lot of jobs
running, \Condor{schedd} could take so long to shut down that job leases
expire before it has a chance to restart and reconnect to the jobs.

\item Fixed a bug that could cause incorrect IP addresses to be advertised
when the \Condor{collector} was on a multi-homed host.

\end{itemize}

\noindent Known Bugs:

\begin{itemize}

\item None.

\end{itemize}

\noindent Additions and Changes to the Manual:

\begin{itemize}

\item None.

\end{itemize}


%%%%%%%%%%%%%%%%%%%%%%%%%%%%%%%%%%%%%%%%%%%%%%%%%%%%%%%%%%%%%%%%%%%%%%
\subsection*{\label{sec:New-7-0-1}Version 7.0.1}
%%%%%%%%%%%%%%%%%%%%%%%%%%%%%%%%%%%%%%%%%%%%%%%%%%%%%%%%%%%%%%%%%%%%%%

\noindent Release Notes:

\begin{itemize}

\item Fixed a bug in Condor's authorization policy reader.  The bug
affects cases where the policy (\MacroNI{ALLOW}/\MacroNI{DENY} and
\MacroNI{HOSTALLOW}/\MacroNI{HOSTDENY} settings) mixes host-based
authorizations with authorizations that refer to the authenticated
user name.  In some cases, this bug would result in host-based
settings not being applied to authenticated users.

\end{itemize}

\noindent New Features:

\begin{itemize}

\item Support for Backfill Jobs is now available on Windows platforms.
For more information on this, please see
section~\ref{sec:Backfill-BOINC-Windows} on
page~\pageref{sec:Backfill-BOINC-Windows}.

\item Condor has been ported to Red Hat Enterprise Linux
5.0 running on the 32-bit x86 architecture and on the 64-bit x86\_64
architecture.

% This feature was added in 6.7, but condor_submit wasn't changed.
% Until now, the user had to set this via the '+' notation in the submit
% file.
\item The command \SubmitCmd{email\_attributes} in a job submit
description file defines a set of job ClassAd attributes whose values
should be included in the e-mail notification of job completion.

\item The configuration variable \Macro{CONDOR\_VIEW\_HOST} may now
contain a port number, and may refer to a
\Condor{collector} daemon running on the same host as the
\Condor{collector} that is forwarding ClassAds.  It is also now possible to
use the forwarded ClassAds for matchmaking purposes.  For example, several
\Condor{collector} daemons could forward ClassAds to 
a single aggregating \Condor{collector} daemon which
a \Condor{negotiator} then uses as its source of information for
matchmaking.

\item \Condor{configure} and \Condor{install} now detect missing
  shared libraries (such as \File{libstdc++.so.5} on Linux), and print
  messages and exit if missing libraries are detected.  The new command
  line option \Opt{--ignore-missing-libs} causes it not to exit
  after the messages have been printed, and to proceed with the
  installation.

\item Added a \Opt{--force} command line option to \Condor{configure}
  (and \Condor{install}) which will turn on \Opt{--overwrite} and
  \Opt{--ignore-missing-libs}.

\item \Condor{configure} now writes simple sh and csh shell scripts
  which can be sourced by their respective shells to set the user's
  \Env{PATH} and \Env{CONDOR\_CONFIG} environment variables.  By default, these
  are created in the root of the Condor installation, but this can be
  changed via the \Opt{--env-scripts-dir} command line option.  Also,
  the creation of these scripts can be disabled with the
  \Opt{--no-env-scripts} command line option.

\end{itemize}

\noindent Configuration Variable Additions and Changes:

\begin{itemize}

\item The new configuration variables \Macro{PREEMPTION\_REQUIREMENTS\_STABLE}
  and \Macro{PREEMPTION\_RANK\_STABLE} are boolean values to
  identify whether or not attributes used within the definition of
  \Macro{PREEMPTION\_REQUIREMENTS} and \Macro{PREEMPTION\_RANK} remain
  unchanged during a negotiation cycle.
  See section~\ref{param:PreemptionRequirementsStable} on
  page~\pageref{param:PreemptionRequirementsStable} for 
  complete definitions.

\item The configuration variable \Macro{STARTER\_UPLOAD\_TIMEOUT}
  changed its default value to 300 seconds.

\item The new configuration variable \Macro{CKPT\_PROBE} 
specifies an internal to Condor
executable which determines information about how a process is laid out
in memory, in addition to other information. This executable is not yet
available on Windows platforms.

\item The new configuration variable 
\MacroNI{CKPT\_SERVER\_CHECK\_PARENT\_INTERVAL} sets an interval
of time between checks by the checkpoint server to see if 
its parent, the \Condor{master} daemon, has gone away unexpectedly.
The checkpoint server shuts itself down if this happens.
The default interval for checking is 120 seconds.
Setting this parameter to 0 disables the check.

\end{itemize}

\noindent Bugs Fixed:

\begin{itemize}

\item Upgrade from PCRE v5.0 to PCRE v7.6, due to security vulnerabilities 
found in PCRE v5.0.

\item Fixed file descriptor leak in the \Condor{schedd} when using the SOAP
interface.

\item Fixed a bug that primarily affected pools with
\MacroNI{MaxJobRetirementTime} (0 by default) set larger than
\MacroNI{REQUEST\_CLAIM\_TIMEOUT} (30 minutes by default).  Since
6.9.3, when the \Condor{schedd} timed out requesting a claim to a slot, the
\Condor{startd} was not made aware of the canceled request.  This
resulted in some wasted time (up to \MacroNI{ALIVE\_INTERVAL}) in
which the \Condor{startd} would wait for a job to run.

\item A problem with \Condor{history} in a Quill environment incorrectly
interpreting the \Opt{-name} option has been fixed.

\item A memory leak that prevented \Condor{load\_history} from running
with large history files has been fixed.

\item A bug in \Condor{history} when running in a quill environment has been fixed.  This bug would cause the command to crash in some situations.

\item The job ClassAd attribute \Attr{EmailAttributes} now works 
for grid universe jobs.

\item On 32-bit Linux platforms, the job queue database file may now exceed 2GB.
Previously, the \Condor{schedd} would halt with an error when trying
to write past the 2GB mark.

\item On 32-bit Linux platforms, \Condor{history} can now read from history
files larger than 2GB \emph{except} when using the \Opt{-backwards} option.

\item Local universe jobs are now scheduled to run more promptly.  Previously,
new local universe jobs would sometimes take up to \MacroNI{SCHEDD\_INTERVAL}
(default 5 minutes) to be considered for running.

\item The memory usage of the \Condor{collector} used to grow over time if
daemons with new names kept joining and then leaving the pool
(for example, in a Glidein pool).
This was due to statistics on dropped updates that
accumulated for all daemons that ever advertised themselves to the
\Condor{collector}.  These statistics are now periodically purged of
information about daemons which have not reported in a long time.  How
long is controlled by \Macro{COLLECTOR\_STATS\_SWEEP}, which
defaults to 2 days.

\item Condor daemons would die when trying to send ClassAd
advertisements to a host name that could not be resolved by DNS.

\item Since 6.9.5, file transfer errors for vanilla, java, or parallel
jobs would sometimes not result in the job going on hold as it should.
This was most likely for very small files that failed to be written
for some reason.

\item The \AdAttr{ImageSize} reported for jobs on AIX was too big by a factor
of 1024.

\item Since 6.9.5, \Condor{glidein} failed in the set up stage, due to the
change in syntax of quoting rules
in the Condor submit description file for gt2 argument strings.

\item Fixed a bug in the \Condor{gridmanager} that could prevent refreshed
X509 proxies from being forwarded to the remote machine for grid universe
jobs of type gt4.

\item Fixed a bug in Condor's authorization policy reader.  The bug
affects cases where the policy (\MacroNI{ALLOW}/\MacroNI{DENY} and
\MacroNI{HOSTALLOW}/\MacroNI{HOSTDENY} settings) mixes host-based
authorizations with authorizations that refer to the authenticated
user name.  In some cases, this bug would result in host-based
settings not being applied to authenticated users.

\item Fixed a bug in \Condor{history} which causes a crash 
when \Condor{quill} is enabled.

\item Fixed a problem affecting the GSI and SSL authentication
methods.  When these methods successfully authenticated the user but
failed to find a mapping of the X509 name to a condor user id, they
were setting the authenticated name to \verb|gsi| and \verb|ssl|
respectively.  However, these names contain no domain, so they could
not be referred to in the authorization policy.  Now these anonymous
mappings are \verb|gsi@unmappeduser| and \verb|ssl@unmappeduser|.
Therefore, configuration to deny access by users who are not explicitly mapped
in the map file appears as:

\begin{verbatim}
DENY_READ = *@unmappeduser
DENY_WRITE = *@unmappeduser
\end{verbatim}

\end{itemize}

\noindent Known Bugs:

\begin{itemize}

\item When using \Condor{compile} with the RHEL5 x86 port of Condor to
produce a standard universe executable, one will see a warning message
about how linking with dynamic libraries is not portable. This warning
is erroneous and should be ignored. It will be fixed in a future version
of Condor.

\end{itemize}

\noindent Additions and Changes to the Manual:

\begin{itemize}

\item The existing configuration variables 
\Macro{SYSTEM\_PERIODIC\_HOLD}, \Macro{SYSTEM\_PERIODIC\_RELEASE}, and
\Macro{SYSTEM\_PERIODIC\_REMOVE} have documented definitions.
See section~\ref{param:SystemPeriodicHold} for definitions.

\item A manual page for \Condor{load\_history} has been added.

\end{itemize}


%%%%%%%%%%%%%%%%%%%%%%%%%%%%%%%%%%%%%%%%%%%%%%%%%%%%%%%%%%%%%%%%%%%%%%
\subsection*{\label{sec:New-7-0-0}Version 7.0.0}
%%%%%%%%%%%%%%%%%%%%%%%%%%%%%%%%%%%%%%%%%%%%%%%%%%%%%%%%%%%%%%%%%%%%%%

\noindent Release Notes:

\begin{itemize}

\item PVM support has been dropped.

\item The time zone for the \Prog{PostgreSQL} 8.2 database
  used with Quill on Windows machines must be explicitly set
  to use an abbreviation.
  This Windows environment variable is \verb@TZ@.
  Proper abbreviations for the value of this variable may be found
  within the \Prog{PostgreSQL} installation in a file,
  \File{share/timezonesets/<continent>.txt}, where
  \verb@<continent>@ is replaced by the continent of the 
  desired time zone.

\end{itemize}


\noindent New Features:

\begin{itemize}

\item The Windows MSI installer now supports VM Universe.

\item Eliminated the ``tarball in a tarball'' in our distribution.
  The contents of \File{release.tar} from the distribution tarball
  (for example, \File{condor-6.9.6-linux-x86-centos45-dynamic.tar.gz}) is now
  included in the distribution tarball.

\item Updated \Condor{configure} to match the above change.  The
  \Opt{--install} option now takes a directory path as its parameter,
  for example \Opt{--install=/path/to/release}.
  It previously took the path to
  the \File{release.tar} tarball.

\item Added \Condor{install}, which is a symlink to \Condor{configure}.
  Invoking 
\begin{verbatim}
    condor_install
\end{verbatim}
  is identical to running
\begin{verbatim}
    condor_configure --install=.
\end{verbatim}

\item Added the option \Opt{--prefix=dir} to \Condor{configure} and
  \Condor{install}.  This is an alias for \Opt{--install-dir=dir}.

\item Added the option \Opt{--backup} option to \Condor{configure} and
  \Condor{install}.  This option renames the target \File{sbin} directory,
  if the \Condor{master} daemon exits while in the target \File{sbin} directory.
  Previous versions of \Condor{configure} did this by default.

\item Changed the default behavior of \Condor{install} to exit with a
  warning if the target \File{sbin} directory exists,
  the \Condor{master} daemon is in the \File{sbin} directory,
  and neither the \Opt{--backup} nor \Opt{--overwrite} options are specified.
  This prevents \Condor{install} from improperly moving an \File{sbin}
  directory out of the way.
  For example,
\begin{verbatim}
    condor_install --prefix=/usr
\end{verbatim}
  will not move \File{/usr/sbin} out of the way unless
  the \Opt{--backup} option is also specified.

\item Updated the usage summary of \Condor{configure} and
  \Condor{install} to be much more readable.

\end{itemize}

\noindent Configuration Variable Additions and Changes:

\begin{itemize}

\item The new configuration variable
  \Macro{DEAD\_COLLECTOR\_MAX\_AVOIDANCE\_TIME} defines the maximum
  time in seconds that a daemon will fail over from a primary
  \Condor{collector} to a secondary \Condor{collector}.
  See section~\ref{param:DeadCollectorMaxAvoidanceTime} on
  page~\pageref{param:DeadCollectorMaxAvoidanceTime} for a
  complete definition.

\end{itemize}

\noindent Bugs Fixed:

\begin{itemize}

\item Fixed a memory leak in the \Condor{procd} daemon on Windows.

\item Fixed a problem that could cause Condor daemons to crash if a
  failure occurred when communicating with the \Condor{procd}.

\item Fixed a couple of problems that were preventing the
  \Condor{startd} from properly removing per-job directories
  when running with PrivSep.

\item The \Condor{startd} will no longer fail to initialize, 
  claiming the \MacroNI{EXECUTE} directory has improper permissions,
  when PrivSep is enabled.

\item Look ups of ClassAd attribute \Attr{CurrentTime} are now
  case-insensitive, just like all other attributes.

\item Fixed problems causing the following error message in the log file:

\footnotesize
\begin{verbatim}
ERROR: receiving new UDP message but found a short message still waiting to be closed (consumed=1). Closing it now.
\end{verbatim}
\normalsize

\item The existence of the executable given in the submit file is now 
  enforced (when transferring the executable and not using VM 
  universe).

\item The copy of \Condor{dagman} that ships with Condor is now automatically 
  added to the list of trusted programs in the Windows Firewall.

\item Removed \SubmitCmd{remove\_kill\_sig} from the submission file
  generated by \Condor{submit\_dag} on Windows.

\item Fixed the algorithm in the \Condor{negotiator} daemon, which
  with large numbers of machine ClassAds (for example, 10,000) 
  was causing long delays at the 
  beginning of each negotiation cycle.

\item Use of \MacroNI{MAX\_CONCURRENT\_UPLOADS} was resulting in a
  connection attempt from the \Condor{shadow} to the \Condor{schedd} with a
  fixed 10 second timeout, which is sometimes too small.  This timeout
  has been increased to be the same as other connection timeouts between
  the \Condor{shadow} and the \Condor{schedd}, and it now respects
  \MacroNI{SHADOW\_TIMEOUT\_MULTIPLIER}, so it can be adjusted if necessary.

\item Fixed a problem with \Macro{MAX\_CONCURRENT\_UPLOADS} and
  \Macro{MAX\_CONCURRENT\_DOWNLOADS}, which was sometimes allowing
  more than the configured number of concurrent transfers to happen.

\item Fixed a bug in the \Condor{schedd} that could cause it to crash due
  to file descriptor exhaustion when trying to send messages to hundreds of
  \Condor{startd}s simultaneously.

\item Fixed a 6.9.4 bug in the \Condor{startd} that would cause it to crash
  when a BOINC backfill job exited.

\item Since 6.9.4, when using glExec, configuring \MacroNI{SLOT<N>\_EXECUTE}
  would cause \Condor{starter} to fail when starting the job.

\item Fixed a bug from 6.9.5 which caused authentication failure for
  the pool password authentication method.

\item Fixed a bug that caused Condor daemons to crash when encountering
  some types of invalid ClassAd expressions.

\item Fixed a bug under Linux that could cause multi-process daemons
  lacking a log lock file to crash while rotating logs that have reached
  their maximum configured size.

\item Fixed a bug under Windows that sometimes caused connection attempts
  between Condor daemons to fail with Windows error number 10056.

\item Fixed a problem in which there are multiple 
  \Condor{collector} daemons in a pool
  for fault tolerance.  If the primary \Condor{collector} failed, the
  \Condor{negotiator} would fail over to the secondary \Condor{collector}
  indefinitely (or until the secondary \Condor{collector} also failed or the
  administrator ran \Condor{reconfig}).  This was a problem for users
  flocking jobs to the pool, because flocking currently only works with
  the primary \Condor{collector}.  Now, the \Condor{negotiator} will fail over
  for a restricted amount of time, up to
  \Macro{DEAD\_COLLECTOR\_MAX\_AVOIDANCE\_TIME} seconds.  The default
  is one hour, but if querying the dead primary \Condor{collector}
  takes very little
  time to fail, the \Condor{negotiator} may retry more frequently
  in order to remain
  responsive to flocked users.

\item Fixed a problem preventing the use of \Condor{q} \Opt{-analyze}
  with the \Opt{-pool} option.

\item Fixed a problem in the \Condor{negotiator} in which machines go
  unassigned when user priorities result in the machines getting split
  into shares that are rounded down to 0.  For example if there are 10
  machines and 100 equal priority submitters, then each submitter was
  getting 0.1 machines, which got rounded down to 0, so no machines were
  assigned to anybody.  The message in the \Condor{negotiator} log in this case
  was this:

\footnotesize
\begin{verbatim}
Over submitter resource limit (0) ... only consider startd ranks
\end{verbatim}
\normalsize

\item Fixed a problem introduced in 6.9.3 that would cause daemons to
  run out of file descriptors if they create sub-processes and are
  configured to use a lock file for the debug log.

\item Standard universe jobs now work properly when using PrivSep.

\item Fixed problem with PrivSep mode where a job that dumps core would
  not get the core file transferred back to the the submit host if the
  \SubmitCmd{transfer\_output\_files} submit option were used.

\item Fixed a bug that caused the \Condor{starter} to crash if a job
called \Condor{chirp} with the \Expr{get\_job\_attr} option.

\end{itemize}

\noindent Known Bugs:

\begin{itemize}

\item None.

\end{itemize}

\noindent Additions and Changes to the Manual:

\begin{itemize}

\item None.

\end{itemize}


% Oct 2009, as we release 7.4, Karen commented out inclusion of the
% 6.9 and 6.8 histories
%%%%%%%%%%%%%%%%%%%%%%%%%%%%%%%%%%%%%%%%%%%%%%%%%%%%%%%%%%%%%%%%%%%%%%%
\section{\label{sec:History-6-9}Development Release Series 6.9}
%%%%%%%%%%%%%%%%%%%%%%%%%%%%%%%%%%%%%%%%%%%%%%%%%%%%%%%%%%%%%%%%%%%%%%

This is the development release series of Condor.
The details of each version are described below.

%%%%%%%%%%%%%%%%%%%%%%%%%%%%%%%%%%%%%%%%%%%%%%%%%%%%%%%%%%%%%%%%%%%%%%
\subsection*{\label{sec:New-6-9-1}Version 6.9.1}
%%%%%%%%%%%%%%%%%%%%%%%%%%%%%%%%%%%%%%%%%%%%%%%%%%%%%%%%%%%%%%%%%%%%%%

\noindent Release Notes:

\begin{itemize}

\item None.

\end{itemize}

\noindent New Features:

\begin{itemize}

\item Improved the scalability of the algorithm used by 
the \Condor{schedd} daemon to find runnable jobs.
This makes a noticeable difference in \Condor{schedd} daemon performance,
when there are on the order of thousands of jobs in the queue.

\item the \Dflag{COMMAND} debugging level has been enhanced to
log many more messages. 

\item Updated the version of DRMAA, which contains several great
improvements regarding scalability and race conditions.

% Gnats PR 774
\item Added the DAGMAN\_SUBMIT\_DEPTH\_FIRST configuration macro which
causes \Condor{dagman} to submit ready nodes in more-or-less depth-first
order if set to \Expr{True}.  (The default behavior is to submit
the ready nodes in breadth-first order.)

\item Added configuration parameter \Macro{USE\_PROCESS\_GROUPS}. If it's
set to False, then Condor daemons on unix machines will not create new 
sessions or process groups. This is intended for use with Glidein, as
we've had reports that some batch systems can't properly track jobs that
create new process groups. The default value is True.

\end{itemize}

\noindent Bugs Fixed:

\begin{itemize}

\item \Condor{configure} used to always make a personal Condor with
\Opt{--install} even when \Opt{--type} called for only "execute" or
"submit" types.  Now, \Condor{configure} honors the \Opt{--type}
argument, even when using \Opt{--install}.
If \Opt{--type} is not specified, the default is to still install a
full personal Condor with the following daemons: Master, Collector,
Negotiator, Schedd, Startd. 

\end{itemize}

\noindent Known Bugs:

\begin{itemize}

\item None.

\end{itemize}

%%%%%%%%%%%%%%%%%%%%%%%%%%%%%%%%%%%%%%%%%%%%%%%%%%%%%%%%%%%%%%%%%%%%%%
\subsection*{\label{sec:New-6-9-0}Version 6.9.0}
%%%%%%%%%%%%%%%%%%%%%%%%%%%%%%%%%%%%%%%%%%%%%%%%%%%%%%%%%%%%%%%%%%%%%%

\noindent Release Notes:

\begin{itemize}

\item The 6.9.0 release contains all of the bug fixes and enhancements
  from the 6.8.x series up to and including version 6.8.2.

% and a few \condor{gridmanager} bug fixes from 6.8.3.  *sigh* we need
% a real solution to this problem (like pointing to issue node ids) ;)

\end{itemize}


\noindent New Features:

\begin{itemize}


\item Preliminary support for using \Prog{glexec} on execute machines
has been added.  This feature causes the \Condor{startd} to spawn the
\Condor{starter} as the user that \Prog{glexec} determines based on
the user's GSI credential.

\item A ``per-job history files'' feature has been added to the
\Condor{schedd}. When enabled, this will cause the \Condor{schedd} to
write out a copy of each job's ClassAd when it leaves the job
queue. The directory to place these files in is determined by the
parameter \Macro{PER\_JOB\_HISTORY\_DIR}. It is the resposibilty of
whatever external entity (e.g. an accounting or monitoring system) is
using these files to remove them as it completes its processing.

\item \Condor{chirp} command now supports writing messages to the userlog.

\item \Condor{chirp} getattr and putattr now send all classad getattr
and putattr commands to the proc 0 classad, which allows multiple proc
parallel jobs to use proc 0 as a scratchpad.

\item Parallel jobs now suport an \Attr{AllRemoteHosts} attribute,
which lists all the hosts across all procs in a cluster.

\item The \Macro{DAGMAN\_ABORT\_DUPLICATES} config macro (which causes
\Condor{dagman} to abort itself if it detects another \Condor{dagman}
running on the same DAG) now defaults to \Expr{True} instead of
\Expr{False}.

\end{itemize}

\noindent Bugs Fixed:

\begin{itemize}

\item None.

\end{itemize}

\noindent Known Bugs:

\begin{itemize}

\item None.

\end{itemize}


%%%%%%%%%%%%%%%%%%%%%%%%%%%%%%%%%%%%%%%%%%%%%%%%%%%%%%%%%%%%%%%%%%%%%%%
\section{\label{sec:History-6-8}Stable Release Series 6.8}
%%%%%%%%%%%%%%%%%%%%%%%%%%%%%%%%%%%%%%%%%%%%%%%%%%%%%%%%%%%%%%%%%%%%%%

This is a stable release series of Condor.
It is based on the 6.7 development series.
All new features added or bugs fixed in the 6.7 series are available
in the 6.8 series.
As usual, only bug fixes (and potentially, ports to new platforms)
will be provided in future 6.8.x releases.
New features will be added in the forthcoming 6.9.x development series.

%%%%%%
% we need a summary of major new features since 6.6.x here.  trying to
% sort through the 21 different 6.7.x releases and all the new
% features is a huge amount of noise.  people just want to see a
% summary of the major new functionality.
% \Todo
%%%%%%

The 6.8.x series supports a different set of platforms than 6.6.x.
Please see the updated table of available platforms in
section~\ref{sec:Availability} on page~\pageref{sec:Availability}.

The details of each version are described below.


%%%%%%%%%%%%%%%%%%%%%%%%%%%%%%%%%%%%%%%%%%%%%%%%%%%%%%%%%%%%%%%%%%%%%%
\subsection*{\label{sec:New-6-8-1}Version 6.8.1}
%%%%%%%%%%%%%%%%%%%%%%%%%%%%%%%%%%%%%%%%%%%%%%%%%%%%%%%%%%%%%%%%%%%%%%

\noindent Release Notes:

\begin{itemize}

\item The PCRE (Perl Compatible Regular Expressions) library used by
Condor is now dynamically linked and shipped as a DLL with Condor for
Windows, rather than being statically linked.

\end{itemize}


\noindent New Features:

\begin{itemize}

% Gnats PR 610
\item Added an optional argument to the \Condor{dagman} ABORT-DAG-ON
command that allows the DAGMan exit code to be specified separately
from the node value that causes the abort; also, a DAG can now be
aborted on a zero exit code from a node.

% I implemented in condor_rm.  Todd is handling condor_schedd implementation.
\item Added \Macro{ALLOW\_FORCE\_RM} setting.  If this expressions evaluates to
TRUE then an \Condor{rm} -f attempt is allowed.  If it evaluated to FALSE, the
attempt is disallowed.  The expression is evaluated in the context of the job
ad.  If not specified the setting defaults to TRUE, matching the behavior of
previous Condor releases.

% Gnats PR 664
\item \Condor{dagman} will now reject DAGs for which any of the node
job user log files are on NFS (because of the unreliability of NFS
file locking, this can cause DAGs to fail).  This feature can be
turned off by setting the DAGMAN\_LOG\_ON\_NFS\_IS\_ERROR configuration
macro to \Expr{False} (the default is \Expr{True}).

\item \Condor{submit} can now be configured to reject jobs for which
the log file is on NFS (to do this, set the LOG\_ON\_NFS\_IS\_ERROR
configuration macro to \Expr{True}).  The default is that \condor{submit}
will simply issue a warning if a log file is on NFS.

\item Added the DAGMAN\_ABORT\_DUPLICATES config macro, which causes
\Condor{dagman} to attempt to detect at startup whether another
\Condor{dagman} is already running on the same DAG; if so, the second
\Condor{dagman} will abort itself.

\end{itemize}

\noindent Bugs Fixed:

\begin{itemize}

\item Fixed a Quill bug that prevented it from running on Windows.  The
symptom was errors in the QuillLog like
\begin{verbatim}
POLLING RESULT: ERROR
\end{verbatim}

\item Fixed a bug in Quill where it would cause errors like
\begin{verbatim}
duplicate key violates unique constraint "history_vertical_pkey"
\end{verbatim}
in the QuillLog and the postgres log file.  These errors would then trigger
a significant slowdown in the performance of Quill and the database.  This
would only happpen when a job attribute would change type from a string
type to a numeric type, or vice versa.

\item When a large number of jobs (roughly 200 or more) are running from a
single schedd and those jobs are using job leases (the default in 6.8), it is
possible the schedd to enter a state where it crashes on startup until all of
the job leases expire.  This bug is fixed.
% The bug is more generic; it potentially hit any user of GenericQuery or
% CondorQuery, but we know of no users encountering the bug in any other cases.

%% This is the change to datathread.C
\item In those unusual cases where Condor is unable to create a new process,
it will shut down cleanly, eliminating a small possibility of data corruption.

\item Fixed a bug with the gt4 and nordugrid grid universe job types that
caused the stdout and stderr of a job to not be transferred correctly if
the filenames given had absolute paths.

% Gnats PR 711
\item \Condor{dagman} now echos warnings from \condor{submit} and
stork\_submit to the \File{dagman.out} file.

\item Fixed a bug introduced in 6.7.20 causing \Condor{ckpt\_server}
to exit immediately after starting up, unless Condor's security
negotation was disabled.

% This was a bug because the default configuration file and manual
% both claimed it defaulted to 1MB.
\item \Macro{MAX\_<SUBSYS>\_LOG} defaults to one megabyte, even if the
setting is missing from the configuration.  Previously it was 64 kilobytes.

% this is the change to condor_secman.C by zmiller
\item Fixed a bug related to non-blocking connect that could occasionally
cause Condor daemons to crash.

% collector.C change: eliminated useless fprintf(stderr).
\item Fixed a rare bug where an exceptionally large query to the
\Condor{collector} could cause it to crash.  The most common cause is a single
schedd restarting and trying to recover a large number of job leases at once.
More than approximately 250 running jobs on a single schedd would be necessary
to trigger this bug.

\item When using the \Macro{JOB\_PROXY\_OVERRIDE\_FILE} configuration
parameter, the X509 proxy will now be properly forwarded for Condor-C jobs.

\item Greatly reduced the chance that a Condor-C job in the REMOVED state
will become HELD due to an expired proxy or failure to talk to the remote
\Condor{schedd}.

\item Fixed a number of error and debug messages added in 6.7.20 that
were incorrectly reporting IP and port numbers.  These messages were
intended to report the peer's address, but they were reporting the
local address of the network socket instead.

\item Fixed a bug introduced in 6.7.20 which could cause Condor daemons to
die with the message ``PANIC -- OUT OF FILE DESCRIPTORS''.  The conditions
causing this were related to failed attempts to send updated status
to the collector, with both non-blocking updates and security negotiation
enabled (the defaults).

\item Under some conditions, when making TCP connections, Condor was
still trying to connect for the full duration of the operation timeout
(often 10 or 20 seconds), even if the connection attempt was refused
(e.g. because the port being accessed is not accepting connections).
Now, the connect operation finishes immediately after the first such
failure, allowing the Condor process to continue with other tasks.

\item Fixed the problems relating to credential cache problems in the Kerberos
authentication mechanism.  The current version of Kerberos is 1.4.3.

% changes to condor_auth_ssl.C by zmiller
\item Fixed the bug in the SSL authentication mechanism that caused the
\Condor{schedd} to crash when submitting a job.

\item Some of the binaries required to use Condor-C on Windows were
mistakenly not included in previous releases of Condor. This has been
fixed.

\item Fixed problem on Windows where the \Condor{startd} could fail to
include some attributes in its ClassAd. This would result in some jobs
incorrectly not being matched to that machine.  This only happened if
\Macro{CREDD\_HOST} was defined and Condor daemons on the execute
machine were unable to authenticate with the \Condor{credd}.

\item Fixed a \Condor{dagman} bug which had prevented the
\MacroU{DAGManJobId} attribute from being expanded in job submit files
(e.g., when used as the value of the \Macro{Priority} command).

\item Fixed a bug in \Condor{submit} that caused parallel universe jobs
submitted via Condor-C to become mpi universe jobs.

\item Fixed a bug which could cause Condor daemons to hang if they try
to write to the standard error stream (stderr) on some platforms.  In
general, this should never happen, but can due to third party
libraries (beyond our control) trying to write error or other messages.

\item Fixed \Condor{status} to report error messages.

\end{itemize}

\noindent Known Bugs:

\begin{itemize}

\item None.

\end{itemize}




%%%%%%%%%%%%%%%%%%%%%%%%%%%%%%%%%%%%%%%%%%%%%%%%%%%%%%%%%%%%%%%%%%%%%%
\subsection*{\label{sec:New-6-8-0}Version 6.8.0}
%%%%%%%%%%%%%%%%%%%%%%%%%%%%%%%%%%%%%%%%%%%%%%%%%%%%%%%%%%%%%%%%%%%%%%

\noindent Release Notes:

\begin{itemize}

\item The default configuration for Condor now requires that
\Macro{HOSTALLOW\_WRITE} be explicitly set.  Condor will refuse
to start if the default configuration is used unmodified.
Existing installations should not need to change anything.  For
those who desire the earlier default, you can set it to "*", but
note that this is potentially a security hole allowing anyone to
submit jobs or machines to your pool.

\item Most Linux distributions are now supported using dynamically
  linked binaries built on a RedHat Enterprise Linux 3 machine.
  Recent security patches to a number of Linux distributions have
  rendered the binaries built on RedHat 9 machines ineffective.
  The download pages have been changed to reflect this, but Linux users
  should be aware of this change.
  The recommended download for most x86 Linux users is now:
  \File{condor-6.8.0-linux-x86-rhel3-dynamic.tar.gz}.

\item Some log messages have been clarified or moved to different
  debugging levels.
  For example, certain messages that looked like errors were printed
  to \MacroNI{D\_ALWAYS}, even though nothing was wrong and the system was
  behaving as expected.

\item The new features and bugs fixed in the rest of this section only
  refer to changes made since the 6.7.20 release, not the last stable
  release (6.6.11).
  For a complete list of changes since 6.6.11, read the 6.7 version
  history in section~\ref{sec:History-6-7} on
  page~\pageref{sec:History-6-7}. 

\end{itemize}


\noindent New Features:

\begin{itemize}

\item Version 1.4 of the Condor DRMAA libraries are now included 
  with the Condor release.
  For more information about DRMAA, see section~\ref{API-DRMAA} on
  page~\pageref{API-DRMAA}.

\item Version 1.0.15 of the Condor GAHP is now used for Condor-G and
  Condor-C. 

% Gnats PR 710
\item Added the \Opt{-outfile\_dir} command-line argument to
\Condor{submit\_dag}.  This allows you to change the directory in which
\Condor{dagman} writes the \File{dagman.out} file.

\item Added a new \Opt{--summary} (also \Opt{-s}) option to the
\Condor{update\_stats} tool.  If enabled, this prevents it from
displaying the entire history for each machine and only displays the
summary info.

\end{itemize}

\noindent Bugs Fixed:

\begin{itemize}

\item Fixed a number of potential static buffer overflows in various
  Condor daemons and libraries.

\item Fixed some small memory leaks in the \Condor{startd},
  \Condor{schedd}, and a potential leak that effected all Condor
  daemons.

\item Fixed a bug in Quill which caused it to crash when certain
long attributes appeared in a job ad.

\item The startd would crash after a reconfig if the address of a
collector had not been resolved since the previous reconfig
(e.g. because DNS was down during that time).

\item Once a Condor daemon failed to lookup the IP address of the
collector (e.g. because DNS was down), it would fail to contact the
collector from that time until the next reconfig.  Now, each time Condor
tries to contact the collector, it generates a fresh DNS query if the
previous attempt failed.

% Gnats PR 707
\item When using Condor-C or the -s or -r command-line options to
\condor{submit}, the job's standard output and error would be placed
in the job's initial working directory, even if the job ad said to
place them in a different directory.

% Gnats PRs 501 and 663
\item Greatly sped up the parsing of large DAGs (by a factor of 50
or so) by using a hash table instead of linear search to find DAG nodes.

% Gnats PR 697
\item Fixed a bug in \Condor{dagman} that caused an EXECUTABLE\_ERROR
event from a node job to abort the DAG instead of just marking the
relevant node as failed.

\item Fixed a bug in \Condor{collector} that caused it to discard
machine ads that don't have an IP address field (either StartdIpAddr
or STARTD\_IP\_ADDR).  The \Condor{startd} will always produce a
StartdIpAddr field, but machine ads published through
\Condor{advertise} may not.

\item When using \MacroNI{BIND\_ALL\_INTERFACES} on a dual-homed
machine, a bug introduced in 6.7.18 was causing Condor daemons to
sometimes incorrectly report their IP addresses, which could cause
jobs to fail to start running.

\item Made the event checking in \Condor{dagman} less strict: 
added the new "allow duplicate events" value to the
\MacroNI{DAGMAN\_ALLOW\_EVENTS} macro (this value is part of the
default); 16 value now also allows terminate event before submit;
changed "allow all events" to "allow almost all events"
(all except "run after terminal event"), so it is more useful.

% Gnats PR 712
\item \Condor{dagman} and \Condor{submit\_dag} now report
\Opt{-NoEventChecks} as ignored rather than deprecated.

\item Fixed a bug in the \Condor{dagman} \Opt{-maxidle} feature:
a shadow exception event now puts the corresponding job into the
idle state in \Condor{dagman}'s internal count.

\item Fixed a problem on Windows where daemons would sometimes crash
when dealing with UNC path names.

\item Fixed a problem where the \Condor{schedd} on Windows would
incorrectly reject a job if the client provided an \Attr{Owner}
attribute that was correct but differed in case from the authenticated
name.

\item Fixed a \Condor{startd} crash introduced in version 6.7.20. This
crash would appear if an execute machine was matched for preemption
but then not claimed in time by the appropriate \Condor{schedd}.

\item Resolved an issue where the \Condor{startd} was unable to clean
up jobs' execute directories on Windows when the \Condor{master} was
started from the command line rather than as a service.

\item Added more patches to Condor's DRMAA interface to make it more
compatible with Sun Grid Engine's DRMAA interface.

\item Removed the unused \MacroNI{D\_UPDOWN} debug level and added the
  \MacroNI{D\_CONFIG} debug level.

\item Fixed a bug that caused \Condor{q} with the \Opt{-l} or \Opt{-xml}
arguments to print out duplicate attributes when using Quill.

\item Fixed a bug that prevented Condor-C jobs (universe grid jobs of type condor)
from submitting correctly if \MacroNI{QUEUE\_ALL\_USERS\_TRUSTED} is set to
True.

\item Fixed a bug that could cause the \Condor{negotiator} to crash if the
pool contains several different versions of the \Condor{schedd} and in the
config file \MacroNI{NEGOTIATOR\_MATCHLIST\_CACHING} is set to True.

\item Changed the default value for config file entry
\MacroNI{NEGOTIATOR\_MATCHLIST\_CACHING} from False to True.  When set to
True, this will instruct the negotiator to safely cache data in order to
improve matchmaking performance.

\item The Condor{master} now recognizes \Condor{quill} as a valid
  Condor daemon without any manual configuration on the part of site
  administrators.
  This simplifies the configuration changes required to enable Quill. 

\item Fixed a rare bug in the \Condor{starter} where if there was a
  failure transferring job output files back to the submitting host,
  it could hang indefinitely, and the job appeared as if it was
  continuing to run.

\end{itemize}


\noindent Known Bugs:

\begin{itemize}

\item There are known scalability problems when using Condor's Kerberos
authentication mechanism in large pools.  If your installation of Condor is
more than a couple dozen machines, and you need to use Kerberos for Condor
authentication, we recommend you wait for Condor version 6.8.1 or use Condor
version 6.7.17 (which does not suffer from these problems).

\item There are known problems with Condor's SSL authentication mechanism.
While the HTTPS support in Condor (which also uses SSL) works fine for the
SOAP/Birdbath interface, there are bugs with the SSL support when SSL is
listed in \MacroNI{SEC\_DEFAULT\_AUTHENTICATION\_METHODS}.  We expect to fix
these issues for version 6.8.1.

\end{itemize}


% Dec 2007, as we release 7.x, Karen commented out the older histories
%%%%%%%%%%%%%%%%%%%%%%%%%%%%%%%%%%%%%%%%%%%%%%%%%%%%%%%%%%%%%%%%%%%%%%%
\section{\label{sec:History-6-7}Development Release Series 6.7}
%%%%%%%%%%%%%%%%%%%%%%%%%%%%%%%%%%%%%%%%%%%%%%%%%%%%%%%%%%%%%%%%%%%%%%

This is the development release series of Condor,
The details of each version are described below.

%%%%%%%%%%%%%%%%%%%%%%%%%%%%%%%%%%%%%%%%%%%%%%%%%%%%%%%%%%%%%%%%%%%%%%
\subsection{\label{sec:New-6-7-0}Version 6.7.0}
%%%%%%%%%%%%%%%%%%%%%%%%%%%%%%%%%%%%%%%%%%%%%%%%%%%%%%%%%%%%%%%%%%%%%%

\noindent New Features:

\begin{itemize}

\item Added support for vanilla and Java jobs to reconnect when the
  connection between the submitting and execution nodes is lost for
  any reason.
  Possible reasons for this disconnect include: network outages,
  rebooting the submit machine, restarting the Condor daemons on the
  submit machine, etc.
  If the execution machine is rebooted or the Condor daemons are
  restarted, reconnection is not possible.
  To take advantage of this reconnect feature, jobs must be submitted
  with a \Attr{JobLeaseDuration}.
  There are new events in the UserLog related to disconnect and
  reconnect.

\item Added a new Condor tool, \Condor{vacate\_job}.
  This command is similar to \Condor{vacate}, except the kinds of
  arguments it takes define jobs in a job queue, not machines to
  vacate.
  For example, a user can vacate a specific job id, all the jobs in a
  given cluster, all the jobs matching a job queue constraint, or even
  all jobs owned by that user.
  The owner of a job can always vacate their own jobs, regardless of
  the pool security policy controlling \Condor{vacate} (which is an
  administrative command which acts directly on machines).
  See the new command reference, section~\ref{man-condor-vacate-job}
  on page~\pageref{man-condor-vacate-job} for details.
  
\item Added a new ``High Availability'' service to the \Condor{master}.
   You can now specify a daemon which can have ``fail over'' capabilities
   (i.e. the master on another machine can start a matching daemon if the
   first one fails).  Currently, this is only available over a shared
   file system (i.e. NFS), and has only been tested for the \Condor{schedd}.

\item Scheduler universe jobs on UNIX can now specify a
  \Attr{HoldKillSig}, the signal that should be sent when the job is
  put on hold.
  If not specified, the default is to use the \Attr{KillSig}, and if
  that is not defined, the job will be sent a SIGTERM.
  The submit file keyword to use for defining this signal is
  \AdAttr{hold\_kill\_sig}, for example,
  \verb@hold_kill_sig = SIGUSR1@.

\item The \Condor{startd} can now support policies on SMP machines
  where each virtual machine (VM) has knowledge of the other VMs on
  the same host.
  For example, if a job starts running on one of the VMs, a job
  running on another VM could immediately be suspended.
  This is accomplished by using the new configuration setting
  \Macro{STARTD\_VM\_EXPRS}, which is a list of ClassAd attribute
  names that should be shared across all VMs on the machine machine. 
  For each VM on the machine, ever attribute in this list is looked
  up in the VM ClassAd, the attribute name is given a prefix
  indicating what VM it came from, and then inserted into the ClassAds
  of all the other VMs.

\item The \Condor{startd} publishes four new attributes into the
  machine ClassAds it generates when it is in the Claimed state:
  \Attr{TotalJobRunTime}, \Attr{TotalJobSuspendTime},
  \Attr{TotalClaimRunTime}, \Attr{TotalClaimSuspendTime}.
  These attributes keep track of the total time the resource was
  either running a job (in the Busy activity) or had a job suspended,
  regardless of how many suspend/resume cycles the job went through.
  The first two attributes (with ``Job'' in the name) keep track for a
  single job (i.e. since the last time the resource was
  Claimed/Idle). 
  The last two attributes (with ``Claim'' in the name) keep track of
  these totals across all jobs that ran under the same claim
  (i.e. since the last state change into the Claimed state).

\item Added a \Opt{-num} option to the \Condor{wait} tool to wait for
   a specified number of jobs to finish.

\item Added a configuration option \Macro{STARTER\_JOB\_ENVIRONMENT}
   so the admin can configure the default environment inherited by
   user jobs.

\item Added a (configurable, defaults to off) feature to the \Condor{schedd}
   to allow backup the spool file before doing anything else.

\item The "Continous" option of the \Condor{startd} ``cron'' jobs is
being depricated.   It's being replaced by two new options which
control separate aspects of it's behaviour:
\begin{itemize}
\item "WaitForExit" specifies the "exit timing" mode
\item "ReConfig" specifies that the job can handle SIGHUPs, and it should 
be sent a SIGHUP when the \Condor{startd} is reconfigured.
\end{itemize}

\item A lot of the items logged by the \Condor{startd} ``cron'' logic,
changed to D\_FULLDEBUG (from D\_ALWAYS), etc.

\item Added \Macro{NEGOTIATOR\_PRE\_JOB\_RANK} and
\Macro{NEGOTIATOR\_POST\_JOB\_RANK}.  These expressions are applied
respectively before and after the user-supplied job rank when deciding
which of the possible matches to choose.  (The existing expression
\Macro{PREEMPTION\_RANK} is applied after
\Macro{NEGOTIATOR\_POST\_JOB\_RANK}.)  The pool administrator may use
these expressions to steer jobs in ways that improve the overall
performance of the pool.  For example, using the pre job rank,
preemption may be avoided as long as there are idle machines, even
when the user-supplied rank expression prefers a machine that happens
to be busy.  Using the post job rank, one could steer jobs towards
machines that are known to be dedicated to batch jobs, or one could
enforce breadth-first instead of depth-first filling of a cluster of
multi-processor machines.

\item Added the ability for Condor to transfer files larger than 2G on
platforms that support large files.  This works automatically for
transferred executables, input files and output files.

\end{itemize}


\noindent Bugs Fixed:

\begin{itemize}

\item Fixed a bug in the \Condor{startd} ``cron'' logic which caused the
\Condor{startd} to except when trying to delete a job that could never
be run (i.e. invalid executable, etc).

\item Fixed a bug in \Condor{startd} ``cron'' logic which caused it to
not detect when the starting of a ``job'' failed.

\item Fixed several bugs in the reconfiguration handling of the
\Condor{startd} ``cron'' logic.  In particular, even if the job has
the "reconfig" option set (or "continuous"), the job(s) won't be sent
a SIGHUP when the startd first starts, or when the job itself is first
run (until it output's it's first "output block", defined by the "-"
separator).

\end{itemize}


\noindent Known Bugs:

\begin{itemize}

\item None.

\end{itemize}

%%%%%%%%%%%%%%%%%%%%%%%%%%%%%%%%%%%%%%%%%%%%%%%%%%%%%%%%%%%%%%%%%%%%%%%
\section{\label{sec:History-6-6}Stable Release Series 6.6}
%%%%%%%%%%%%%%%%%%%%%%%%%%%%%%%%%%%%%%%%%%%%%%%%%%%%%%%%%%%%%%%%%%%%%%

This is a stable release series of Condor.
It is based on the 6.5 development series.
All new features added or bugs fixed in the 6.5 series are available
in the 6.6 series.
The details of each version are described below.

%%%%%%%%%%%%%%%%%%%%%%%%%%%%%%%%%%%%%%%%%%%%%%%%%%%%%%%%%%%%%%%%%%%%%%
\subsection{\label{sec:New-6-6-1}Version 6.6.1}
%%%%%%%%%%%%%%%%%%%%%%%%%%%%%%%%%%%%%%%%%%%%%%%%%%%%%%%%%%%%%%%%%%%%%%

\noindent New Features:

\begin{itemize}

\item Added full support (including standard universe jobs with
  checkpointing and remote system calls) for Linux i386 RedHat 8.x
  (using gcc/g++ version 3.2.x and glibc version 2.2.93). 

\item The time it takes \Condor{dagman} to submit jobs has been
      reduced slightly to improve up the startup time of large DAGs.

\item In order to help reduce load on the \Condor{schedd} when
      \Condor{dagman} is submitting jobs, there is a new config
      variable, \Macro{DAGMAN\_SUBMIT\_DELAY}, to specify the number
      of seconds \Condor{dagman} will sleep before submitting each
      job.

\item Enabled the ``update statistics'' in the \Condor{collector} by
      default in both the executable and in the default configuration.

\item Command-line arguments to \Condor{dagman} are now handled
      case-insensitively.

% commented out since i'm not sure we want to make a big deal of this
% new config file knob. -derek 1/12/04
% \item Added a new configuration macro
%     \Macro{MAX\_CLAIM\_ALIVES\_MISSED}, described on
%     page~\pageref{param:MaxClaimAlivesMissed}.
%     This setting controls how many keep alive messages a startd is
%     willing to miss before it releases the claim from a given schedd.

\item Added support for Condor-G and strong authentication to Condor
  for IRIX 6.5, but removed support for checkpointing and remote
  system calls.
  We plan to add support in Condor for IRIX's kernel-level
  checkpointing in a future release.


\end{itemize}

\noindent Bugs Fixed:

\begin{itemize}

\item Fixed a bug in the standard universe where C++ code that threw an 
exception would result in abortion of the executable instead of the
delivery of the exception. This bug affects Condor version 6.6.0 for
Redhat 7.x.

\item Fixed bug whereby \Condor{dagman}, if removed from the queue via
      \Condor{rm}, could fail to remove all of its submitted jobs if
      any of their submit events had not yet appeared in the userlog.

\item Fixed the bug whereby \Condor{preen} could potentially remove
  files related to a valid Computing on Demand (COD) claim on an
  otherwise idle machine.

\item Fixed the faulty argument parsing in \Condor{rm},
  \Condor{release}, and \Condor{hold}.
  Before you could accidentally type \verb@condor_rm -analyze@, and it
  would remove all of your jobs.
  Now it gives an error.

\item On Windows, when you type a command like
  \verb@condor_reconfig.exe@ instead of \verb@condor_reconfig@, you no
  longer get an error.

\item Fixed a bug on Windows that would cause ``GetCursorPos() failed''
  to appear repeatedly in the StartLog. The startd now uses a different
  function to track mouse activity that does not have a tendency to fail.

\item Fixed a bug on Windows that would prevent some \Condor{shadow}
  daemons from obtaining a lock to their log file under heavy load, and
  thus causing them to EXCEPT().

\item Fixed a bug on Windows where file transfers would incorrectly fail
because of bad permissions when using domain accounts with nested groups,
or when UNC paths were used.

\item Fixed the bug where the \Condor{starter} would fail to transfer
  back core files created by Vanilla, Java and MPI universe jobs.
  This bug was introduced in Condor version 6.5.2.
  Now, Condor correctly transfers back any core files created by
  faulty user jobs in any job universe.

\item In some circumstances, \Condor{history} would fail to read
  information about some jobs, and would report errors. In particular,
  when jobs had large environments, it would fail. This has been
  corrected.

\item The error message in the SchedLog that indicates that swap space
  has been depleted has been rephrased so it appears to be
  significant. 

\item Fixed a \Condor{negotiator} bug that could, in certain rare
      circumstances, cause a \Condor{schedd} to hang for five minutes
      while trying to communicate with it.

\item Fixed a rare bug affecting \Condor{dagman} when job-throttling
      was enabled: if \Condor{dagman} was removed from the queue
      together with some of its own jobs (e.g., via \verb@condor_rm -a@),
      it would quickly submit new jobs to replace them before
      recognizing that it needs to exit.  It now shuts down
      immediately without submitting and then removing these
      unnecessary jobs.

\item Fixed a potential security problem that was introduced in Condor
  version 6.5.5 when the \Macro{REQUIRE\_LOCAL\_CONFIG\_FILE}
  configuration setting was added.
  This setting used to default to FALSE if it was not defined in the
  configuration files.
  It now defaults to TRUE.
  If administrators define local configuration files for the machines
  in their pool, it should be a fatal error if those files don't exist
  unless the administrators actively disable this check by defining 
  \MacroNI{REQUIRE\_LOCAL\_CONFIG\_FILE} to be FALSE.

\item Fixed a bug on Windows that would cause the \Condor{startd} to
EXCEPT() if the \Condor{starter} exited and left orphaned processes to
be cleaned up. This bug first appeared in 6.5.0.

\item Fixed a bug on Windows that would cause graceful shutdowns on
Windows (such as when \verb@condor_vacate@  is called) to fail to
complete.

\item The gahp\_server helper program, which provides Globus services to
Condor-G, was always dynamically linked, even in statically-linked releases. 
The statically linked distributions of Condor now include a static gahp\_server.

\item Fixed minor bug in parsing XML user log files that contain empty
  strings. 

\end{itemize}

\noindent Known Bugs:

\begin{itemize}

\item Submission of MPI jobs from a Unix machine to run on Windows
machines (or vice versa) fails for machine\_count > 1.  This is
not a new bug.  Cross-platform submission of MPI jobs between
Unix and Windows has always had this problem.

\end{itemize}


%%%%%%%%%%%%%%%%%%%%%%%%%%%%%%%%%%%%%%%%%%%%%%%%%%%%%%%%%%%%%%%%%%%%%%
\subsection{\label{sec:New-6-6-0}Version 6.6.0}
%%%%%%%%%%%%%%%%%%%%%%%%%%%%%%%%%%%%%%%%%%%%%%%%%%%%%%%%%%%%%%%%%%%%%%

\noindent New Features:

\begin{itemize}

\item The \Condor{dagman} debugging log now reports the total number
      of ``Un-Ready'' Nodes (i.e. those waiting for unfinished
      dependencies) in its periodic summaries.  In the past, the
      omission of this state led to confusion because the total of all
      reported job states didn't always match the total number of jobs
      in the DAG.

\item Most Condor commands (\Condor{on}, \Condor{off},
  \Condor{restart}, \Condor{reconfig}, \Condor{vacate},
  \Condor{checkpoint}, \Condor{reschedule}) now support a \Opt{-all}
  command-line option to specify which daemons to act on.
  This is more efficient and much easier to use than previous methods
  for accomplishing the same effect.
  Using \Opt{-all} with \Condor{off} correctly leaves the existing
  \Condor{master} processes running on each host, so that a subsequent
  \Condor{on} would work.
  See section~\ref{sec:Pool-Shutdown-and-Restart} on
  page~\pageref{sec:Pool-Shutdown-and-Restart} for more details on
  proper use of \Opt{-all} with \Condor{off} and \Condor{on}

\end{itemize}

\noindent Bugs Fixed:

\begin{itemize}

\item Fixed a bug under Solaris 8 with Update 6+, and Solaris 9 where
Condor would incorrectly report the console and mouse idle times as zero.

\item The standard-universe fetch\_files feature was not cleaning up
temporary files on the execution machine.

\item In rare circumstances, a Linux kernel bug results in conflicting
information about system boot time (\File{/proc/stat} and
\File{/proc/uptime}). 
Specifically, the "btime" field in \File{/proc/stat} suddenly jumps to
the present moment and then stays at that value.  This
was resulting in incorrect estimation of process ages, which caused
Condor's estimation of CondorLoadAvg to be completely wrong.  A more
robust heuristic is now being used.

\item A long configuration line with with continuation lines can cause the
config file parser to not properly skip the leading whitespace from
the continued lines.  This has been corrected.

\item The Grid Monitor now will automatically probe for and work with
``unknown'' batch systems.

\item Fixed a bug where under certain circumstances \Condor{dagman}
      would fail to detect an unsuccessful invocation of
      \Condor{submit}, and would instead report the job as
      successfully submitted with job id 0.0.

\item Fixed a bug which was causing problems when a periodic\_remove
expression for a scheduler universe job evaluates to true.  Under
these conditions, the schedd did not log the job terminiation to the
job log.  Addtionally, the schedd would exit with an error status.

\item Fixed a recently-introduced \Condor{dagman} bug where the number
      of node retries (specified with the RETRY keyword) wasn't being
      updated after some failures; instead, the node would be allowed
      to retry indefinitely if it kept failing.

\item Fixed a recently-introduced bug where shutting down the
      \Condor{schedd} caused \Condor{dagman} to remove all its jobs
      from the queue and write a rescue file, rather than simply
      exiting so that it could recover automatically upon restart.

\item Changed the default ``Periodic Expression Interval'' parameter
(PERIODIC\_EXPR\_INTERVAL) from 60 seconds to 300 seconds.

\item Whenever \Condor{reconfig} was used to re-configure multiple
  daemons which included the \Condor{collector} for a pool, the
  command would start to fail after the \Condor{collector} was
  reconfigured due to problems with security sessions in Condor's
  strong authentication code.
  This situation no longer causes problems for the \Condor{reconfig}
  tool, and it can properly re-configure multiple daemons at once,
  even if one of them is the \Condor{collector} for a pool.

\item Most Condor commands (\Condor{on}, \Condor{off},
  \Condor{restart}, \Condor{reconfig}, \Condor{vacate},
  \Condor{checkpoint}, \Condor{reschedule}) now check to make sure
  they are not sending a duplicate command if the user specifies the
  same target machine or daemon twice.  For example:
\begin{verbatim}
     condor_reconfig hostname1 hostname2 hostname1
\end{verbatim}
  will only send a single reconfig command to \verb@hostname1@.

\item Fixed a bug in the HPUX version of Condor which was causing the
startd to occasionally abort operation.  This has been in Condor since
version 6.1.1.

\item The Condor daemons will no longer overwhelm NIS servers
when large numbers of daemons are running. Condor now caches
uid and group information internally, and refreshes the
cache entries on a specified interval (which defaults to 5
minutes). See section~\ref{param:PasswdCacheRefresh} on
page~\pageref{param:PasswdCacheRefresh} for more details.

\end{itemize}

\noindent Known Bugs:

\begin{itemize}

\item The \Condor{preen} program does not know about Computing on
  Demand (COD) claims.
  If there are no regular Condor jobs on a given machine, but there
  are COD claims, and \Condor{preen} is spawned, it will remove files
  related to the COD claims.
  In version 6.6.0, sites using COD are encouraged to disable
  \Condor{preen} by commenting out the \MacroNI{PREEN} setting in the
  config files.
  This bug has been fixed in Condor version 6.6.1.

\item Normally, if a user's job crashes and creates a core file on a
  remote execution machine, the \Condor{starter} will automatically
  transfer the core file back to the submit machine.
  However, beginning in Condor version 6.5.2, if a vanilla, Java, or
  MPI universe job creates a core file, the \Condor{starter} will fail
  to transfer it back.
  This bug will be fixed in version 6.6.1.
  
\item There are a few bugs related to Condor tools failing to
  correctly locate the \Condor{negotiator} daemon.
  These bugs usually show up if a site is using non-standard ports for
  the central manager daemon.
  However, some of the bugs show up regardless of if the negotiator is
  listening on the standard port or not. 

  \begin{itemize}
    \item \verb@condor_config_val -negotitator@ queries the
          \Condor{collector}, instead of querying the
          \Condor{negotiator} like it should.  

    \item Using the \Opt{-pool} option to \verb@condor_q -analyze@
          will not work.
          The tool will fail to find and query the \Condor{negotiator}
          for user priorities which it needs to determine why jobs may
          not be running.

    \item The Condor tools that support either the \Opt{-negotiator}
          or \Opt{-collector} options do not work when a user also
          specifies the \Opt{-pool} to define a remote pool to
          communicate with.
          The tools print a somewhat confusing message in this case.

    \item Most Condor tools that support \verb@-pool hostname@ will
          also recogize \verb@-pool hostname:port@ if the remote
          \Condor{collector} is listening on a non-standard port.
          However, the \Condor{findhost} tool does not work if given a
          \Opt{-pool} option that includes a port.

  \end{itemize}

\end{itemize}

\begin{center}
\begin{table}[hbt]
\begin{tabular}{|ll|} \hline
\emph{Architecture} & \emph{Operating System} \\ \hline \hline
Hewlett Packard PA-RISC (both PA7000 and PA8000 series) & HPUX 10.20 \\ \hline
Sun SPARC Sun4m,Sun4c, Sun UltraSPARC & Solaris 2.6, 2.7, 8, 9 \\ \hline
Silicon Graphics MIPS (R5000, R8000, R10000) & IRIX 6.5 \\ \hline
Intel x86 & Red Hat Linux 7.1, 7.2, 7.3 \\
 & Red Hat Linux 8 (clipped) \\ \hline
 & Red Hat Linux 9 (clipped) \\ \hline
 & Windows NT 4.0 Workstation and Server (clipped) \\ \hline
 & Windows 2000 Professional and Server, 2003 Server (clipped) \\ \hline
 & Windows XP Professional (clipped) \\ \hline
ALPHA & Digital Unix 4.0 \\
 & Red Hat Linux 7.1, 7.2, 7.3 (clipped) \\ \hline
 & Tru64 5.1 (clipped) \\ \hline
PowerPC & Macintosh OS X (clipped) \\
Itanium & Red Hat Linux 7.1, 7.2, 7.3 (clipped) \\
\end{tabular}
\caption{\label{6.6.0-supported-platforms}Condor version 6.6.0 supported platforms}
\end{table}
\end{center}


% Feb 2007 -- still in the manual source, just not incorporating
% these old histories into the finished product, thereby
% reducing the size of the manual by 200 pages
%%%%%%%%%%%%%%%%%%%%%%%%%%%%%%%%%%%%%%%%%%%%%%%%%%%%%%%%%%%%%%%%%%%%%%%
\section{\label{sec:History-6-5}Stable Release Series 6.5}
%%%%%%%%%%%%%%%%%%%%%%%%%%%%%%%%%%%%%%%%%%%%%%%%%%%%%%%%%%%%%%%%%%%%%%

This is the development release series of Condor,
The details of each version are described below.


%%%%%%%%%%%%%%%%%%%%%%%%%%%%%%%%%%%%%%%%%%%%%%%%%%%%%%%%%%%%%%%%%%%%%%
\subsection{\label{sec:New-6-5-0}Version 6.5.0}
%%%%%%%%%%%%%%%%%%%%%%%%%%%%%%%%%%%%%%%%%%%%%%%%%%%%%%%%%%%%%%%%%%%%%%
\noindent New Features:
\begin{itemize}

\item A new log\_xml option has been added to condor\_submit. It is
documented in the condor\_submit portion of the manual.

\item A new DAGMan option to produce dot files was added. Dot is a
program that creates visualizations of DAGs. This feature is
documented in Section~\ref{sec:DAGMan}.

\item The email report from condor_preen is now less cryptic, and
more self-explanatory.

\end{itemize}

\noindent Bugs Fixed:
\begin{itemize}

\item 

\item 

\end{itemize}

\noindent Known Bugs:
\begin{itemize}
\item 

\item 

\end{itemize}

%%%%%%%%%%%%%%%%%%%%%%%%%%%%%%%%%%%%%%%%%%%%%%%%%%%%%%%%%%%%%%%%%%%%%%%
\section{\label{sec:History-6-4}Stable Release Series 6.4}
%%%%%%%%%%%%%%%%%%%%%%%%%%%%%%%%%%%%%%%%%%%%%%%%%%%%%%%%%%%%%%%%%%%%%%

This is the stable release series of Condor.
New features will be added and tested in the 6.5 development series. 
The details of each version are described below.
%%%%%%%%%%%%%%%%%%%%%%%%%%%%%%%%%%%%%%%%%%%%%%%%%%%%%%%%%%%%%%%%%%%%%%
\subsection{\label{sec:New-6-4-3}Version 6.4.3}
%%%%%%%%%%%%%%%%%%%%%%%%%%%%%%%%%%%%%%%%%%%%%%%%%%%%%%%%%%%%%%%%%%%%%%

\noindent New Features:
\begin{itemize}

\item Added a \Opt{-hold} and \Opt{-held} option to \Condor{q} which 
displays the reason that the job had been held.

\end{itemize}

\noindent Bugs Fixed:
\begin{itemize}

\item Fixed a bug where more than one space between arguments to a job
in the java universe would result in it being invoked with and incorrect
list arguments.

\item Removed renaming of the executable to ``condor_exec'' in the java
universe. This fixes a bug where the JVM was looking at its path to determine
its installation directory.

\item Fixed a bug and resulting null pointer exception in the java universe
because under certain conditions, Condor would invoke the JVM incorrectly.

\item Fixed serveral error reporting messages to be more precise.

\item When the NIS environment was being used, the \Condor{starter} daemon
would produce heavy amounts of NIS traffic. This has been fixed.

\item Binary characters in the \File{StarterLog} file and a possible
segmentation fault have been fixed.

\item Fixed \Cmd{select}{2} in the standard universe on our Linux ports.

\item Fixed a small bug in \Condor{q} that was displaying the wrong
username for ``niceuser'' jobs.

\item Fixed a bug where, in the standard universe, you could not open a file
whose name had spaces in it.

\item Fixed a bug in DAGMan where pre and post scripts would fail to
run if the DAG description file had extra whitespace.
Also, reworded the error messages DAGMan produces when it fails to
parse the DAG description file to be more clear and helpful for
solving the problem.

\item Fixed some misleading error messages in the Condor log files
when there were permission problems trying to execute a program. 

\end{itemize}

\noindent Known Bugs:
\begin{itemize}

\item You may not open a file in the standard universe whose name contains a
colon ``:''.

\end{itemize}

%%%%%%%%%%%%%%%%%%%%%%%%%%%%%%%%%%%%%%%%%%%%%%%%%%%%%%%%%%%%%%%%%%%%%%
\subsection{\label{sec:New-6-4-2}Version 6.4.2}
%%%%%%%%%%%%%%%%%%%%%%%%%%%%%%%%%%%%%%%%%%%%%%%%%%%%%%%%%%%%%%%%%%%%%%
\noindent New Features:
\begin{itemize}

\item. This release mirrored the Condor-G release, and has no new features.

\end{itemize}

\noindent Bugs Fixed:
\begin{itemize}
\item None.

\end{itemize}
\noindent Known Bugs:
\begin{itemize}

\item None.

\end{itemize}

%%%%%%%%%%%%%%%%%%%%%%%%%%%%%%%%%%%%%%%%%%%%%%%%%%%%%%%%%%%%%%%%%%%%%%
\subsection{\label{sec:New-6-4-1}Version 6.4.1}
%%%%%%%%%%%%%%%%%%%%%%%%%%%%%%%%%%%%%%%%%%%%%%%%%%%%%%%%%%%%%%%%%%%%%%
\noindent New Features:
\begin{itemize}

\item None.

\end{itemize}

\noindent Bugs Fixed:
\begin{itemize}

\item Users are now allowed to answer ``none'' when prompted by the
installer to provide a Java JVM path. This avoids an endless loop and
leaves the Java abilities of Condor unconfigured.

\end{itemize}

\noindent Known Bugs:
\begin{itemize}

\item None.

\end{itemize}

%%%%%%%%%%%%%%%%%%%%%%%%%%%%%%%%%%%%%%%%%%%%%%%%%%%%%%%%%%%%%%%%%%%%%%
\subsection{\label{sec:New-6-4-0}Version 6.4.0}
%%%%%%%%%%%%%%%%%%%%%%%%%%%%%%%%%%%%%%%%%%%%%%%%%%%%%%%%%%%%%%%%%%%%%%
\noindent New Features:
\begin{itemize}

\item 

\item 

\end{itemize}

\noindent Bugs Fixed:
\begin{itemize}

\item 

\item 

\end{itemize}

\noindent Known Bugs:
\begin{itemize}
\item 

\item 

\end{itemize}



%%%%%%%%%%%%%%%%%%%%%%%%%%%%%%%%%%%%%%%%%%%%%%%%%%%%%%%%%%%%%%%%%%%%%%%
\section{\label{sec:History-6-3}Development Release Series 6.3}
%%%%%%%%%%%%%%%%%%%%%%%%%%%%%%%%%%%%%%%%%%%%%%%%%%%%%%%%%%%%%%%%%%%%%%

This is the second development release series of Condor.

It contains numerous enhancements over the 6.2 stable series.
For example:

\begin{itemize}

\item 
Condor DAGMan is dramatically more reliable and efficient, and offers
a number of new features.

\end{itemize}

The 6.3 series has many other improvements over the 6.2 series, and
may be available on newer platforms.  The new features, bugs fixed,
and known bugs of each version are described below in detail.


%%%%%%%%%%%%%%%%%%%%%%%%%%%%%%%%%%%%%%%%%%%%%%%%%%%%%%%%%%%%%%%%%%%%%%
\subsubsection{\label{sec:New-6-3-1}Version 6.3.1}
%%%%%%%%%%%%%%%%%%%%%%%%%%%%%%%%%%%%%%%%%%%%%%%%%%%%%%%%%%%%%%%%%%%%%%

\begin{itemize}

\item
More Condor DAGMan improvements and bug fixes:

\begin{itemize}

\item
Added a new event to the Condor userlog at the completion of a POST
script.  This allows DAGMan, during recovery, to know which POST
scripts have finished succesfully, so it no longer has to re-run them
all to make sure.

\item
Implemented separate \Arg{-MaxPre} and \Arg{-MaxPost} options to limit
the number of simultaneously running PRE and POST scripts.  The
\Arg{-MaxScripts} option is still available, and is equivalent to
setting both \Arg{-MaxPre} and \Arg{-MaxPost} to the same value.

\item
Fixed a bug whereby DAGMan would clean up its lock file without
creating a rescue file when killed with SIGTERM.

\item
DAGMan no longer aborts the DAG if it encounters executable error or
job aborted events in the userlog, but rather marks the corresponding
DAG nodes as ``failed'' so the rest of the DAG can continue.

\end{itemize}

\end{itemize}


%%%%%%%%%%%%%%%%%%%%%%%%%%%%%%%%%%%%%%%%%%%%%%%%%%%%%%%%%%%%%%%%%%%%%%
\subsubsection{\label{sec:New-6-3-0}Version 6.3.0}
%%%%%%%%%%%%%%%%%%%%%%%%%%%%%%%%%%%%%%%%%%%%%%%%%%%%%%%%%%%%%%%%%%%%%%

\begin{itemize}

\item
Many Condor DAGMan improvements and bug fixes:

\begin{itemize}

\item
PRE and POST scripts now run asynchronously, rather than synchronously
as in the past.  As a result, DAGMan now supports a \Arg{-MaxScripts}
option to limit the number of simultaneously running PRE and POST
scripts.

\item
Whether or not POST scripts are always executed after failed jobs is
now configurable with the \Arg{-NoPostFail} argument.

\item
Added a \Arg{-r} flag to \Condor{submit\_dag} to submit DAGMan to a
remote \Condor{schedd}.

\item
Made the arguments to \Condor{submit\_dag} case-insensitive.

\item
Fixed a variety of bugs in DAGMan's event handling, so DAGMan should
no longer hang indefinitely after failed jobs, or mistake one job's
userlog events for those of another.

\item
DAGMan's error handling and logging output have been substantially
clarified and improved.  For example, DAGMan now prints a list of
failed jobs when it exits, rather than just saying ``some jobs
failed''.

\item
Jobs submitted by a \Condor{dagman} job now have \AdAttr{DAGManJobId}
and \AdAttr{DAGNodeName} in the job classad.

\item
Fixed a \Condor{submit\_dag} bug preventing the submission of DAGMan
Rescue files.

\item
Improved the handling of userlog errors (less crashing, more coping).

\item
Fixed a bug when recovering from the userlog after a crash or reboot.

\item
Fixed bugs in the handling of \Arg{-MaxJobs}.

\end{itemize}

\item
Added a \Arg{-a line} argument to \Condor{submit} to add a line to the
submit file before processing (overriding the submit file).

\item
Added a \Arg{-dag} flag to \Condor{q} to format and sort DAG jobs
sensibly under their DAGMan master job.

\end{itemize}

\noindent Known Bugs:

\begin{itemize}

\item None.

\end{itemize}

%%%%%%%%%%%%%%%%%%%%%%%%%%%%%%%%%%%%%%%%%%%%%%%%%%%%%%%%%%%%%%%%%%%%%%%
\section{\label{sec:History-6-2}Stable Release Series 6.2}
%%%%%%%%%%%%%%%%%%%%%%%%%%%%%%%%%%%%%%%%%%%%%%%%%%%%%%%%%%%%%%%%%%%%%%

This is the second stable release series of Condor.
All of the new features developed in the 6.1 series are now considered
stable, supported features of Condor.
New releases of 6.2.0 should happen infrequently and will only include
bug fixes and support for new platforms.
New features will be added and tested in the 6.3 development series. 
The details of each version are described below.

%%%%%%%%%%%%%%%%%%%%%%%%%%%%%%%%%%%%%%%%%%%%%%%%%%%%%%%%%%%%%%%%%%%%%%
\subsection{\label{sec:New-6-2-1}Version 6.2.1}
%%%%%%%%%%%%%%%%%%%%%%%%%%%%%%%%%%%%%%%%%%%%%%%%%%%%%%%%%%%%%%%%%%%%%%

\noindent New Features:

\begin{itemize}

\item The \Condor{userlog} command is now available on Windows NT.

\item Jobs run in stand-alone checkpointing mode can now take a -\_condor\_nowarn
argument, which silences the warnings from the system call library when you
perform a checkpoint-unsafe action, such as opening a file for reading and
writing.

\end{itemize}

\noindent Bugs Fixed:

\begin{itemize}

\item The entries in the \AdAttr{environment} 
option in a submit file now correctly override the variables brought in
from the \AdAttr{getenv} option on Windows NT.
In previous version of CondorNT, the job would get an environment with the
variable defined multiple times. This bug did not affect UNIX versions of 
Condor.

\item Some service packs of Windows NT had bugs that prevented Condor
from determining the file permissions on input and output files. 6.2.1
uses a different set of API's to determine the permissions and works 
properly across all service packs

\item In versions of Condor previous to 6.2.0, the registry would slowly
grow on Windows NT and sometimes become corrupted. This was fixed in 6.2.0,
but if a previously-corrupted registry was detected Condor aborted. In 6.2.1,
this has been turned into a warning, as it doesn't need to be a fatal error.

\item Fixed a memory-corruption bug in the \Condor{collector}

\item PVM resources in Condor were unable to have more than one
\Attr{@} symbol in a name. 

\item The \Attr{TRANSFER\_FILES} is now set to \Attr{ON\_EXIT}
on UNIX by default for the vanilla universe. Previously, users submitting
from UNIX to NT needed to explicitly enable it or include the executable in
the list of input files for the job to run.

\item If \Attr{TRANSFER\_FILES} was set to \AdAttr{TRUE}
files created during the job's run would be transfered whenever the job was 
vacated and transfered to the next machine the job ran on, but would not be
transfered back to the submit machine when the job finally exited for the last time.

\item Determining the current working directory was broken in stand-alone
checkpointing. 

\item A job's standard output and standard error can now go to the same file.

\item When the \Macro{START\_HAS\_BAD\_UTMP} is set to TRUE, the
\Condor{startd} now detects activity on the 
\begin{verbatim}/dev/pts\end{verbatim} devices.  

\item The \Condor{negotiator} in 6.2.0 could incorrectly reject a job
that should have been successfully matched if it previously rejected a 
job. If the same jobs were sent to the \Condor{negotiator} in a different
order, the match that should succeed would. In 6.2.1, the order is no longer
important, and previous rejections will not prevent future matches.

\item The getdents, getdirents, and statfs system calls now work correctly in 
cross-platform submissions.

\item \Condor{compile} is better able to detect which version of Linux
it is running on and which flags it should pass to the linker. This should
help Condor users on non-Red Hat distributions.

\item Fixed a bug in the \Condor{startd} that would cause the daemon
to crash if you set the \Macro{POLLING\_INTERVAL} macro to a value
greater than 60.

\item In \Condor{q}, dash-arguments (e.g., -pool, -run, etc.) were being
parsed incorrectly such that the same arguments specified without a
dash would be interpreted as if the dash were present, making it
impossible to specify ``pool'' or ``globus'' or ``run'' as an owner
argument.

\item Fixed bug in \Condor{submit} that would cause certain submit
file directives to be silently ignored if you used the wrong attribute
name.  
Now, all submit file attributes can use the same names you see in the
job ClassAd (what you'd see with \Condor{q} \Opt{-long}.
For example, you can now use ``CoreSize = 0''  or ``core\_size = 0''
in your submit file, and either one would be recognized.

\item A static limit on the number of clusters the \Condor{schedd}
would accept from the \Condor{negotiator} was removed.

\item On Windows NT, if a job's log file was in a non-existent location,
both the \Condor{submit} and the \Condor{schedd} would crash.

\item Encounting unsupported system calls could cause Condor to corrupt the
signal state of the job. 

\item Fixed some of the error messages in \Condor{submit} so that they
are all consistently formatted.

\item Fixed a bug in the Linux standard universe where \Cmd{calloc}{2}
would not return zero filled memory.

\item \Condor{rm}, \Condor{hold} and \Condor{release} will now return
a non-zero exit status on failure, and only return 0 on success.
Previously, they always returned status 0.

\item If a user accidentally put \begin{verbatim}notify_user = false\end{verbatim} in their submit file, Condor used to treat that
as a valid entry.
Now, \Condor{submit} prints out a warning in this case, telling the
user that they probably want to use 
\begin{verbatim}notification = never\end{verbatim} instead.

\end{itemize}

\noindent Known Bugs:

\begin{itemize}

\item It may be possible to checkpoint with an open socket on IRIX 6.2.
On restart, the job will abort and go back into the queue. 

\end{itemize}

%%%%%%%%%%%%%%%%%%%%%%%%%%%%%%%%%%%%%%%%%%%%%%%%%%%%%%%%%%%%%%%%%%%%%%
\subsubsection{\label{sec:New-6-2-0}Version 6.2.0}
%%%%%%%%%%%%%%%%%%%%%%%%%%%%%%%%%%%%%%%%%%%%%%%%%%%%%%%%%%%%%%%%%%%%%%

\noindent New Features Over the 6.0 Release Series
\begin{itemize}

\item Support for running multiple jobs on SMP (Symmetric Multi-Processor)
machines.

\end{itemize}

\noindent New Features Over the Last Development Series: 6.1.17
\begin{itemize}

\item If \Attr{CkptArch} isn't specified in the job submission file's
\Attr{Requirements} attribute, then automatically add this expression:

\begin{verbatim}
CkptRequirements = ((CkptArch == Arch) || (CkptArch =?= UNDEFINED)) &&
	((CkptOpSys == OpSys) || (CkptOpSys =?= UNDEFINED))
\end{verbatim}

to the \Attr{Requirements} expression. This allows for users who specify
a heterogeneous submission to not have to worry about having their checkpoints
incorrectly starting up on architectures for which they were not designed
to run.

\item The \Macro{APPEND\_REQ\_<universe>} config file entries now get
appended to the beginning of the expressions before Condor adds internal
default expressions.  This allows the sysadmin to override any default
policy that Condor enforces.

\item There is now a single \Macro{APPEND\_REQUIREMENTS} attribute
that will get appended to all universe's \Attr{Requirements}
expressions unless a specific \Macro{APPEND\_REQ\_STANDARD} or
\Macro{APPEND\_REQ\_VANILLA} expression is defined.

\item Increased certain networking parameters to help alleviate the 
\Condor{shadow}'s inability to contact the \Condor{schedd} during heavy load
of the system.

\item Added a \Condor{glidein} man page to the manual.

\item Some of the log messages in the \Condor{startd} were modified to
be more clear and to provide more information.

\item Added a new attribute to the \Condor{startd} ClassAd when the
machine is claimed, \AdAttr{RemoteOwner}.

\end{itemize}

\noindent Bugs fixed since 6.1.17
\begin{itemize}

\item On NT, the Registry would increase in size while Condor was
servicing jobs. This has been fixed.

\item Added \File{utmpx} support for Solaris 2.8 to fix a problem where
\AdAttr{KeyBoardIdle} wasn't being set correctly.

\item When doing a \Condor{hold} under NT, the job was removed instead of
held. This has been fixed.

\item When using the \Arg{-master} argument to\Condor{restart}, the
\Condor{master} used to exit instead of restarting.
Now, the \Condor{master} correctly restarts itself in this case.

\end{itemize}

\noindent Known Bugs:
\begin{itemize}

\item \Attr{STARTD\_HAS\_BAD\_UTMP} does not work if set to True on Solaris 
2.8.  However, since \File{utmpx} support is enabled, you shouldn't
need to do this normally.

\item \Condor{kbdd} doesn't work properly under Compaq Tru64 5.1, and
as a result, resources may not leave the ``Unclaimed'' state
regardless of keyboard or pty activity.  Compaq Tru64 5.0a and earlier
do work properly.

\end{itemize}

%%%%%%%%%%%%%%%%%%%%%%%%%%%%%%%%%%%%%%%%%%%%%%%%%%%%%%%%%%%%%%%%%%%%%%%
\section{\label{sec:History-6-1}Development Release Series 6.1}
%%%%%%%%%%%%%%%%%%%%%%%%%%%%%%%%%%%%%%%%%%%%%%%%%%%%%%%%%%%%%%%%%%%%%%

This was the first development release series.
It contains numerous enhancements over the 6.0 stable series.
For example:

\begin{itemize}
\item Support for running multiple jobs on SMP machines
\item Enhanced functionality for pool administrators
\item Support for PVM, MPI and Globus jobs
\item Support for \Term{Flocking} jobs across different Condor pools
\end{itemize}

The 6.1 series has many other improvements over the 6.0 series, and  
is available on more platforms.  
The new features, bugs fixed, and known bugs of each version are
described below in detail.

%%%%%%%%%%%%%%%%%%%%%%%%%%%%%%%%%%%%%%%%%%%%%%%%%%%%%%%%%%%%%%%%%%%%%%
\subsection*{\label{sec:New-6-1-17}Version 6.1.17}
%%%%%%%%%%%%%%%%%%%%%%%%%%%%%%%%%%%%%%%%%%%%%%%%%%%%%%%%%%%%%%%%%%%%%%

This version is the 6.2.0 ``release candidate''.  
It was publically released in Feburary of 2001, and it will be released
as 6.2.0 once it is considered ``stable'' by heavy testing at the 
UW-Madison Computer Science Department Condor pool.

\noindent New Features:

\begin{itemize}

\item Hostnames in the HOSTALLOW and HOSTDENY entries are now case-insensitive.

\item It is now possible to submit NT jobs from a UNIX machine.

\item The NT release of Condor now supports a USE\_VISIBLE\_DESKTOP parameter. 
If true, Condor will allow the job to create windows on the desktop of the
execute machine and interact with the job. This is particularly useful for 
debugging why an application will not run under Condor.

\item The \Condor{startd} contains support for the new MPI dedicated 
scheduler that will appear in the 6.3 development series. This will allow
you to use your 6.2 Condor pool with the new scheduler.

\item Added a \Opt{mixedcase} option to \Condor{config\_val} to allow 
for overriding the default of lowercasing all the config names

\item Added a pid\_snapshot\_interval option to the config file to
control how often the \Condor{startd} should examine the running 
process family. It defaults to 50 seconds.

\end{itemize}

\noindent Bugs Fixed:

\begin{itemize}

\item Fixed a bug with the \Condor{schedd} reaching the MAX\_JOBS\_RUNNING
mark and properly calculating Scheduler Universe jobs for preemption.

\item Fixed a bug in the \Condor{schedd} loosing track of \Condor{startd}s 
in the initial claiming phase. This bug affected all platforms, but was most
likely to manifest on Solaris 2.6

\item CPU Time can be greater than wall clock time in Multi-threaded
apps, so do not consider it an error in the UserLog.

\item \Condor{restart} \Opt{-master} now works correctly.
 
\item Fixed a rare condition in the \Condor{startd} that could corrupt
memory and result in a signal 11 (SIGSEGV, or segmentation violation).

\item Fixed a bug that would cause the ``execute event'' to not be
logged to the UserLog if the binary for the job resided on AFS.

\item Fixed a race-condition in Condor's PVM support on SMP machines
(introduced in version 6.1.16) that caused PVM tasks to be associated
with the wrong daemon.

\item Better handling of checkpointing on large-memory Linux machines.

\item Fixed random occasions of job completion email not being sent.

\item It is no longer possible to use \Condor{user\_prio} to set a priority of less
than 1.

\item Fixed a bug in the job completion email statistics.
Run Time was being underreported when the job completed after doing a
periodic checkpoint.

\item Fixed a bug that caused CondorLoadAvg to get stuck at 0.0 on
Linux when the system clock was adjusted.

\item Fixed a \Condor{submit} bug that caused all machine\_count
commands after the first queue statement to be ignored for PVM jobs.

\item PVM tasks now run as the user when appropriate instead of always
running under the UNIX ``nobody'' account.

\item Fixed support for the PVM group server.

\item PVM uses an environment variable to communicate with it's children
instead of a file in /tmp. This file previously could become overwritten
by mulitple PVM jobs.

\item \Condor{stats} now lives in the ``bin'' directory instead of ``sbin''.

\end{itemize}

\noindent Known Bugs:

\begin{itemize}

\item The \Condor{negotiator} can crash if the Accountantnew.log file becomes
corrupted. This most often occurs if the Central Manager runs out of diskspace. 

\end{itemize}

%%%%%%%%%%%%%%%%%%%%%%%%%%%%%%%%%%%%%%%%%%%%%%%%%%%%%%%%%%%%%%%%%%%%%%
\subsection*{\label{sec:New-6-1-16}Version 6.1.16}
%%%%%%%%%%%%%%%%%%%%%%%%%%%%%%%%%%%%%%%%%%%%%%%%%%%%%%%%%%%%%%%%%%%%%%

\noindent New Features:

\begin{itemize}

\item Condor now supports multiple pvmds per user on a machine.  Users
can now submit more than one PVM job at a time, PVM tasks can now run
on the submission machine, and multiple PVM tasks can run on SMP
machines.  \Condor{submit} no longer inserts default job requirements
to restrict PVM jobs to one pvmd per user on a machine.  This new
functionality requires the \Condor{pvmd} included in this (and future)
Condor releases.  If you set ``PVM\_OLD\_PVMD = True'' in the Condor
configuration file, \Condor{submit} will insert the default PVM job
requirements as it did in previous releases.  You must set this if you
don't upgrade your \Condor{pvmd} binary or if your jobs flock with pools
that user an older \Condor{pvmd}.

\item The NT release of Condor no longer contains debugging
information.
This drastically reduces the size of the binaries you must install.  

\end{itemize}

\noindent Bugs Fixed:

\begin{itemize}

\item The configuration files shipped with version 6.1.15 contained a
number of errors relating to host-based security, the configuration of
the central manager, and a few other things.
These errors have all been corrected.

\item Fixed a memory management bug in the \Condor{schedd} that could
cause it to crash under certain circumstances when machines were taken
away from the schedd's control.

\item Fixed a potential memory leak in a library used by the
\Condor{startd} and \Condor{master} that could leak memory while
Condor jobs were executing.

\item Fixed a bug in the NT version of Condor that would result in
faulty reporting of the load average.

\item The \Condor{shadow.pvm} should now correctly return core files
when a task or \Condor{pvmd} crashes.

\item This release fixes a memory error introduced in version
6.1.15 that could crash the \Condor{shadow.pvm}.

\item Some \Condor{pvmd} binaries in previous releases included
debugging code we added that could cause the \Condor{pvmd} to crash.
This release includes new \Condor{pvmd} binaries for all platforms
with the problematic debugging code removed.

\item Fixed a bug in the \Opt{-unset} options to \Condor{config\_val}
that was introduced in version 6.1.15.
Both \Opt{-unset} and \Opt{-runset} work correctly, now.

\end{itemize}

\noindent Known Bugs:

\begin{itemize}

\item None.

\end{itemize}

%%%%%%%%%%%%%%%%%%%%%%%%%%%%%%%%%%%%%%%%%%%%%%%%%%%%%%%%%%%%%%%%%%%%%%
\subsection*{\label{sec:New-6-1-15}Version 6.1.15}
%%%%%%%%%%%%%%%%%%%%%%%%%%%%%%%%%%%%%%%%%%%%%%%%%%%%%%%%%%%%%%%%%%%%%%

\noindent New Features:

\begin{itemize}

\item In the job submit description file passed to \Condor{submit}, 
a new style of macro (with two dollar-signs) can reference attributes
from the machine ClassAd.  This new style macro can be used in the
job's \MacroNI{Executable}, \MacroNI{Arguments}, or \MacroNI{Environment}
settings in the submit description file.  For example, if you have both
Linux and Solaris machines in your pool, the following submit description
file will run either foo.INTEL.LINUX or foo.SUN4u.SOLARIS27 as appropiate,
and will pass in the amount of memory available on that machine on the
command line:
\begin{verbatim}
	executable = foo.$$(Arch).$$(Opsys)
	arguments = $$(Memory)
	queue
\end{verbatim}

\item The \DCPerm{CONFIG} security access level now controls the
modification of daemon configurations using \Condor{config\_val}.  For
more information about security access levels, see
section~\ref{sec:Host-Security} on
page~\pageref{sec:Host-Security}.

\item The \Macro{DC\_DAEMON\_LIST} macro now indicates to the
\Condor{master} which processes in the \Macro{DAEMON\_LIST} use
Condor's DaemonCore inter-process communication mechanisms.  This
allows the \Condor{master} to monitor both processes developed with or
without the Condor DaemonCore library.

\item The new \Macro{NEGOTIATE\_ALL\_JOBS\_IN\_CLUSTER} macro can be
use to configure the \Condor{schedd} to not assume (for efficiency)
that if one job in a cluster can't be scheduled, then no other jobs in
the cluster can be scheduled.
If \Macro{NEGOTIATE\_ALL\_JOBS\_IN\_CLUSTER} is set to True, the
\Condor{schedd} will now always try to schedule each individual job in
a cluster.

\item The \Condor{schedd} now automatically adds any machine it is
matched with to its HOSTALLOW\_WRITE list.
This simplifies setting up a machine for flocking, since the
submitting user doesn't have to know all the machines where the job
might execute, they only have to know what central manager they wish
to flock to.
Submitting users must trust a central manager they report to, so this
doesn't impact security in any way.

\item Some static limits relating to the number of jobs which can be 
simultaneously started by the \Condor{schedd} has been removed.

\item The default Condor config file(s) which are installed by
the installation program have been re-organized for greater 
clarity and simplicity.  

\end{itemize}

\noindent Bugs Fixed:

\begin{itemize}

\item In the STANDARD Universe, jobs submitted to Condor could segfault
if they opened multiple files with the same name.  Usually this bug
was exposed when users would submit jobs without specifying a file
for either stdout or stderr; in this case, both would default to 
\File{/dev/null}, and this could trigger the problem.

\item The Linux 2.2.14 kernel, which is used by default with Red Hat 6.2,
has a serious bug can cause the machine to lock up when 
the same socket is used for repeated connection attempts.   Thus, 
previous versions of Condor could cause the 2.2.14 kernel to hang
(lots of other applications could do this as well).  The Condor Team
recommends that you upgrade your kernel to 2.2.16 or later.  However,
in v6.1.15 of Condor, a patch was added to the Condor networking
layer so that Condor would not trigger this Linux kernel bug.

\item If no email address was specified when the job was submitted
with \Condor{submit}, completion email was being sent to 
user@submit-machine-hostname.  This is not the correct behavior.  Now 
email goes by default to user@uid-domain, where uid-domain is
defined by the \MacroNI{UID\_DOMAIN} setting in the config file.

\item The \Condor{master} can now correctly shutdown and restart the
\Condor{checkpoint\_server}.

\item Email sent when a SCHEDULER Universe job compeltes now has the
correct From: header.

\item In the STANDARD universe, jobs which call sigsuspend() will 
now receive the correct return value.

\item Abnormal error conditions, such as the hard disk on the submit
machine filling up, are much less likely to result in a job disappearing
from the queue.

\item The \Condor{checkpoint\_server} now correctly reconfigures when
a \Condor{reconfig} command is received by the \Condor{master}.

\item Fixed a bug with how the \Condor{schedd} associates jobs with
machines (claimed resources) which would, under certain circumstances,
cause some jobs to remain idle until other jobs in the queue complete
or are preempted.

\item A number of PVM universe bugs are fixed in this release.
Bugs in how the \Condor{shadow.pvm} exited, which caused jobs to hang
at exit or to run multiple times, have been fixed.
The \Condor{shadow.pvm} no longer exits if there is a problem starting
up PVM on one remote host.
The \Condor{starter.pvm} now ignores the periodic checkpoint command
from the startd.  Previously, it would vacate the job when it received
the periodic checkpoint command.
A number of bugs with how the \Condor{starter.pvm} handled
asynchronous events, which caused it to take a long time to clean up
an exited PVM task, have been fixed.
The \Condor{schedd} now sets the status correctly on multi-class PVM
jobs and removes them from the job queue correctly on exit.
\Condor{submit} no longer ignores the machine\_count command for PVM
jobs.
And, a problem which caused pvm\_exit() to hang was diagnosed:
PVM tasks which call pvm\_catchout() to catch the output of
child tasks should be sure to call it again with a NULL argument to
disable output collection before calling pvm\_exit().

\item The change introduced in 6.1.13 to the \Condor{shadow} regarding
when it logged the execute event to the user log produced situations
where the shadow could log other events (like the shadow exception
event) before the execute event was logged.
Now, the \Condor{shadow} will always log an execute event before it
logs any other events.
The timing is still improved over 6.1.12 and older versions, with the
execute event getting logged after the bulk of the job initialization
has finished, right before the job will actually start executing.
However, you will no longer see user logs that contain a ``shadow
exception'' or ``job evicted'' message without a ``job executing''
event, first.

\item \Syscall{stat} and varient calls now go through the file table to
get the correct logical size and access times of buffered files.
Before, \Syscall{stat} used to return zero size on a buffered file that had
not yet been synced to disk.

\end{itemize}

\noindent Known Bugs:

\begin{itemize}

\item On IRIX 6.2, C++ programs compiled with GNU C++ (g++) 2.7.2 and
linked with the Condor libraries (using \Condor{compile}) will not
execute the constructors for any global objects.
There is a work-around for this bug, so if this is a problem for you,
please send email to \Email{condor-admin@cs.wisc.edu}.

\item In HP-UX 10.20, \Condor{compile} will not work correctly with HP's
C++ compiler. 
The jobs might link, but they will produce incorrect output, or die with
a signal such as SIGSEGV during restart after a checkpoint/vacate cycle.
However, the GNU C/C++ and the HP C compilers work just fine.

\item The \Syscall{getrusage} call does not work always as expected in
STANDARD Universe jobs.  
If your program uses \Syscall{getrusage}, it 
could decrease incorrectly by a second
across a checkpoint and restart.  In addition, the time it takes
Condor to restart from a checkpoint is included in the usage times
reported by \Syscall{getrusage}, and it probably should not be.

\end{itemize}


%%%%%%%%%%%%%%%%%%%%%%%%%%%%%%%%%%%%%%%%%%%%%%%%%%%%%%%%%%%%%%%%%%%%%%
\subsection*{\label{sec:New-6-1-14}Version 6.1.14}
%%%%%%%%%%%%%%%%%%%%%%%%%%%%%%%%%%%%%%%%%%%%%%%%%%%%%%%%%%%%%%%%%%%%%%

\noindent New Features:

\begin{itemize}

\item Initial supported added for Red Hat Linux 6.2 (i.e. glibc 2.1.3).

\end{itemize}

\noindent Bugs Fixed:

\begin{itemize}

\item In version 6.1.13, periodic checkpoints would not occur (see the
Known Bugs section for v6.1.13 listed below).  This bug, which only
impacts v6.1.13, has been fixed.

\end{itemize}

\noindent Known Bugs:

\begin{itemize}

\item The \Syscall{getrusage} call does not work properly inside
``standard'' jobs.  
If your program uses \Syscall{getrusage}, it will not report correct values
across a checkpoint and restart.
If your program relies on proper reporting from \Syscall{getrusage}, you
should either use version 6.0.3 or 6.1.10.

\item While Condor now supports many networking calls such as
\Syscall{socket} and \Syscall{connect}, (see the description below of this
new feature added in 6.1.11), on Linux, we cannot at this time support
\Syscall{gethostbyname} and a number of other database lookup calls.
The reason is that on Linux, these calls are implemented by bringing in a
shared library that defines them, based on whether the machine is using
DNS, NIS, or some other database method.
Condor does not support the way in which the C library tries to explicitly
bring in these shared libraries and use them.
There are a number of possible solutions to this problem, but the Condor
developers are not yet agreed on the best one, so this limitation might not
be resolved by 6.1.14.

\item In HP-UX 10.20, \Condor{compile} will not work correctly with HP's
C++ compiler. 
The jobs might link, but they will produce incorrect output, or die with
a signal such as SIGSEGV during restart after a checkpoint/vacate cycle.
However, the GNU C/C++ and the HP C compilers work just fine.

\item When a program linked with the Condor libraries (using \Condor{compile})
is writing output to a file, \Syscall{stat}--and variant calls,
will return zero for the size of the file if the program has not yet
read from the file or flushed the file descriptors.
This is a side effect of the file buffering code in Condor and will be
corrected to the expected semantic.

\item On IRIX 6.2, C++ programs compiled with GNU C++ (g++) 2.7.2 and
linked with the Condor libraries (using \Condor{compile}) will not
execute the constructors for any global objects.
There is a work-around for this bug, so if this is a problem for you,
please send email to \Email{condor-admin@cs.wisc.edu}.

\end{itemize}
%%%%%%%%%%%%%%%%%%%%%%%%%%%%%%%%%%%%%%%%%%%%%%%%%%%%%%%%%%%%%%%%%%%%%%
\subsection*{\label{sec:New-6-1-13}Version 6.1.13}
%%%%%%%%%%%%%%%%%%%%%%%%%%%%%%%%%%%%%%%%%%%%%%%%%%%%%%%%%%%%%%%%%%%%%%

\noindent New Features:

\begin{itemize}

\item Added \Macro{DEFAULT\_IO\_BUFFER\_SIZE} and
\Macro{DEFAULT\_IO\_BUFFER\_BLOCK\_SIZE} to config parameters to allow
the administrator to set the default file buffer sizes for user jobs
in \Condor{submit}.

\item There is no longer any difference in the configuration file
syntax between ``macros'' (which were specified with an ``='' sign)
and ``expressions'' (which were specified with a ``:'' sign).  
Now, all config file entries are treated and referenced as macros. 
You can use either ``='' or ``:'' and they will work the same way. 
There is no longer any problem with forward-referencing macros
(referencing macros you haven't yet defined), so long as they are
eventually defined in your config files (even if the forward reference
is to a macro defined in another config file, like the local config
file, for example).

\item \Condor{vacate} now supports a \Opt{-fast} option that forces
Condor to hard-kill the job(s) immediately, instead of waiting for
them to checkpoint and gracefully shutdown.

\item \Condor{userlog} now displays times in days+hours:minutes format
instead of total hours or total minutes.

\item The \Condor{run} command provides a simple front-end to
\Condor{submit} for submitting a shell command-line as a vanilla
universe job.

\item Solaris 2.7 SPARC, 2.7 INTEL have been added to the
list of ports that now support remote system calls and checkpointing.

\item Any mail being sent from Condor now shows up as having been sent from
the designated Condor Account, instead of root or ``Super User''.

\item The \Condor{submit} ``hold'' command may be used to submit jobs
to the queue in the hold state.  Held jobs will not run until released
with \Condor{release}.

\item It is now possible to use checkpoint servers in remote pools
when flocking even if the local pool doesn't use a checkpoint server.
This is now the default behavior (see the next item).

\item \Macro{USE\_CKPT\_SERVER} now defaults to True if a checkpoint
server is available.  It is usually more efficient to use a checkpoint
server near the execution site instead of storing the checkpoint back
to the submission machine, especially when flocking.

\item All Condor tools that used to expect just a hostname or address 
(\Condor{checkpoint}, \Condor{off}, \Condor{on}, \Condor{restart},
\Condor{reconfig}, \Condor{reschedule}, \Condor{vacate}) to specify
what machine to effect, can now take an optional \Opt{-name} or
\Opt{-addr} in front of each target.
This provides consistancy with other Condor tools that require the
\Opt{-name} or \Opt{-addr} options.
For all of the above mentioned tools, you can still just provide
hostnames or addresses, the new flags are not required.

\item Added \Opt{-pool} and \Opt{-addr} options to \Condor{rm},
\Condor{hold} and \Condor{release}.

\item When you start up the \Condor{master} or \Condor{schedd} as any
user other than ``root'' or ``condor'' on Unix, or ``SYSTEM'' on NT,
the daemon will have a default \Attr{Name} attribute that includes
both the username of the user who the daemon is running as and the
full hostname of the machine where it is running.

\item Clarified our Linux platform support.  We now officially
support the Red Hat 5.2 and 6.x distributions, and although other Linux
distributions (especially those with similar libc versions) may work,
they are not tested or supported.

\item The schedd now periodically updates the run-time counters in the
job queue for running jobs, so if the schedd crashes, the counters
will remain relatively up-to-date.  This is controlled by the
\Macro{WALL\_CLOCK\_CKPT\_INTERVAL} parameter.

\item The \Condor{shadow} now logs the ``job executing'' event in the
user log after the binary has been successfully transfered, so that
the events appear closer to the actual time the job starts running.
This can create some somewhat unexpected log files.  
If something goes wrong with the job's initialization, you might see
an ``evicted'' event before you see an ``executing'' event.

\end{itemize}

\noindent Bugs Fixed:

\begin{itemize}

\item Fixed how we internally handle file names for user jobs. This
fixes a nasty bug due to changing directories between checkpoints.

\item Fixed a bug in our handling of the \Macro{Arguments} macro in
the command file for a job. If the arguments were extremely long, or
there were an extreme number of them, they would get corrupted when the
job was spawned.

\item Fixed DAGMan. It had not worked at all in the previous release.

\item Fixed a nasty bug under Linux where file seeks did not work
correctly when buffering was enabled.

\item Fixed a bug where \Condor{shadow} would crash while sending job
completion e-mail forcing a job to restart multiple times and the user
to get multiple completion messages.

\item Fixed a long standing bug where Fortran 90 would occasionally
truncate its output files to random sizes and fill them with zeros.

\item Fixed a bug where \Syscall{close} did not propogate its return
value back to the user job correctly.

\item If a SIGTERM was delivered to a \Condor{shadow}, it used to
remove the job it was running from the job queue, as if \Condor{rm}
had been used.
This could have caused jobs to leave the queue unexpectedly.
Now, the \Condor{shadow} ignores SIGTERM (since the \Condor{schedd}
knows how to gracefully shutdown all the shadows when it gets a
SIGTERM), so jobs should no longer leave the queue prematurely.
In addition, on a SIGQUIT, the shadow now does a fast shutdown, just
like the rest of the Condor daemons.

\item Fixed a number of bugs which caused checkpoint restarts
to fail on some releases of Irix 6.5 (for example, when migrating from
a mips4 to a mips3 CPU or when migrating between machines with
different pagesizes).

\item Fixed a bug in the implementation of the \Syscall{stat} family
of remote system calls on Irix 6.5 which caused file opens in Fortran
programs to sometimes fail.

\item Fixed a number of problems with the statistics reported in the
job completion email and by \Condor{q} \Opt{-goodput}, including the
number of checkpoints and total network usage.  Correct values will
now be computed for all new jobs.

\item Changes in \Macro{USE\_CKPT\_SERVER} and
\Macro{CKPT\_SERVER\_HOST} no longer cause problems for jobs in the
queue which have already checkpointed.

\item Many of the Condor administration tools had a bug where they
would suffer a segmentation violation if you specified a \Opt{-pool} 
option and did not specify a hostname.
This case now results in an error message instead.

\item Fixed a bug where the \Condor{schedd} could die with a
segmentation violation if there was an error mapping an IP address
into a hostname.

\item Fixed a bug where resetting the time in a large negative direction
caused the \Condor{negotiator} to have a floating point error on some
platforms.

\item Fixed \Condor{q}'s output so that certain arguments are not ignored.

\item Fixed a bug in \Condor{q} where issuing a \Opt{-global} with a
fairly restrictive \Opt{-constraint} argument would cause garbage to be
printed to the terminal sometimes.

\item Fixed a bug which caused jobs to exit without completing a
checkpoint when preempted in the middle of a periodic checkpoint.
Now, the jobs will complete their periodic checkpoint in this case
before exiting.
\end{itemize}

\noindent Known Bugs:

\begin{itemize}

\item Periodic checkpoints do not occur.  Normally, when the config
file attribute \Macro{PERIODIC\_CHECKPOINT} evaluates to True, 
Condor performs a periodic checkpoint of the running job.  This
bug has been fixed in v6.1.14.  \Note there is a work-around to permit
periodic checkpoints to occur in v6.1.13: include the attribute name
``PERIODIC\_CHECKPOINT'' to the attributes 
listed in the \Macro{STARTD\_EXPRS} entry in the config file.

\item The \Syscall{getrusage} call does not work properly inside
``standard'' jobs.  
If your program uses \Syscall{getrusage}, it will not report correct values
across a checkpoint and restart.
If your program relies on proper reporting from \Syscall{getrusage}, you
should either use version 6.0.3 or 6.1.10.

\item While Condor now supports many networking calls such as
\Syscall{socket} and \Syscall{connect}, (see the description below of this
new feature added in 6.1.11), on Linux, we cannot at this time support
\Syscall{gethostbyname} and a number of other database lookup calls.
The reason is that on Linux, these calls are implemented by bringing in a
shared library that defines them, based on whether the machine is using
DNS, NIS, or some other database method.
Condor does not support the way in which the C library tries to explicitly
bring in these shared libraries and use them.
There are a number of possible solutions to this problem, but the Condor
developers are not yet agreed on the best one, so this limitation might not
be resolved by 6.1.14.

\item In HP-UX 10.20, \Condor{compile} will not work correctly with HP's
C++ compiler. 
The jobs might link, but they will produce incorrect output, or die with
a signal such as SIGSEGV during restart after a checkpoint/vacate cycle.
However, the GNU C/C++ and the HP C compilers work just fine.

\item When writing output to a file, \Syscall{stat}--and variant calls,
will return zero for the size of the file if the program has not yet
read from the file or flushed the file descriptors,
This is a side effect of the file buffering code in Condor and will be
corrected to the expected semantic.

\item On IRIX 6.2, C++ programs compiled with GNU C++ (g++) 2.7.2 and
linked with the Condor libraries (using \Condor{compile}) will not
execute the constructors for any global objects.
There is a work-around for this bug, so if this is a problem for you,
please send email to \Email{condor-admin@cs.wisc.edu}.

\end{itemize}

%%%%%%%%%%%%%%%%%%%%%%%%%%%%%%%%%%%%%%%%%%%%%%%%%%%%%%%%%%%%%%%%%%%%%%
\subsection*{\label{sec:New-6-1-12}Version 6.1.12}
%%%%%%%%%%%%%%%%%%%%%%%%%%%%%%%%%%%%%%%%%%%%%%%%%%%%%%%%%%%%%%%%%%%%%%

Version 6.1.12 fixes a number of bugs from version 6.1.11.
If you linked your ``standard'' jobs with version 6.1.11, you should
upgrade to 6.1.12 and re-link your jobs (using \Condor{compile}) as soon as
possible.

\noindent New Features:

\begin{itemize}

\item None.

\end{itemize}

\noindent Bugs Fixed:

\begin{itemize}

\item A number of system calls that were not being trapped by the Condor
libraries in version 6.1.11 are now being caught and sent back to the
submit machine.
Not having these functions being executed as remote system calls prevented
a number of programs from working, in particular Fortran programs, and
many programs on IRIX and Solaris platforms.

\item Sometimes submitted jobs report back as having no owner and have
\Bold{-????-} in the status line for the job. This has been fixed.

\item \Condor{q} \Opt{-io} has been fixed in this release.

\end{itemize}

\noindent Known Bugs:

\begin{itemize}

\item The \Syscall{getrusage} call does not work properly inside
``standard'' jobs.  
If your program uses \Syscall{getrusage}, it will not report correct values
across a checkpoint and restart.
If your program relies on proper reporting from \Syscall{getrusage}, you
should either use version 6.0.3 or 6.1.10.

\item While Condor now supports many networking calls such as
\Syscall{socket} and \Syscall{connect}, (see the description below of this
new feature added in 6.1.11), on Linux, we cannot at this time support
\Syscall{gethostbyname} and a number of other database lookup calls.
The reason is that on Linux, these calls are implemented by bringing in a
shared library that defines them, based on whether the machine is using
DNS, NIS, or some other database method.
Condor does not support the way in which the C library tries to explicitly
bring in these shared libraries and use them.
There are a number of possible solutions to this problem, but the Condor
developers are not yet agreed on the best one, so this limitation might not
be resolved by 6.1.13.

\item In HP-UX 10.20, \Condor{compile} will not work correctly with HP's
C++ compiler. 
The jobs might link, but they will produce incorrect output, or die with
a signal such as SIGSEGV during restart after a checkpoint/vacate cycle.
However, the GNU C/C++ and the HP C compilers work just fine.

\item When writing output to a file, \Syscall{stat}--and variant calls,
will return zero for the size of the file if the program has not yet
read from the file or flushed the file descriptors,
This is a side effect of the file buffering code in Condor and will be
corrected to the expected semantic.

\item On IRIX 6.2, C++ programs compiled with GNU C++ (g++) 2.7.2 and
linked with the Condor libraries (using \Condor{compile}) will not
execute the constructors for any global objects.
There is a work-around for this bug, so if this is a problem for you,
please send email to \Email{condor-admin@cs.wisc.edu}.

\item The \Opt{-format} option in \Condor{q} has no effect when querying
remote machines with the \Opt{-n} option.

\item \Condor{dagman} does not work at all in this release. 
The behaviour of its failure is to exit immediately with a success and
to not perform any work. It will be fixed in the next release of Condor.

\end{itemize}


%%%%%%%%%%%%%%%%%%%%%%%%%%%%%%%%%%%%%%%%%%%%%%%%%%%%%%%%%%%%%%%%%%%%%%
\subsection*{\label{sec:New-6-1-11}Version 6.1.11}
%%%%%%%%%%%%%%%%%%%%%%%%%%%%%%%%%%%%%%%%%%%%%%%%%%%%%%%%%%%%%%%%%%%%%%

\noindent New Features:

\begin{itemize}

\item \Condor{status} outputs information for held jobs instead of
MaxRunningJobs when supplied with \Opt{-schedd} or \Opt{-submitter}.

\item \Condor{userprio} now prints 4 digit years (for Y2K compiance). 
If you give a two digit date, it also will assume that 1/1/00 is 1/1/2000
and not 1/1/1900.

\item IRIX 6.5 has been added to the list of ports that now support
remote system calls and checkpointing.

\item \Condor{q} has been fixed to be faster and much more memory
efficient.  This is much more obvious when getting the queue from
\Condor{schedd}'s that have more than 1000 jobs.

\item Added support for support for socket() and pipe() in standard
jobs.  Both sockets and pipes are created on the executing machine.
Checkpointing is deferred anytime a socket or pipe is open.

\item Added limited support for select() and poll() in standard jobs.
Both calls will work only on files opened locally.

\item Added limited support for fcntl() and ioctl() in standard jobs.
Both calls will be performed remotely if the control-number is understood
and the third argument is an integer.

\item Replaced buffer implementation in standard jobs.
The new buffer code reads and writes variable sized chunks.
It will never issue a read to satisfy a write.  Buffering is enabled
by default.

\item Added extensive feedback on I/O performance in the user's email.

\item Added \Opt{-io} option to \Condor{q} to show I/O statistics.

\item Removed libckpt.a and libzckpt.a.  To build for standalone
checkpointing, just do a regular \Condor{compile}.
No -standalone option is necessary.

\item The checkpointing library now only re-opens files when they are
actually used.  If files or other needed resources cannot be found
at restart time, the checkpointer will fail with a verbose error.

\item The \Attr{RemoteHost} and \Attr{LastRemoteHost} attributes in
the job classad now contain hostnames instead IP address and port
numbers.  The \Opt{-run} option of older versions of \Condor{q} is not
compatible with this change.

\item Condor will now automatically check for compatibility between
the version of the Condor libraries you have linked into a standard
job (using \Condor{compile}) and the version of the \Condor{shadow}
installed on your submit machine.
If they are incompatible, the \Condor{shadow} will now put your job on
hold.  
Unless you set ``Notification = Never'' in your submit file, Condor
will also send you email explaining what went wrong and what you can
do about it.

\item All Condor daemons and tools now have a \Attr{CondorPlatform}
string, which shows which platform a given set of Condor binaries was
built for.
In all places that you used to see \Attr{CondorVersion}, you will now
see both \Attr{CondorVersion} and \Attr{CondorPlatform}, such as in
each daemon's ClassAd, in the output to a \Opt{-version} option (if
supported), and when running \Prog{ident} on a given Condor binary. 
This string can help identify situations where you are running the 
wrong version of the Condor binaries for a given platform (for
example, running binaries built for Solaris 2.5.1 on a Solaris 2.6
machine).   

\item Added commented-out settings in the default
\File{condor\_config} file we ship for various SMP-specific settings
in the \Condor{startd}.
Be sure to read section~\ref{sec:Configuring-SMP} on ``Configuring the
Startd for SMP Machine'' on page~\pageref{sec:Configuring-SMP} for
details about using these settings. 

\item \Condor{rm}, \Condor{hold}, and \Condor{release} all support
\Opt{-help} and \Opt{-version} options now.

\end{itemize}

\noindent Bugs Fixed:

\begin{itemize}

\item A race condition which could cause the \Condor{shadow} to not
exit when its job was removed has been fixed.
This bug would cause jobs that had been removed with \Condor{rm} to
remain in the queue marked as status ``X'' for a long time.
In addition, Condor would not shutdown quickly on hosts that had hit
this race condition, since the \Condor{schedd} wouldn't exit until all
of its \Condor{shadow} children had exited.

\item A signal race condition during restart of a Condor job has
been fixed.

\item In a Condor linked job, \Syscall{getdomainname} is now
supported. 

\item IRIX 6.5 can give negative time reports for how long a process has been
running. We account for that now in our statistics about usage times.

\item The \Condor{status} memory error introduced in version 6.1.10
has been fixed.

\item The \Macro{DAEMON\_LIST} configuration setting is now case
insensitive.

\item Fixed a bug where the \Condor{schedd}, under rare circumstances,
cause another schedd's jobs not to be matched.

\item The free disk space is now properly computed on Digital Unix.
This fixed problems where the \Attr{Disk} attribute in the
\condor{startd} classad reported incorrect values.

\item The config file parser now detects incremental macro definitions
correctly (see section~\ref{sec:Config-File-Macros} on
page~\pageref{sec:Config-File-Macros}).  Previously, when a macro (or
expression) being defined was a substring of a macro (or expression)
being referenced in its definition, the reference would be erroneously
marked as an incremental definition and expanded immediately.  The
parser now verifies that the entire strings match.

\end{itemize}

\noindent Known Bugs:

\begin{itemize}

\item The output for \condor{q} \Opt{-io} is incorrect and will likely show
zeroes for all values.  A fixed version will appear in the next release.

\end{itemize}

%%%%%%%%%%%%%%%%%%%%%%%%%%%%%%%%%%%%%%%%%%%%%%%%%%%%%%%%%%%%%%%%%%%%%%
\subsection*{\label{sec:New-6-1-10}Version 6.1.10}
%%%%%%%%%%%%%%%%%%%%%%%%%%%%%%%%%%%%%%%%%%%%%%%%%%%%%%%%%%%%%%%%%%%%%%

\noindent New Features:

\begin{itemize}

\item \Condor{q} now accepts \texttt{-format} parameters like \Condor{status}

\item \Condor{rm}, \Condor{hold} and \Condor{release} accept
  \texttt{-constraint} parameters like \Condor{status}

\item \Condor{status} now sorts displayed totals by the first column.
(This feature introduced a bug in \Condor{status}.  See ``Known Bugs''
below.)

\item Condor version 6.1.10 introduces ``clipped'' support for Sparc
Solaris version 2.7.
This version does not support checkpointing or remote system calls.
Full support for Solaris 2.7 will be released soon.

\item Introduced code to enable Linux to use the standard C library's
I/O buffering again, instead of relying on the Condor I/O buffering
code (which is still in beta testing).  

\end{itemize}

\noindent Bugs Fixed:

\begin{itemize}

\item The bug in checkpointing introduced in version 6.1.9 has been
fixed.
Checkpointing will now work on all platforms, as it always used to.  
Any jobs linked with the 6.1.9 Condor libraries will need to be
relinked with \Condor{compile} once version 6.1.10 has been installed
at your site. 

\end{itemize}

\noindent Known Bugs:

\begin{itemize}

\item The \AdAttr{CondorLoadAvg} attribute in the \Condor{startd} has
some problems in the way it is computed.
The CondorLoadAvg is somewhat inaccurate for the first minute a job
starts running, and for the first minute after it completes.
Also, the computation of CondorLoadAvg is very wrong on NT.
All of this will be fixed in a future version.

\item A memory error may cause \Condor{status} to die with SIGSEGV
(segmentation violation) when displaying totals or cause incorrect
totals to be displayed.  This will be fixed in version 6.1.11.

\end{itemize}


%%%%%%%%%%%%%%%%%%%%%%%%%%%%%%%%%%%%%%%%%%%%%%%%%%%%%%%%%%%%%%%%%%%%%%
\subsection*{\label{sec:New-6-1-9}Version 6.1.9}
%%%%%%%%%%%%%%%%%%%%%%%%%%%%%%%%%%%%%%%%%%%%%%%%%%%%%%%%%%%%%%%%%%%%%%

\noindent New Features:

\begin{itemize}

\item Added full support for Linux 2.0.x and 2.2.x kernels using
libc5, glibc20 and glibc21.
This includes support for Red Hat 6.x, Debian 2.x and other popular
Linux distributions.
Whereas the Linux machines had once been fragmented across libc5 and
GNU libc, they have now been reunified.
This means there is no longer any need for the ``LINUX-GLIBC'' OpSys
setting in your pool: all machines will now show up as ``LINUX''.
Part of this reunification process was the removal of dynamically
linked user jobs on Linux.
\Condor{compile} now forces static linking of your Standard Universe
Condor jobs. 
Also, please use \Condor{compile} on the same machine on which you
compiled your object files.

\item Added \Condor{qedit} utility to allow users to modify job
attributes after submission.  See the new manual page on
page~\pageref{man-condor-qedit}.

\item Added \OptArg{{-runfor}{minutes}} option to daemonCore to have
the daemon gracefully shut down after the given number of minutes.

\item Added support for statfs(2) and fstatfs(2) in user jobs. We support 
only the fields
\textit{f\_bsize, f\_blocks, f\_bfree, f\_bavail, f\_files, f\_ffree} from
the structure statfs. This is still in the experimental stage.

\item Added the \Opt{-direct} option to \Condor{status}.
After you give \Opt{-direct}, you supply a hostname, and
\Condor{status} will query the \Condor{startd} on the specified host
and display information directly from there, instead of querying the
\Condor{collector}.
See the manual page on page~\pageref{man-condor-submit} for details. 

\item Users can now define \Macro{NUM\_CPUS} to override the automatic
computation of the number of CPUs in your machine.
Using this config setting can cause unexpected results, and is not
recommended. 
This feature is only provided for sites that specifically want this
behavior and know what they are doing.

\item The \Opt{-set} and \Opt{-rset} options to \Condor{config\_val}
have been changed to allow administrators to set both macros and
expressions.
Previously, \Condor{config\_val} assumed you wanted to set
expressions.
Now, these two options each take a single argument, the string
containing exactly what you would put into the config file, so you can
specify you want to create a macro by including an ``='' sign, or an
expression by including a ``:''.
See section~\ref{sec:Intro-to-Config-Files} on
page~\pageref{sec:Intro-to-Config-Files} for details on macros
vs. expressions.
See the \Condor{config\_val} man page on
page~\pageref{man-condor-config-val} for details on
\Condor{config\_val}.  

\item If the directory you specified for LOCK (which holds lock files
used by Condor) doesn't exist, Condor will now try to create that
directory for you instead of giving up right away.

\item If you change the \Attr{COLLECTOR\_HOST} setting and reconfig
the \Condor{startd}, the startd will ``invalidate'' its ClassAds at
the old collector before it starts reporting to the new one.

\end{itemize}

\noindent Bugs Fixed:

\begin{itemize}

\item Fixed a major bug dealing with the group access a Condor job is
started with.
Now, Condor jobs are started with all the groups the job's owner is
in, not just their default group.
This also fixes a security hole where user jobs could be started up in
access groups they didn't belong to.

\item Fixed a bug where there was a needless limitation on the number of open
file descriptors a user job could have.

\item Fixed a standalone checkpointing bug where we weren't blocking signals
in critical sections and causing file table corruption at checkpoint
time.

\item Fixed a linker bug on Digital Unix 4.0 concerning fortran where
the linker would fail on \_\_uname and \_\_sigsuspend.

\item Fixed a bug in \Condor{shadow} that would send incorrect job
completion email under Linux.

\item Fixed a bug in the remote system call of \Syscall{fchdir} that caused
a garbage file descriptor to be used in Standard Universe jobs.

\item Fixed a bug in the \Condor{shadow} which was causing \Condor{q}
\Opt{-goodput} to display incorrect values for some jobs.

\item Fixed some minor bugs and made some minor enhancements in the
\Condor{install} script.
The bugs included a typo in one of the questions asked, and incorrect
handling for the answers of a few different questions.
Also, if DNS is misconfigured on your system, \Condor{install} will
try a few ways to find your fully qualified hostname, and if it still
can't determine the correct hostname, it will prompt the user for it. 
In addition, we now avoid one installation step in cases were it is
not needed. 

\item Fixed a rare race condition that could delay the completion of
large clusters of short running jobs. 

\item Added more checking to the various arguments that might be
passed to \Condor{status}, so that in the case of bad input,
\Condor{status} will print an error message and exit, instead of
performing a segmentation fault.
Also, when you use the \Opt{-sort} option, \Condor{status} will only
display ClassAds where the attributes you use to sort are defined.

\item Fixed a bug in the handling of the config files created by
using the \Opt{-set} or \Opt{-rset} options to \Condor{config\_val}.
Previously, if you manually deleted the files that were created, you
could cause the affected Condor daemon to have a segmentation fault.
Now, the daemons simply exit with a fatal error but still have a
chance to clean up.

\item Fixed a bug in the \Opt{-negotiator} option for most Condor
tools that was causing it to get the wrong address.

\item Fixed a couple of bugs in the \Condor{master} that could cause
improper shutdowns. 
There were cases during shutdown where we would restart a daemon
(because we previously noticed a new executable, for example).
Now, once you begin a shutdown, the \Condor{master} will not restart
anything. 
Also, fixed a rare bug that could cause the \Condor{master} to stop
checking the timestamps on a daemon.

\item Fixed a minor bug in the \Opt{-owner} option to
\Condor{config\_val} that was causing \Condor{init} not to work.

\item Fixed a bug where the \Condor{startd}, while it was already
shutting down, was allowing certain actions to succeed that should
have failed.
For example, it allowed itself to be matched with a user looking for
available machines, or to begin a new PVM task.

\end{itemize}

\noindent Known Bugs:

\begin{itemize}

\item The \AdAttr{CondorLoadAvg} attribute in the \Condor{startd} has
some problems in the way it is computed.
The CondorLoadAvg is somewhat inaccurate for the first minute a job
starts running, and for the first minute after it completes.
Also, the computation of CondorLoadAvg is very wrong on NT.
All of this will be fixed in a future version.

\item There is a serious bug in checkpointing when using Condor's
I/O buffering for ``standard'' jobs.
By default, Linux uses Condor buffering in version 6.1.9 for all
standard jobs.
The bug prevents checkpointing from working more than once.
This renders the \Condor{vacate} and \Condor{checkpoint} commands
useless, and jobs will just be killed without a checkpoint when
machine owners come back to their machines.

\end{itemize}


%%%%%%%%%%%%%%%%%%%%%%%%%%%%%%%%%%%%%%%%%%%%%%%%%%%%%%%%%%%%%%%%%%%%%%
\subsection*{\label{sec:New-6-1-8}Version 6.1.8}
%%%%%%%%%%%%%%%%%%%%%%%%%%%%%%%%%%%%%%%%%%%%%%%%%%%%%%%%%%%%%%%%%%%%%%

\begin{itemize}

\item Added \Term{file\_remaps} as command in the job submit file given to
STANDARD universe jobs.
A Job can now specify that it would like to have files be remapped
from one file to another.
In addition you can specify that files should be read from the local machine
by specifing them.
See the \Condor{submit} manual page on page~\pageref{man-condor-submit} for
more details.

\item Added \Term{buffer\_size} and \Term{buffer\_block\_size} so that STANDARD
universe jobs can specify that they wish to have I/O buffering turned on.
Without buffering, all I/O requests in the STANDARD universe are sent back
over the network to be executed on the submit machine.  
With buffering, read ahead, write behind, and seek batch buffering is
performed to minimize network traffic and latency.
By default, jobs do not specify buffering, however, for many situations buffering
can drastically increase throughput.  See the \Condor{submit} manual page
on page~\pageref{man-condor-submit} for more details.

\item The \Condor{schedd} is much more memory efficient handling clusters
with hundreds/thousands of jobs.  
If you submit large clusters, your submit machine will only use a fraction
of the amount of RAM it used to require.  
\Note The memory savings will only be realized for new clusters submitted
after the upgrade to v6.1.8 -- clusters which previously existed in the
queue at upgrade time will still use the same amount of RAM in the
\Condor{schedd}.

\item Submitting jobs, especially submitting large clusters containing many
jobs, is much faster.

\item Added a \Opt{-goodput} option to \Condor{q}, which displays
statistics about the execution efficiency of STANDARD universe jobs.

\item Added FS\_REMOTE method of user authentication to possible values
of the configuration option \Macro{AUTHENTICATION\_METHODS} to fix problems
with using the \Opt{-r} remote scheduler option of \Condor{submit}.
Additionally, the user authentication protocol has changed, so previous
versions of Condor programs cannot co-exist with this new protocol.

\item Added a new utility and documentation for \Condor{glidein} which uses 
Globus resources to extend your local pool to use remote Globus machines as 
part of your Condor pool.

\item Fixed more bugs in the handling of the stat() system call
and its relatives on Linux with glibc.
This was causing problems mainly with Fortran I/O, though other I/O
related problems on glibc Linux will probably be solved now.

\item Fixed a bug in various Condor tools (\Condor{status},
\Condor{user\_prio}, \Condor{config\_val}, and \Condor{stats}) that
would cause them to seg fault on bad input to the \Opt{-pool} option. 

\item Fixed a bug with the \Opt{-rset} option to \Condor{config\_val} which
could crash the Condor daemon whose configuration was being changed.

\item Added \Term{allow\_startup\_script} command to the job submit
description file which is given to \Condor{submit}.  This allows the
submission of a startup script to the STANDARD universe.  See 

\item Fixed a bug in the \Condor{schedd} where it would get into an
infinite loop if the persistant log of the job queue got corrupted.  
The \Condor{schedd} now correctly handles corrupted log files.

\item The full release tar file now contains a \File{dagman}
subdirectory in the \File{examples} directory.
This subdirectory includes an example DAGMan job, including a README
(in both ASCII and HTML), a Makefile, and so on.

\item Condor will now insert an environment variable, \Env{CONDOR\_VM}, into
the environment of the user job.  
This variable specifies which SMP ``virtual machine'' the job was started on.
It will equal either vm1, vm2, vm3, \Dots , depending upon which virtual
machine was matched.
On a non-SMP machine, \Env{CONDOR\_VM} will always be set to vm1.

\item Fixed some timing bugs introduced in v6.1.6 which could occur when
Condor tries to simultaneously start a large number of jobs submitted from a
single machine.

\item Fixed bugs when Condor is told to gracefully shutdown; Condor no
longer starts up new jobs when shutting down.  Also, the \Condor{schedd}
progressively checkpoints running jobs during a graceful shutdown instead of
trying to vacate all the job simultaneously.  The rate at which the shutdown
occurs is controlled by the \Macro{JOB\_START\_DELAY} configuration
parameter (see page~\pageref{param:JobStartDelay}).

\item Fixed a bug which could cause the \Condor{master} process to exit if
the Condor daemons have been hung for a while by the operating system (if,
for instance, the LOG directory was placed on an NFS volume and the NFS
server is down for an extended period).

\item Previously, removing a large number of jobs with \Condor{rm} would
result in the \Condor{schedd} being unresponsive for a period of time
(perhaps leading to timeouts when running \Condor{q}).  The \Condor{schedd}
has been improved to multitask the removal of jobs while servicing new
requests.

\item Added new configuration parameter \Macro{COLLECTOR\_SOCKET\_BUFSIZE}
which controls the size of TCP/IP buffers used by the \Condor{collector}.
For more info, see section~ref{param:CollectorSocketBufsize} on
page~pageref{param:CollectorSocketBufsize}.

\item Fixed a bug with the \Opt{-analyze} option to \Condor{q}: in some
cases, the RANK expression would not be evaluated correctly.  This could
cause the output from \Opt{-analyze} to be in error.

\item When running on a multi-CPU (SMP) Hewlett-Packard machine, fixed bugs
computing the system load average.

\item Fixed bug in \Condor{q} which could cause the RUN\_TIME reported to
be temporarily incorrect when jobs first start running. 

\item The \Condor{startd} no longer rapidly sends multiple ClassAds one
right after another to the Central Manager when its state/activity is in
rapid transition.  Also, on SMP machines, the \Condor{startd} will only send
updates for 4 nodes per second (to avoid overflowing the central manager when
reporting the state of a very large SMP machine with dozens of CPUs).

\item Reading a parameter with \Condor{config\_val} is now allowed from any
machine with Host-IP READ permission.
Previsouly, you needed ADMINISTRATOR permission.  
Of course, setting a parameter still requires ADMINISTRATOR permission.

\item Worked around a bug in the StreamTokenizer Java class from Sun
that we use in the CondorView client Java applet.
The bug would cause errors if usernames or hostnames in your pool
contained ``-'' or ``\_'' characters.
The CondorView applet now gets around this and properly displays all
data, including entries with the ``bad'' characters.

\end{itemize}

%%%%%%%%%%%%%%%%%%%%%%%%%%%%%%%%%%%%%%%%%%%%%%%%%%%%%%%%%%%%%%%%%%%%%%
\subsection*{\label{sec:New-6-1-7}Version 6.1.7}
%%%%%%%%%%%%%%%%%%%%%%%%%%%%%%%%%%%%%%%%%%%%%%%%%%%%%%%%%%%%%%%%%%%%%%

\Note Version 6.1.7 only adds support for platforms not supported in
6.1.6.  
There are no bug fixes, so there are no binaries released for any
other platforms. 
You do not need 6.1.7 unless you are using one of the two platforms we
released binaries for.

\begin{itemize}

\item Added ``clipped'' support for Alpha Linux machines running the
2.0.X kernel and glibc 2.0.X (such as Red Hat 5.X).
We do not yet support checkpointing and remote system calls on this
platform, but we can start ``vanilla'' jobs.
See section~\ref{sec:Choosing-Universe} on
page~\pageref{sec:Choosing-Universe} for details on vanilla
vs. standard jobs.

\item Re-added support for Intel Linux machines running the 2.0.X
Linux kernel, glibc 2.0.X, using the GNU C compiler (gcc/g++ 2.7.X) or
the EGCS compilers (versions 1.0.X, 1.1.1 and 1.1.2).
This includes Red Hat 5.X, and Debian 2.0.
\Bold{Red Hat 6.0 and Debian 2.1 are not yet supported, since they use
glibc 2.1.X and the 2.2.X Linux kernel.}
Future versions of Condor will support all combinations of kernels,
compilers and versions of libc.

\end{itemize}


%%%%%%%%%%%%%%%%%%%%%%%%%%%%%%%%%%%%%%%%%%%%%%%%%%%%%%%%%%%%%%%%%%%%%%
\subsection*{\label{sec:New-6-1-6}Version 6.1.6}
%%%%%%%%%%%%%%%%%%%%%%%%%%%%%%%%%%%%%%%%%%%%%%%%%%%%%%%%%%%%%%%%%%%%%%

\begin{itemize}

\item Added \Term{file\_remaps} as command in the job submit file given to
\Condor{submit}.
This allows the user to explicitly specify where to find a given file (e.g.
either on the submit or execute machine), as well as remap file access to a
different filename altogether.

\item Changed the way that \Condor{master} spawns daemons and
\Condor{preen} which allows you to specify command line arguments for
any of them, though a \MacroNI{SUBSYS\_ARGS} setting.
Previously, when you specified \Macro{PREEN}, you added the command
line arguments directly to that setting, but that caused some
problems, and only worked for \Condor{preen}.
\Bold{Once you upgrade to version 6.1.6, if you continue to use your
old \File{condor\_config} files, you must change the \Macro{PREEN}
setting to remove any arguments you have defined and place those
arguments into a separate config setting, \Macro{PREEN\_ARGS}.}
See section~\ref{sec:Master-Config-File-Entries}, ``\condor{master}
Config File Entries'', on
page~\pageref{sec:Master-Config-File-Entries} for more details.

\item Fixed a very serious bug in the Condor library linked in with
\Condor{compile} to create standard jobs that was causing
checkpointing to fail in many cases.  
Any jobs that were linked with the 6.1.5 Condor libraries should
probably be removed, re-linked, and re-submitted. 

\item Fixed a bug in \Condor{userprio} that was introduced in version
6.1.5 that was preventing it from finding the address of the
\Condor{negotiator} for your pool.

\item Fixed a bug in \Condor{stats} that was introduced in version
6.1.5 that was preventing it from finding the address of the
\Condor{collector} for your pool.

\item Fixed a bug in the way the \Opt{-pool} option was handled by
many Condor tools that was introduced in version 6.1.5. 


\item \Condor{q} now displays job \emph{allocation time} by default, instead
of displaying CPU time.  
Job allocation time, or RUN\_TIME, is the amount of wall-clock time the job
has spent running.  
Unlike CPU time information which is only updated when a job is
checkpointed, the allocation time displayed by \Condor{q} is continuously
updated, even for vanilla universe jobs.  
By default, the allocation time displayed will be the total time across all
runs of the job.  
The new \Opt{-currentrun} option to \Condor{q} can be used to display the
allocation time for solely the current run of the job.
Additionally, the \Opt{-cputime} option can be used to view job CPU times as
in earlier versions of Condor.

\item \Condor{q} will display an error message if there is a timeout
fetching the job queue listing from a \condor{schedd}.  Previously,
\Condor{q} would simply list the queue as empty upon a communication error.

\item The \condor{schedd} daemon has been updated to verify all queue access
requests via Condor's IP/Host-Based Security mechanism (see
section~\ref{sec:Host-Security}).

\item Fixed a bug on platforms which require the \Condor{kbdd} (currently
Digital Unix and IRIX).  
This bug could have allowed Condor to start a job within the first five
minutes after the Condor daemons had been started, even if there is a user
typing on the keyboard.

\item \Condor{release} now gives an error message if the user tries to
release a job which either does not exist or is not in the hold state.

\item Added a new config file parameter, \Macro{USER\_JOB\_WRAPPER}, which
allows administrators to specify a file to act as a ``wrapper'' script
around all jobs started by Condor. 
See inside section~\ref{param:UserJobWrapper}, on 
page~\pageref{sec:Starter-Config-File-Entries}, for more details.

\item \Condor{dagman} now permits the backslash character (``\Bs'') to be used
as a line-continuation character for DAG Input Files, just like the
\condor{config} files.

\item The Condor version string is now included in all Condor
libraries.
You can now run \Prog{ident} on any program linked with
\Condor{compile} to view which version of the Condor libraries you
linked with.
In addition, the format of the version string changed in 6.1.6.
Now, the identifier used is ``CondorVersion'' instead of ``Version''
to prevent any potential ambiguity.
Also, the format of the date changed slightly.

\item The SMP startd can now handle dynamic reconfiguration of the
number of each type of virtual machine being reported.
This allows you, during the normal running of the startd, to increase
or decrease the number of CPUs that Condor is using.
If you reconfigure the startd to use less CPUs than it currently has
under its control, it will first remove CPUs that have no Condor jobs
running on them.
If more CPUs need to be evicted, the startd will checkpoint jobs and
evict them in reverse rank order (using the startd's \Macro{Rank}
expression).
So, the lower the value of the rank, the more likely a job will be
kicked off.

\item The SMP startd contrib module's \Condor{starter} no longer makes
a call that was causing warning messages about ``ERROR: Unknown System
Call (-58) - system call not supported by Condor'' when used with the
6.0.X \Condor{shadow}.
This was a harmless call, but removing the call prevents the error
message.

\item The SMP contrib module now includes the \Condor{checkpoint} and
\Condor{vacate} programs, which allow you to vacate or checkpoint jobs
on individual CPUs on the SMP, instead of checkpointing or vacating
everything.  
You can now use ``\condor{vacate} vm1@hostname'' to just vacate the
first virtual machine, or ``\condor{vacate} hostname'' to vacate all
virtual machines. 

\item Added support for SMP Digital Unix (Alpha) machines.

\item Fixed a bug that was causing an overflow in the computation of
free disk and swap space on Digital Unix (Alpha) machines.

\item The \Condor{startd} and \Condor{schedd} now can ``invalidate''
their classads from the collector.
So, when a daemon is shut down, or a machine is reconfigured to 
advertise fewer virtual machines, those changes will be instantly
visible with \Condor{status}, instead of having to wait 15 minutes for
the stale classads to time-out.

\item The \Condor{schedd} no longer forks a child process (a ``schedd
agent'') to claim available \Condor{startd}s.  
You should no longer see multiple \condor{schedd} processes running on
your machine after a negotiation cycle.
This is now accomplished in a non-blocking manner within the
\Condor{schedd} itself.

\item The startd now adds an \Attr{VirtualMachineID} attribute to
each virtual machine classad it advertises.
This is just an integer, starting at 1, and increasing for every
different virtual machine the startd is representing.
On regular hosts, this is the only ID you will ever see.
On SMP hosts, you will see the ID climb up to the number of different
virtual machines reported.
This ID can be used to help write more complex policy expressions on
SMP hosts, and to easily identify which hosts in your pool are in fact
SMP machines.

\item Modified the output for \Condor{q} -run for scheduler and PVM
universe jobs.  The host where the scheduler universe job is running
is now displayed correctly.  For PVM jobs, a count of the current
number of hosts where the job is running is displayed.

\item Fixed the \Condor{startd} so that it no longer prints lots of
ProcAPI errors to the log file when it is being run as non-root.

\item \Macro{FS\_PATHNAME} and \Macro{VOS\_PATHNAME} are no longer
used.  AFS support now works similar to NFS support, via the
\Macro{FILESYSTEM\_DOMAIN} macro.

\item Fixed a minor bug in the \File{Condor.pm} perl module that was
causing it to be case-sensitive when parsing the Condor submit file.
Now, the perl module is properly case-insensitive, as indicated in the
documentation.

\end{itemize}

%%%%%%%%%%%%%%%%%%%%%%%%%%%%%%%%%%%%%%%%%%%%%%%%%%%%%%%%%%%%%%%%%%%%%%
\subsection*{\label{sec:New-6-1-5}Version 6.1.5}
%%%%%%%%%%%%%%%%%%%%%%%%%%%%%%%%%%%%%%%%%%%%%%%%%%%%%%%%%%%%%%%%%%%%%%

\begin{itemize}

\item Fixed a nasty bug in \Condor{preen} that would cause it to
remove files it shouldn't remove if the \Condor{schedd} and/or
\Condor{startd} were down at the time \Condor{preen} ran.
This was causing jobs to mysteriously disappear from the job queue.

\item Added preliminary support to Condor for running on machines with
multiple network interfaces.
On such machines, users can specify the IP address Condor should use
in the \Macro{NETWORK\_INTERFACE} config file parameter on each host. 
In addition, if the pool's central manager is on such a machine, users
should set the \Macro{CM\_IP\_ADDR} parameter to the ip address you wish
to use on that machine.
See section~\ref{sec:Multiple-Interfaces} on
page~\pageref{sec:Multiple-Interfaces} for more details.

\item The support for multiple network interfaces introduced bugs in
\Condor{userprio}, \Condor{stats}, CondorPVM, and the \Opt{-pool}
option to many Condor tools.
All of these will be fixed in version 6.1.6.

\item Fixed a bug in the remote system call library that was
preventing certain Fortran operations from working correctly on
Linux.  

\item The Linux binaries for GLIBC we now distribute are compiled on a
Red Hat 5.2 machine.
If you're using this version of Red Hat, you might have better luck
with the dynamically linked version of Condor than previous releases
of Condor.
Sites using other GLIBC Linux distributions should continue to use the
statically linked version of Condor.

\item Fixed a bug in the \Condor{shadow} that could cause it to die
with signal 11 (segmentation violation) under certain rare
circumstances. 

\item Fixed a bug in the \Condor{schedd} that could cause it to die
with signal 11 (segmentation violation) under certain rare
circumstances. 

\item Fixed a bug in the \Condor{negotiator} that could cause it to
die with signal 8 (floating point exception) on Digital Unix
machines. 

\item The following shadow parameters have been added to control
checkpointing: \Macro{COMPRESS\_PERIODIC\_CKPT},
\Macro{COMPRESS\_VACATE\_CKPT}, \Macro{PERIODIC\_MEMORY\_SYNC},
\Macro{SLOW\_CKPT\_SPEED}.  See
section~\ref{sec:Shadow-Config-File-Entries} on
page~\pageref{sec:Shadow-Config-File-Entries} for more details.
In addition, the shadow now honors the \Attr{CkptWanted} flag in a job
classad, and if it is set to ``False'', the job will never
checkpoint.

\item Fixed a bug in the \Condor{startd} that could cause it to
report negative values for the CondorLoadAvg on rare occasions. 

\item Fixed a bug in the \Condor{startd} that could cause it to die
with a fatal exception in situations where the act of getting claimed
by a remote schedd failed for some reason.  
This resulted in the \Condor{startd} exiting on rare occasions with a
message in its log file to the effect of \texttt{ERROR ``Match timed
out but not in matched state''}.

\item Fixed a bug in the \Condor{schedd} that under rare circumstances
could cause a job to be left in the ``Running'' state even after the
\Condor{shadow} for that job had exited.

\item Fixed a bug in the \Condor{schedd} and various tools that
prevented remote read-only access to the job queue from working.
So, for example, \texttt{condor\_q -name foo}, if run on any machine
other than foo, wouldn't display any jobs from foo's queue. 
This fix re-enables the following options to \Condor{q} to work:
\Opt{submitter}, \Opt{name}, \Opt{global}, etc.

\item Changed the \Condor{schedd} so that when starting jobs, it
always sorts on the cluster number, in addition to the date the jobs
were enqueued and the process number within clusters, so that if many
clusters were submitted at the same time, the jobs are started in
order.

\item Fixed a bug in \Condor{compile} that was modifying the
\Env{PATH} environment variable by adding things to the front of it.
This would potentially cause jobs to be compiled and linked with a
different version of a compiler than they thought they were getting.  

\item Minor change in the way the \Condor{startd} handles the
\Dflag{LOAD} and \Dflag{KEYBOARD} debug flags.  
Now, each one, when set, will only display every
\Macro{UPDATE\_INTERVAL}, regardless of the startd state.
If you wish to see the values for keyboard activity or load average
every \Macro{POLLING\_INTERVAL}, you must enable \Dflag{FULLDEBUG}. 

\end{itemize}

%%%%%%%%%%%%%%%%%%%%%%%%%%%%%%%%%%%%%%%%%%%%%%%%%%%%%%%%%%%%%%%%%%%%%%
\subsection*{\label{sec:New-6-1-4}Version 6.1.4}
%%%%%%%%%%%%%%%%%%%%%%%%%%%%%%%%%%%%%%%%%%%%%%%%%%%%%%%%%%%%%%%%%%%%%%

\begin{itemize}

\item Fixed a bug in the socket communication library used by Condor
that was causing daemons and tools to die on some platforms (notably,
Digital Unix) with signal 8, SIGFPE (floating point exception).

\item Fixed a bug in the usage message of many Condor tools that
mentioned a \Opt{-all} option that isn't yet supported. 
This option will be supported in future versions of Condor.

\item Fixed a bug in the filesystem authentication code used to
authenticate operations on the job queue that left empty temporary
files in /tmp.  
These files are now properly removed after they are used.

\item Fixed a minor bug in the totals \Condor{status} displays when
you use the \Opt{ckptsrvr} option.

\item Fixed a minor syntax error in the \Condor{install} script that
would cause warnings.

\item the \File{Condor.pm} Perl module is now included in the
\File{lib} directory of the main release directory.

\end{itemize}

%%%%%%%%%%%%%%%%%%%%%%%%%%%%%%%%%%%%%%%%%%%%%%%%%%%%%%%%%%%%%%%%%%%%%%
\subsection*{\label{sec:New-6-1-3}Version 6.1.3}
%%%%%%%%%%%%%%%%%%%%%%%%%%%%%%%%%%%%%%%%%%%%%%%%%%%%%%%%%%%%%%%%%%%%%%

\Note There are a lot of new, unstable features in 6.1.3.  
PLEASE do not install all of 6.1.3 on a production pool.
Almost all of the bug fixes in 6.1.3 are in the \Condor{startd} or
\Condor{starter}, so, unless you really know what you're doing, we
recommend you just upgrade SMP-Startd contrib module, not the entire
6.1.3 release. 

\begin{itemize}

\item Owners can now specify how the SMP-Startd partitions the system
resources into the different types and numbers of virtual machines,
specifying the number of CPUs, megs of RAM, megs of swap space, etc.,
in each.
Previously, each virtual machine reported to Condor from an SMP
machine always had one CPU, and all shared system resources were
evenly divided among the virtual machines.

\item Fixed a bug in the reporting of virtual memory and disk space on
SMP machines where each virtual machine represented was advertising
the total in the system for itself, instead of its own share.
Now, both the totals, and the virtual machine-specific values are
advertised.  

\item Fixed a bug in the \Condor{starter} when it was trying to
suspend jobs.
While we always killed all of the processes when we were trying to
vacate, if a vanilla job forked, the starter would sometimes not
suspend some of the children processes.
In addition, we could sometimes miss a standard universe job for
suspending as well.
This is all fixed.

\item Fixed a bug in the SMP-Startd's load average computation that
could cause processes spawned by Condor to not be associated w/ the
Condor load average.
This would cause the startd to over-estimate the owner's load average,
and under-estimate the Condor load, which would cause a cycle of
suspending and resuming a Condor job, instead of just letting it run.

\item Fixed a bug in the SMP-Startd's load average computation that
could cause certain rare exceptions to be treated as fatal, when in
fact, the Startd could recover from them.

\item Fixed a bug in the computation of the total physical memory on
some platforms that was resulting in an overflow on machines with
lots of ram (over 1 gigabyte).

\item Fixed some bugs that could cause \Condor{starter} processes to
be left as zombies underneath the \Condor{startd} under very rare
conditions.  

\item For sites using AFS, if there are problems in the
\Condor{startd} computing the AFS cell of the machine it's running on,
the startd will exit with an error message at start-up time.

\item Fixed a minor bug in \Condor{install} that would lead to a
syntax error in your config file given a certain set of installation
options.  

\item Added the \Opt{-maxjobs} option to the \Condor{submit\_dag}
script that can be used to specify the maximum number of jobs Condor
will run from a DAG at any given time.
Also, \Condor{submit\_dag} automatically creates a ``rescue DAG''.
See section~\ref{sec:DAGMan} on page~\pageref{sec:DAGMan} for details
on DAGMan.

\item Fixed bug in ClassAd printing when you tried to display an
integer or float attribute that didn't exist in the given ClassAd. 
This could show up in \Condor{status}, \Condor{q}, \Condor{history},
etc. 

\item Various commands sent to the Condor daemons now have separate
debug levels associated with them.
For example, commands such as ``keep-alives'', and the command sent
from the \Condor{kbdd} to the \Condor{startd} are only seen in the
various log files if \Dflag{FULLDEBUG} is turned on, instead of
\Dflag{COMMAND}, which the default and now enabled for all daemons on
all platforms by default.
Administrators retaining their old configuration when upgrading to
this version are encouraged to enable \Dflag{COMMAND} in the
\Macro{SCHEDD\_DEBUG} setting.  
In addition, for IRIX and Digital Unix machines, it should be enabled
in the \Macro{STARTD\_DEBUG} setting as well.
See section~\ref{sec:Daemon-Logging-Config-File-Entries} on
page~\pageref{sec:Daemon-Logging-Config-File-Entries} for details on
debug levels in Condor.

\item New debug levels added to Condor: 
\begin{itemize}
\item \Dflag{NETWORK}, used by various daemons in Condor to report
various network statistics about the Condor daemons. 
\item \Dflag{PROCFAMILY}, used to report information about various
families of processes that are monitored by Condor.
For example, this is used in the \Condor{startd} when monitoring the
family of processes spawned by a given user job for the purposes of
computing the Condor load average.
\item \Dflag{KEYBOARD}, used by the \Condor{startd} to print out
statistics about remote tty and console idle times in the
\Condor{startd}.
This information used to be logged at \Dflag{FULLDEBUG}, along with
everything else, so now, you can see just the idle times, and/or have
the information stored to a separate file.
\end{itemize}

\item Added a \Opt{-run} option to \Condor{q}, which displays
information for running jobs, including the remote host where each job
is running.

\item Macros can now be incrementally defined.  See
section~\ref{sec:Config-File-Macros} on
page~\pageref{sec:Config-File-Macros} for more details.

\item \Condor{config\_val} can now be used to set configuration
variables.  See the man page on page~\pageref{man-condor-config-val}
for more details.

\item The job log file now contains a record of network activity.  The
evict, terminate, and shadow exception events indicate the number of
bytes sent and received by the job for the specific run.  
The terminate event additionally indicates totals for the life of the
job.

\item \Macro{STARTER\_CHOOSES\_CKPT\_SERVER} now defaults to true.
See section~\ref{param:StarterChoosesCkptServer} on
page~\pageref{param:StarterChoosesCkptServer} for more details.

\item The infrastructure for authentication within Condor has been
overhauled, allowing for much greater flexibility in supporting new
forms of authentication in the future.
This means that the 6.1.3 schedd and queue management tools (like
\Condor{q}, \Condor{submit}, \Condor{rm} and so on) are incompatible
with previous versions of Condor.

\item Many of the Condor administration tools have been improved to
allow you to specify the ``subsystem'' you want them to effect.  
For example, you can now use ``\condor{reconfig} -startd'' to just
have the startd reconfigure itself.
Similarly, \condor{off}, \condor{on} and \condor{restart} can now all 
work on a single daemon, instead of machine-wide.
See the man pages (section~\ref{sec:command-reference} on
page~\pageref{sec:command-reference}) or run any command with \Opt{-help}
for details. 
\Note The usage message in 6.1.3 incorrectly reports \Opt{-all} as a
valid option.

\item Fixed a bug in the Condor tools that could cause a segmentation
violation in certain rare error conditions.

\end{itemize}

%%%%%%%%%%%%%%%%%%%%%%%%%%%%%%%%%%%%%%%%%%%%%%%%%%%%%%%%%%%%%%%%%%%%%%
\subsection*{\label{sec:New-6-1-2}Version 6.1.2}
%%%%%%%%%%%%%%%%%%%%%%%%%%%%%%%%%%%%%%%%%%%%%%%%%%%%%%%%%%%%%%%%%%%%%%

\begin{itemize}

\item Fixed some bugs in the \Condor{install} script.
Also, enhanced \Condor{install} to customize the path to perl in
various perl scripts used by Condor.

\item Fixed a problem with our build environment that left some files
out of the \File{release.tar} files in the binary releases on some
platforms. 

\item \Condor{dagman}, ``DAGMan'' (see section~\ref{sec:DAGMan} on 
page~\pageref{sec:DAGMan} for details) is now included in the
development release by default.

\item Fixed a bug in the computation of the total physical memory in
HPUX machines that was resulting in an overflow on machines with
lots of ram (over 1 gigabyte).
Also, if you define ``MEMORY'' in your config file, that value will
override whatever value Condor computes for your machine.

\item Fixed a bug in \Condor{starter.pvm}, the PVM version of the
Condor starter (available as an optional ``Contrib module''), when you
disabled \Macro{STARTER\_LOCAL\_LOGGING}.
Now, having this set to ``False'' will properly place debug messages
from \Condor{starter.pvm} into the \File{ShadowLog} file of the
machine that submitted the job (as opposed to the \File{StarterLog}
file on the machine executing the job).  

\end{itemize}


%%%%%%%%%%%%%%%%%%%%%%%%%%%%%%%%%%%%%%%%%%%%%%%%%%%%%%%%%%%%%%%%%%%%%%
\subsection*{\label{sec:New-6-1-1}Version 6.1.1}
%%%%%%%%%%%%%%%%%%%%%%%%%%%%%%%%%%%%%%%%%%%%%%%%%%%%%%%%%%%%%%%%%%%%%%

\begin{itemize}

\item Fixed a bug in the \Condor{startd} where we compute the load
average caused by Condor that was causing us to get the wrong values.
This could cause a cycle of continuous job suspends and job resumes.

\item Beginning with this version, any jobs linked with the Condor
checkpoint libraries will use the zlib compression code (used by gzip
and others) to compress periodic checkpoints before they are written
to the network.  
These compressed checkpoints are uncompressed at startup time.  
This saves network bandwidth, disk space, as well as time (if the
network is the bottleneck to checkpointing, which it usually is). 
In future versions of Condor, all checkpoints will probably be
compressed, but at this time, it is only used for periodic
checkpoints.  
Note, you have to relink your jobs with the \Condor{compile} command
to have this feature enabled.
Old jobs (not relinked) will continue to run just fine, they just
won't be compressed.

\item \Condor{status} now has better support for displaying checkpoint
server ClassAds. 

\item More contrib modules from the development series are now
available, such as the checkpoint server, PVM support, and the
CondorView server.  

\item Fixed some minor bugs in the UserLog code that were causing
problems for DAGMan in exceptional error cases.

\item Fixed an obscure bug in the logging code when \Dflag{PRIV} was
enabled that could result in incorrect file permissions on log files. 

\end{itemize}

%%%%%%%%%%%%%%%%%%%%%%%%%%%%%%%%%%%%%%%%%%%%%%%%%%%%%%%%%%%%%%%%%%%%%%
\subsection*{\label{sec:New-6-1-0}Version 6.1.0}
%%%%%%%%%%%%%%%%%%%%%%%%%%%%%%%%%%%%%%%%%%%%%%%%%%%%%%%%%%%%%%%%%%%%%%

\begin{itemize}

\item Support has been added to the \condor{startd} to run multiple
jobs on SMP machines.
See section~\ref{sec:Configuring-SMP} on
page~\pageref{sec:Configuring-SMP} for details about setting up and
configuring SMP support.

\item The expressions that control the \condor{startd} policy for
vacating, jobs has been simplified.
See section~\ref{sec:Configuring-Policy} on
page~\pageref{sec:Configuring-Policy} for complete details on the new
policy expressions, and section~\ref{sec:V60-Policy-diffs} on
page~\pageref{sec:V60-Policy-diffs} for an explanation of what's
different from the version 6.0 expressions.

\item We now perform better tracking of processes spawned by Condor.
If children die and are inherited by init, we still know they belong
to Condor.
This allows us to better ensure we don't leave processes lying around
when we need to get off a machine, and enables us to have a much more
accurate computation of the load average generated by Condor (the
\Attr{CondorLoadAvg} as reported by the \Condor{startd}). 

\item The \condor{collector} now can store historical information
about your pool state.
This information can be queried with the \Condor{stats} program (see
the man page on page~\pageref{man-condor-stats}), which is used by the
\Condor{view} Java GUI, which is available as a separate contrib
module.

\item Condor jobs can now be put in a ``hold'' state with the
\Condor{hold} command.
Such jobs remain in the job queue (and can be viewed with \Condor{q}),
but there will not be any negotiation to find machines for them.
If a job is having a temporary problem (like the permissions are 
wrong on files it needs to access), the job can be put on hold until
the problem can be solved.
Jobs put on hold can be released with the \Condor{release} command.

\item \condor{userprio} now has the notion of \Term{user factors} as a
way to create different groups of users in different priority levels.
See section~\ref{sec:UserPrio} on page~\pageref{sec:UserPrio} for
details.
This includes the ability to specify a local priority domain, and all
users from other domains get a much worse priority.

\item Usage statistics by user is now available from
\condor{userprio}.
See the man page on page~\pageref{man-condor-userprio} for details. 

\item The \condor{schedd} has been enhanced to enable ``flocking'',
where it seeks matches with machines in multiple pools if its requests
cannot be serviced in the local pool.
See section~\ref{sec:Flocking} on page~\pageref{sec:Flocking} for more
details.

\item The \condor{schedd} has been enhanced to enable \condor{q} and
other interactive tools better response time.

\item The \condor{schedd} has also been enhanced to allow it to check
the permissions of the files you specify for input, output, error and
so on.  
If the schedd doesn't have the required access rights to the files,
the jobs will not be submitted, and \Condor{submit} will print an
error message.

\item When you perform a \Condor{rm} command, and the job you removed
was using a ``user log'', the remove event is now recorded into the
log. 

\item Two new attributes have been added to the job classad when it 
begins executing: \Attr{RemoteHost} and \Attr{LastRemoteHost}.
These attributes list the IP address and port of the startd that is
either currently running the job, or the last startd to run the job
(if it's run on more than one machine). 
This information helps users track their job's execution more closely,
and allows administrators to troubleshoot problems more effectively. 

\item The performance of checkpointing was increased by using larger
buffers for the network I/O used to get the checkpoint file on and off
the remote executing host (this helps for all pools, with or without
checkpoint servers). 

\end{itemize}


%%%%%%%%%%%%%%%%%%%%%%%%%%%%%%%%%%%%%%%%%%%%%%%%%%%%%%%%%%%%%%%%%%%%%%%
\section{\label{sec:History-6-0}Stable Release Series 6.0}
%%%%%%%%%%%%%%%%%%%%%%%%%%%%%%%%%%%%%%%%%%%%%%%%%%%%%%%%%%%%%%%%%%%%%%

6.0 is the first version of Condor with \Term{ClassAds}.
It contains many other fundamental enhancements over version 5.
It is also the first official stable release series, with a
development series (6.1) simultaneously available.


%%%%%%%%%%%%%%%%%%%%%%%%%%%%%%%%%%%%%%%%%%%%%%%%%%%%%%%%%%%%%%%%%%%%%%
\subsection*{\label{sec:New-6-0-3}Version 6.0.3}
%%%%%%%%%%%%%%%%%%%%%%%%%%%%%%%%%%%%%%%%%%%%%%%%%%%%%%%%%%%%%%%%%%%%%%

\begin{itemize}

\item Fixed a bug that was causing the hostname of the submit machine
that claimed a given execute machine to be incorrectly reported by the
\Condor{startd} at sites using NIS.

\item Fixed a bug in the \Condor{startd}'s benchmarking code that
could cause a floating point exception (SIGFPE, signal 8) on very,
very fast machines, such as newer Alphas.

\item Fixed an obscure bug in \Condor{submit} that could happen when
you set a requirements expression that references the ``Memory''
attribute.
The bug only showed up with certain formations of the requirement
expression.

\end{itemize}


%%%%%%%%%%%%%%%%%%%%%%%%%%%%%%%%%%%%%%%%%%%%%%%%%%%%%%%%%%%%%%%%%%%%%%
\subsection*{\label{sec:New-6-0-2}Version 6.0.2}
%%%%%%%%%%%%%%%%%%%%%%%%%%%%%%%%%%%%%%%%%%%%%%%%%%%%%%%%%%%%%%%%%%%%%%

\begin{itemize}

\item Fixed a bug in the \Syscall{fcntl} call for Solaris 2.6 that was
causing problems with file I/O inside Fortran jobs.

\item Fixed a bug in the way the \Macro{DEFAULT\_DOMAIN\_NAME}
parameter was handled so that this feature now works properly.  

\item Fixed a bug in how the \Macro{SOFT\_UID\_DOMAIN} config file
parameter was used in the \Condor{starter}.
This feature is also documented in the manual now (see
section~\ref{param:SoftUidDomain} on
page~\pageref{param:SoftUidDomain}).

\item You can now set the RunBenchmarks expression to ``False'' and
the \Condor{startd} will never run benchmarks, not even at startup
time. 

\item Fixed a bug in \Syscall{getwd} and \Syscall{getcwd} for sites
that use the NFS automounter.
his bug was only present if user programs tried to call
\Syscall{chdir} themselves.
Now, this is supported. 

\item Fixed a bug in the way we were computing the available virtual
memory on HPUX 10.20 machines.

\item Fixed a bug in \Condor{q} -analyze so it will correctly identify
more situations where a job won't run.

\item Fixed a bug in \Condor{status} -format so that if the requested 
attribute isn't available for a given machine, the format string
(including spaces, tabs, newlines, etc) is still printed, just the
value for the requested attribute will be an empty string. 

\item Fixed a bug in the \Condor{schedd} that was causing
\Condor{history} to not print out the first ClassAd attribute of all
jobs that have completed

\item Fixed a bug in \Condor{q} that would cause a segmentation fault
if the argument list was too long.

\end{itemize}

%%%%%%%%%%%%%%%%%%%%%%%%%%%%%%%%%%%%%%%%%%%%%%%%%%%%%%%%%%%%%%%%%%%%%%
\subsection*{\label{sec:New-6-0-1}Version 6.0.1}
%%%%%%%%%%%%%%%%%%%%%%%%%%%%%%%%%%%%%%%%%%%%%%%%%%%%%%%%%%%%%%%%%%%%%%

\begin{itemize}

\item Fixed bugs in the \Syscall{getuid}), \Syscall{getgid},
\Syscall{geteuid}, and \Syscall{getegid} system calls. 

\item Multiple config files are now supported as a list specified via
the \Macro{LOCAL\_CONFIG\_FILE} variable. 

\item \Macro{ARCH} and \Macro{OPSYS} are now automatically determined
on all machines (including HPUX 10 and Solaris). 

\item Machines running IRIX now correctly suspend vanilla jobs.

\item \Condor{submit} doesn't allow root to submit jobs.

\item The \Condor{startd} now notices if you have changed
\Macro{COLLECTOR\_HOST} on reconfig.

\item Physical memory is now correctly reported on Digital Unix when
daemons are not running as root. 

\item New \MacroU{SUBSYSTEM} macro in configuration files that changes
based on which daemon is reading the file (i.e. STARTD, SCHEDD, etc.) 
See section~\ref{sec:Condor-Subsystem-Names}, ``Condor Subsystem
Names'' on page~\pageref{sec:Condor-Subsystem-Names} for a complete
list of the subsystem names used in Condor.

\item Port to HP-UX 10.20.  

\item \Syscall{getrusage} is now a supported system call.  
This system call will allow you to get resource usage about the entire
history of your condor job.

\item Condor is now fully supported on Solaris 2.6 machines (both
Sparc and Intel). 

\item Condor now works on Linux machines with the GNU C library.  
This includes machines running Red Hat 5.x and Debian 2.0. 
In addition, there seems to be a bug in Red Hat that was causing the
output from \Condor{config\_val} to not appear when used in scripts
(like \Condor{compile}).
We put in explicit calls to flush the I/O buffers before
\Condor{config\_val} exits, which seems to solve the problem.

\item Hooks have been added to the checkpointing library to help
support the checkpointing of PVM jobs.

\item Condor jobs can now send signals to themselves when running in
the standard universe.
You do this just as you normally would:
\begin{verbatim}
        kill( getpid(), signal_number )
\end{verbatim}
Trying to send a signal to any other process will result in
\Syscall{kill} returning -1.

\item Support for NIS has been improved on Digital Unix and IRIX.

\item Fixed a bug that would cause the negotiator on IRIX machines to
never match jobs with available machines.  

\end{itemize}

%%%%%%%%%%%%%%%%%%%%%%%%%%%%%%%%%%%%%%%%%%%%%%%%%%%%%%%%%%%%%%%%%%%%%%
\subsection*{\label{sec:New-6-0-pl4}Version 6.0 pl4}
%%%%%%%%%%%%%%%%%%%%%%%%%%%%%%%%%%%%%%%%%%%%%%%%%%%%%%%%%%%%%%%%%%%%%%

\Note Back in the bad old days, we used this evil ``patch level''
version number scheme, with versions like ``6.0pl4''.
This has all gone away in the current versions of Condor. 

\begin{itemize}

\item Fixed a bug that could cause a segmentation violation in the 
\Condor{schedd} under rare conditions when a \Condor{shadow} exited.

\item Fixed a bug that was preventing any core files that user jobs
submitted to Condor might create from being transferred back to the
submit machine for inspection by the user who submitted them.

\item Fixed a bug that would cause some Condor daemons to go into an
infinite loop if the "ps" command output duplicate entries.
This only happens on certain platforms, and even then, only under rare
conditions.
However, the bug has been fixed and Condor now handles this case
properly.

\item Fixed a bug in the \Condor{shadow} that would cause a
segmentation violation if there was a problem writing to the user log
file specified by "log = filename" in the submit file used with
\Condor{submit}.

\item Added new command line arguments for the Condor daemons to support
saving the PID (process id) of the given daemon to a file, sending a
signal to the PID specified in a given file, and overriding what
directory is used for logging for a given daemon.
These are primarily for use with the \Condor{kbdd} when it needs to be
started by XDM for the user logged onto the console, instead of
running as root.
See section~\ref{sec:kbdd} on ``Installing the \Condor{kbdd}'' on
page~\pageref{sec:kbdd} for details.

\item Added support for the \Macro{CREATE\_CORE\_FILES} config file
parameter.  
If this setting is defined, Condor will override whatever limits you
have set and in the case of a fatal error, will either create core
files or not depending on the value you specify ("true" or "false").

\item Most Condor tools (\Condor{on}, \Condor{off},
\Condor{master\_off}, \Condor{restart}, \Condor{vacate},
\Condor{checkpoint}, \Condor{reconfig}, \Condor{reconfig\_schedd},
\Condor{reschedule}) can now take the IP address and port you want to
send the command to directly on the command line, instead of only
accepting hostnames. 
This IP/port must be passed in a special format used in Condor (which
you will see in the daemon's log files, etc).
It is of the form: \Sinful{ip.address:port}.  
For example: \Sinful{123.456.789.123:4567}.

\end{itemize}

%%%%%%%%%%%%%%%%%%%%%%%%%%%%%%%%%%%%%%%%%%%%%%%%%%%%%%%%%%%%%%%%%%%%%%
\subsection*{\label{sec:New-6-0-pl3}Version 6.0 pl3}
%%%%%%%%%%%%%%%%%%%%%%%%%%%%%%%%%%%%%%%%%%%%%%%%%%%%%%%%%%%%%%%%%%%%%%

\begin{itemize}

\item Fixed a bug that would cause a segmentation violation if a
machine was not configured with a full hostname as either the official
hostname or as any of the hostname aliases.

\item If your host information does not include a fully qualified
hostname anywhere, you can specify a domain in the
\Macro{DEFAULT\_DOMAIN\_NAME} parameter in your global config file
which will be appended to your hostname whenever Condor needs to use a
fully qualified name.

\item All Condor daemons and most tools now support a "-version"
option that displays the version information and exits.

\item The \Condor{install} script now prompts for a short description
of your pool, which it stores in your central manager's local config
file as \Macro{COLLECTOR\_NAME}.
This description is used to display the name of your pool when sending
information to the Condor developers.

\item When the \Condor{shadow} process starts up, if it is configured
to use a checkpoint server and it cannot connect to the server, the
shadow will check the \Macro{MAX\_DISCARDED\_RUN\_TIME} parameter.  
If the job in question has accumulated more CPU minutes than this
parameter, the \Condor{shadow} will keep trying to connect to the
checkpoint server until it is successful.
Otherwise, the \Condor{shadow} will just start the job over from
scratch immediately.

\item If Condor is configured to use a checkpoint server, it will only
use the checkpoint server.
Previously, if there was a problem connecting to the checkpoint
server, Condor would fall back to using the submit machine to store
checkpoints.
However, this caused problems with local disks filling up on machines
without much disk space.

\item Fixed a rare race condition that could cause a segmentation
violation if a Condor daemon or tool opened a socket to a daemon and
then closed it right away.

\item All TCP sockets in Condor now have the "keep alive" socket option
enabled.
This allows Condor daemons to notice if their peer goes away in a hard
crash.

\item Fixed a bug that could cause the \Condor{schedd} to kill jobs
without a checkpoint during its graceful shutdown method under certain
conditions.

\item The \Condor{schedd} now supports the
\Macro{MAX\_SHADOW\_EXCEPTIONS} parameter.
If the \Condor{shadow} processes for a given match die due to a fatal
error (an exception) more than this number of times, the
\Condor{schedd} will now relinquish that match and stop trying to
spawn \Condor{shadow} processes for it.

\item The "-master" option to \Condor{status} now displays the \Attr{Name}
attribute of all \Condor{master} daemons in your pool, as opposed
to the \Attr{Machine} attribute.
This helps for pools that have submit-only machines joining them, for
example.

\end{itemize}

%%%%%%%%%%%%%%%%%%%%%%%%%%%%%%%%%%%%%%%%%%%%%%%%%%%%%%%%%%%%%%%%%%%%%%
\subsection*{\label{sec:New-6-0-pl2}Version 6.0 pl2}
%%%%%%%%%%%%%%%%%%%%%%%%%%%%%%%%%%%%%%%%%%%%%%%%%%%%%%%%%%%%%%%%%%%%%%

\begin{itemize}

\item In patch level 1, code was added to more accurately find the
full hostname of the local machine.
Part of this code relied on the resolver, which on many platforms is a
dynamic library.
On Solaris, this library has needed many security patches and the
installation of Solaris on our development machines produced binaries
that are incompatible with sites that haven't applied all the security
patches.
So, the code in Condor that relies on this library was simply removed
for Solaris.

\item Version information is now built into Condor.
You can see the \Attr{CondorVersion} attribute in every daemon's
ClassAd. 
You can also run the UNIX command "ident" on any Condor binary to see
the version. 

\item Fixed a bug in the "remote submit" mode of \Condor{submit}.
The remote submit wasn't connecting to the specified schedd, but was
instead trying to connect to the local schedd.

\item Fixed a bug in the \Condor{schedd} that could cause it to exit
with an error due to its log file being locked improperly under
certain rare circumstances.

\end{itemize}

%%%%%%%%%%%%%%%%%%%%%%%%%%%%%%%%%%%%%%%%%%%%%%%%%%%%%%%%%%%%%%%%%%%%%%
\subsection*{\label{sec:New-6-0-pl1}Version 6.0 pl1}
%%%%%%%%%%%%%%%%%%%%%%%%%%%%%%%%%%%%%%%%%%%%%%%%%%%%%%%%%%%%%%%%%%%%%%

\begin{itemize}

\item \Condor{kbdd} bug patched: On Silicon Graphics and DEC Alpha
ports, if your X11 server is using Xauthority user authentication, and
the \Condor{kbdd} was unable to read the user's \File{.Xauthority}
file for some reason, the \Condor{kbdd} would fall into an infinite 
loop.

\item When using a Condor Checkpoint Server, the protocol between the
Checkpoint Server and the \Condor{schedd} has been made more robust
for a faulty network connection. Specifically, this improves
reliability when submitting jobs across the Internet and using a
remote Checkpoint Server.

\item Fixed a bug concerning \Macro{MAX\_JOBS\_RUNNING}: The parameter
\MacroNI{MAX\_JOBS\_RUNNING} in the config file controls the maximum
number of simultaneous \Condor{shadow} processes allowed on your
submission machine.
The bug was the number of shadow processes could, under certain
conditions, exceed the number specified by
\MacroNI{MAX\_JOBS\_RUNNING}. 

\item Added new parameter \Macro{JOB\_RENICE\_INCREMENT} that can be
specified in the config file.
This parameter specifies the UNIX nice level that the \Condor{starter}
will start the user job.
It works just like the \Cmd{renice}{1} command in UNIX. 
Can be any integer between 1 and 19; a value of 19 is the lowest
possible priority.

\item Improved response time for \Condor{userprio}.

\item Fixed a bug that caused periodic checkpoints to happen more
often than specified.

\item Fixed some bugs in the installation procedure for certain
environments that weren't handled properly, and made the documentation
for the installation procedure more clear.

\item Fixed a bug on IRIX that could allow vanilla jobs to be started
as root under certain conditions.
This was caused by the non-standard uid that user "nobody" has on
IRIX.
Thanks to Chris Lindsey at NCSA for help discovering this bug.

\item On machines where the \File{/etc/hosts} file is misconfigured to
list just the hostname first, then the full hostname as an alias,
Condor now correctly finds the full hostname anyway.

\item The local config file and local root config file are now only
found by the files listed in the \Macro{LOCAL\_CONFIG\_FILE} and
\Macro{LOCAL\_ROOT\_CONFIG\_FILE} parameters in the global config
files.
Previously, \File{/etc/condor} and user condor's home directory
(\~condor) were searched as well.
This could cause problems with submit-only installations of Condor at
a site that already had Condor installed.

\end{itemize}

%%%%%%%%%%%%%%%%%%%%%%%%%%%%%%%%%%%%%%%%%%%%%%%%%%%%%%%%%%%%%%%%%%%%%%
\subsection*{\label{sec:New-6-0-pl0}Version 6.0 pl0}
%%%%%%%%%%%%%%%%%%%%%%%%%%%%%%%%%%%%%%%%%%%%%%%%%%%%%%%%%%%%%%%%%%%%%%

\begin{itemize}

\item Initial Version 6.0 release.

\end{itemize}

