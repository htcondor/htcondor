%%%%%%%%%%%%%%%%%%%%%%%%%%%%%%%%%%%%%%%%%%%%%%%%%%%%%%%%%%%%%%%%%%%%%%
\section{\label{sec:gotchas}Upgrading from the 7.4 series to the 7.6 series of Condor}
%%%%%%%%%%%%%%%%%%%%%%%%%%%%%%%%%%%%%%%%%%%%%%%%%%%%%%%%%%%%%%%%%%%%%%

\index{upgrading!items to be aware of}
Upgrading from the 7.4 series of Condor to this 7.6 series will
bring many features introduced in the 7.5 series of Condor.
While nothing will replace reading through the version histories,
here is a list of some new features and release notes to
be aware of.

\begin{itemize}

\item  Condor's RPMs now use FHS locations for files.
See section~\ref{sec:install-rpms} for updated installation information. 

\item  The locations of \emph{many} executables within the release directories
have changed.

\item  Condor's file transfer mechanism now provided limited support 
for the transfer of directories.

\item  The feature set once known as the Startd Cron or as Hawkeye
is now called \Term{Daemon ClassAd Hooks}.  Besides the new name,
the mechanisms have been updated and significantly revised.
See section~\ref{sec:daemon-classad-hooks} for documentation.

\item  Condor DAGMan provides new features.
See section~\ref{sec:DAGMan} for documentation.

\item  Grid jobs targeted at CREAM, EC2 servers and Eucalyptus systems
 are now supported.

\item  For those who compile Condor from the source code,
rather than using packages of pre-built executables, 
Condor is now built using \Prog{cmake} instead of \Prog{imake}.

\item  Quill is now available in the source code, 
and it is not incorporated as part of pre-built and released executables.

\item  The \Condor{job\_router} is no longer limited to routing vanilla
universe jobs.

\item  Condor uses the New ClassAd language internally.

\item  The \Condor{shared\_port} daemon allows Condor daemons 
to share a single network port. 
This makes opening access to Condor through a firewall easier and safer. 
It also increases the scalability of a submit node by decreasing port usage.

\end{itemize}

