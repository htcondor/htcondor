%%%      PLEASE RUN A SPELL CHECKER BEFORE COMMITTING YOUR CHANGES!
%%%      PLEASE RUN A SPELL CHECKER BEFORE COMMITTING YOUR CHANGES!
%%%      PLEASE RUN A SPELL CHECKER BEFORE COMMITTING YOUR CHANGES!
%%%      PLEASE RUN A SPELL CHECKER BEFORE COMMITTING YOUR CHANGES!
%%%      PLEASE RUN A SPELL CHECKER BEFORE COMMITTING YOUR CHANGES!

%%%%%%%%%%%%%%%%%%%%%%%%%%%%%%%%%%%%%%%%%%%%%%%%%%%%%%%%%%%%%%%%%%%%%%
\section{\label{sec:History-7-7}Development Release Series 7.7}
%%%%%%%%%%%%%%%%%%%%%%%%%%%%%%%%%%%%%%%%%%%%%%%%%%%%%%%%%%%%%%%%%%%%%%

This is the development release series of Condor.
The details of each version are described below.


%%%%%%%%%%%%%%%%%%%%%%%%%%%%%%%%%%%%%%%%%%%%%%%%%%%%%%%%%%%%%%%%%%%%%%
\subsection*{\label{sec:New-7-7-2}Version 7.7.2}
%%%%%%%%%%%%%%%%%%%%%%%%%%%%%%%%%%%%%%%%%%%%%%%%%%%%%%%%%%%%%%%%%%%%%%

\noindent Release Notes:

\begin{itemize}

\item Condor version 7.7.2 not yet released.
%\item Condor version 7.7.2 released on Month Date, 2011.

\end{itemize}


\noindent New Features:

\begin{itemize}

\item The dedicated scheduler used to attempt to change the
\Attr{Scheduler} attribute on all parallel job processes in a durable fashion,
resulting in an \Procedure{fsync} for each process.
This has been changed to be not durable, 
improving scalability by reducing the 
number of \Procedure{fsync} calls without impacting correctness. 
\Ticket{2367}

% gittrack #2364
\item In privsep mode, when an error is encountered when trying to
  switch to the user account chosen for running the job, the error
  message has been improved to make debugging easier.  Now, the error
  message distinguishes between safety check failures for the uid,
  tracking group id, primary group id, and supplementary group ids.

% commit c3c0fe27f59e1807d77ce5709e965ed17629c223
% commit 7c0e456f220670af3624b0192244093edcc768c0
% commit 5732223317b7c459123aefdf6670fe4b825f7fe8
% commit a46a255129ef2ec5c5a977ae1fc4fdb8683cfe6a
% commit 74c9ac5086fd18527319e58e6d48ce7947951f0c
% commit cbdf2f2a514d03f976bdaaf3d9c347fc3e705de2
% commit bdb12c225d7f49fdcf1cd4396861cf630e8e76fc
% commit 46dc3c3b930d9af0924f72377c66db3714d34330
% commit 390506265d15b304f923ea8d44017b18018c97f6
% commit e45222055d1541fcf28bf2ccbadec2790296ea0f
% commit e8dd11b3b3a1c299cd202b7c4f4461fe253d2c3e
% commit 44647837f105a057032d6a9048333818c15f57b8
% commit d542c6c8f45f593eda489928a5a163c06bbcfeab
% commit 7090363927f47fa35d0a7aea9b53b863e8eea858
% commit 13ffba584f2d84527b3655d77c2d219f2bda136f
% commit 7086d8947b4eba370e290a925190093e673f5ec4
% commit 296a8686220827ea8f893c63e1cf1d12ee5cc026
% commit 60e46cc4466d77507495ace11f10efed3084e39a
% commit 68a8b39d11c39b1920747e80392fd20fc0b65133
% commit f282314060ee8fa43504943e4ea96e013288718e
% commit f31f33e08d9309d585a365509eb992cf045c54ae
% commit 8487f22d60b2878516c8fdbd2c2bbdb9122c037d
% commit 07cccd34924f347acc0f0044a724b9074056b318
% commit b9a0f59111a04fcc2c81e8bf14b44786fc0aba5f
% commit 32cdfb006ba46cb099442a7257f27f6ada0f347b
% commit af3822ab2abe8781c532bdc217cdb6033329023b
% commit 0736933d3ded193d84ffd604fe2a30d66b04f291
% commit 4a55dcc8f433393939e6a2a154eb903cffc91f52
% commit 16ccf652ad67416ce800bd703a48468dce490a45
% commit 0c624cbf807fd23261aac540e78d9e2c663c4f84
% commit 5fad7f81d9377f87cca4c0a44c68585250dc12e7
% commit 69fceba06015c2a40c122fc829bb0ad58de176b4
% commit 6c35cdd174f5459d39edf5b8a4a529a18c1fce5a
% commit bfb2f58403a7e183f7f2e630e0d5488fec6b9f0d
% commit 37e2f9153dcaeb3d081a0b70cbf3715f19961ccf
% commit 22652297344f093790a0a529512e2ea95b09c079
% commit 06e795068dcc8f63affa96cf2ca6dc1492b3e489
% commit 4d283d8f17ad168f12efeda5178a2c8b7aa87cf3
\item The behavior of DAGMan is changed, such that, by default, POST scripts
  will be run regardless of the return value from the PRE script of the same
  node.  The previous behavior of not running the POST script can be restored by
  either adding the \Opt{-PostIgnorePre} option to \Condor{submit\_dag}, or by
  setting the new configuration variable \Macro{DAGMAN\_POST\_IGNORE\_PRE} to
  \Expr{False}. \Ticket{2057}

\end{itemize}

\noindent Configuration Variable and ClassAd Attribute Additions and Changes:

\begin{itemize}

\item Configuration variable \Macro{COLLECTOR\_ADDRESS\_FILE} is now set 
in the example configuration,
similar to \MacroNI{MASTER\_ADDRESS\_FILE}.
This configuration variable is required when \Macro{COLLECTOR\_HOST} 
has the port set to 0, which means to select any available port.
In other environments, it should have no visible impact.
\Ticket{2375}

\end{itemize}

\noindent Bugs Fixed:

\begin{itemize}

\item Fixed a bug existing since April 2001,
in which on \Condor{schedd} startup, with parallel universe jobs, 
the job queue sanity checking code would change the \Attr{Scheduler}
attribute on jobs,
only to have the attribute changed later by the dedicated scheduler.
\Ticket{2367}


\end{itemize}

\noindent Known Bugs:

\begin{itemize}

\item None.

\end{itemize}

\noindent Additions and Changes to the Manual:

\begin{itemize}

\item None.

\end{itemize}


%%%%%%%%%%%%%%%%%%%%%%%%%%%%%%%%%%%%%%%%%%%%%%%%%%%%%%%%%%%%%%%%%%%%%%
\subsection*{\label{sec:New-7-7-1}Version 7.7.1}
%%%%%%%%%%%%%%%%%%%%%%%%%%%%%%%%%%%%%%%%%%%%%%%%%%%%%%%%%%%%%%%%%%%%%%

\noindent Release Notes:

\begin{itemize}

\item Condor version 7.7.1 not yet released.
%\item Condor version 7.7.1 released on Month Date, 2011.

\end{itemize}


\noindent New Features:

\begin{itemize}

% gittrac #2122
\item The new \Arg{PRE\_SKIP} key word in DAGMan changes the
behavior of DAG node execution such that the node's job and POST script
may be skipped based on the exit value of the PRE script.
See section ~\ref{dagman:SCRIPT} for details.

% uncomment item, if it appears in 7.7.1
% gittrac #659 
%\item Filip Krikava supplied a patch that limits the number of 
%file descriptors that DAGMan has open at a time.
%The reason for creating this capability is that
%DAGman tends to fail on wide DAGs, where many jobs are independent,
%rather than being linear, where jobs have many dependencies.

% gittrac #1874
\item Condor now uses the OS-provided versions of OpenSSL and Kerberos,
rather than including it's own versions.\Ticket{1874}

\end{itemize}

\noindent Configuration Variable and ClassAd Attribute Additions and Changes:

\begin{itemize}

\item None.

\end{itemize}

\noindent Bugs Fixed:

\begin{itemize}

% gittrac #2329
\item ``\Arg{PERIODIC}\_$\ast$'' (e.g., \Arg{PERIODIC\_HOLD} and 
\Arg{PERIODIC\_RELEASE}) expressions could inadvertently cause a claim to
be released if shadow exits before waiting for final update from the starter. 

% gittrac #2350
\item \Condor{submit} previously could incorrectly detect references
  in the requirements expression to special attributes such as
  \Attr{Memory} when the name of the attribute happened to appear in a
  string literal or as part of the name of some other attribute.  The
  detection of references to various special attributes influences the
  automatic requirements which are appended to the job requirements.

% gittrac #2360
\item In rare cases, CCB requests could cause the server to hang for
20s while waiting for all of the request to arrive.

\end{itemize}

\noindent Known Bugs:

\begin{itemize}

\item None.

\end{itemize}

\noindent Additions and Changes to the Manual:

\begin{itemize}

\item None.

\end{itemize}


%%%%%%%%%%%%%%%%%%%%%%%%%%%%%%%%%%%%%%%%%%%%%%%%%%%%%%%%%%%%%%%%%%%%%%
\subsection*{\label{sec:New-7-7-0}Version 7.7.0}
%%%%%%%%%%%%%%%%%%%%%%%%%%%%%%%%%%%%%%%%%%%%%%%%%%%%%%%%%%%%%%%%%%%%%%

\noindent Release Notes:

\begin{itemize}

\item Condor version 7.7.0 released on July 29, 2011.
This developer release contains all bug fixes from Condor version 7.6.2.

\end{itemize}


\noindent New Features:

\begin{itemize}

\item A full port of Condor is available for RedHat Enterprise Linux 6
on the x86\_64 processor.
A full port includes support for the standard universe.

\item The matchmaking attributes \Attr{SubmitterUserResourcesInUse}
and \Attr{RemoteUserResourcesInUse} are now biased by slot weights.

% gittrac #1971
\item \Condor{submit} now accepts the new command line option \Opt{-addr},
naming the IP address of the \Condor{schedd} to submit to.

\item The \Condor{vm\_gahp} now is dynamically linked to libvirt.  
We believe this makes it more portable.

\item Programs \Condor{reschedule\_schedd} and \Condor{master\_off}
are no longer part of the distribution.
These programs were replaced many years ago by the more general
\Condor{reschedule} and \Condor{off} commands.

\item On Windows platforms, improved the ability of the \Condor{starter}
and \Condor{shadow} daemons to clean up the execute directory,
if jobs have changed the ACLs or permissions on files they have created.

\item \Condor{submit} now sets a default value for job ClassAd attribute
\Attr{RequestMemory}.

\item The submission performance of cream grid jobs has been substantially
improved by batching submit requests.

\item \Condor{q} \Opt{-better} now has cleaner output, and informs
the user when negotiation has not yet occurred.

\item Implemented many improvements to the Condor \Prog{init} scripts.

\item Deltacloud support has been updated to deltacloud version 0.8.

% gittrac #1960
\item As of Condor version 7.6.0,
vm universe submit description files no longer support
automatic creation of cdrom images from text input file.
Users must now explicitly create ISO images and transfer them
with the job.

\item \Condor{q} now supports the new option \Opt{-stream-results}.
  When this option is specified, \Condor{q} displays results as they
  are fetched from the job queue, rather than buffering up the query
  results before displaying anything.

% gittrac #1871 
% gittrac #2295
\item The new submit description file command \SubmitCmd{stack\_size} 
  applies to Linux jobs that are not running in the standard universe. 
  It sets the allocation of stack space to be other than the default
  value, which is unlimited.
  It also advertises the job ClassAd attribute \AdAttr{StackSize}.

% gittrac #1550
\item The new ClassAd function \Code{stringListsIntersect} evaluates to 
  \Expr{True} if two strings of delimited elements have any matching elements,
  and it evaluates to \Expr{False} otherwise.

% gittrac #1821
\item The grid universe now supports the \SubmitCmd{ec2} resource type,
  which uses the EC2 Query (REST) API to start virtual machines on cloud
  resources.

% gittrac #2090 
\item The behavior of DAGMan has changed, 
such that if multiple definitions of a VARS macroname 
for a specific node within a DAG input exist,
a warning is written to the log, of the format
\begin{verbatim}
Warning: VAR <macroname> is already defined in job <JobName>
Discovered at file "<DAG input file name>", line <line number>
\end{verbatim}
See section ~\ref{dagman:VARS} for details.

% gittrac #2297
\item The version number for ClassAds now matches the Condor version number. 

% gittrac #2259
\item When \Prog{glexec} fails to execute a job,
diagnostic error messages produced by \Prog{glexec} used to be discarded.
These error messages are now displayed in the log of the \Condor{starter} 
and in the job's hold reason. 

% gittrac #2185
\item New submit description file commands
\SubmitCmd{periodic\_hold\_reason}, \SubmitCmd{periodic\_hold\_subcode},
\SubmitCmd{on\_exit\_hold\_reason}, and \SubmitCmd{on\_exit\_hold\_subcode}
permit the job to set a hold reason string and subcode number.
Similarly, the system job policy can specify the reason and subcode 
using \Macro{SYSTEM\_PERIODIC\_HOLD\_REASON} and 
\Macro{SYSTEM\_PERIODIC\_HOLD\_SUBCODE}.
In addition, the \Condor{hold} command now accepts a \Opt{-subcode} option,
which is used to set the job attribute \Attr{HoldReasonSubCode}. 

\item If the \Condor{shadow} cannot write to the user log, 
the job is now put on hold.

\end{itemize}


\noindent Configuration Variable and ClassAd Attribute Additions and Changes:

\begin{itemize}

\item The new configuration variable \Macro{NEGOTIATOR\_UPDATE\_AFTER\_CYCLE}
defaults to \Expr{False}.
If set to \Expr{True}, it will force the \Condor{negotiator} daemon
to publish an update ClassAd to the \Condor{collector} at the end of 
every negotiation cycle. 
This is useful if monitoring cycle-based statistics.

% gittrac #1960
% gittrac #2004
\item The new configuration variable \Macro{SHADOW\_RUN\_UNKNOWN\_USER\_JOBS}
defaults to \Expr{False}.
When \Expr{True}, it allows the \Condor{shadow} daemon to run jobs remotely 
submitted from users not in the local password file.

\item The configuration variables for security 
\Macro{DENY\_CLIENT} and \Macro{HOSTDENY\_CLIENT}
now also look for the prefixes \Expr{TOOL} and \Expr{SUBMIT}.
 
% gittrac #1249
\item \Macro{CONDOR\_VIEW\_HOST} is now a comma and/or white space separated
list of hosts, in order to support more than one CondorView host.

\item For a job with an X.509 proxy credential, the new job ClassAd
attribute \AdAttr{X509UserProxyEmail} is the email address extracted
from the proxy.

% gittrac 2067
\item On Linux execute machines with kernel version more recent than 2.6.27,
the proportional set size (PSS) in Kbytes summed across all
processes in the job is now reported in the attribute
\AdAttr{ProportionalSetSizeKb}.  If the execute machine does not
support monitoring of PSS or PSS has not yet been measured, this
attribute will be undefined.  PSS differs from \AdAttr{ImageSize} in
how memory shared between processes is accounted.  The PSS for one
process is the sum of that process' memory pages divided by the
number of processes sharing each of the pages.  \AdAttr{ImageSize} is
the same, except there is no division by the number of processes
sharing the pages.

% gittrac #1755
\item The new configuration variable \Macro{DAGMAN\_USE\_STRICT} 
turns warnings into errors, as defined in section~\ref{param:DAGManUseStrict}.

% gittrac #2006
\item The \Condor{schedd} now publishes performance-related statistics.
  Page~\pageref{sec:Scheduler-ClassAd-Attributes} in Appendix A contains
  definitions for these new attributes:
  \begin{itemize}
    \item \Attr{DetectedMemory}
    \item \Attr{DetectedCpus}
    \item \Attr{UpdateInterval}
    \item \Attr{WindowedStatWidth}
    \item \Attr{ExitCode<N>}
    \item \Attr{ExitCodeCumulative<N>}
    \item \Attr{JobsSubmitted}
    \item \Attr{JobsSubmittedCumulative}
    \item \Attr{JobsStarted}
    \item \Attr{JobsStartedCumulative}
    \item \Attr{JobsCompleted}
    \item \Attr{JobsCompletedCumulative}
    \item \Attr{JobsExited}
    \item \Attr{JobsExitedCumulative}
    \item \Attr{ShadowExceptions}
    \item \Attr{ShadowExceptionsCumulative}
    \item \Attr{JobSubmissionRate}
    \item \Attr{JobStartRate}
    \item \Attr{JobCompletionRate}
    \item \Attr{MeanTimeToStart}
    \item \Attr{MeanTimeToStartCumulative}
    \item \Attr{MeanRunningTime}
    \item \Attr{MeanRunningTimeCumulative}
    \item \Attr{SumTimeToStartCumulative}
    \item \Attr{SumRunningTimeCumulative}
  \end{itemize}

% gittrac #1930
\item For Windows platforms, the \Condor{startd} now publishes the 
ClassAd attribute \Attr{DotNetVersions},
containing a comma separated list of installed .NET versions.

\end{itemize}

\noindent Bugs Fixed:

\begin{itemize}

\item Fixed a bug in which the \Condor{startd} daemon can get stuck in a
loop trying to execute an invalid, 
that is non-existent, Daemon ClassAd Hook job.

\item Fixed bug that would cause the \Condor{startd} daemon to incorrectly
report Benchmarking activity instead of Idle activity,
when there is a problem launching the benchmarking programs.

\item On Windows only, fixed a rare bug that could cause
a sporadic access violation when a Condor daemon spawned another process.

\item Fixed a bug introduced in Condor version 7.5.5,
which caused the \Condor{schedd} to die managing parallel jobs.

% commented out, as this bug fix is listed in the 7.6.1 version history.
% \item Fixed bug present throughout ClassAds,
% where expressions expecting a floating point value returned an error,
% if they got a boolean value.  This is common in \MacroNI{RANK} expressions.

\item The \Condor{startd} daemon now looks up the \Condor{kbdd} daemon address
on every update.  
This fixed problems if the \Condor{kbdd} daemon is restarted 
during the \Condor{startd} lifespan.

\item Fixed bug in \Condor{hold} that happened if the hold
reason contained a double quote character.

\item Fixed a bug introduced in Condor version 7.5.6 that
caused any Daemon ClassAd hook job with non-empty value for
\MacroNI{STARTD\_CRON\_<JobName>\_ARGS},
\MacroNI{SCHEDD\_CRON\_<JobName>\_ARGS}
or \MacroNI{BENCHMARKS\_<JobName>\_ARGS} to fail.
Also, the specification of 
\MacroNI{STARTD\_CRON\_<JobName>\_ENV},
\MacroNI{SCHEDD\_CRON\_<JobName>\_ENV},
or \MacroNI{BENCHMARKS\_<JobName>\_ENV} for these jobs was ignored.

\item Fixed bug in the RPM \Prog{init} script. 
A status request would always report Condor as inactive, 
and a shutdown request would not report failure if there was a
timeout shutting down Condor.

\item File transfer plug-ins now have a correctly set environment.

\item Fixed a problem with detecting IBM Java Virtual Machines whose
version strings have embedded newline characters.

\item \Condor{q} \Opt{-analyze} now works with ClassAd built-in functions.

\item Fixed bug in \Condor{q} \Opt{-run}, such that it displays
the host name correctly for local and scheduler universe jobs.

\item Standalone checkpointing now works with compressed checkpoints again.
This had been broken in Condor version 7.5.4.

%gittrac 1962
\item On Windows, \Prog{net stop condor} would sometimes cause the
\Condor{master} daemon to crash.  This is now fixed.

% gittrac #1928
\item \AdAttr{JobUniverse} was effectively a required attribute for
  jobs created via the Fetch Work hook,
  due to the need to set the \MacroNI{IS\_VALID\_CHECKPOINT\_PLATFORM}
  expression, such that it would not evaluate to \Expr{Undefined}.
  Now the default \MacroNI{IS\_VALID\_CHECKPOINT\_PLATFORM} expression
  evaluates to \Expr{True} when \AdAttr{JobUniverse} is not defined.

% gittrac #1943
\item When there are multiple cpus but only one slot, the slot name no
longer begins with \Expr{slot1@}.

% gittrac #1805 
\item The tool \Condor{advertise} seemed to be trying too hard to resolve
host names. This was fixed to only do the minimally necessary 
number of look ups.

\end{itemize}

\noindent Known Bugs:

\begin{itemize}

\item None.

\end{itemize}

\noindent Additions and Changes to the Manual:

\begin{itemize}

\item None.

\end{itemize}

