%%%%%%%%%%%%%%%%%%%%%%%%%%%%%%%%%%%%%%%%%%%%%%%%%%%%%%%%%%%%%%%%%%%%%%
\section{\label{sec:to-8.6}Upgrading from the 8.4 series to the 8.6 series of HTCondor}
%%%%%%%%%%%%%%%%%%%%%%%%%%%%%%%%%%%%%%%%%%%%%%%%%%%%%%%%%%%%%%%%%%%%%%

\index{upgrading!items to be aware of}
Upgrading from the 8.4 series of HTCondor to the 8.6 series
will bring new features introduced in the 8.5 series of HTCondor.
These new features include:

%TEMPTEMP>>>
\begin{verbatim}
* 8.5.8:
The starter puts all jobs in a cgroup by default;
Added condor_submit commands that support job retries;
condor_qedit defaults to the current user's jobs;
Ability to add SCRIPTS, VARS, etc. to all nodes in a DAG with a single command;
Able to conditionally add Docker volumes for certain jobs.
Initial support for Singularity containers;
A 64-bit Windows release.

* 8.5.7:
The schedd can perform job ClassAd transformations;
Specifying dependencies between DAGMan splices is much more flexible;
Simplification of DAG node priorities?;
The second argument of the ClassAd ? : operator may be omitted;
Many usability improvements in condor_q and condor_status;
condor_q and condor_status can produce JSON, XML, and new ClassAd output;
To prepare for a 64-bit Windows release, HTCondor identifies itself as X86;
Automatically detect Daemon Core daemons and pass localname to them.

* 8.5.6:
The -batch output for condor_q is now the default;
Python bindings for job submission and machine draining;
Numerous Docker usability changes;
New options to limit condor_history results to jobs since last invocation;
Shared port daemon can be used with high availability and replication;
ClassAds can be written out in JSON format;
More flexible ordering of DAGMan commands;
Efficient PBS and SLURM job monitoring;
Simplified leases for grid universe jobs.

* 8.5.5:
Improvements for scalability of EC2 grid universe jobs;
Docker Universe jobs advertises remote user and system CPU time;
Improved systemd support;
The master can now run an administrator defined script at shutdown;
DAGMan includes better support for the batch name feature.

* 8.5.4:
Fixed a bug that delays schedd response when significant attributes change;
Fixed a bug where the group ID was not set in Docker universe jobs;
Limit update rate of various attributes to not overload the collector;
To make job router configuration easier, added implicit "target" scoping;
To make BOSCO work, the blahp does not generate limited proxies by default;
condor_status can now display utilization per machine rather than per slot;
Improve performance of condor_history and other tools.

* 8.5.3:
Use IPv6 (and IPv4) interfaces if they are detected;
Prefer IPv4 addresses when both are available;
Count Idle and Running jobs in Submitter Ads for Local and Scheduler universes;
Can submit jobs to SLURM with the new "slurm" type in the Grid universe;
HTCondor is built and linked with Globus 6.0.

* 8.5.2:
condor_q now defaults to showing only the current user's jobs;
condor_q -batch produces a single line report for a batch of jobs;
Docker Universe jobs now report and update memory and network usage;
immutable and protected job attributes;
improved performance when querying a HTCondor daemon's location;
Added the ability to set ClassAd attributes within the DAG file;
DAGMan now provides event timestamps in dagman.out.

* 8.5.1:
the shared port daemon is enabled by default;
the condor_startd now records the peak memory usage instead of recent;
the condor_startd advertises CPU submodel and cache size;
authorizations are automatically setup when "Match Password" is enabled;
added a schedd-constraint option to condor_q.
With the shared port daemon enabled, all HTCondor daemons share a single
inbound network port. This change makes it much easier to construct a
firewall configuration that allows HTCondor use.

* 8.5.0:
multiple enhancements to the python bindings;
the condor_schedd no longer changes the ownership of spooled job files;
spooled job files are visible to only the user account by default;
the condor_startd records when jobs are evicted by preemption or draining.
\end{verbatim}
%<<<TEMPTEMP
\begin{itemize}

\item (none)

\end{itemize}

Upgrading from the 8.4 series of HTCondor to the 8.6 series will
also introduce changes that administrators and users of sites running from an older
HTCondor version should be aware of when planning an upgrade.
Here is a list of items that administrators should be aware of.

\begin{itemize}

\item Shared port (see section~\ref{sec:shared-port-daemon}) is now
enabled by default; set \MacroNI{USE\_SHARED\_PORT} to \Expr{False} to
disable it.  Note that this configuration macro does not control the HAD or
replication daemon's use of shared port; use \MacroNI{HAD\_USE\_SHARED\_PORT}
or \MacroNI{REPLICATION\_USE\_SHARED\_PORT} instead.  See
\ref{sec:HA-configuration} for more details on how to configure HAD (and/or
the replication daemon) to work with shared port, since just activating
shared port without any other configuration change will not work.

\item To mitigate performance problems, \MacroNI{LOWPORT} and
\MacroNI{HIGHPORT} no longer restrict outbound port ranges on Windows.  To
re-enable this functionality, set \MacroNI{OUT\_LOWPORT} and
\MacroNI{OUT\_HIGHPORT}.  \Ticket{4711}

\item Cgroups (see section~\ref{sec:CGroupTracking}) are now enabled by default.  This means that if you
have partitionable slots, jobs need to get \SubmitCmd{request\_memory}
correct.

\item By default, \Condor{q} queries only the current user's jobs,
unless the current user is a queue superuser or the
\MacroNI{CONDOR\_Q\_ONLY\_MY\_JOBS} configuration macro is set to
\Expr{False}.

% \item By default, \Condor{q} is now invoked with the \Opt{-batch}
% option, which displays a single line for each batch of jobs,
% rather than one line per job.

\end{itemize}

