%%%%%%%%%%%%%%%%%%%%%%%%%%%%%%%%%%%%%%%%%%%%%%%%%%%%%%%%%%%%%%%%%%%%%%
\section{\label{sec:java-install}Installing Java Support in Condor}
%%%%%%%%%%%%%%%%%%%%%%%%%%%%%%%%%%%%%%%%%%%%%%%%%%%%%%%%%%%%%%%%%%%%%%

\index{installation!Java}
\index{Java}

Any Condor execution site may be coupled with a
\index{Java Virtual Machine}
\index{JVM}
Java Virtual Machine (JVM) in order to make it Java-capable.
However, Condor must be informed of some details of the
installation in order to execute programs correctly.

Begin by installing a Java distribution according to the vendor's
instructions.  We have successfully used the Sun Java Developer's
Kit, but any distribution should suffice.  Your machine may have
been delivered with a JVM already installed -- it is frequently
found in \File{/usr/bin/java}.

Next, edit your \Condor{config} file.  Adjust the \Expr{JAVA}
setting to point to the JVM binary, typically \File{/usr/bin/java}.
Now, restart the \Condor{startd} on that host.  For example,

\begin{verbatim}
% condor_restart -startd <hostname>
\end{verbatim}

The \Condor{startd} should take a few moments to test the Java
installation, query its properties, and advertise itself as
Java-capable.  If the setup succeeded, then \Condor{status} will
tell you the host is now Java-capable:

\begin{verbatim}
% condor_status -java <hostname>
\end{verbatim}

If, after a few tries, this command does not give any output,
then the startd is having difficulty executing the JVM.
The exact cause of the problem depends on the details of the
JVM, your local installation, and a variety of other factors.
We can't offer you any specific advice on these matters, but
we can outline a general strategy for solving these problems.

To begin debugging the problem, examine the
\Expr{STARTD\_LOG} on that machine.  For example:

\begin{verbatim}
% condor_config_val STARTD_LOG
/var/home/condor/log/StartLog
% more /var/home/condor/log/StartLog
\end{verbatim}

This log will show the command used to start the JVM, and
any error messages should it fail.  Two common problems
are encountered here.

The most common problem is that the JVM
must be explicitly directed to the location
of the standard class files.  If this is the case, then
edit the \Expr{JAVA\_CLASSPATH\_DEFAULT} setting, add the
path to the standard libraries, and restart the \Condor{startd}.
For example, you might have to change this:

\begin{verbatim}
JAVA_CLASSPATH_DEFAULT = $(RELEASE_DIR)/lib .
\end{verbatim}

To this:

\begin{verbatim}
JAVA_CLASSPATH_DEFAULT = /usr/java/lib/classes.zip $(RELEASE_DIR)/lib .
\end{verbatim}

Another common problem is that the JVM requires some extensive 
setup information involving environment variables, path settings,
and the like. If this is the case, we recommended that you put
the necesary information in a wrapper script, and then direct Condor to
execute the wrapper instead of the JVM directly.  For example,
you might set:

\begin{verbatim}
JAVA = /usr/local/bin/java.wrapper
\end{verbatim}

And then create \File{/usr/local/bin/java.wrapper} with:

\begin{verbatim}
#!/bin/sh

LD_LIBRARY_PATH=/usr/java/lib:/usr/lib
export LD_LIBRARY_PATH

exec /usr/bin/java "$@"
\end{verbatim}
