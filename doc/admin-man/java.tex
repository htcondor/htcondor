%%%%%%%%%%%%%%%%%%%%%%%%%%%%%%%%%%%%%%%%%%%%%%%%%%%%%%%%%%%%%%%%%%%%%%
\section{\label{sec:java-install}Installing Java Support in Condor}
%%%%%%%%%%%%%%%%%%%%%%%%%%%%%%%%%%%%%%%%%%%%%%%%%%%%%%%%%%%%%%%%%%%%%%

\index{installation!Java}
\index{Java}

Compiled Java programs may be executed (under Condor) on
any
execution site with a
\index{Java Virtual Machine}
\index{JVM}
Java Virtual Machine (JVM).
To do this,
Condor must be informed of some details of the
JVM installation.

Begin by installing a Java distribution according to the vendor's
instructions.
We have successfully used the Sun Java Developer's Kit,
but any distribution should suffice.
Your machine may have
been delivered with a JVM already installed -- installed code
is frequently found in \File{/usr/bin/java}.

Condor's configuration includes the location of the installed
JVM.
Edit the configuration file.
Modify the \Expr{JAVA} entry to point to the JVM binary,
typically \File{/usr/bin/java}.
Restart the \Condor{startd} daemon on that host.  For example,

\begin{verbatim}
% condor_restart -startd bluejay
\end{verbatim}

The \Condor{startd} daemon takes a few moments to test the Java
installation, query its properties, and then advertise the machine
to the pool as Java-capable.
If the set up succeeded, then \Condor{status} will
tell you the host is now Java-capable by printing the Java
vendor and the version number:

\begin{verbatim}
% condor_status -java bluejay
\end{verbatim}

After a suitable amount of time, if this command does not give any output,
then the \Condor{startd} daemon is having difficulty executing the JVM.
The exact cause of the problem depends on the details of the
JVM, the local installation, and a variety of other factors.
We cannot offer any specific advice on these matters, but
we can provide an approach to solving the problem.

To begin debugging the problem, set the \Condor{startd} daemon
to full debugging mode on the machine.
Within the configuration file, modify the entry \MacroNI{STARTD\_LOG}:
\begin{verbatim}
STARTD_DEBUG = D_FULLDEBUG
\end{verbatim}

Restart the \Condor{startd} daemon (again):
\begin{verbatim}
% condor_restart -startd bluejay
\end{verbatim}

Wait for the \Condor{startd} daemon to (again) test the installation,
and then examine the \Expr{STARTD\_LOG}.
This log will show the command used to start the JVM, and
any error messages should it fail.

% editted to this point in the file

A common problem encountered is that the JVM
must be explicitly directed to the location
of the standard class files.
If this is the case, then
edit the configuration file to modify the
\MacroNI{JAVA\_CLASSPATH\_DEFAULT} entry.
An example of the unmodified configuration file entry might be:
\begin{verbatim}
JAVA_CLASSPATH_DEFAULT = $(LIB) $(LIB)/scimark2lib.jar .
\end{verbatim}
and the modified configuration entry might be:
\begin{verbatim}
JAVA_CLASSPATH_DEFAULT = /usr/java/lib/classes.zip $(LIB) $(LIB)/scimark2.lib.jar .
\end{verbatim}
After this change, (again) restart the \Condor{startd} daemon.


Another common problem occurs when the JVM requires extensive 
set up information involving environment variables, path settings,
and the like.
If this is the case, a solution places
the necessary information in a wrapper script.
Condor is then directed to
execute the wrapper script instead of the JVM directly.
For example, to set Condor to execute the wrapper script instead
of the JVM directly, the \MacroNI{JAVA} configuration
file entry becomes:

\begin{verbatim}
JAVA = /usr/local/bin/java.wrapper
\end{verbatim}

The wrapper script must also be created.
Here is an example of what the script (\File{/usr/local/bin/java.wrapper})
might look like:

\begin{verbatim}
#!/bin/sh

LD_LIBRARY_PATH=/usr/java/lib:/usr/lib
export LD_LIBRARY_PATH

exec /usr/bin/java "$@"
\end{verbatim}
