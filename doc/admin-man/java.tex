%%%%%%%%%%%%%%%%%%%%%%%%%%%%%%%%%%%%%%%%%%%%%%%%%%%%%%%%%%%%%%%%%%%%%%
\section{\label{sec:java-install}Java Support Installation}
%%%%%%%%%%%%%%%%%%%%%%%%%%%%%%%%%%%%%%%%%%%%%%%%%%%%%%%%%%%%%%%%%%%%%%

\index{installation!Java}
\index{Java}

Compiled Java programs may be executed (under Condor) on
any
execution site with a
\index{Java Virtual Machine}
\index{JVM}
Java Virtual Machine (JVM).
To do this,
Condor must be informed of some details of the
JVM installation.

Begin by installing a Java distribution according to the vendor's
instructions.
We have successfully used the Sun Java Developer's Kit,
but any distribution should suffice.
Your machine may have
been delivered with a JVM already installed -- installed code
is frequently found in \File{/usr/bin/java}.

Condor's configuration includes the location of the installed
JVM.
Edit the configuration file.
Modify the \Macro{JAVA} entry to point to the JVM binary,
typically \File{/usr/bin/java}.
Restart the \Condor{startd} daemon on that host.  For example,

\begin{verbatim}
% condor_restart -startd bluejay
\end{verbatim}

The \Condor{startd} daemon takes a few moments to exercise the Java
capabilites of the \Condor{starter}, query its properties,
and then advertise the machine
to the pool as Java-capable.
If the set up succeeded, then \Condor{status} will
tell you the host is now Java-capable by printing the Java
vendor and the version number:

\begin{verbatim}
% condor_status -java bluejay
\end{verbatim}

After a suitable amount of time, if this command does not give any output,
then the \Condor{starter}  is having difficulty executing the JVM.
The exact cause of the problem depends on the details of the
JVM, the local installation, and a variety of other factors.
We can offer only limited advice on these matters,
but here is an approach to solving the problem.

To reproduce the test that the \Condor{starter} is attempting,
try running the Java \Condor{starter} directly.  To find
where the \Condor{starter} is installed, run this command:

\begin{verbatim}
% condor_config_val STARTER
\end{verbatim}

This command prints out the path to the \Condor{starter},
perhaps something like this:

\begin{verbatim}
/usr/condor/sbin/condor_starter
\end{verbatim}

Use this path to execute the \Condor{starter} directly
with the \Arg{-classad} argument.
This tells the starter to run its tests and display its properties.

\begin{verbatim}
/usr/condor/sbin/condor_starter -classad
\end{verbatim}

This command will display a short list of cryptic properties, such as:

\begin{verbatim}
IsDaemonCore = True
HasFileTransfer = True
HasMPI = True
CondorVersion = "$CondorVersion: 6.2$"
\end{verbatim}

If the Java configuration is correct, there will also
be a short list of Java properties, such as:

\begin{verbatim}
JavaVendor = "Sun Microsystems Inc."
JavaVersion = "1.2.2"
JavaMFlops = 9.279696
HasJava = True
\end{verbatim}

If the Java installation is incorrect, then any error
messages from the shell or Java will be printed
on the error stream instead.

One identified difficulty occurs when the machine has
a large quantity of physical RAM, and this quantity
exceeds the Java limitations.
This is a known problem for the Sun JVM.
Condor appends the maximum amount of system RAM
to the Java Maxheap Argument,
and sometimes this value is larger than the JVM allows.
The end result is that Condor believes that the JVM on the machine
is faulty,
resulting in nothing showing up as a result of executing
the command \ShortExpr{condor\_status -java}.

The way to work around this particular problem is to modify
the configuration file for those machines that may execute
Java universe jobs.
The \Macro{JAVA\_MAXHEAP\_ARGUMENT} macro is explicitly set
to null in the configuration, to prevent Condor from appending
the machine-specific, but too-big value.
Then the Java Maxheap Argument is set (again, in the configuration)
to the maximum value allowed for the JVM on that platform,
using the \Macro{JAVA\_EXTRA\_ARGUMENTS} configuration variable.
Note that the name of the switch that regulates the Java
Maxheap Argument is different for different vendors' JVM.

The following is an example of the configuration fix for the Sun JVM:

\footnotesize
\begin{verbatim}
# First set JAVA_MAXHEAP_ARGUMENT to null, to disable the default of max RAM
JAVA_MAXHEAP_ARGUMENT =
# Now set the argument with the Sun-specific maximum allowable value
JAVA_EXTRA_ARGUMENTS = -Xmx1906m
\end{verbatim}
\normalsize


