%%%%%%%%%%%%%%%%%%%%%%%%%%%%%%%%%%%%%%%%%%%%%%%%%%%%%%%%%%%%%%%%%%%%%%
\subsection{\label{sec:EventD}
Condor Event Daemon}
%%%%%%%%%%%%%%%%%%%%%%%%%%%%%%%%%%%%%%%%%%%%%%%%%%%%%%%%%%%%%%%%%%%%%%

\index{daemon!eventd}
\index{event daemon}
\index{contrib module!event daemon}

The event daemon is an administrative tool for scheduling events in a
Condor pool.
Every \Macro{EVENTD\_INTERVAL}, for each defined event, the event
daemon (eventd) computes an estimate of the time required to complete or
prepare for the event.  If the time required is less than the time
between the next interval and the start of the event, the event daemon
activates the event.

Currently, this daemon supports \Macro{SHUTDOWN} events, which place machines
in the owner state during scheduled times.
The eventd causes machines to vacate jobs one at a time
in anticipation of \Macro{SHUTDOWN} events.
Scheduling this improves performance, because the machines
do not all attempt to checkpoint their jobs at the same time.
To determine the estimate of the time required to complete a \Macro{SHUTDOWN}
event, the \Attr{ImageSize} values for all running standard universe jobs are
totalled and then divided by the maximum bandwidth specified for this
event.

When a \Macro{SHUTDOWN} event is activated, the eventd contacts all startd
daemons
that match constraints given in the configuration file,
and instructs them to shut down.
In response to this instruction,
the startd on any machine not running a job will immediately transition to
the owner state.
Any machine currently running a job will continue to run the
job, but will not start any new job.
The eventd then sends a vacate command to the each startd
that is currently running a job.
Once the job is vacated, the startd transitions to the
owner state.

\Condor{eventd} must run on a machine with administrator
access to your pool.
See section~\ref{sec:Host-Security} on
page~\pageref{sec:Host-Security} for full details about IP/host-based
security in Condor.

%%%%%%%%%%%%%%%%%%%%%%%%%%%%%%%%%%%%%%%%%%%%%%%%%%%%%%%%%%%%%%%%%%%%%%
\subsubsection{\label{sec:EventD-Installation}
Installing the Event Daemon} 
%%%%%%%%%%%%%%%%%%%%%%%%%%%%%%%%%%%%%%%%%%%%%%%%%%%%%%%%%%%%%%%%%%%%%%

\Condor{eventd} requires version 6.1.3 or later of
\Condor{startd}.
So, you should first install either the latest version of the SMP
\Condor{startd} contrib module or the latest release of Condor version
6.1.

First, download the \Condor{eventd} contrib module.
Uncompress and untar the file, to have a directory that
contains a \File{eventd.tar}.
The \File{eventd.tar} acts much like the \File{release.tar} file from
a main release.
This archive contains the files:
\begin{verbatim}
	sbin/condor_eventd
	etc/examples/condor_config.local.eventd
\end{verbatim}
These are all new files, not found in the main release, so you can
safely untar the archive directly into your existing release
directory.
The file \File{\condor{eventd}} is the eventd binary.
The example configuration file is described below.

%%%%%%%%%%%%%%%%%%%%%%%%%%%%%%%%%%%%%%%%%%%%%%%%%%%%%%%%%%%%%%%%%%%%%%
\subsubsection{\label{sec:EventD-Configuration}
Configuring the Event Daemon} 
%%%%%%%%%%%%%%%%%%%%%%%%%%%%%%%%%%%%%%%%%%%%%%%%%%%%%%%%%%%%%%%%%%%%%%

The file \File{etc/examples/condor\_config.local.eventd} contains an
example configuration.
To define events, first set the \Macro{EVENT\_LIST} macro.
This macro contains a list of macro names which define the individual
events.
The definition of individual events depends on the type of the event.
Currently, there is only one event type: \Macro{SHUTDOWN}.
The format for \Macro{SHUTDOWN} events is
\begin{verbatim}
	SHUTDOWN DAY TIME DURATION BANDWIDTH CONSTRAINT RANK
\end{verbatim}
\verb@TIME@ and \verb@DURATION@ are specified in an hours:minutes format.

For example:
\index{event daemon!example configuration}
\begin{verbatim}
EVENT_LIST	= TestEvent, TestEvent2
TestEvent	= SHUTDOWN W 16:00 1:00 2.5 TestEventConstraint TestEventRank
TestEvent2	= SHUTDOWN F 14:00 0:30 6.0 TestEventConstraint2 TestEventRank
TestEventConstraint		= (Arch == "INTEL")
TestEventConstraint2		= (True)
TestEventRank			= (0 - ImageSize)
\end{verbatim}

In this example, the \verb@TestEvent@ is a \Macro{SHUTDOWN} type event, which
specifies that all machines whose startd ads match the constraint
\verb@Arch == "INTEL"@ should be shutdown for one hour starting at
16:00 every Wednesday, and no more than 2.5 Mbytes/s of bandwidth
should be used to vacate jobs in anticipation of the shutdown
event.  According to the \verb@TestEventRank@, jobs will be vacated in
reverse order of their \Attr{ImageSize} (larger jobs first, smaller jobs
last).  \verb@TestEvent2@ is a \Macro{SHUTDOWN} type event, which specifies
that all machines should be shutdown for 30 minutes starting at
14:00 every Friday, and no more than 6.0 Mbytes/s of bandwidth should
be used to vacate jobs in anticipation of the shutdown event.

Note that the \Macro{DAEMON\_LIST} macro (described in
section~\ref{sec:Master-Config-File-Entries}) is defined in the
section of settings you may want to customize.
If you want the event daemon managed by the \Condor{master}, the
\Macro{DAEMON\_LIST} entry must contain both 
\Attr{MASTER} and \Attr{EVENTD}.
Verify that this macro is set to run the correct daemons on
this machine.  By default, the list also includes
\Attr{SCHEDD} and \Attr{STARTD}.

See section~\ref{sec:Eventd-Config-File-Entries} on
page~\pageref{sec:Eventd-Config-File-Entries} for a description of
optional event daemon parameters.

%%%%%%%%%%%%%%%%%%%%%%%%%%%%%%%%%%%%%%%%%%%%%%%%%%%%%%%%%%%%%%%%%%%%%%
\subsubsection{\label{sec:Start-EventD} 
Starting the Event Daemon} 
%%%%%%%%%%%%%%%%%%%%%%%%%%%%%%%%%%%%%%%%%%%%%%%%%%%%%%%%%%%%%%%%%%%%%%

To start an event daemon once it is configured to run on a given
machine, restart Condor on that given machine to enable
the \Condor{master} to notice the new configuration.
Send a \Condor{restart} command from any machine
with administrator access to your pool.
See section~\ref{sec:Host-Security} on
page~\pageref{sec:Host-Security} for full details about IP/host-based
security in Condor.

