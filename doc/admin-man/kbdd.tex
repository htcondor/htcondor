%%%%%%%%%%%%%%%%%%%%%%%%%%%%%%%%%%%%%%%%%%%%%%%%%%%%%%%%%%%%%%%%%%%%%%
\subsection{\label{sec:kbdd}Installing the \Condor{kbdd}}
%%%%%%%%%%%%%%%%%%%%%%%%%%%%%%%%%%%%%%%%%%%%%%%%%%%%%%%%%%%%%%%%%%%%%%

The condor keyboard daemon (\Condor{kbdd}) monitors X events on
machines where the operating system does not provide a way of
monitoring the idle time of the keyboard or mouse.  In particular,
this is necessary on Digital Unix machines and IRIX machines.  

\Note If you are running on Solaris, Linux, or HP/UX, you do
not need to use the keyboard daemon.

Although great measures have been taken to make this daemon as robust
as possible, the X window system was not designed to facilitate such a
need, and thus is less then optimal on machines where many users log
in and out on the console frequently.

In order to work with X authority, the system by which X authorizes
processes to connect to X servers, the condor keyboard daemon needs to
run with super user privileges.  Currently, the daemon assumes that X
uses the \Env{HOME} environment variable in order to locate a file
named \File{.Xauthority}, which contains keys necessary to connect to
an X server.  The keyboard daemon attempts to set this environment
variable to various users home directories in order to gain a
connection to the X server and monitor events.  This may fail to work
on your system, if you are using a non-standard approach.  If the
keyboard daemon is not allowed to attach to the X server, the state of
a machine may be incorrectly set to idle when a user is, in fact,
using the machine.

In some environments, the keyboard daemon will not be able to connect to the X
server because the user currently logged into the system keeps their
authentication token for using the X server in a place that no local user on
the current machine can get to.  This may be the case if you are running AFS
and have the user's X authority file in an AFS home directory.  There may also
be cases where you cannot run the daemon with super user privileges because of
political reasons, but you would still like to be able to monitor X activity.
In these cases, you will need to change your XDM configuration in order to
start up the keyboard daemon with the permissions of the currently logging in
user.  Although your situation may differ, if you are running X11R6.3, you
will probably want to edit the files in /usr/X11R6/lib/X11/xdm.  The Xsession
file should have the keyboard daemon startup at the end, and the Xreset file
should have the keyboard daemon shutdown.  As of patch level 4 of Condor
version 6.0, the keyboard daemon has some additional command line options to
facilitate this.  The -l option can be used to write the daemons log file to a
place where the user running the daemon has permission to write a file.  We
recommend something akin to \$HOME/.kbdd.log since this is a place where every
user can write and won't get in the way.  The -pidfile and -k options allow
for easy shutdown of the daemon by storing the process id in a file.  You will
need to add lines to your XDM config that look something like this:

\begin{verbatim}
	condor_kbdd -l $HOME/.kbdd.log -pidfile $HOME/.kbdd.pid
\end{verbatim}

This will start the keyboard daemon as the user who is currently logging in
and write the log to a file in the directory \$HOME/.kbdd.log/.  Also, this
will save the process id of the daemon to ~/.kbdd.pid, so that when the user
logs out, XDM can simply do a:

\begin{verbatim}
	condor_kbdd -k $HOME/.kbdd.pid
\end{verbatim}

This will shutdown the process recorded in ~/.kbdd.pid and exit.

To see how well the keyboard daemon is working on your system, review
the log for the daemon and look for successful connections to the X
server.  If you see none, you may have a situation where the keyboard
daemon is unable to connect to your machines X server.  If this
happens, please send mail to \Email{condor-admin@cs.wisc.edu} and let
us know about your situation.

