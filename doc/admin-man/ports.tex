%%%%%%%%%%%%%%%%%%%%%%%%%%%%%%%%%%%%%%%%%%%%%%%%%%%%%%%%%%%%%%%%%%%%%%%%%%%
\subsection{\label{sec:Port-Details}Port Usage in Special Evironments }
%%%%%%%%%%%%%%%%%%%%%%%%%%%%%%%%%%%%%%%%%%%%%%%%%%%%%%%%%%%%%%%%%%%%%%%%%%

\index{port usage}

%%%%%%%%%%%%%%%%%%%%%%%%%%%%%%%%%%%%%%%%%%%%%%%%%%%%%%%%%%%%%%%%%%%%%%%%%%%
\subsubsection{\label{sec:Ports-NonStandard}Non Standard Ports for Central Managers}
%%%%%%%%%%%%%%%%%%%%%%%%%%%%%%%%%%%%%%%%%%%%%%%%%%%%%%%%%%%%%%%%%%%%%%%%%%%
\index{port usage!nonstandard ports for central managers}
By default,
Condor uses port 9618 for the \Condor{collector} daemon
and 9614 for the \Condor{negotiator} daemon.
To use non standard port numbers for these daemons,
the configuration variables that tell Condor these communication
details are modified.
Instead of
\begin{verbatim}
CONDOR_HOST = machX.cs.wisc.edu
COLLECTOR_HOST = $(CONDOR_HOST)
NEGOTIATOR_HOST = $(CONDOR_HOST)
\end{verbatim}
the configuration might be
\begin{verbatim}
CONDOR_HOST = machX.cs.wisc.edu
COLLECTOR_HOST = $(CONDOR_HOST):9650
NEGOTIATOR_HOST = $(CONDOR_HOST):9651
\end{verbatim}

If a non-standard port is defined, the same value of
\MacroNI{COLLECTOR\_HOST} (including the port) must be used for all
machines in the Condor pool.
Therefore, this setting should be modified in the global
\File{condor\_config} file, or the value should be duplicated across
all configuration files in the pool if a single \File{condor\_config}
is not being shared.

On pools that are made up from only one machine, you can configure the
\Condor{collector} and \Condor{negotiator} to use a dynamically
assigned port given out by the operating system.
This prevents any possible port conflict with other services on the
same machine.
However, dynamically assigned ports should only be used on
single-machine Condor pools, and only if the
\Macro{COLLECTOR\_ADDRESS\_FILE} and \Macro{NEGOTIATOR\_ADDRESS\_FILE}
settings have also been enabled.
This mechanism allows all of the Condor daemons and tools running on
the same machine to find the real port for the \Condor{collector} and
\Condor{negotiator}, even if these ports are not defined in the
configuration file and are not known in advance.

To enable these daemons to use a dynamic port, the port number should
be set to 0 in the \Macro{COLLECTOR\_HOST} and
\Macro{NEGOTIATOR\_HOST} settings.
For example:
\begin{verbatim}
COLLECTOR_HOST = $(CONDOR_HOST):0
NEGOTIATOR_HOST = $(CONDOR_HOST):0
COLLECTOR_ADDRESS_FILE = $(LOG)/.collector_address
NEGOTIATOR_ADDRESS_FILE = $(LOG)/.negotiator_address
\end{verbatim}

For more information about the \MacroNI{COLLECTOR\_ADDRESS\_FILE}
setting, see section~\ref{param:SubsysAddressFile} on
page~\pageref{param:SubsysAddressFile}.
For more information about the \MacroNI{COLLECTOR\_HOST} or
\MacroNI{NEGOTIATOR\_HOST} settings, please see
section~\ref{param:CollectorHost} on
page~\pageref{param:CollectorHost}.



%%%%%%%%%%%%%%%%%%%%%%%%%%%%%%%%%%%%%%%%%%%%%%%%%%%%%%%%%%%%%%%%%%%%%%%%%%%
\subsubsection{\label{sec:Ports-Firewalls}Firewalls}
%%%%%%%%%%%%%%%%%%%%%%%%%%%%%%%%%%%%%%%%%%%%%%%%%%%%%%%%%%%%%%%%%%%%%%%%%%%

\index{port usage!firewalls}
If a Condor pool is completely behind a firewall,
then no special consideration is needed.
However, if there is a firewall between the machines within
a Condor pool, then
configuration variables may be set to force the usage of
specific ports and to utilize a specific range of ports.

By default,
Condor uses port 9618 for the \Condor{collector} daemon,
9614 for the \Condor{negotiator} daemon,
and system-assigned (apparently random) ports for everything else.
See section~\ref{sec:Ports-NonStandard},
if non standard ports are to be used for the
\Condor{collector} \Condor{negotiator} daemons.

The configuration variables
\Macro{HIGHPORT} and \Macro{LOWPORT} facilitate setting a restricted
range of ports that Condor will use.
This may be useful if behind a firewall.
The configuration macros
\MacroNI{HIGHPORT} and \MacroNI{LOWPORT} only affect these
system-assigned ports, but will restrict them to the range specified.
These configuration variables are fully defined
in section~\ref{sec:Condor-wide-Config-File-Entries}.
Note that both \MacroNI{HIGHPORT} and \MacroNI{LOWPORT} must be at least 1024.

% From Alain, Karen is awaiting confirmation on 7/7/03
% If a machine is an execution machine, it needs several ports. By default,
% you can have one job per CPU on your machine (this can be changed by
% defining virtual CPUs). Someone will need to estimate this better than me,
% but you need something like:
% 
% 1 port per job that can run (condor_starter)
% 1 port for condor_master
% 1 port for condor_startd (advertises machine)
%
% It possible that we need more than that.
% 
% If a machine is an submission machine, you'll need:
% 
% 1 port for condor_master
% 1 port for schedd
% 1 port per active job that you have

%%%%%%%%%%%%%%%%%%%%%%%%%%%%%%%%%%%%%%%%%%%%%%%%%%%%%%%%%%%%%%%%%%%%%%%%%%%
\subsubsection{\label{sec:Ports-MultipleCollectors}Multiple Collectors}
%%%%%%%%%%%%%%%%%%%%%%%%%%%%%%%%%%%%%%%%%%%%%%%%%%%%%%%%%%%%%%%%%%%%%%%%%%%
\index{port usage!multiple collectors}
\Todo


%%%%%%%%%%%%%%%%%%%%%%%%%%%%%%%%%%%%%%%%%%%%%%%%%%%%%%%%%%%%%%%%%%%%%%%%%%%
\subsubsection{\label{sec:Ports-Conflicts}Port Conflicts}
%%%%%%%%%%%%%%%%%%%%%%%%%%%%%%%%%%%%%%%%%%%%%%%%%%%%%%%%%%%%%%%%%%%%%%%%%%%
\index{port usage!conflicts}
\Todo

