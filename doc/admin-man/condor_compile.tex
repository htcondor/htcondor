In order to take advantage of two major condor features: checkpointing and
remote system calls, users of the condor system need to relink their
binaries.  The condor system can be set up in such a way that this can be made
nearly transparent to users.  

The basic idea here is to replace the system linker (ld) with the condor
linker.  Then, when a program is to be linked, the condor linker figures out
whether this binary will be for condor, or for a normal binary.  If it is to
be a normal compile, the old ld is called.  If this binary is to be linked for
condor, the script performs the necessary operations in order to prepare a
binary that can be used with condor.  In order to differentiate between normal
builds and condor builds, all a user need do is prefix condor_compile before
there build command whether that be make, gcc, or any other command.

In order to install this condor_compile facility, the following steps need
to be followed.
	
	1. Rename the system linker from ld to ld.real.
	2. Copy the condor linker to the location of the previous ld.
	3. Set the owner of the linker to root.
	4. Set the permissions on the new linker to 755.

The actual commands that you must execute depend upon the system that you
are on.  The location of the system linker (ld), is as follows:

	Operating System              Location of ld (ld-path)
	Linux                         /usr/bin
	Solaris 2.X, HP-UX 10.x       /usr/ccs/bin
	OSF/1 (Digital Unix)          /usr/lib/cmplrs/cc

On these platforms, issue the following commands (as root), where [ld-path] is
replaced by the path to your systems linker.

	mv /[ld-path]/ld /[ld-path]/ld.real
        cp /usr/local/condor/lib/ld /[ld-path]/ld
        chown root /[ld-path]/ld
        chmod 755 /[ld-path]/ld

On IRIX, things are more complicated in that there are multiple ld binaries
that need to be moved, and symbolic links need to be made in order to convince
the linker to work, since it looks at the name of it's own binary in order
to figure out what to do.

	mv /usr/lib/ld /usr/lib/ld.real
        mv /usr/lib/uld /usr/lib/uld.real
        cp /usr/local/condor/lib/ld /usr/lib/ld
        ln /usr/lib/ld /usr/lib/uld
        chown root /usr/lib/ld /usr/lib/uld
        chmod 755 /usr/lib/ld /usr/lib/uld
        mkdir /usr/lib/condor
        chown root /usr/lib/condor
        chmod 755 /usr/lib/condor
        ln -s /usr/lib/uld.real /usr/lib/condor/uld
        ln -s /usr/lib/uld.real /usr/lib/condor/old_ld

If you remove condor from your system latter on, linking will continue to
work, since the condor linker will always resolve that you are compiling
normal binaries and simply call the real ld.  In the interest of simplicity,
it is recomended that you reverse the above changes by moving your ld.real
linker back to it's former position as ld, overwriting the condor linker.  On
IRIX, you need to do this for both linkers, and you will probably want to
remove the symbolic links as well.


