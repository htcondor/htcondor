%%%%%%%%%%%%%%%%%%%%%%%%%%%%%%%%%%%%%%%%%%%%%%%%%%%%%%%%%%%%%%%%%%%%%%
\subsection{\label{sec:Contrib-CondorView-Install}
Configuring The CondorView Server}
%%%%%%%%%%%%%%%%%%%%%%%%%%%%%%%%%%%%%%%%%%%%%%%%%%%%%%%%%%%%%%%%%%%%%%

\index{CondorView!Server}
The CondorView server is an alternate use of the
\Condor{collector}
that logs information on disk, providing a 
persistent, historical database of pool state.
This includes machine state, as well as the state of jobs submitted by
users.

You may configure your existing \Condor{collector} to act as the
CondorView collector.  This is the simplest configuration, because you
do not need to do anything other than turn on logging of historical
information.  Configuring a new \Condor{collector} to act as the
CondorView collector is not much more complicated and it offers the
advantage that you can use the same CondorView collector for several pools
if you wish to aggregate information into one place.

The following sections describe how to configure a machine to run a
CondorView server and to configure your pool to send updates to it. 


%%%%%%%%%%%%%%%%%%%%%%%%%%%%%%%%%%%%%%%%%%%%%%%%%%%%%%%%%%%%%%%%%%%%%%
\subsubsection{\label{sec:CondorView-Server-Setup}
Configuring a Machine to be a CondorView Server} 
%%%%%%%%%%%%%%%%%%%%%%%%%%%%%%%%%%%%%%%%%%%%%%%%%%%%%%%%%%%%%%%%%%%%%%

\index{CondorView!configuration}

To configure the CondorView collector, you have to add a few settings
to the configuration file for the \Condor{collector} chosen to act
as the CondorView collector.
These settings are described in detail in the Condor \VersionNotice\ 
Administrator's Manual, in
section~\ref{sec:Collector-Config-File-Entries} on
page~\pageref{sec:Collector-Config-File-Entries}.
A short explanation of the entries you must customize is provided
below.

\begin{description}

\item[\Macro{POOL\_HISTORY\_DIR}] This is the directory where
historical data will be stored.
This directory must be writable by whatever user the CondorView
collector is running as (usually the user condor).  
There is a configurable limit to the maximum space required for all
the files created by the CondorView server called
(\Macro{POOL\_HISTORY\_MAX\_STORAGE}). 

\Note This directory should be separate and different from the
\File{spool} or \File{log} directories already set up for
Condor.
There are a few problems putting these files into either of those
directories.

\item[\Macro{KEEP\_POOL\_HISTORY}] This is a boolean value that determines
if the CondorView collector should store the historical information.
It is false by default, which is why you must specify it as true in
your local configuration file to enable data collection.

\end{description}

Once these settings are in place in the configuration file for your
CondorView server host, you must create the directory you specified
in \MacroNI{POOL\_HISTORY\_DIR} and make it writable by the user your
CondorView collector is running as.
This is the same user that owns the \File{CollectorLog} file in
your \File{log} directory. The user is usually condor.

If you are using your existing \Condor{collector} as the CondorView collector,
no further configuration is needed.  If you wish to run a different
\Condor{collector} to act as the CondorView collector, you must configure
Condor to automatically start it.

If you have chosen a separate host for the CondorView collector,
starting it up is simply a matter of adding \MacroNI{COLLECTOR} to
\MacroNI{DAEMON\_LIST} and restarting condor on that host.  If you wish
to run the CondorView collector on the same host as another collector,
you must ensure that the two collectors use different network ports.
Here is an example configuration in which the main collector and the
CondorView collector are started up by the same \Condor{master} on
the same machine.  In this example, the CondorView collector uses
port 12345.

\begin{verbatim}
        VIEW_SERVER = $(COLLECTOR)
	VIEW_SERVER_ARGS = -f -p 12345
        VIEW_SERVER_ENVIRONMENT = "_CONDOR_COLLECTOR_LOG=$(LOG)/ViewServerLog"
        DAEMON_LIST = MASTER, NEGOTIATOR, COLLECTOR, VIEW_SERVER
\end{verbatim}

For this change to take effect, you must re-start the
\Condor{master} on this host (which you can do with the
\Condor{restart} command, if you run the command with
administrator access to your pool.


%%%%%%%%%%%%%%%%%%%%%%%%%%%%%%%%%%%%%%%%%%%%%%%%%%%%%%%%%%%%%%%%%%%%%%
\subsubsection{\label{sec:CondorView-Pool-Setup}
Configuring a Pool to Report to the CondorView Server} 
%%%%%%%%%%%%%%%%%%%%%%%%%%%%%%%%%%%%%%%%%%%%%%%%%%%%%%%%%%%%%%%%%%%%%%

For the CondorView server to function, configure the existing collector to
forward ClassAd updates to it.  This is only necessary if you have chosen to
have the CondorView collector be a different collector from the existing
collector for the pool.
All the Condor daemons in the pool send their ClassAd updates to the
regular \Condor{collector}, which in turn will forward them on to the
CondorView server.

Define the following configuration variable:
\begin{verbatim}
  CONDOR_VIEW_HOST = full.hostname[:portnumber]
\end{verbatim}
where \verb@full.hostname@ is the full host name of the machine 
running the CondorView collector.
The full host name is optionally followed by a colon and
port number.  This is only necessary if you have configured the CondorView
collector to use a port number other than the default.

Place this setting in the configuration file used by the existing collector.
It is ok to put it in the global configuration file if you wish.  The
CondorView collector will ignore this setting (as it should) if it sees
that it is being asked to forward ClassAds to itself.

Once the CondorView server is running and you have made the above
change, you can send a
\Condor{reconfig} to your main \Condor{collector} for the change to
take effect so it will begin forwarding updates.  You should then be
able to query the CondorView collector to verify that it is working.
Example:

\begin{verbatim}
condor_status -pool condor.view.host[:portnumber]
\end{verbatim}
