%%%%%%%%%%%%%%%%%%%%%%%%%%%%%%%%%%%%%%%%%%%%%%%%%%%%%%%%%%%%%%%%%%%%%%
\subsection{\label{sec:Install-PVM-Condor}
Installing PVM Support in Condor} 
%%%%%%%%%%%%%%%%%%%%%%%%%%%%%%%%%%%%%%%%%%%%%%%%%%%%%%%%%%%%%%%%%%%%%%

\index{contrib module!PVM}
\index{PVM contrib module}
\index{installation!PVM contrib module}

To install the PVM contrib module, you must first download the
appropriate binary module for whatever platform(s) you plan to use for 
Condor-PVM.
You can find all of the Condor binary modules at
\Url{http://www.cs.wisc.edu/condor/downloads}.

\Note The PVM contrib module version must match with your installed
Condor version.

Once you have downloaded each module, uncompressed and untarred it,
you will be left with a directory that contains a \File{pvm.tar},
\File{README} and so on.
The \File{pvm.tar} acts much like the \File{release.tar} file for a
main release.
It contains all the binaries and supporting files you would install in
your release directory to enable Condor-PVM:
\begin{verbatim}
        sbin/condor_pvmd
        sbin/condor_pvmgs
        sbin/condor_shadow.pvm
        sbin/condor_starter.pvm
\end{verbatim}

You must install these files in the release directory for the platform
they were built for.  
Since these files do not exist in a main release, you can safely untar
the \File{pvm.tar} directly into the appropriate release directory.
You do not need to worry about shutting down Condor, moving files out
of the way, and so on.
Once the \File{pvm.tar} file has been untarred into the release
directory, you are done installing the PVM contrib module.
You will now be able to submit PVM jobs to your Condor pool.  

For complete documentation on using PVM in Condor, see the
section~\ref{sec:PVM} on page~\pageref{sec:PVM} entitled ``Parallel
Applications in Condor: Condor-PVM''. 
