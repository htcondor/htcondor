%%%%%%%%%%%%%%%%%%%%%%%%%%%%%%%%%%%%%%%%%%%%%%%%%%%%%%%%%%%%%%%%%%%%%%
\section{\label{sec:Pool-Management}Pool Management}
%%%%%%%%%%%%%%%%%%%%%%%%%%%%%%%%%%%%%%%%%%%%%%%%%%%%%%%%%%%%%%%%%%%%%%

There are a number of administrative tools Condor provides to help you
manage your pool.
The following sections describe various tasks you might wish to
perform on your pool and explains how to most efficiently do them. 

All of the commands described in this section must be run from a
machine listed in the \Macro{HOSTALLOW\_ADMINISTRATOR} setting in
your config files, so that the IP/host-based security allows the
administrator commands to be serviced.
See section~\ref{sec:Host-Security} on
page~\pageref{sec:Host-Security} for full details about IP/host-based
security in Condor.

%%%%%%%%%%%%%%%%%%%%%%%%%%%%%%%%%%%%%%%%%%%%%%%%%%%%%%%%%%%%%%%%%%%%%%
\subsection{\label{sec:Pool-Shutdown-and-Restart}
Shutting Down and Restarting your Condor Pool}
%%%%%%%%%%%%%%%%%%%%%%%%%%%%%%%%%%%%%%%%%%%%%%%%%%%%%%%%%%%%%%%%%%%%%%

There are a couple of situations where you might want to shutdown and
restart your entire Condor pool.
In particular, when you want to install new binaries, it is generally
best to make sure no jobs are running, shutdown Condor, and then
install the new daemons.

%%%%%%%%%%%%%%%%%%%%%%%%%%%%%%%%%%%%%%%%%%%%%%%%%%%%%%%%%%%%%%%%%%%%%%
\subsubsection{\label{sec:Pool-Shutdown}Shutting Down your Condor Pool}
%%%%%%%%%%%%%%%%%%%%%%%%%%%%%%%%%%%%%%%%%%%%%%%%%%%%%%%%%%%%%%%%%%%%%%

The best way to shutdown your pool is to take advantage of the remote
administration capabilities of the \Condor{master}.
The first step is to save the IP address and port of the
\Condor{master} daemon on all of your machines to a file, so that 
even if you shutdown your \Condor{collector}, you can still send
administrator commands to your different machines.
You do this with the following command:
\begin{verbatim}
        % condor_status -master -format "%s\n" MasterIpAddr > addresses
\end{verbatim}

The first step to shutting down your pool is to shutdown any currently
running jobs and give them a chance to checkpoint.
Depending on the size of your pool, your network infrastructure, and
the image-size of the standard jobs running in your pool, you may want
to make this a slow process, only vacating one host at a time.
You can either shutdown hosts that have jobs submitted (in which case
all the jobs from that host will try to checkpoint simultaneously), or
you can shutdown individual hosts that are running jobs.
To shutdown a host, simply send:
\begin{verbatim}
        % condor_off hostname
\end{verbatim}
where ``hostname'' is the name of the host you want to shutdown.
This will only work so long as your \Condor{collector} is still
running.
Once you have shutdown Condor on your central manager, you will have
to rely on the \File{addresses} file you just created.

If all the running jobs are checkpointed and stopped, or if you're not
worried about the network load put in effect by shutting down
everything at once, it is safe to turn off all daemons on all machines
in your pool.
You can do this with one command, so long as you run it from a blessed
administrator machine:
\begin{verbatim}
        % condor_off `cat addresses`
\end{verbatim}
where \File{addresses} is the file where you saved your master
addresses. 
\Condor{off} will shutdown all the daemons, but leave the
\Condor{master} running, so that you can send a \Condor{on} in the
future.  

Once all of the Condor daemons (except the \Condor{master}) on each
host is turned off, you're done.
You are now safe to install new binaries, move your checkpoint server
to another host, or any other task that requires the pool to be
shutdown to successfully complete.

\Note If you are planning to install a new \Condor{master} binary, be
sure to read the following section for special considerations with
this somewhat delicate task.

%%%%%%%%%%%%%%%%%%%%%%%%%%%%%%%%%%%%%%%%%%%%%%%%%%%%%%%%%%%%%%%%%%%%%%
\subsubsection{\label{sec:New-Master}Installing a New \condor{master}}
%%%%%%%%%%%%%%%%%%%%%%%%%%%%%%%%%%%%%%%%%%%%%%%%%%%%%%%%%%%%%%%%%%%%%%

If you are going to be installing a new \Condor{master} binary, there
are a few other steps you should take.
If the \Condor{master} restarts, it will have a new port it is
listening on, so your \File{addresses} file will be stale information.
Moreover, when the master restarts, it doesn't know that you sent it a
\Condor{off} in its past life, and will just start up all the daemons
it's configured to spawn unless you explicitly tell it otherwise.

If you just want your pool to completely restart itself whenever the
master notices its new binary, neither of these issues are of any
concern and you can skip this (and the next) section.
Just be sure installing the new master binary is the last thing you
install, and once you put the new binary in place, the pool will
restart itself over the next 5 minutes (whenever all the masters
notice the new binary, which they each check for once every 5 minutes
by default).

However, if you want to have absolute control over when the rest of
the daemons restart, you must take a few steps.

\begin{enumerate}
\item Put the following setting in your global config file:
\begin{verbatim}
        START_DAEMONS = False
\end{verbatim}
This will make sure that when the master restarts itself that it 
doesn't also start up the rest of its daemons.
\item Install your new \Condor{master} binary.
\item Start up Condor on your central manager machine.
You will have to do this manually by logging into the machine and
sending commands locally.
First, send a \Condor{restart} to make sure you've got the new master,
then send a \Condor{on} to start up the other daemons (including, most
importantly, the \Condor{collector}).
\item Wait 5 minutes, such that all the masters have a chance to
notice the new binary, restart themselves, and send an update with
their new address.  Make sure that: 
\begin{verbatim}
        % condor_status -master
\end{verbatim}
lists all the machines in your pool.
\item Remove the special setting from your global config file.
\item Recreate your \File{addresses} file as described above:
\begin{verbatim}
        % condor_status -master -format "%s\n" MasterIpAddr > addresses
\end{verbatim}
\end{enumerate}

Once the new master is in place, and you're ready to start up your
pool again, you can restart your whole pool by simply following the
steps in the next section.

%%%%%%%%%%%%%%%%%%%%%%%%%%%%%%%%%%%%%%%%%%%%%%%%%%%%%%%%%%%%%%%%%%%%%%
\subsubsection{\label{sec:Pool-Restart}Restarting your Condor Pool}
%%%%%%%%%%%%%%%%%%%%%%%%%%%%%%%%%%%%%%%%%%%%%%%%%%%%%%%%%%%%%%%%%%%%%%

Once you are done performing whatever tasks you need to perform and
you're ready to restart your pool, you simply have to send a
\Condor{on} to all the \Condor{master} daemons on each host.
You can do this with one command, so long as you run it from a blessed
administrator machine:
\begin{verbatim}
        % condor_on `cat addresses`
\end{verbatim}
That's it.  All your daemons should now be restarted, and your pool
will be back on its way.

%%%%%%%%%%%%%%%%%%%%%%%%%%%%%%%%%%%%%%%%%%%%%%%%%%%%%%%%%%%%%%%%%%%%%%
\subsection{\label{sec:Reconfigure-Pool}Reconfiguring Your Condor Pool}
%%%%%%%%%%%%%%%%%%%%%%%%%%%%%%%%%%%%%%%%%%%%%%%%%%%%%%%%%%%%%%%%%%%%%%

If you change a global config file setting and want to have all your
machines start to use the new setting, you must send a
\Condor{reconfig} command to each host.
You can do this with one command, so long as you run it from a blessed
administrator machine:
\begin{verbatim}
        % condor_reconfig `condor_status -master`
\end{verbatim}

\Note If your global config file is not shared among all your machines
(using a shared filesystem), you will need to make the change to each
copy of your global config file before sending the \Condor{reconfig}.
