%%%%%%%%%%%%%%%%%%%%%%%%%%%%%%%%%%%%%%%%%%%%%%%%%%%%%%%%%%%%%%%%%%%%%%%%%%%
\subsection{\label{sec:High-Availability}
Configuring Condor for the High Availability of Daemons} 
%%%%%%%%%%%%%%%%%%%%%%%%%%%%%%%%%%%%%%%%%%%%%%%%%%%%%%%%%%%%%%%%%%%%%%%%%%

\index{High Availability}
In the case that a key machine no longer functions,
Condor can be configured such that another machine assumes
the key functions.
This is called \emph{High Availability}.

While \emph{High Availability} is generally applicable,
there is currently one specialized case for its use.
For a pool where all jobs are submitted through
a single machine in the pool,
and there are lots of jobs,
this machine becoming nonfunctional means that
jobs stop running.
The \Condor{schedd} daemon maintains the job queue.
No job queue due to having a nonfunctional machine
implies that no jobs can be run.
This situation is worsened by using one machine
as the single submission point.
For each Condor job (taken from the queue) that is executed,
a \Condor{shadow} process runs on the machine where submitted
to handle input/output functionality.
If this machine becomes nonfunctional, none of the jobs can
continue.
The entire pool stops running jobs.

The goal of \emph{High Availability} in this special case is
to transfer the \Condor{schedd} daemon to run on another
designated machine.
Jobs caused to stop without finishing can be restarted from the
beginning, or can continue execution using the most recent checkpoint.
New jobs can enter the job queue.
Without \emph{High Availability},
the job queue would remain intact, but further progress on jobs
would wait until the machine running the \Condor{schedd} daemon
became available (after fixing whatever caused it to become
unavailable).

Condor uses its flexible configuration mechanisms to allow
the transfer of the \Condor{schedd} daemon from one machine
to another.
The configuration specifies
which machines are chosen to run the \Condor{schedd} daemon.
To prevent multiple \Condor{schedd} daemons from running at the same time,
a lock (semaphore-like) is held over the job queue.
This synchronizes  the situation in which control is
transferred to a secondary machine,
and the primary machine returns to functionality.
Configuration variables also determine time intervals at which 
the lock expires,
and periods of time that pass between polling to check
for expired locks.

To specify a single machine that would take over, if the
machine running the \Condor{schedd} daemon stops working,
the following additions are made to the local configuration
of any and all machines that are able to run the \Condor{schedd} daemon
(becoming the single pool submission point): 

\begin{verbatim}
MASTER_HA_LIST = SCHEDD
SPOOL = /share/spool
HA_LOCK_URL = file:/share/spool
\end{verbatim}

Configuration macro \Macro{MASTER\_HA\_LIST} identifies the 
\Condor{schedd} daemon as the daemon that is to be watched
to make sure that it is running.
Each machine with this configuration must have access to the
lock (the job queue) which synchronizes which single machine does run the
\Condor{schedd} daemon.
This lock and the job queue must both be located in a shared file space,
and is currently specified only with a file URL.
The configuration specifies the shared space
(\MacroNI{SPOOL}),
and the URL of the lock.

As Condor starts on machines that are configured to run
the single \Condor{schedd} daemon, 
the \Condor{master} daemon of the
first machine that looks at (polls) the lock
notices that no lock is held.
This implies that no \Condor{schedd} daemon is running.
This \Condor{master} daemon acquires the lock
and runs the \Condor{schedd} daemon.
Other machines with this same capability to run the
\Condor{schedd} daemon look at (poll) the lock, 
but do not run the daemon, as the lock is held.
The machine running the \Condor{schedd} daemon renews the
lock periodically.

If the machine running the \Condor{schedd} daemon fails to renew
the lock (because the machine is not functioning),
the lock times out (becomes stale).
The lock is released by the \Condor{master} daemon
if \Condor{off} or \Prog{condor\_off -schedd} is
executed, or when the \Condor{master} daemon knows that the
\Condor{schedd} daemon is no longer running.
As other machines capable of running the \Condor{schedd} daemon
look at the lock (poll), one machine will be the first
to notice that the lock has timed out or been released.
This machine (correctly) interprets this situation as the
\Condor{schedd} daemon is no longer running.
This machine's \Condor{master} daemon then acquires the lock
and runs the \Condor{schedd} daemon.

See 
section~\ref{sec:Master-Config-File-Entries},
in the section on \Condor{master} Configuration File Macros
for details relating to the configuration variables
used to set timing and polling intervals.
