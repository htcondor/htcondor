%%%%%%%%%%%%%%%%%%%%%%%%%%%%%%%%%%%%%%%%%%%%%%%%%%%%%%%%%%%%%%%%%%%%%%%%%%%
\subsection{\label{sec:Multiple-Platforms}Configuring Condor for
Multiple Platforms} 
%%%%%%%%%%%%%%%%%%%%%%%%%%%%%%%%%%%%%%%%%%%%%%%%%%%%%%%%%%%%%%%%%%%%%%%%%%

Beginning with Condor version 6.0.1, you can use a single, global
config file for all platforms in your Condor pool, with only
platform-specific settings placed in separate files.  This greatly
simplifies administration of a heterogeneous pool by allowing you to
change platform-independent, global settings in one place, instead of
separately for each platform.  This is made possible by the
\Macro{LOCAL\_CONFIG\_FILE} parameter being treated by Condor as a
list of files, instead of a single file.  Of course, this will only
help you if you are using a shared filesystem for the machines in your
pool, so that multiple machines can actually share a single set of
configuration files.

If you have multiple platforms, you should put all
platform-independent settings (the vast majority) into your regular
condor\_config file, which would be shared by all platforms.  This
global file would be the one that is found with the
\Env{CONDOR\_CONFIG} environment variable, user condor's home
directory, or \File{/etc/condor/condor\_config}.

You would then set the \MacroNI{LOCAL\_CONFIG\_FILE} parameter from that
global config file to specify both a platform-specific config file and
optionally, a local, machine-specific config file (this parameter is
described in section~\ref{sec:Condor-wide-Config-File-Entries} on
``Condor-wide Config File Entries'').

The order in which you specify files in the
\MacroNI{LOCAL\_CONFIG\_FILE} parameter is important, because settings
in files at the beginning of the list are overridden if the same
settings occur in files later in the list.  So, if you specify the
platform-specific file and then the machine-specific file, settings in
the machine-specific file would override those in the
platform-specific file (which is probably what you want).  

%%%%%%%%%%%%%%%%%%%%%%%%%%%%%%%%%%%%%%%%%%%%%%%%%%%%%%%%%%%%%%%%%%%%%%%%%%%
\subsubsection{\label{sec:Specify-Platform-Files}Specifying a
Platform-Specific Config File} 
%%%%%%%%%%%%%%%%%%%%%%%%%%%%%%%%%%%%%%%%%%%%%%%%%%%%%%%%%%%%%%%%%%%%%%%%%%%

To specify the platform-specific file, you could simply use the
\Macro{ARCH} and \Macro{OPSYS} parameters which are defined
automatically by Condor.  For example, if you had Intel Linux
machines, Sparc Solaris 2.6 machines, and SGIs running IRIX 6.x, you
might have files named:

\begin{verbatim}
        condor_config.INTEL.LINUX
        condor_config.SUN4x.SOLARIS26
        condor_config.SGI.IRIX6
\end{verbatim}

Then, assuming these three files were in the directory held in the
\Macro{ETC} macro, and you were using machine-specific config files in
the same directory, named by each machine's hostname, your
\Macro{LOCAL\_CONFIG\_FILE} parameter would be set to:

\begin{verbatim}
  LOCAL_CONFIG_FILE = $(ETC)/condor_config.$(ARCH).$(OPSYS), \
                      $(ETC)/$(HOSTNAME).local
\end{verbatim}

Alternatively, if you are using AFS, you can use an ``@sys link'' to
specify the platform-specific config file and let AFS resolve this
link differently on different systems.  For example, perhaps you have
a soft linked named ``condor\_config.platform'' that points to
``condor\_config.@sys''.  In this case, your files might be named:

\begin{verbatim}
        condor_config.i386_linux2
        condor_config.sun4x_56
        condor_config.sgi_64
        condor_config.platform -> condor_config.@sys
\end{verbatim}

and your \MacroNI{LOCAL\_CONFIG\_FILE} parameter would be set to:

\begin{verbatim}
  LOCAL_CONFIG_FILE = $(ETC)/condor_config.platform, \
                      $(ETC)/$(HOSTNAME).local
\end{verbatim}

%%%%%%%%%%%%%%%%%%%%%%%%%%%%%%%%%%%%%%%%%%%%%%%%%%%%%%%%%%%%%%%%%%%%%%%%%%%
\subsubsection{\label{sec:Platform-Specific-Settings}Platform-Specific
Config File Settings}
%%%%%%%%%%%%%%%%%%%%%%%%%%%%%%%%%%%%%%%%%%%%%%%%%%%%%%%%%%%%%%%%%%%%%%%%%%%

The only settings that are truly platform-specific are:

\begin{description}

\item[\Macro{RELEASE\_DIR}] Full path to where you have installed your
  Condor binaries.  While the config files may be shared among
  different platforms, the binaries certainly cannot.  Therefore, you
  must still maintain separate release directories for each platform
  in your pool.  See section~\ref{sec:Condor-wide-Config-File-Entries}
  on ``Condor-wide Config File Entries'' for details.

\item[\Macro{MAIL}] The full path to your mail program.  See
  section~\ref{sec:Condor-wide-Config-File-Entries} on ``Condor-wide
  Config File Entries'' for details.

\item[\Macro{CONSOLE\_DEVICES}] Which devices in /dev should be
  treated as ``console devices''.  See
  section~\ref{sec:Startd-Config-File-Entries} on ``\condor{startd}
  Config File Entries'' for details.

\item[\Macro{DAEMON\_LIST}] Which daemons the \Condor{master} should
  start up.  The only reason this setting is platform-specific is
  because on Alphas running Digital Unix and SGIs running IRIX, you
  must use the \Condor{kbdd}, which is not needed on other platforms.
  See section~\ref{sec:Master-Config-File-Entries} on
  ``\condor{master} Config File Entries'' for details.

\end{description}

Reasonable defaults for all of these settings will be found in the
default config files inside a given platform's binary distribution
(except the \MacroNI{RELEASE\_DIR}, since it is up to you where you want
to install your Condor binaries and libraries).  If you have multiple
platforms, simply take one of the condor\_config files you get from
either running \Condor{install} or from the
\Release{etc/examples/condor\_config.generic} file, take these settings
out and save them into a platform-specific file, and install the
resulting platform-independent file as your global config file.  Then,
find the same settings from the config files for any other platforms
you are setting up and put them in their own platform specific files.
Finally, set your \MacroNI{LOCAL\_CONFIG\_FILE} parameter to point to
the appropriate platform-specific file, as described above.

Not even all of these settings are necessarily going to be different.
For example, if you have installed a mail program that understands the
``-s'' option in \File{/usr/local/bin/mail} on all your platforms, you
could just set \Macro{MAIL} to that in your global file and not define
it anywhere else.  If you've only got Digital Unix and IRIX machines,
the \Macro{DAEMON\_LIST} will be the same for each, so there's no
reason not to put that in the global config file (or, if you have no
IRIX or Digital Unix machines, \MacroNI{DAEMON\_LIST} won't have to be
platform-specific either).

%%%%%%%%%%%%%%%%%%%%%%%%%%%%%%%%%%%%%%%%%%%%%%%%%%%%%%%%%%%%%%%%%%%%%%%%%%%
\subsubsection{\label{sec:Other-Uses-for-Platform-Files}Other Uses for
Platform-Specific Config Files} 
%%%%%%%%%%%%%%%%%%%%%%%%%%%%%%%%%%%%%%%%%%%%%%%%%%%%%%%%%%%%%%%%%%%%%%%%%%%

It is certainly possible that you might want other settings to be
platform-specific as well.  Perhaps you want a different startd policy
for one of your platforms.  Maybe different people should get the
email about problems with different platforms.  There's nothing
hard-coded about any of this.  What you decide should be shared and
what should not is entirely up to you and how you lay out your config
files.

Since the \Macro{LOCAL\_CONFIG\_FILE} parameter can be an arbitrary
list of files, you can even break up your global, platform-independent
settings into separate files.  In fact, your global config file might
only contain a definition for \MacroNI{LOCAL\_CONFIG\_FILE}, and all
other settings would be handled in separate files.  

You might want to give different people permission to change different
Condor settings.  For example, if you wanted some user to be able to
change certain settings, but nothing else, you could specify those
settings in a file which was early in the \MacroNI{LOCAL\_CONFIG\_FILE}
list, give that user write permission on that file, then include all
the other files after that one.  That way, if the user was trying to
change settings she/he shouldn't, they would simply be overridden.  

As you can see, this mechanism is quite flexible and powerful.  If you
have very specific configuration needs, they can probably be met by
using file permissions, the \MacroNI{LOCAL\_CONFIG\_FILE} setting, and
your imagination.

