%%%%%%%%%%%%%%%%%%%%%%%%%%%%%%%%%%%%%%%%%%%%%%%%%%%%%%%%%%%%%%%%%%%%%%
\section{\label{sec:NT-Install}NT Installation of Condor}
%%%%%%%%%%%%%%%%%%%%%%%%%%%%%%%%%%%%%%%%%%%%%%%%%%%%%%%%%%%%%%%%%%%%%%

\index{installation!NT|(}
\index{NT installation|(}
This section contains the instructions for installing the NT version
Condor at your site.  
The installation sets up a default configuration which
may be further customized.

Please read the copyright and disclaimer information in 
section~\ref{sec:copyright-and-disclaimer} on
page~\pageref{sec:copyright-and-disclaimer} of the manual, or in the
file 
\File{LICENSE.TXT}, before proceeding.  Installation and
use of Condor is acknowledgement that you have read and agreed to these
terms.

The Condor NT executable for distribution is packaged in
a single file such as:
\begin{verbatim}
  condor-6.3.0-WINNT40-x86.exe
\end{verbatim}

\index{NT installation!initial file size}
This file is approximately 5 Mbytes in size, and may be
removed once Condor is fully installed.

Before installing Condor, please consider joining the condor-world mailing
list.  Traffic on this list is kept to an absolute minimum.  It is only
used to announce new releases of Condor.  To subscribe, send a message
to \Email{majordomo@cs.wisc.edu} with the body:
\begin{verbatim}
   subscribe condor-world 
\end{verbatim}

%%%%%%%%%%%%%%%%%%%%%%%%%%%%%%%%%%%%%%%%%%%%%%%%%%%%%%%%%%%%%%%%%%%%%%
\subsection{\label{sec:NT-Preparing-to-Install}Preparing to Install
Condor under NT} 
%%%%%%%%%%%%%%%%%%%%%%%%%%%%%%%%%%%%%%%%%%%%%%%%%%%%%%%%%%%%%%%%%%%%%%

\index{NT installation!preparation}
Before you install the NT version of Condor at your site,
there are only two
decisions to make about the basic layout of your pool.

\begin{enumerate}
\item What machine will be the central manager?
\item Do I have enough disk space for Condor?
\end{enumerate}

If you feel that you already know the answers to these questions,
skip to the NT Installation Procedure section below,
section~\ref{sec:nt-install-procedure} on
page~\pageref{sec:nt-install-procedure}.
If you are unsure, read on.

%%%%%%%%%%%%%%%%%%%%%%%%%%%%%%%%%%%%%%%%%%%%%%%%%%%%%%%%%%%%%%%%%%%%%%
\subsubsection{What machine will be the central manager?}
%%%%%%%%%%%%%%%%%%%%%%%%%%%%%%%%%%%%%%%%%%%%%%%%%%%%%%%%%%%%%%%%%%%%%%

One machine in your pool must be the central manager.
This is the
centralized information repository for the Condor pool and is also the
machine that matches available machines with waiting
jobs.  If the central manager machine crashes, any currently active
matches in the system will keep running, but no new matches will be
made.  Moreover, most Condor tools will stop working.  Because of the
importance of this machine for the proper functioning of Condor, we
recommend you install it on a machine that is likely to stay up all the
time, or at the very least, one that will be rebooted quickly if it
does crash.  Also, because all the services will send updates (by
default every 5 minutes) to this machine, it is advisable to consider
network traffic and your network layout when choosing the central
manager.

For Personal Condor, your machine will act as your central manager.

Install Condor on the central manager before intalling
on the other machines within the pool.

%%%%%%%%%%%%%%%%%%%%%%%%%%%%%%%%%%%%%%%%%%%%%%%%%%%%%%%%%%%%%%%%%%%%%%
\subsubsection{Do I have enough disk space for Condor?}
%%%%%%%%%%%%%%%%%%%%%%%%%%%%%%%%%%%%%%%%%%%%%%%%%%%%%%%%%%%%%%%%%%%%%%

\index{NT installation!required disk space}
The Condor release directory takes up a fair amount of space.
The size requirement for the release
directory is approximately 20 Mbytes.

%%%%%%%%%%%%%%%%%%%%%%%%%%%%%%%%%%%%%%%%%%%%%%%%%%%%%%%%%%%%%%%%%%%%%%
\subsection{\label{sec:nt-install-procedure}
Installation Procedure}
%%%%%%%%%%%%%%%%%%%%%%%%%%%%%%%%%%%%%%%%%%%%%%%%%%%%%%%%%%%%%%%%%%%%%%

% condor MUST be run as local system
% 
%  root == adminstrator
%  to install, must be running with administrator privileges
%  the kernel runs as == local system

Installation of Condor must
be done by a user with administrator privileges.
After installation, Condor will be run as local system.

Download Condor.
Run the downloaded file by double clicking on it.
This starts the installation process.
The Condor installation is completed by answering
questions and choosing options within the following steps.


\begin{description}
\item[If Condor is already installed.]

     For upgrade purposes, you may be running the installation of Condor
     after it has been previously installed.
     In this case, a dialog box will appear before the
     installation of Condor proceeds.
     The question asks if you wish to preserve your current
     Condor configuration files.
     Answer yes or no, as appropriate.

     If you answer no, then there will be a second question
     that asks if you want to use answers
     given during the previous intallation
     as default answers.

\item[STEP 1: License Agreement.]

     The first step in installation of Condor
     is a welcome and license agreement.
     You are reminded that it is best to run the installation
     when no other Windows programs are running.
     You are given the option of cancelling the installation
     in order to exit other Windows programs.
     You are asked to agree to the license.
     Answer yes or no.
     After agreeing to the license agreement,
     fill in the name and company information requested,
     or use the defaults as given.

     After agreeing to licensing, you will see a window that
     shows the installation unpacking various parts of the
     Condor system.  This will be followed by a setup window.

\item[STEP 2: The Condor Pool.]

     The Condor installation needs different
     information depending on whether the installation
     is to create a new pool or join an existing
     Condor pool.

     As you create a new Condor pool, the installation
     assumes this machine is the central manager.
     For the creation of a new Condor pool, enter
     information about the pool:
     \begin{description}
     \item[Name of the pool]
     \item[hostname]
     \item[Do you want to keep statistics?]
       Answer yes or no, as appropriate.
       If yes, then the maximum amount of data accumulated will
       be 10 Mbytes.
       A configurable quantity, \Macro{POOL\_HISTORY\_MAX\_STORAGE}
       sets the maximum amount of data, and it
       defaults to 10 Mbytes.
       If no, then the CondorView client will not have data to display.
     \item[Size of pool]
       Condor needs to know if this a Personal Condor installation,
       or if there will be more than one machine in the pool.
       Personal Condor
       implies that there is only one machine in the pool.
       For Personal Condor, several of the following
       steps are omitted.
     \end{description}

     An installation on a machine that will be part of an
     existing Condor
     pool requests the hostname of the central manager
     of the pool.

\item[STEP 3: Machine's use of Condor.] 

     This step is omitted for the installation of Personal Condor.

     Each machine within a Condor pool may either
     submit jobs or execute submitted jobs, or both
     submit and execute jobs.
     This step allows the installation on this machine
     to choose if the machine will only submit jobs,
     only execute submitted jobs, or both.
     The common case is both, so the default is both.

\item[STEP 4: Where will Condor be installed?]

\index{NT installation!location of files}

It is strongly recommended that Condor be installed in the
location shown as the default in the dialog box:
\File{C:\Bs Condor}.

Installation on the local disk is chosen for several reasons.

Condor runs as local system, and local system
has no network privileges.
Therefore, Condor should be installed on a local hard drive
as opposed to a network drive (file server),
because accesses to a network drive can not be
authenticated.

A second reason for installation on the local disk is that
the NT usage of drive letters has implications for where
Condor is placed.
The drive letter used must be dedicated for all users,
and it must remain at all times for Condor.

To place Condor on a hard drive that is not local,
a dependency must be added to the service control
manager such that Condor starts after file services
are available.

%  !! goes in C:/condor   (default)
%  !! advice is really should go on local hard drive,
%  as opposed to a network drive (also called file server)
%  Because,
%    1. Condor runs as local system, and accesses to a network
%      drive can't be authenticated  -- local system has
%      no network privileges.
%    2.  it is likely that you don't have this set up:
%    (and you need it to make it work)
%    you can add a dependency in the service control manager
%    that condor should start after the file services are
%    available
%    3. drive letters are "system-wide"
%    Must have dedicated letter (for all users), that remains
%    intact for all time, or condor won't know where
%    things are and can't get access (without its "letter")


\item[STEP 5: Where should Condor send e-mail if things go wrong?]

     Various parts of Condor will send e-mail to a Condor administrator
     if something goes wrong that needs human attention.
     You specify the e-mail address and the SMTP relay host
     of this administrator.

\item[STEP 6: The domain.]

% not really used right now.  "Things that suck about NT."
% UNIX has 2 domains:  file system domain and userID domain
% NT has only 1:  a combination, and so going back to letter
% drives, things get screwed up.
     This step is omitted for the installation of Personal Condor.

     Enter the machine's domain.
     It will be used for both a file system domain and
     a user ID (UID) domain.

\item[STEP 7: Access permissions.]
     This step is omitted for the installation of Personal Condor.

     Machines within the Condor pool will need
     various types of access permission. 
     The three categories of permission are read, write,
     and administrator. Enter the machines to be given
     access permissions.

     \begin{description}
     \item[Read]
     Read access allows a machine to obtain information about
     Condor such as the status of machines in the pool and the
     job queues.
     All machines in the pool should be given read access. 
     In addition, giving read access to *.cs.wisc.edu 
     will allow the Condor team to obtain information about
     your Condor pool in the event that debugging is needed.
     \item[Write]
     All machines in the pool should be given write access. 
     It allows machines to send information to Condor, for
     example, a ClassAd update.
     Note that any machine with both read and write access
     may join the Condor pool.
     \item[Administrator]
     A machine with administrator access will be allowed more
     extended permission to to things such as
     change user priorities, modify the job queue,
     turn Condor services on and off,
     and restart Condor.
     The central manager should be given administrator access
     and is the default listed.
     \end{description}

\item[STEP 8: Option for starting Condor jobs.]
     Condor will execute submitted jobs on machines based on
     a preference given at installation.
     Three options are given, and the first is most commonly used
     by Condor pools.
     This specification may be changed or refined in
     the machine ClassAd requirements attribute.

     The three choices:
     \begin{description}
     \item[After 15 minutes of no console activity and low CPU activity.]
     Console activity is the use of the mouse or keyboard.
     Low CPU activity is defined as a load of less than 30\Percent. 
     \item[After 15 minutes of no console activity.]
     \item[Always run Condor jobs.]
     \end{description}

\item[STEP 9: Option for stopping Condor jobs.]
     This step is omitted if Condor jobs are always run as
     the option chosen in STEP 8.

     If Condor is executing a job and the user returns,
     Condor needs to know what to do with the partially completed job.
     There are currently two options for the job.

     \begin{description}
     \item[The job is killed 5 minutes after your return.]
     The job is suspended immediately once there is console activity.
     If the activity continues, then the job is
     killed after 5 minutes, losing ?
     There is no checkpointing, so the job will
     be restarted from the beginning at a later time.
     The job will be placed back into the queue.
     \item[ Suspend job, leaving it in memory.]
     At a later time, when a Condor job may again be run,
     the suspended job is continued, starting from where
     it was when suspended.
     (this takes up swap space)
     \end{description}

%    Advice on which to choose goes here.
     Killing a job is less intrusive on the workstation owner
     than leaving it in memory for a later time.
     A suspended job left in memory will require swap space,
     possibly a scarce resource.
     Leaving a job in memory has the benefit that accumulated
     run time is not lost for a partially completed job.

\item[STEP 10: Review entered information.]
     Check that the entered information is correctly entered.
     You have the option to return to previous dialog boxes to fix entries.
\end{description}

\index{installation!NT|)}
\index{NT installation|)}

%%%%%%%%%%%%%%%%%%%%%%%%%%%%%%%%%%%%%%%%%%%%%%%%%%%%%%%%%%%%%%%%%%%%%%
\subsection{\label{nt-installed-now-what}
Condor is installed... now what?}
%%%%%%%%%%%%%%%%%%%%%%%%%%%%%%%%%%%%%%%%%%%%%%%%%%%%%%%%%%%%%%%%%%%%%%
\index{NT installation!starting Condor}

After the installation of Condor is completed, Condor can
be started.
From the Start menu, choose Settings.
From the Settings menu, choose Control Panel.
From the Control Panel, choose Services.
From Services, choose Condor, and Start.

Run the Task Manager (Control-Shift-Escape) to check that Condor
services, such as
\Condor{master}, ?, and ? are running.

%%%%%%%%%%%%%%%%%%%%%%%%%%%%%%%%%%%%%%%%%%%%%%%%%%%%%%%%%%%%%%%%%%%%%%
\subsection{\label{nt-running-now-what}
Condor is running... now what?}
%%%%%%%%%%%%%%%%%%%%%%%%%%%%%%%%%%%%%%%%%%%%%%%%%%%%%%%%%%%%%%%%%%%%%%

Once Condor services are running, try building
and submitting some test jobs.  See \File{C:examples/README}
for details.

