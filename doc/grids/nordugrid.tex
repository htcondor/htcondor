\index{NorduGrid|(}

%%%%%%%%%%%%%%%%%%%%%%%%%%%%%%%%%%%%%%%%%%%%%%%%%%%%%%%%%%%%%%%%%%%%%%%%%%%
\subsection{\label{sec:NorduGrid}The nordugrid grid\_type }
%%%%%%%%%%%%%%%%%%%%%%%%%%%%%%%%%%%%%%%%%%%%%%%%%%%%%%%%%%%%%%%%%%%%%%%%%%%

NorduGrid is a project to develop a free grid middleware called Advanced  
Resource Connector (ARC). See the NorduGrid web page
(\URL{http://www.nordugrid.org}) for more information about the NorduGrid
software.

You can submit jobs to NorduGrid resources using the \SubmitCmd{nordugrid}
\SubmitCmd{grid\_type} of the grid universe. The semantics are the same as
for other \SubmitCmd{grid\_type}s, except as noted here.

The command \SubmitCmd{nordugrid\_resource} is required. It specifies the
hostname of the NorduGrid resource to which the job should be submitted.

NorduGrid uses X.509 credentials for authentication, usually in the form a
proxy certificate. For more information about proxies and certificates,
please consult the Alliance PKI pages at
\URL{http://archive.ncsa.uiuc.edu/SCD/Alliance/GridSecurity/}.
\Condor{submit} will look in the default locations for the proxy. You can 
override this with the \SubmitCmd{x509userproxy} command.

NorduGrid uses a form of RSL (see the Condor-G section for a description
of RSL) to describe jobs. You can use the \SubmitCmd{nordugrid\_rsl}
command to add additional attribute settings to the job RSL that Condor
constructs. The format of the \SubmitCmd{nordugrid\_rsl} is
\begin{verbatim}
nordugrid_rsl = (name=value)(name=value)
\end{verbatim}
