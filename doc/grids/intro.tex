%%%%%%%%%%%%%%%%%%%%%%%%%%%%%%%%%%%%%%%%%%%%%%%%%%%%%%%%%%%%%%%%%%%%%%%%%%%
\section{\label{sec:grids-intro}An Introduction}
%%%%%%%%%%%%%%%%%%%%%%%%%%%%%%%%%%%%%%%%%%%%%%%%%%%%%%%%%%%%%%%%%%%%%%%%%%%

Grids refer to computing resources all over the world. 
Condor's purpose of providing computing cycles extends naturally
to grids.
To this end, Condor has the grid universe for jobs.
Within this universe, jobs may be specified for a variety
of grids.

An easy extension allows Condor jobs submitted within one pool
of machines to execute on another (separate) Condor pool.
Condor calls this flocking.
If a machine within the pool where a job is submitted is not
available to run the job,
the job makes its way to another pool.
This is enabled by the configuration of the pools.

Condor-C allows the use grid computing resources
wherever Condor is running and configured to allow
jobs.
Jobs submitted to Condor-C may relocate from one machine's
job queue to another machine's job queue.

Condor-G provides
grid computing features utilizing Globus software
(\URL{http://www.globus.org/}).
Globus provides infrastructure for authentication, authorization,
and remote job submission (including data transfer) on Grid resources.
Condor-G provides all of Condor's job submission features,
but for these far-removed resources.

