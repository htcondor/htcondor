%%%%%%%%%%%%%%%%%%%%%%%%%%%%%%%%%%%%%%%%%%%%%%%%%%%%%%%%%%%%%%%%%%%%%%%%%%%
\section{\label{sec:grids-intro}An Introduction}
%%%%%%%%%%%%%%%%%%%%%%%%%%%%%%%%%%%%%%%%%%%%%%%%%%%%%%%%%%%%%%%%%%%%%%%%%%%

A goal of grid computing is to allow utilization of resources that
span many administrative domains.
Although a Condor pool may include
resources owned and controlled by many different people,
it still
represents a certain amount of organization and cooperation.
If researchers from different organizations are collaborating,
it often is not feasible to combine all of their computers
into a single, large Condor pool.
Thus, grid computing provides ways for users to make use of
resources outside their local Condor pool.
Condor's history of working with non-dedicated and
distributively-owned resources
is a good fit for the grid computing environment.

Condor both has its own native mechanisms for grid computing,
and works well with other grid systems.

One native mechanism allows Condor jobs submitted within one pool
of machines to execute on another separate Condor pool.
Condor calls this \Term{flocking}.
Insufficient resources to run jobs in the local Condor pool
cause
Condor to look for available resources in remote Condor pools.
This is enabled by the configuration of the pools.

% Karen has partially editted to this point in this file.
With flocking, the user's local Condor scheduler remains directly
responsible for locating and talking with machines that will run its
jobs. 

With other mechanisms, the local scheduler delegates
responsibility to another scheduler.
The local scheduler is still
ultimately responsible for the job, but the remote scheduler is 
temporarily responsible for scheduling the job and keeps the local
scheduler apprised of the job's status.
These mechanisms are
collectively known and invoked with
the grid universe.
Within the grid universe, there
are multiple \Opt{grid\_type}s. Each \Opt{grid\_type}
represents a kind of remote
scheduling system with its own language that the local scheduler must
speak to interact with it.

The grid universe and \Expr{grid\_type = condor} 
is called Condor-C.
It allows the local Condor scheduler to submit a job to
a remote Condor scheduler (usually in a different Condor pool).
%Condor-C allows the use grid computing resources
%wherever Condor is running and configured to allow
%jobs.
Jobs submitted to Condor-C may relocate from one machine's
job queue to another machine's job queue.

Condor-G provides
grid computing features utilizing Globus software
(\URL{http://www.globus.org/}).
Globus provides infrastructure for authentication, authorization,
and remote job submission (including data transfer) on Grid resources.
Condor-G provides all of Condor's job submission features,
but for these far-removed resources.

Glidein builds on top of Condor-G.
It uses Condor-G to start up the Condor
execute daemons as a user job on remote machines.
The daemons then join
the local Condor pool, where they appear as ordinary
Condor resources that your jobs can run on.


With all these different options for grid computing, 
users may 
wonder which option to use.
The differences in the options themselves generally lead
to a choice of Condor's grid computing solution.

% Commented out by Karen, as she has no idea what these are.
% grid_type of batch and unicore do not exist in the manual, yet. . .
%Some situations preclude the use of some options completely. For example,
%if you're using the standard universe, you can't use the globus, batch, or
%unicore grid\_types by themselves. If the remote machines are available
%only via globus, that's your only option. But in most cases, you'll have
%some choice to make.

A quick comparison of the features and limitations of the different
mechanisms is shown in Table~\ref{grid-features}.

% grid computing feature/limitation table
\begin{center}
\begin{table}[hbt]
\begin{tabular}{|l||c|c|c|c|c|c|c|} \hline
 & \emph{Non-Vanilla Jobs} & \emph{Split Local/Remote} &
\emph{Late Binding} & \emph{Disconnect} & \emph{Private Network} &
\emph{Non-Condor} & \emph{No Config} \\ \hline \hline
Flocking & Yes & Yes & Yes & No & No & No & No \\ \hline
Condor-C & Maybe & No & No & Yes & Yes & Maybe & Maybe \\ \hline
Condor-G & No & No & No & Yes & Yes & Yes & Yes \\ \hline
Glidein & Yes & Yes & Yes & No & No & Yes & No \\ \hline
Batch & No & No & No & Yes & Yes & Yes & Yes \\ \hline
Unicore & No & No & No & Yes & Yes & Yes & Yes \\ \hline
\end{tabular}
\caption{\label{grid-features}Features and Limitations of Grid Mechanisms}
\end{table}
\end{center}

Non-Vanilla Jobs: Flocking and Glidein allow you to use all of the
non-grid Condor universes, allowing you to take advantage of all their
special features. With Condor-C, it depends on what type of job the job
becomes on the remote scheduler. If it becomes a Condor-G job, it has the
features and limitations of Condor-G. If it becomes a standard universe
job, the remote i/o calls will be directed to the remote machine, not the
original submit machine. The other mechanisms offer a vanilla-like
environment for the job.

Split Local/Remote: Flocking and Glidein allow jobs to be dynamically
split become resources in your local pool and remote resources. The
other mechanisms require you to declare that your jobs will be going to
a remote scheduler, precluding them from running on machines in the local
pool if they happen to become available.

Late Binding: With Flocking and Glidein, jobs are sent directly to an
execute machine when it's ready to start executing them. With the other
mechanisms, jobs are sent to a remote scheduler where they can sit idle
waiting for an execute machine. This can lead to load imbalances where
jobs sit idle at one remote scheduler while another remote scheduler has
access to available machines.

Disconnect: Flocking and Glidein require relatively continuous network
connectivity. While they can be configured to withstand short-lived
disconnection (standard universe can't withstand any disconnection),
new jobs won't be started during the disconnection and long-lived
disconnection can lead to aborting of running jobs. The other mechanisms
work well with network disconnect. All jobs forwarded to a remote
scheduler will continue to run normally.

% Once GCB is supported, mention it as a work-around for private networks
Private Network: Flocking and Glidein require that the scheduler and
execute machine be able to establish network connections to each other
(in both directions). This won't work if the execute machine is on a
private network. Firewalls can also pose problems, although it's possible
to configure them to allow the necessary connections. The other 
mechanisms can work with private networks, provided the remote scheduler
can communicate with the execute machines.

Non-Condor: Flocking requires that any remote resources are managed by
Condor. Condor-C requires a Condor scheduler to be running remotely,
though the resources can be managed by a different system that Condor
knows how to talk to.

No Config: Flocking requires the administrator of the remote pool to 
allow your scheduler access. Glidein requires the administrator of your
local pool to allow your Glidein daemons access. Condor-C requires you
to start a Condor scheduler on the remote machine (if there isn't one
there already) or the administrator of the remote pool to enable a
non-local authentication method for the remote Condor scheduler.

