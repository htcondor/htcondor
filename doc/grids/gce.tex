%%%%%%%%%%%%%%%%%%%%%%%%%%%%%%%%%%%%%%%%%%%%%%%%%%%%%%%%%%%%%%%%%%%%%%%%%%%
\subsection{\label{sec:Gce}The GCE Grid Type }
%%%%%%%%%%%%%%%%%%%%%%%%%%%%%%%%%%%%%%%%%%%%%%%%%%%%%%%%%%%%%%%%%%%%%%%%%%%
\index{Google Compute Engine}
\index{GCE grid jobs}
\index{grid computing!submitting jobs to GCE}
\index{grid type!gce}

HTCondor jobs may be submitted to the Google Compute Engine (GCE)
cloud service.
GCE is an on-line commerical service that provides
the rental of computers by the hour to run computational applications.
Its runs virtual machine images that have been uploaded to Google's
servers.
More information about Google Compute Engine is available at
\URL{http://cloud.google.com/Compute}.

%%%%%%%%%%%%%%%%%%%%%%%%%%%%%%%%%%%%%%%%%%%%%%%%%%%%%%%%%%%%%%%%%%%%%%%%%%%
\subsubsection{\label{sec:Gce-submit}GCE Job Submission}
%%%%%%%%%%%%%%%%%%%%%%%%%%%%%%%%%%%%%%%%%%%%%%%%%%%%%%%%%%%%%%%%%%%%%%%%%%%

HTCondor jobs are submitted to the GCE service
with the \SubmitCmd{grid} universe, setting the
\SubmitCmd{grid\_resource} command to \SubmitCmd{gce}, followed 
by the service's URL, your GCE project, and the GCE zone you 
wish to use.
The result should look something like this:
\begin{verbatim}
grid_resource = gce https://www.googleapis.com/compute/v1 my_proj us-central1-a
\end{verbatim}

Since the job is a virtual machine image,
most of the submit description file commands
specifying input or output files are not applicable.
The \SubmitCmd{executable} command is still required,
but its value is ignored. 
It can be used to identify different jobs in the output of \Condor{q}.

The VM image for the job must already reside in Google's Cloud Storage
service and be registered with GCE.
In the submit description file,
provide the full URL identifier for the image using \SubmitCmd{gce\_image}.

This grid type requires you to grant HTCondor permission to use your
Google account. The easiest way to do this is to use the \Prog{gcutil}
command-line tool distributed by Google.
You can find \Prog{gcutil} and documentation for it at
\URL{https://developers.google.com/compute/docs/gcutil}.
Once you have installed it, run \Prog{gcutil auth} and follow its
directions.
Once you are done, the tool will write authorization credentials to the
file \File{.gcutil\_auth} in your HOME directory.
You can use the \Arg{--credentials\_file} command-line argument to tell
\Prog{gcutil} to use a different file.

Once you have an authorization file, you must specify its location in
your submit description file, using \SubmitCmd{gce\_auth\_file}:
\begin{verbatim}
gce_auth_file = /path/to/auth-file
\end{verbatim}

GCE allows the choice of different hardware configurations 
for instances to run on.
Select which configuration to use for the \SubmitCmd{gce} grid type
with the \SubmitCmd{gce\_machine\_type} submit description file command.
HTCondor provides no default.

Each virtual machine instance can be given a unique set of metadata,
which consists of name/value pairs, similar to the environment variables
of regular jobs.
The instance can query its metadata via a well-known address.
This makes it easy for many instances to share the same VM image,
but perform different work.
This data can be specified to HTCondor in one of two ways.
First, the data can be provided directly in the submit description file 
using the \SubmitCmd{gce\_metadata} command.
The value should be a comma-separated list of name=value settings, like this:
\begin{verbatim}
gce_metadata = setting1=foo,setting2=bar
\end{verbatim}

Second, the data can be
stored in a file, and the file name is specified with the
\SubmitCmd{gce\_metadata\_file} submit description file command.
This second option allows a wider range of characters to be used in the
metadata values.
Each name=value pair should be on its own line.
No whitespace is removed from the lines, except the newline that
separates entries.

Both options can be used at the same time. Don't use the same
metadata name in both places.

%%%%%%%%%%%%%%%%%%%%%%%%%%%%%%%%%%%%%%%%%%%%%%%%%%%%%%%%%%%%%%%%%%%%%%%%%%%
\subsubsection{\label{sec:Gce-config}GCE Configuration Variables}
%%%%%%%%%%%%%%%%%%%%%%%%%%%%%%%%%%%%%%%%%%%%%%%%%%%%%%%%%%%%%%%%%%%%%%%%%%%

The following configuration parameters are specific to the \SubmitCmd{gce}
grid type. The values given are the defaults. You can set different values
in your HTCondor configuration files.

\footnotesize
\begin{verbatim}
GCE_GAHP     = $(SBIN)/gce_gahp
GCE_GAHP_LOG = /tmp/GceGahpLog.$(USERNAME)
\end{verbatim}
\normalsize
