\index{Condor-C|(}

%%%%%%%%%%%%%%%%%%%%%%%%%%%%%%%%%%%%%%%%%%%%%%%%%%%%%%%%%%%%%%%%%%%%%%%%%%%
\section{\label{sec:Condor-C}Condor-C}
%%%%%%%%%%%%%%%%%%%%%%%%%%%%%%%%%%%%%%%%%%%%%%%%%%%%%%%%%%%%%%%%%%%%%%%%%%%

\index{grid computing!Condor-C}
Condor-C allows jobs in one machine's job queue to
be moved to another machine's job queue.
These machines may be far removed from each other,
providing powerful grid computation mechanisms,
while requiring only Condor software and its configuration.

While jobs reside in a queue,
outages that affect networks and machines are somewhat
irrelevant.
An expected usage
sets up Personal Condor on a laptop,
submits some jobs that are sent to a Condor pool,
waits until the jobs are staged on the pool,
then turns off the laptop.
When the laptop reconnects at a later time,
any results can be pulled back.

Condor-C scales gracefully when compared with Condor's flocking
mechanism.
The machine upon which jobs are submitted
maintains a single process and network connection to a remote machine,
without regard to the number
of jobs queued or running.

%%%%%%%%%%%%%%%%%%%%%%%%%%%%%%%%%%%%%%%%%%%%%%%%%%%%%%%%%%%%%%%%%%%%%%%%%%%
\subsection{\label{sec:Condor-C-Config}Condor-C Configuration}
%%%%%%%%%%%%%%%%%%%%%%%%%%%%%%%%%%%%%%%%%%%%%%%%%%%%%%%%%%%%%%%%%%%%%%%%%%%
\index{Condor-C!configuration}
There are two aspects to configuration to enable the
submission and execution of Condor-C jobs.
These two aspects correspond to the endpoints of the 
communication: there is the machine from which jobs are
submitted, and there is the remote machine upon which the
jobs are placed in the queue (executed).

Configuration of a machine from which jobs are submitted
requires two extra configuration variables:

\footnotesize
\begin{verbatim}
CONDOR_GAHP=$(SBIN)/condor_c-gahp
C_GAHP_LOG=/tmp/CGAHPLog.$(USERNAME)
\end{verbatim}
\normalsize

The acronym GAHP stands for Grid ASCII Helper Protocol.
A GAHP server provides grid-related services for a
variety of underlying middle-ware systems.
The configuration variable \Macro{CONDOR\_GAHP}
gives a full path to the GAHP server utilized by Condor-C.
The configuration variable \Macro{C\_GAHP\_LOG} defines
the location of the log that the Condor GAHP server writes.

A submit machine must also have a \Condor{collector} daemon to which the
\Condor{schedd} daemon can submit a query.
The query is for the location (IP address and port)
of the intended remote machine's \Condor{schedd} daemon.
This facilitates communication between the two machines.

The machine upon which jobs are executed 
must also be configuration correctly.
This machine must be running a \Condor{schedd} daemon.
Unless specified explicitly in a submit file, 
\MacroUNI{CONDOR\_HOST} must point to a 
\Condor{collector} daemon that it can write to,
and the machine upon which jobs are submitted can read from.
This facilitates communication between the two machines.

An important aspect of configuration is the security 
configuration relating to authentication.
Condor-C on the remote machine relies on an
authentication protocol to
know the identity of the user under which to run a job.
The following is a working example
of the portion of configuration that sets up authentication.

\footnotesize
\begin{verbatim}
SEC_DEFAULT_NEGOTIATION = OPTIONAL
SEC_DEFAULT_AUTHENTICATION_METHODS = CLAIMTOBE
\end{verbatim}
\normalsize


%%%%%%%%%%%%%%%%%%%%%%%%%%%%%%%%%%%%%%%%%%%%%%%%%%%%%%%%%%%%%%%%%%%%%%%%%%%
\subsection{\label{sec:Condor-C-Submit}Condor-C Job Submission}
%%%%%%%%%%%%%%%%%%%%%%%%%%%%%%%%%%%%%%%%%%%%%%%%%%%%%%%%%%%%%%%%%%%%%%%%%%%
\index{Condor-C!job submission}
\index{universe!grid}
Job submission of Condor-C jobs is the same as for any Condor job.
The \Opt{universe} is \Expr{grid},
and the \Opt{grid\_type} is \Expr{condor}. 
The remote machine where the job goes is specified in
the submit description file with \Opt{remote\_schedd}.
Its value is the same as the machine ClassAd attribute
\Attr{Name} on the remote machine.

The following represents a minimal submit description file for
a job.

\footnotesize
\begin{verbatim}
   # minimal submit description file for a Condor-C job
   universe = grid
   grid_type = condor
   executable = myjob
   output = myoutput
   error = myerror
   log = mylog

   remote_schedd = joe@remotemachine.example.com
   +remote_jobuniverse = 5
   +remote_requirements = True
   queue
\end{verbatim}
\normalsize

The remote machine needs to understand attributes of the job.
These are specified in the submit description file using the '+'
syntax, followed by the string \AdStr{remote\_}.
At a minimum, this will be the job's \Opt{universe} and the job's
\Opt{requirements}.
The \Opt{universe} is specified using an integer assigned for
a job ClassAd \Attr{JobUniverse}. 

As Condor-C is a recent addition to Condor,
the universes, associated integer assignments,
and notes about the existence of functionality are given in 
Table~\ref{working-remote-universes}.
The note "untested" implies that
submissions under the given universe have not yet
been throughly tested.
They may already work.

% universes, and whether they work in Condor-C
\begin{center}
\begin{table}[hbt]
\begin{tabular}{|l|l|l}
\textbf{Universe Name} & \textbf{Value} & \textbf{Notes}\\ \hline \hline
standard  & 1 & untested \\ \hline
PVM       & 4 & untested \\ \hline
vanilla   & 5 & works well \\ \hline
scheduler & 7 & does \emph{not} work \\ \hline
MPI       & 8 & untested \\ \hline
grid      & 9 & \\
 & grid\_type = condor & works well \\
 & grid\_type = gt2  & untested \\
 & grid\_type = gt3 & untested \\
 & grid\_type = nordugrid & untested \\
 & grid\_type = oracle & untested \\ \hline
java & 10 & untested \\ \hline
\end{tabular}
\caption{\label{working-remote-universes}Functionality of remote job universes with Condor-C}
\end{table}
\end{center}

File transfer of a job's executable, \File{stdin}, \File{stdout}, and
\File{stderr} are automatic.
%If other files are needed,
%then . . .
% FINISH THIS PARAGRAPH!

For communication between \Condor{schedd} daemons on the submit
and remote machines,
a location of the remote \Condor{schedd} daemon is needed.
By default, the submit machine queries its \Condor{collector}
daemon for the needed information.
However, a newly defined command for use within the submit
description file can override the default.
An example of this submit command is
\footnotesize
\begin{verbatim}
remote_pool = machine1.example.com
\end{verbatim}
\normalsize
When a command such as this appears within the submit description
for the job,
it defines a different \Condor{collector} daemon 
(assumed to be listening at a standard IP address and port)
to ask for location information about the remote machine.

%%%%%%%%%%%%%%%%%%%%%%%%%%%%%%%%%%%%%%%%%%%%%%%%%%%%%%%%%%%%%%%%%%%%%%%%%%%
\subsection{\label{sec:Condor-C-Limits}Current Limitations in Condor-C}
%%%%%%%%%%%%%%%%%%%%%%%%%%%%%%%%%%%%%%%%%%%%%%%%%%%%%%%%%%%%%%%%%%%%%%%%%%%
\index{Condor-C!limitations}
Submitting jobs to run under the grid universe has not yet
been perfected.
The following is a list of known limitations with Condor-C:

\begin{enumerate}
  \item{Condor-C currently does not work as root.
  The \Condor{schedd} daemon running on the remote machine can not be
  executed as root.
  A permissions problem prevents it from working correctly.  }

  \item{Authentication methods other than
  \Expr{CLAIMTOBE}, such as \Expr{GSI} and \Expr{KERBEROS}, are 
  untested, and may not yet work.}

  \item{Platforms that run a Windows operating system
are not yet supported as either a submit or remote execute
machine.}
\end{enumerate}

\index{Condor-C|)}


